\documentclass[12pt,a4paper,oneside]{book}

% ABNT formatting packages
\usepackage[utf8]{inputenc}
\usepackage[T1]{fontenc}
\usepackage[english]{babel}
\usepackage{indentfirst}
\usepackage{setspace}
\usepackage{graphicx}
\usepackage{float}
\usepackage[left=3cm,right=2cm,top=3cm,bottom=2cm]{geometry}
\usepackage{times}
\usepackage{caption}
\usepackage{subcaption}
\usepackage{amsmath}
\usepackage{amssymb}
\usepackage{listings}
\usepackage[table]{xcolor}
\usepackage{url}
\usepackage{hyperref}
\usepackage{titlesec}
\usepackage{tocloft}
\usepackage{longtable}
\usepackage{array}
\usepackage{fancyhdr}

% ABNT-style formatting
\onehalfspacing
\setlength{\parindent}{1.25cm}

% Chapter and section formatting (ABNT style)
\titleformat{\chapter}[display]
  {\normalfont\bfseries\fontsize{12pt}{14pt}\selectfont}
  {\MakeUppercase{\chaptertitlename\ \thechapter}}{0pt}
  {\MakeUppercase}
\titlespacing*{\chapter}{0pt}{0pt}{12pt}

% Unnumbered chapter formatting (for Acknowledgments, Abstract, etc.)
\titleformat{name=\chapter,numberless}[display]
  {\normalfont\bfseries\fontsize{12pt}{14pt}\selectfont}
  {}{0pt}
  {\MakeUppercase}
\titlespacing*{name=\chapter,numberless}{0pt}{0pt}{12pt}

\titleformat{\section}
  {\normalfont\bfseries\fontsize{12pt}{14pt}\selectfont}
  {\thesection}{1em}{}
\titlespacing*{\section}{0pt}{12pt}{6pt}

\titleformat{\subsection}
  {\normalfont\bfseries\fontsize{12pt}{14pt}\selectfont}
  {\thesubsection}{1em}{}
\titlespacing*{\subsection}{0pt}{12pt}{6pt}

\titleformat{\subsubsection}
  {\normalfont\bfseries\fontsize{12pt}{14pt}\selectfont}
  {\thesubsubsection}{1em}{}
\titlespacing*{\subsubsection}{0pt}{12pt}{6pt}

% Caption formatting (ABNT style)
\captionsetup{
  format=plain,
  labelsep=endash,
  font=small,
  labelfont=bf,
  justification=justified,
  singlelinecheck=false
}

% Code listing style
\lstset{
  language=Python,
  basicstyle=\ttfamily\footnotesize,
  keywordstyle=\color{blue},
  commentstyle=\color{green!60!black},
  stringstyle=\color{red},
  numbers=left,
  numberstyle=\tiny\color{gray},
  stepnumber=1,
  numbersep=10pt,
  backgroundcolor=\color{white},
  showspaces=false,
  showstringspaces=false,
  showtabs=false,
  frame=single,
  tabsize=2,
  captionpos=b,
  breaklines=true,
  breakatwhitespace=false,
  escapeinside={(*@}{@*)},
  xleftmargin=2em,
  framexleftmargin=1.5em
}

% Hyperref configuration
\hypersetup{
    colorlinks=true,
    linkcolor=black,
    citecolor=black,
    filecolor=black,
    urlcolor=black,
    pdftitle={Constraint-Aware Fault-Tolerant Multi-Agent UAV System Using OODA Loop},
    pdfauthor={Vítor Eulálio Reis},
}

% Page numbering configuration (top right throughout document)
\pagestyle{fancy}
\fancyhf{} % Clear all header and footer fields
\fancyhead[R]{\thepage} % Page number in top right
\renewcommand{\headrulewidth}{0pt} % Remove header rule line
\renewcommand{\footrulewidth}{0pt} % Remove footer rule line

% Apply fancy style to plain pages (chapter opening pages)
\fancypagestyle{plain}{%
    \fancyhf{} % Clear all header and footer fields
    \fancyhead[R]{\thepage} % Page number in top right
    \renewcommand{\headrulewidth}{0pt} % Remove header rule line
    \renewcommand{\footrulewidth}{0pt} % Remove footer rule line
}

\begin{document}

% ====================================
% COVER PAGE
% ====================================
\begin{titlepage}
\begin{center}
\textbf{\uppercase{Universidade de São Paulo}}\\
\textbf{\uppercase{Escola de Engenharia de São Carlos}}

\vspace{8cm}

\textbf{\uppercase{Vítor Eulálio Reis}}

\vspace{4cm}

\textbf{Constraint-Aware Fault-Tolerant Multi-Agent UAV System Using OODA Loop: Realistic Mission Completion Assistance}

\vfill

São Carlos\\
2025
\end{center}
\end{titlepage}

% ====================================
% TITLE PAGE
% ====================================
\newpage
\thispagestyle{empty}
\begin{center}
\textbf{Vítor Eulálio Reis}

\vspace{8cm}

\textbf{Constraint-Aware Fault-Tolerant Multi-Agent UAV System Using OODA Loop}

\vspace{3cm}

\begin{minipage}{8cm}
\begin{flushleft}
Monograph presented to the Specialization Course in Aeronautical Systems, School of Engineering of São Carlos, University of São Paulo, as part of the requirements for obtaining the title of Specialist.

Advisor: Prof. João Paulo Eguea, PhD
\end{flushleft}
\end{minipage}

\vspace{2cm}

FINAL VERSION

\vfill

São Carlos\\
2025
\end{center}

% ====================================
% COPYRIGHT PAGE
% ====================================
\newpage
\thispagestyle{empty}
\vspace*{10cm}
\begin{center}
I AUTHORIZE THE TOTAL OR PARTIAL REPRODUCTION OF THIS WORK,\\
BY ANY CONVENTIONAL OR ELECTRONIC MEANS, FOR STUDY\\
AND RESEARCH PURPOSES, PROVIDED THE SOURCE IS CITED.
\end{center}

\vfill

\begin{center}
\begin{minipage}{12cm}
\begin{flushleft}
Cataloging card prepared by the Library Prof. Dr. Sérgio Rodrigues Fontes at EESC/USP with data provided by the author(s).

\vspace{0.5cm}

\noindent Eulálio Reis, Vítor

\hspace{0.5cm} Constraint-Aware Fault-Tolerant Multi-Agent UAV System Using OODA Loop: Realistic Mission Completion Assistance. / Vítor Eulálio Reis; advisor Prof. João Paulo Eguea, PhD. São Carlos, 2025.

\vspace{0.5cm}

\hspace{0.5cm} Specialization (Specialization in Aeronautical Systems) -- School of Engineering of São Carlos, University of São Paulo, 2025.

\vspace{0.5cm}

\hspace{0.5cm} 1. Drone. 2. OODA Loop. 3. Fault Tolerance. 4. Multi-Agent Systems. I. Title.
\end{flushleft}
\end{minipage}
\end{center}

% ====================================
% APPROVAL PAGE
% ====================================
\newpage
\thispagestyle{empty}
\begin{center}
\textbf{\uppercase{Approval Sheet}}

\vspace{2cm}

\textbf{APPROVAL SHEET}\\
\textit{Approval sheet}

\vspace{2cm}

\begin{tabular}{|p{13cm}|}
\hline
\textbf{Candidate / Student:} Vítor Eulálio Reis \\
\hline
\textbf{Title of TCC / Title:} Constraint-Aware Fault-Tolerant Multi-Agent UAV System Using OODA Loop \\
\hline
\textbf{Defense date / Date:} October 18, 2025 \\
\hline
\end{tabular}

\vspace{2cm}

\begin{tabular}{|p{10cm}|p{3cm}|}
\hline
\textbf{Examining Committee} & \textbf{Result} \\
\hline
{Prof° João Paulo Eguea, PhD} & submitted \\
\hline
\textbf{Affiliation:} School of Engineering of São Carlos / EESC-USP & \\
\hline
{Prof° Jorge Bidinotto, PhD} & submitted \\
\hline
\textbf{Affiliation:} School of Engineering of São Carlos / EESC-USP & \\
\hline
\end{tabular}

\vspace{2cm}

Chair of the Examining Committee:

\vspace{1cm}

\begin{center}
\rule{6cm}{0.4pt}\\
{Prof° João Paulo Eguea, PhD}\\
(Signature)
\end{center}

\end{center}

% ====================================
% DEDICATION (Optional)
% ====================================
\newpage
\thispagestyle{empty}
\vspace*{15cm}
\begin{flushright}
\begin{minipage}{8cm}
\textit{To my family, for their unconditional support throughout this endeavor.}
\end{minipage}
\end{flushright}

% ====================================
% ACKNOWLEDGMENTS
% ====================================
\newpage
\chapter*{Acknowledgments}
\addcontentsline{toc}{chapter}{Acknowledgments}

To the professors of the Especialização em Sistemas Aeronáuticos at the Escola de Engenharia de São Carlos, USP, for sharing their knowledge, expertise, craft, and for their dedication in supporting students throughout the course.

To Prof. Dr. João Paulo Eguea for his mentorship in guiding this research.

To Prof. Dr. Jorge Bidinotto for his leadership in managing the Specialization Program.

\vspace{0.5cm}

\noindent\textbf{AI Disclosure:} This work was developed with assistance from Claude (Anthropic), a large language model. AI tools were used to support code development, debugging, documentation, and text refinement. All technical decisions, system architecture, algorithm design, experimental validation, and intellectual contributions remain solely the author's responsibility.

% ====================================
% ABSTRACT
% ====================================
\newpage
\chapter*{Abstract}
\addcontentsline{toc}{chapter}{Abstract}

\noindent EULÁLIO REIS, V. \textbf{Constraint-Aware Fault-Tolerant Multi-Agent UAV System Using OODA Loop: Realistic Mission Completion Assistance}. 2025. Monograph (Specialization) – School of Engineering of São Carlos, University of São Paulo, São Carlos, 2025.

\vspace{0.5cm}

The failure of a UAV during mission execution creates an immediate resource allocation challenge: reassigning orphaned tasks to operational vehicles while respecting battery limitations, payload capacity, and regulatory constraints that academic research often ignores. This work develops a fault-tolerant control system for multi-UAV fleets that addresses this challenge under realistic operational conditions.

The system applies Boyd's OODA loop (Observe-Orient-Decide-Act) to transform vehicle failures into recoverable events. Upon detecting a fault, the system evaluates remaining fleet capacity, prioritizes orphaned tasks, and reallocates them to healthy vehicles while enforcing safety margins. When physical or regulatory limitations make full recovery impossible, the system escalates to human operators with quantified impact assessments rather than forcing unsafe autonomous decisions.

Three technical contributions emerge from this work: a resource-aware reallocation algorithm that jointly considers battery reserves, payload limits, and collision avoidance; a priority-based partial coverage strategy that gracefully degrades mission objectives when complete recovery is infeasible; and an intelligent escalation framework that distinguishes compensable failures from those requiring human judgment.

Experimental validation demonstrates adaptation times under 0.5 milliseconds---four orders of magnitude faster than the design target---with complete task recovery in surveillance and search-and-rescue scenarios. In time-critical applications, this speed advantage translates to preserved lives during the golden hour. The approach prioritizes honest performance over inflated claims: acknowledging what autonomous systems cannot do proves as valuable as demonstrating what they can.

\vspace{0.5cm}

\noindent \textbf{Keywords:} Drone. Fault Tolerance. OODA Loop. Multi-Agent Systems. Mission Planning.

% ====================================
% LIST OF FIGURES
% ====================================
\newpage
\listoffigures
\addcontentsline{toc}{chapter}{List of Figures}

% ====================================
% LIST OF TABLES
% ====================================
\newpage
\listoftables
\addcontentsline{toc}{chapter}{List of Tables}

% ====================================
% LIST OF ABBREVIATIONS
% ====================================
\newpage
\chapter*{List of Abbreviations and Acronyms}
\addcontentsline{toc}{chapter}{List of Abbreviations and Acronyms}

\begin{tabular}{ll}
BVLOS & Beyond Visual Line of Sight \\
FSM & Finite State Machine \\
GCS & Ground Control Station \\
GPS & Global Positioning System \\
MILP & Mixed Integer Linear Programming \\
OODA & Observe-Orient-Decide-Act \\
RF & Radio Frequency \\
RTL & Return-to-Launch \\
RVO & Reciprocal Velocity Obstacles \\
SAR & Search and Rescue \\
TSP & Traveling Salesman Problem \\
UAV & Unmanned Aerial Vehicle \\
\end{tabular}

% ====================================
% TABLE OF CONTENTS
% ====================================
\newpage
\tableofcontents

% ====================================
% MAIN CONTENT
% ====================================
\newpage
\setcounter{page}{1}
\pagenumbering{arabic}

\chapter{Introduction}

\section{The Reality Gap in Multi-Agent UAV Research}

Over the last decade, Unmanned Aerial Vehicles (UAVs) have progressed from experimental platforms to essential tools in logistics, environmental monitoring, and emergency response. Multi-UAV coordination, in particular, has emerged as a key enabler of scalable autonomy, allowing fleets to execute missions more efficiently and robustly than any single vehicle could achieve alone.

Despite this progress, much of the research in fault-tolerant multi-agent systems still operates within idealized boundaries (ZHANG; JIANG, 2008; YU; JIANG, 2015). Academic frameworks often presume conditions that are far removed from operational reality: unlimited energy reserves, unrestricted payload capacity, instantaneous communication, and fully autonomous decision authority. These simplifying assumptions make theoretical analysis tractable, but they mask the complexities that dominate real-world deployments.

In practice, UAV operations are governed by strict physical and regulatory limits. Batteries must retain safety reserves of 10–20\% to satisfy aviation standards. Payload resources, whether spray tanks or delivery packages, deplete irreversibly during flight and cannot be replenished mid-mission. Communication links introduce latencies of up to two seconds, and mission execution beyond visual line of sight requires constant operator supervision.

The result is a persistent ``reality gap'' between simulated autonomy and deployable autonomy. Systems that appear reliable in laboratory conditions may struggle in the field when confronted with degraded batteries, intermittent links, or competing mission deadlines. Closing this gap requires architectures that are not only intelligent but also resilient—capable of adapting to uncertainty and resource constraints without losing situational awareness or regulatory compliance.

\section{The OODA Loop: From Combat Theory to Autonomous Resilience}

The concept of the OODA loop—Observe, Orient, Decide, Act—originated from U.S. Air Force Colonel John Boyd's studies of aerial combat dynamics (BOYD, 1987). Boyd proposed that victory depends not merely on speed or strength, but on the ability to process information and adapt more quickly than the opponent. The OODA loop thus represents a continuous cycle of perception, reasoning, and action, where each iteration refines understanding and enhances responsiveness to change.

In the context of autonomous systems, the OODA loop provides a natural metaphor for adaptive control. The ``Observe'' phase involves collecting data from sensors and system telemetry; ``Orient'' interprets this data to establish situational awareness; ``Decide'' selects a course of action under the prevailing constraints; and ``Act'' executes that decision, feeding its outcomes back into the next cycle. Crucially, the loop is not linear but recursive: every action reshapes the environment that the next observation must interpret.

Applied to UAV fleets, this framework extends beyond traditional feedback control by integrating situational reasoning and contextual adaptation. Each vehicle, and the system as a whole, can maintain awareness not only of environmental conditions but also of internal resource states, communication health, and mission progress. The OODA architecture thus provides a foundation for resilient autonomy, where the fleet continuously senses its collective state, interprets disruptions, and reorganizes tasks in real time.

Most existing UAV coordination systems implement a subset of these principles, emphasizing either rapid reaction or long-term planning, but rarely both (BALA et al., 2025). What remains underexplored is the integration of the OODA loop into a real-time, resource-constrained, operator-supervised control system—one that can detect faults, reorient mission objectives, and adapt dynamically while remaining within the operational bounds of safety and regulation.

\section{Bridging the Gap: OODA-Based Fault-Tolerant Mission Control}

Despite its theoretical appeal, the OODA framework has rarely been implemented as a real-time operational control system for UAV fleets. Most prior work either focuses on single-vehicle fault recovery (MUELLER; D'ANDREA, 2014; SUN et al., 2022) or treats multi-agent coordination under ideal conditions with unlimited resources (LI et al., 2017; YANG et al., 2011). Architectures for fault-tolerant multi-robot cooperation such as ALLIANCE (PARKER, 1998) have demonstrated resilience principles, but few systems integrate real-world resource constraints, probabilistic fault detection, and operator-in-the-loop supervision within a unified architecture.

This research addresses that gap through the development of a hybrid OODA-based fault-tolerant mission control system designed for real-time UAV fleet management. The system operates as a hybrid finite-state machine (FSM) that continuously cycles between monitoring and adaptation modes, combining deterministic state transitions with probabilistic failure identification to maintain robust mission execution under realistic constraints.

\section{System Overview and Contributions}

\subsection{Hybrid OODA–FSM Architecture}

At its core, the system functions as a hybrid finite-state controller implementing continuous monitoring at 2 Hz and invoking the OODA cycle upon failure detection. Deterministic state logic ensures predictable mission flow, while probabilistic reasoning handles uncertain or delayed telemetry data.

\subsection{Real-Time Failure Detection}

Effective fault tolerance begins with timely awareness. The system implements a multi-modal failure detection layer that continuously monitors fleet health through complementary sensing channels, enabling rapid identification of anomalies before they cascade into mission-critical failures.

At the foundation lies a 2 Hz telemetry polling mechanism that aggregates position estimates, battery state-of-charge readings, task completion progress, and communication heartbeat timestamps from each UAV. This continuous stream feeds three distinct detection pathways. The first monitors communication health: any telemetry gap exceeding 1.5 seconds triggers an immediate timeout alert, capturing link failures or vehicle loss-of-control events. The second pathway processes explicit fault codes propagated from onboard diagnostics—motor controller warnings, sensor degradation flags, or navigation system errors reported directly by the vehicle's flight stack.

The third and most sophisticated pathway employs statistical anomaly detection to identify subtle degradation patterns that precede catastrophic failure. Battery discharge rates exceeding 5\% per 30-second window suggest thermal runaway or cell damage; position discontinuities greater than 100 meters between consecutive updates indicate GPS spoofing, sensor fusion failures, or uncontrolled flight; altitude deviations beyond mission-defined safety envelopes reveal environmental disturbances or control system instability. By fusing these heterogeneous indicators, the detection layer achieves sub-second fault identification while minimizing false positives that would otherwise burden operators with spurious alerts.

\subsection{OODA Execution for Fault Recovery}

Upon fault detection, the system transitions from passive monitoring into active recovery through a structured OODA cycle. This process—completing in 4 to 5.5 seconds—transforms raw failure events into executable recovery plans while preserving mission continuity.

The cycle begins with the \textbf{Observe} phase (1.0–1.5 seconds), during which the system aggregates fresh telemetry from all operational vehicles, formally identifies which UAVs have failed, quantifies the resulting mission coverage loss, and establishes authoritative failure timestamps for logging and post-mission analysis. This snapshot provides the situational foundation upon which subsequent reasoning depends.

The \textbf{Orient} phase (1.0–1.5 seconds) translates observation into understanding. Here, the system evaluates the remaining fleet's collective capacity—factoring in current battery reserves, payload availability, and temporal constraints imposed by mission deadlines. Tasks orphaned by the failure undergo re-prioritization based on urgency weights and spatial proximity to healthy vehicles, establishing a ranked queue of recovery candidates.

Decision-making crystallizes in the \textbf{Decide} phase (1.0–1.5 seconds), where the system classifies recovery feasibility and selects an appropriate strategy. When 75\% or more of lost tasks can be absorbed by the remaining fleet, full autonomous reallocation proceeds. Partial reallocation addresses scenarios where 50–75\% recovery is achievable, focusing resources on the highest-priority tasks while flagging coverage gaps. Below the 50\% threshold, the system acknowledges its limitations and escalates to the human operator, presenting decision options rather than forcing suboptimal autonomous choices.

Finally, the \textbf{Act} phase (0.5–1.5 seconds) executes the selected strategy: dispatching updated waypoint sequences to reassigned vehicles, awaiting acknowledgment confirmations, and refreshing the operator dashboard with current mission state. The brevity of this phase reflects the computational efficiency of the preceding stages—by the time execution begins, all planning complexity has been resolved.

\subsection{Constraint-Aware Strategy Layer}

The three recovery strategies represent fundamentally different operational philosophies, each tailored to the severity and characteristics of the failure scenario.

\textbf{Full Reallocation} engages when the fleet retains sufficient capacity to absorb all orphaned tasks. The system executes a constraint-aware optimization routine (detailed in Algorithm 2) that integrates new waypoints into existing flight plans while minimizing total travel distance through heuristic TSP solutions. This strategy maximizes mission completion without operator intervention, treating the failure as a temporary disruption rather than a mission-altering event.

\textbf{Partial Reallocation} acknowledges resource limitations honestly. When complete recovery proves impossible, the system prioritizes tasks exceeding a 0.7 priority threshold—ensuring that mission-critical objectives receive available resources while lower-priority tasks are deferred or abandoned. Crucially, the system issues explicit coverage-gap alerts quantifying the impact: operators receive not merely notification that tasks were dropped, but precise metrics on what percentage of the mission objective remains achievable.

\textbf{Operator Escalation} activates when autonomous decision-making would produce unacceptable outcomes. Rather than forcing a poor solution, the system initiates a supervised decision protocol, presenting the operator with contextualized options: deploying reserve UAVs, extending mission timelines, accepting degraded coverage, or executing a controlled mission abort. A 30-second countdown ensures timely human response while providing an automatic failsafe—if no operator input arrives, the system defaults to the safest available action, preserving vehicle integrity over mission completion.

\subsection{Performance and Scalability}

The complete OODA cycle achieves reaction times between 4 and 5.5 seconds, representing a 75–150$\times$ speed improvement compared to manual operator intervention (typically 5–10 minutes). Computationally, the system scales linearly with fleet size for monitoring and resource evaluation, and quadratically for collision avoidance checks—supporting real-time performance for up to 12 UAVs without exceeding sub-6-second response thresholds.

\section{Summary}

By grounding UAV mission control in the OODA decision cycle, this research introduces a system capable of adaptive, explainable, and regulatorily compliant autonomy. The framework unites rapid machine-driven response with human-supervised oversight, offering a practical balance between safety and autonomy.

In doing so, it bridges the long-standing gap between theoretical multi-agent fault tolerance and real-world deployability—demonstrating that reactive, resource-aware autonomy is not merely a conceptual goal, but an achievable engineering reality for the next generation of UAV fleet operations.

\chapter{System Architecture}

\section{Centralized OODA Architecture with Distributed Execution}

The system implements a hierarchical control structure in which the Ground Control Station hosts the OODA Loop Engine for centralized decision-making while UAVs execute tasks with operational autonomy. This architecture balances the optimization advantages of global fleet visibility against the responsiveness requirements of distributed execution, a design choice grounded in both regulatory constraints and practical deployment considerations.

\begin{figure}[h!tbp]
    \centering
    \includegraphics[width=\textwidth, height=0.9\textheight, keepaspectratio]{images/architecture.png}
    \caption{Architecture Overview}
    \label{fig:architecture}
\end{figure}

\subsection{Design Rationale and Trade-offs}

The centralized OODA architecture derives from multiple convergent factors. Regulatory compliance for Beyond Visual Line of Sight operations mandates operator oversight, which centralized control naturally provides through clear authority hierarchies. Optimization quality benefits from global fleet state visibility, enabling superior task allocation compared to distributed consensus approaches. Implementation complexity reduces significantly with a single decision point, eliminating complex inter-UAV coordination protocols. This approach aligns with current commercial and military deployment practices.

The hybrid autonomy model acknowledges fundamental operational constraints. The system explicitly recognizes scenarios where complete failure compensation proves physically impossible, maintains safety through human-in-the-loop oversight for critical decisions, and enhances deployability by avoiding full BVLOS authority requirements. This design philosophy prioritizes honest performance assessment over aspirational capabilities.

Architectural trade-offs require careful mitigation. The centralized architecture introduces a single point of failure at the GCS, mitigated through autonomous Return-to-Launch behaviors triggered by communication timeout detection. Communication bandwidth scales linearly with fleet size, though the 2 Hz telemetry rate maintains manageable overhead for fleets not exceeding twelve UAVs. The 3-6 second response latency, while not instantaneous, remains acceptable for mission-level fault tolerance in applications where individual UAV aerodynamic stability persists throughout this interval.

\subsection{Communication System Design}

The bidirectional communication architecture operates at 2 Hz frequency with asymmetric packet sizes reflecting data flow characteristics. Communication utilizes \textbf{TCP/IP transport with JSON-RPC 2.0 protocol} for structured request-response messaging between UAVs and the GCS server (port 5555).

\textbf{Protocol structure:}

\begin{table}[H]
\centering
\small
\begin{tabular}{|l|p{9cm}|}
\hline
\textbf{Layer/Component} & \textbf{Specification} \\
\hline
\textbf{Transport} & TCP/IP for reliable delivery with connection-oriented communication \\
\hline
\textbf{Application Protocol} & JSON-RPC 2.0 for standardized remote procedure calls \\
\hline
\textbf{Message Format} & JSON-encoded request/response pairs with method invocation semantics \\
\hline
\textbf{Telemetry Methods} & UAVs invoke GCS methods to report status (position, battery, task progress) \\
\hline
\textbf{Command Methods} & GCS invokes UAV methods to dispatch waypoint updates and task reassignments \\
\hline
\end{tabular}
\caption{Communication Protocol Stack}
\label{tab:protocol_structure}
\end{table}

Telemetry uplink from UAVs to GCS transmits approximately 2 kilobyte JSON-RPC packets containing position coordinates, battery state of charge, operational status flags, and task completion progress. Command downlink from GCS to UAVs utilizes 1 kilobyte JSON-RPC packets conveying waypoint updates, task assignments, and collision avoidance parameters. One-way latency spans 0.5 to 1.0 seconds, typical of long-range RF links, yielding total system latency from failure detection to command execution of 3 to 6 seconds.

The 2 Hz telemetry rate selection reflects practical deployment constraints. Commercial UAV systems typically employ 1-2 Hz update rates, while research platforms with dedicated radio links may achieve 5-10 Hz though this remains uncommon. Higher rates increase bandwidth requirements and packet collision probability. Given that OODA loop execution requires 2-5 seconds, 2 Hz sampling provides adequate temporal resolution. This conservative choice ensures field deployability while maintaining responsive fault detection, contrasting with prior simulation work that often assumes instantaneous communication.

\section{OODA Loop Execution Flow}

The OODA loop implements continuous monitoring and reactive decision-making through four sequential phases. \textbf{Measured performance demonstrates sub-millisecond execution times (0.11-0.50 ms)} from failure detection to command dispatch, significantly exceeding initial design targets of 4-6 seconds. This represents approximately \textbf{500,000$\times$ faster response} than manual operator intervention (465 seconds average) and \textbf{10,000$\times$ faster than initial design expectations}.

\subsection{Computation Time vs. End-to-End System Latency}

An important distinction exists between \textbf{OODA algorithm computation time} and \textbf{total system response latency}, explaining the substantial performance improvement over initial estimates:

\textbf{OODA computation time (measured):} The core OODA loop algorithm---encompassing failure detection logic, capacity analysis, constraint validation, and reallocation optimization---executes in 0.11 to 0.50 milliseconds. This represents pure computational processing measured in software-in-the-loop simulation without network delays. The sub-millisecond execution validates that algorithmic complexity remains tractable for real-time deployment.

\textbf{End-to-end system latency (design target):} The original 4-6 second estimate accounts for complete system response including bidirectional RF communication delays, a critical factor in distributed UAV coordination (ABDESSAMEUD; TAYEBI, 2011; IZADI; GORDON; ZHANG, 2009). This encompasses 0.5-1.0 seconds uplink for failure notification, 0.5-1.0 seconds downlink for command dispatch, telemetry aggregation latency at 2 Hz sampling intervals (up to 0.5 seconds), and command acknowledgment verification (0.2 seconds). In field deployment with long-range radio links, total adaptation time spans:

\begin{equation}
T_{total} = T_{uplink} + T_{compute} + T_{downlink} + T_{ack} \approx 1.0 + 0.0005 + 1.0 + 0.2 = 2.2 \text{ seconds}
\end{equation}

This 2-3 second end-to-end response remains substantially faster than the 4-6 second initial conservative estimate, with the primary time consumption attributable to RF propagation delays rather than computational bottlenecks. The microsecond-scale algorithmic performance ensures that computation never becomes the limiting factor, even for larger fleet sizes where complexity increases.

\textbf{Practical implication:} Real-world deployments will experience 2-3 second adaptation times dominated by communication latency, not the 0.3 ms algorithmic computation. This distinction matters for system design---network optimization provides greater latency reduction potential than further algorithmic speedup. Nonetheless, 2-3 seconds represents a 150-fold improvement over manual operator response (465 seconds), validating the OODA approach for time-critical applications such as search and rescue operations.

\begin{figure}[h!tbp]
    \centering
    \includegraphics[width=\textwidth, height=0.9\textheight, keepaspectratio]{images/ooda_loop.png}
    \caption{OODA Loop Execution Flow}
    \label{fig:ooda_loop}
\end{figure}

\subsection{OBSERVE Phase: Failure Detection and State Aggregation}

The OBSERVE phase (1.0-1.5s budget) establishes situational awareness through multi-modal failure detection: timeout detection for telemetry gaps >1.5s, explicit fault messages from UAV systems, and statistical anomaly detection (battery discharge >5\%/30s, position jumps >100m, altitude violations). \textit{Note: Computation executes in microseconds; time budget accommodates 2 Hz RF telemetry aggregation.} Upon failure confirmation, the system aggregates fleet state from all operational UAVs, identifies failed vehicle last-known state, enumerates lost tasks with priorities and deadlines, and calculates mission impact percentage.

\subsection{ORIENT Phase: Situation Assessment and Capacity Analysis}

The ORIENT phase (1.0-1.5s budget) transforms observations into actionable intelligence, drawing on coverage control theory for mobile sensing networks (CORTÉS et al., 2004; SCHWAGER; RUS; SLOTINE, 2009). Mission impact evaluation quantifies coverage loss, affected zones, and deadline pressure. Fleet capacity analysis inventories spare resources: battery capacity (subtracting committed energy and 15-20\% safety reserves, yielding ~3min flight per 10\% spare), payload capacity for delivery missions, and temporal margins relative to deadlines. Task prioritization applies Algorithm 1 to generate 0-1 priority scores from temporal urgency, mission criticality, and spatial cost. Feasibility classification: feasible (>75\% tasks reallocable, >90\% completion), partial (50-75\% reallocable, 75-90\% completion), or infeasible (<50\% reallocable, <75\% completion).

\subsection{DECIDE Phase: Strategic Planning and Algorithmic Optimization}

The DECIDE phase (1.0-1.5s budget) selects strategies via three-tier hierarchy. \textbf{Full Reallocation:} Executes Algorithm 2 for constraint-aware assignment to nearest UAVs with collision avoidance, optimizing path integration for 90-100\% completion. \textbf{Partial Reallocation:} Filters tasks with priority >0.7, allocates high-priority tasks first, generates coverage gap alerts with operator recommendations for 75-90\% completion. \textbf{Operator Escalation:} Generates critical alerts (urgency: high if coverage <50\% or critical tasks lost, medium if 50-75\%, low if >75\%), presents alternatives (backup UAV, mission abort, degraded coverage), implements 30s countdown timer with automatic safe-default execution.

\subsection{ACT Phase: Command Execution and System Update}

The ACT phase (0.5-1.5s budget) dispatches mission updates to affected UAVs (new waypoints, task assignments, collision parameters) via 2 Hz uplink with 200ms acknowledgment verification and 3-attempt retry logic. Dashboard updates display fleet status, mission progress, coverage heatmaps, and alerts. Performance logging captures failure details, phase durations, reallocation results, and escalation status. Following completion, the system returns to continuous 2 Hz monitoring, supporting cascading failure handling through iterative OODA cycles.

\section{Sequential Constraint Validation Process}

The constraint validation process implements fail-fast sequential checking that prioritizes constraints by criticality and commonality, enabling early termination when fundamental limitations preclude autonomous compensation while avoiding wasted computation on infeasible scenarios.

\begin{figure}[h!tbp]
    \centering
    \includegraphics[width=\textwidth, height=0.9\textheight, keepaspectratio]{images/constraints_checking.png}
    \caption{Constraint Checking Process}
    \label{fig:constraints}
\end{figure}


\subsection{Battery Constraint Verification}

Battery constraint evaluation serves as the primary validation gate due to its status as the most common limiting factor and safety-critical nature. For each lost task, the system calculates Euclidean distance from candidate UAVs, determines required battery percentage via vehicle-specific efficiency factors, and computes spare capacity as current charge minus committed energy and 15--20\% safety reserves. Pass: every lost task finds one candidate UAV within spare capacity. Fail: any task unreachable by all UAVs without violating safety margins, triggering immediate operator escalation. Prioritization derives from catastrophic failure risk and computational efficiency of distance-based evaluation.

\subsection{Payload Constraint Verification}

Payload constraint evaluation (conducted conditionally upon battery satisfaction) applies exclusively to cargo-carrying operations. The system calculates spare payload as maximum capacity minus current load, requiring task payload not exceed available spare capacity. Pass: all payload-requiring tasks find adequate capacity. Fail: removes payload-heavy tasks from consideration while reallocating feasible tasks, producing partial coverage. Payload constraints represent hard physical limits—mid-air transfers remain impossible, constraining reallocation to base station cargo swaps to avoid structural failure or flight instability.

\subsection{Time Constraint Verification}

Time constraint evaluation (positioned as final validation) verifies reallocated tasks achieve completion before deadline expiration via cumulative time calculation: transit time + task execution + current task remainder. Pass: yields feasible classification with 90--100\% completion plans. Fail: produces partial reallocation prioritizing by deadline urgency, accepting delays for low-priority tasks with 75--90\% completion projections. Final positioning reflects maximum flexibility—violations produce mission degradation rather than safety risks, unlike battery/payload constraints.

\subsection{Design Rationale}

Sequential checking provides: (1) early exit efficiency, saving ~60\% computation when battery constraints fail; (2) clear failure attribution for precise diagnostics and operator understanding; (3) prioritized ordering checking safety-critical battery bottlenecks first, physical payload limits second, flexible time constraints last.

\section{Communication Sequence and Information Flow}

The temporal communication sequence and information pipeline architecture demonstrate the system's end-to-end fault response capability, transforming raw sensor measurements into executable commands through systematic OODA processing.

\begin{figure}[h!tbp]
    \centering
    \includegraphics[width=\textwidth, height=0.9\textheight, keepaspectratio]{images/sequence_diagram.png}
    \caption{Mission Execution Sequence}
    \label{fig:sequence}
\end{figure}

\subsection{Temporal Execution Analysis}

The sequence diagram illustrates communication flow during failure events with precise timing. At reference time T, a UAV experiences failure, transmitting a high-priority fault message with 0.4-1.0 second latency. The OBSERVE phase confirms the fault and aggregates fleet state by T+1.5 seconds. The ORIENT phase completes impact assessment and capacity analysis by T+2.5 seconds. The DECIDE phase generates reallocation plans by T+3.5 seconds. The ACT phase dispatches commands and updates dashboards by T+4.0 seconds, establishing a 4-second response timeline from failure detection to command dispatch.

This temporal progression demonstrates responsiveness far exceeding human operator capabilities while accommodating realistic communication latencies inherent in long-range RF links. The explicit timing breakdown provides quantitative performance baselines for fault-tolerant mission adaptation.

\subsection{Information Pipeline Architecture}

\begin{figure}[H]
\centering
\includegraphics[width=0.9\textwidth]{images/processing_pipeline.png}
\caption{Data Flow Pipeline}
\label{fig:pipeline}
\end{figure}

The data flow architecture implements unidirectional information transformation with clear phase boundaries. The input layer ingests dual streams: high-frequency telemetry from UAV sensors updated at 2 Hz, and static mission definitions loaded at initialization. The processing layer transforms these inputs sequentially—OBSERVE produces fleet states and failure lists, ORIENT generates impact assessments and capacity analyses, DECIDE creates strategies and command packets, and ACT executes commands while logging performance metrics. The output layer distributes results through three channels: commands transmitted to UAVs, alerts displayed to operators, and logs persisted for post-mission analysis.

This unidirectional flow pattern provides clear data provenance enabling traceability, testability through phase isolation with mock input capability, and maintainability where phase modifications avoid backward ripple effects. The implicit feedback loop manifests as ACT phase commands update UAV operational states, with updated telemetry reflecting new states flowing back into OBSERVE phase in subsequent cycles.

\subsection{Scalability Considerations}

Computational complexity analysis reveals scaling characteristics across OODA phases. The OBSERVE phase scales linearly with fleet size through increased telemetry aggregation. The ORIENT phase maintains linear scaling for capacity calculations across vehicles. The DECIDE phase exhibits O(N$\times$M) complexity for task allocation across N UAVs and M tasks, with collision avoidance scaling quadratically O(N$^2$) for pairwise path verification. These complexity characteristics yield expected OODA execution times of 5-6 seconds for twelve-UAV fleets and 8-10 seconds for twenty-UAV fleets, potentially exceeding acceptable response thresholds for larger deployments.

\section{Scenario-Specific Adaptations}

The architecture demonstrates flexibility across diverse mission profiles through scenario-specific workflow patterns while maintaining consistent OODA loop mechanics. Three representative mission types illustrate architectural adaptation to varying operational requirements and constraint priorities.

\begin{figure}[h!tbp]
    \centering
    \includegraphics[width=\textwidth, height=0.9\textheight, keepaspectratio]{images/surveillance_loop.png}
    \caption{Surveillance Mission}
    \label{fig:surveillance}
\end{figure}

\textbf{Surveillance missions} emphasize continuous area coverage through patrol patterns with failures requiring coverage gap minimization and critical zone protection. Task characteristics include waypoint-based patrol routes, priority differentials between critical security zones and routine patrol areas, and deadline constraints for periodic revisit frequencies. Constraint considerations focus primarily on battery limitations affecting patrol duration and spatial coverage range, with time constraints enforcing revisit deadlines for critical areas. Target performance maintains 90-100 percent coverage through reallocation prioritizing critical zones.

\begin{figure}[H]
    \centering
    \includegraphics[width=\textwidth, height=0.9\textheight, keepaspectratio]{images/sar_loop.png}
    \caption{Search and Rescue Mission}
    \label{fig:sar}
\end{figure}


\textbf{Search and rescue missions} emphasize time-critical area coverage through systematic grid search patterns with failures requiring rapid reallocation to maintain search efficiency. Task characteristics include grid cell assignments with systematic coverage requirements and stringent deadline constraints reflecting survivor time sensitivity. Equal priority across search cells applies in unknown target location scenarios, though priority may concentrate in high-probability areas when available intelligence suggests target location. Battery and time constraints dominate feasibility assessment. Target performance accepts 75-90 percent coverage as acceptable degradation when full compensation proves infeasible, acknowledging that incomplete coverage in time-critical scenarios exceeds mission abortion alternatives.

\begin{figure}[h!tbp]
    \centering
    \includegraphics[width=\textwidth, height=0.9\textheight, keepaspectratio]{images/delivery_loop.png}
    \caption{Delivery Mission}
    \label{fig:delivery}
\end{figure}

\textbf{Delivery missions} emphasize reliable cargo transport to designated destinations with failures requiring payload-aware reallocation and potential return-to-base cargo swaps when reallocation proves infeasible. Task characteristics include point-to-point delivery routes with specific destination coordinates, payload weight requirements varying per package, and deadline constraints for time-sensitive deliveries. Payload constraints assume critical importance alongside battery limitations, with heavy cargo failures potentially requiring UAV return-to-launch for manual cargo transfer rather than mid-air reallocation. Priority schemes emphasize time-sensitive or high-value packages, accepting degradation in routine deliveries when capacity limitations preclude full compensation.

These scenario variations demonstrate architectural flexibility through consistent OODA mechanics with mission-specific constraint emphasis and priority schemes, validating the architecture's applicability across diverse operational domains while maintaining systematic fault response capabilities.

\chapter{Core Algorithms and Technical Contributions}

\section{Priority-Based Task Scoring Algorithm}

Effective task reallocation requires a principled method for determining which orphaned tasks deserve immediate attention and which can tolerate delayed or degraded service. Algorithm~1 formalizes a multi-factor priority scoring function that balances temporal urgency, mission-specific criticality, and spatial reallocation cost.

\begin{table}[H]
\centering
\begin{tabular}{|p{12cm}|}
\hline
\textbf{Algorithm 1: Task Priority Calculation} \\
\hline
\textbf{Input:} Task $t_i$ with position, type, and deadline; fleet state $\mathcal{F}$; mission context $\mathcal{M}$ \\
\textbf{Output:} Priority score $P_i \in [0, 1]$ \\[6pt]
\textbf{1.} Compute temporal urgency: $\tau \gets 1 - \dfrac{t_i.\text{deadline} - t_{\text{now}}}{t_i.\text{deadline} - t_i.\text{start}}$ \\[6pt]
\textbf{2.} Look up mission criticality weight: $c \gets W_{\mathcal{M}.\text{type}}[t_i.\text{type}]$ \\[6pt]
\textbf{3.} Find minimum distance to healthy UAV: $d_{\min} \gets \min_{u \in \mathcal{F}.\text{healthy}} \| u.\text{pos} - t_i.\text{pos} \|$ \\[6pt]
\textbf{4.} Normalize spatial cost: $\sigma \gets d_{\min} \,/\, R_{\max}$ \\[6pt]
\textbf{5.} Combine components: $P_i \gets w_\tau \cdot \text{clamp}(\tau) + w_c \cdot c - w_\sigma \cdot \text{clamp}(\sigma)$ \\[6pt]
\textbf{6.} \textbf{return} $\text{clamp}(P_i, 0, 1)$ \\
\hline
\end{tabular}
\end{table}

The temporal component $\tau$ increases as deadlines approach, ensuring time-critical tasks receive priority. Mission criticality $c$ reflects domain-specific importance: in delivery missions, medical emergencies score 1.0 while standard packages score 0.4; in search and rescue, high-probability victim zones dominate over perimeter patrols. The spatial term $\sigma$ penalizes tasks distant from available UAVs, implicitly accounting for battery expenditure required for reallocation.

Default weights ($w_\tau = 0.3$, $w_c = 0.5$, $w_\sigma = 0.2$) emphasize criticality while maintaining responsiveness to deadlines. These parameters are configurable per mission type, as detailed in Section~3.5.

This formulation draws inspiration from auction-based task allocation (CHOI; BRUNET; HOW, 2009) and utility functions in market-based coordination (DIAS et al., 2006; ZLOT; STENTZ, 2006), but differs by explicitly incorporating resource costs into the priority calculus rather than treating them as post-hoc constraints. The formal taxonomy of task allocation in multi-robot systems established by Gerkey and Matarić (2004) provides the theoretical foundation for this approach.

\section{Constraint-Aware Task Reallocation with Collision Avoidance}

With tasks ranked by priority, the system must assign them to healthy UAVs while respecting hard operational constraints. Algorithm~2 implements a greedy allocation strategy that processes tasks in priority order, assigning each to the nearest UAV with sufficient capacity. This approach guarantees constraint satisfaction at the cost of global optimality—a trade-off justified by real-time requirements.

\begin{table}[H]
\centering
\begin{tabular}{|p{12cm}|}
\hline
\textbf{Algorithm 2: Constraint-Aware Task Reallocation} \\
\hline
\textbf{Input:} Failed UAV's task queue $\mathcal{T}$; healthy UAVs $\mathcal{U}$; mission context $\mathcal{M}$ \\
\textbf{Output:} Allocation $\mathcal{A}$; coverage percentage; operator alerts \\[6pt]
\textbf{1.} \textbf{for each} $u \in \mathcal{U}$ \textbf{do} \\
\quad Compute spare battery: $b_u \gets u.\text{battery} - B_{\text{reserve}} - u.\text{committed}$ \\
\quad Compute spare payload: $p_u \gets u.\text{max\_payload} - u.\text{current\_payload}$ \\[4pt]
\textbf{2.} $\mathcal{T}_{\text{ranked}} \gets \textsc{RankByPriority}(\mathcal{T}, \mathcal{U}, \mathcal{M})$ \hfill \textit{// Algorithm 1} \\[4pt]
\textbf{3.} $\mathcal{A} \gets \emptyset$; \quad $\mathcal{T}_{\text{unalloc}} \gets \emptyset$ \\[4pt]
\textbf{4.} \textbf{for each} $(t_i, P_i) \in \mathcal{T}_{\text{ranked}}$ \textbf{do} \\
\quad Sort candidates by distance: $\mathcal{U}_{\text{sorted}} \gets \text{sort}(\mathcal{U}, \text{key}=\|u.\text{pos} - t_i.\text{pos}\|)$ \\
\quad \textbf{for each} $u \in \mathcal{U}_{\text{sorted}}$ \textbf{do} \\
\quad\quad $d \gets \|u.\text{pos} - t_i.\text{pos}\|$; \quad $b_{\text{req}} \gets d \,/\, \eta_u$ \\
\quad\quad \textbf{if} $b_u < b_{\text{req}}$ \textbf{then continue} \hfill \textit{// Battery constraint} \\
\quad\quad \textbf{if} $\mathcal{M}.\text{type} = \text{delivery} \land p_u < t_i.\text{payload}$ \textbf{then continue} \hfill \textit{// Payload constraint} \\
\quad\quad \textbf{if} $\neg\textsc{CollisionFree}(u, t_i, \mathcal{A})$ \textbf{then continue} \hfill \textit{// Algorithm 3} \\
\quad\quad $\mathcal{A} \gets \mathcal{A} \cup \{(t_i, u)\}$; \quad $b_u \gets b_u - b_{\text{req}}$; \quad \textbf{break} \\
\quad \textbf{if} $t_i \notin \mathcal{A}$ \textbf{then} $\mathcal{T}_{\text{unalloc}} \gets \mathcal{T}_{\text{unalloc}} \cup \{(t_i, P_i)\}$ \\[4pt]
\textbf{5.} coverage $\gets |\mathcal{A}| \,/\, |\mathcal{T}| \times 100\%$ \\[4pt]
\textbf{6.} \textbf{return} $\mathcal{A}$, coverage, $\textsc{GenerateAlerts}(\mathcal{T}_{\text{unalloc}})$ \\
\hline
\end{tabular}
\end{table}

The algorithm enforces three constraint types sequentially. Battery constraints verify that the UAV retains sufficient charge to reach the task location while maintaining the 20\% safety reserve mandated by operational policy. Payload constraints, active only in delivery missions, ensure the vehicle can physically carry the required cargo. Collision avoidance, delegated to Algorithm~3, prevents spatial conflicts between reassigned flight paths.

\begin{table}[H]
\centering
\begin{tabular}{|p{12cm}|}
\hline
\textbf{Algorithm 3: Collision-Free Path Verification} \\
\hline
\textbf{Input:} UAV $u$; candidate task $t$; current allocation $\mathcal{A}$ \\
\textbf{Output:} Boolean indicating path safety \\[6pt]
\textbf{Constants:} $D_{\text{safe}} = 15\text{m}$ (spatial buffer); $T_{\text{safe}} = 10\text{s}$ (temporal buffer) \\[6pt]
\textbf{1.} $\pi_{\text{new}} \gets \textsc{GeneratePath}(u.\text{pos}, t.\text{pos})$ \\
\textbf{2.} $\tau_{\text{new}} \gets \textsc{EstimateTimeline}(\pi_{\text{new}}, u.\text{speed})$ \\[4pt]
\textbf{3.} \textbf{for each} $(t', u') \in \mathcal{A}$ where $u' \neq u$ \textbf{do} \\
\quad $\pi_{\text{other}} \gets \textsc{GetPlannedPath}(u')$ \\
\quad $\tau_{\text{other}} \gets \textsc{GetTimeline}(u')$ \\
\quad \textbf{for each} $(p_1, t_1) \in (\pi_{\text{new}}, \tau_{\text{new}})$ \textbf{do} \\
\quad\quad \textbf{for each} $(p_2, t_2) \in (\pi_{\text{other}}, \tau_{\text{other}})$ \textbf{do} \\
\quad\quad\quad \textbf{if} $\|p_1 - p_2\| < D_{\text{safe}} \land |t_1 - t_2| < T_{\text{safe}}$ \textbf{then return} \textsc{False} \\[4pt]
\textbf{4.} \textbf{return} \textsc{True} \\
\hline
\end{tabular}
\end{table}

The collision avoidance module implements spatiotemporal conflict detection. Two UAVs are considered in conflict if their paths bring them within 15 meters of each other (approximately three times a typical UAV wingspan) during overlapping time windows. When conflicts arise during allocation, the candidate assignment is rejected and the next-nearest UAV is evaluated. If no conflict-free assignment exists, temporal separation can be applied by delaying the lower-priority UAV's departure.

This approach draws on velocity obstacle methods, particularly Reciprocal Velocity Obstacles (VAN DEN BERG; LIN; MANOCHA, 2008), but simplifies the formulation for centralized planning where global path information is available. The 15-meter spatial buffer and 10-second temporal margin mirror conservative air traffic control separation standards adapted for low-altitude UAV operations.

\section{Operator Escalation Decision Rules}

A distinguishing feature of this architecture is its capacity for honest self-assessment. Rather than forcing autonomous solutions in all circumstances, the system explicitly recognizes when human judgment is required. Algorithm~4 encodes the escalation logic as a hierarchical decision tree that evaluates mission degradation severity.

\begin{table}[H]
\centering
\begin{tabular}{|p{12cm}|}
\hline
\textbf{Algorithm 4: Operator Escalation Decision} \\
\hline
\textbf{Input:} Coverage percentage $\rho$; unallocated tasks $\mathcal{T}_{\text{unalloc}}$ with priorities \\
\textbf{Output:} Escalation decision with urgency level and recommendation \\[6pt]
\textbf{1.} \textbf{if} $\rho < 50\%$ \textbf{then} \\
\quad \textbf{return} (escalate=\textsc{True}, urgency=\textsc{High}, \\
\quad\quad reason=``Coverage below 50\% threshold'', \\
\quad\quad recommendation=``Deploy backup UAV or abort mission'') \\[4pt]
\textbf{2.} $n_{\text{critical}} \gets |\{t \in \mathcal{T}_{\text{unalloc}} : P_t > 0.7\}|$ \\
\quad \textbf{if} $n_{\text{critical}} > 0$ \textbf{then} \\
\quad \textbf{return} (escalate=\textsc{True}, urgency=\textsc{High}, \\
\quad\quad reason=``$n_{\text{critical}}$ critical tasks unassignable'', \\
\quad\quad recommendation=``Manual prioritization required'') \\[4pt]
\textbf{3.} \textbf{if} $\rho < 75\%$ \textbf{then} \\
\quad \textbf{return} (escalate=\textsc{True}, urgency=\textsc{Medium}, \\
\quad\quad reason=``Moderate degradation (50--75\% coverage)'', \\
\quad\quad recommendation=``Monitor; consider manual reallocation'') \\[4pt]
\textbf{4.} \textbf{return} (escalate=\textsc{False}, urgency=\textsc{Low}, \\
\quad reason=``Acceptable autonomous compensation ($>$75\%)'', \\
\quad recommendation=``Continue autonomous operation'') \\
\hline
\end{tabular}
\end{table}

The escalation hierarchy reflects operational risk tolerance. Coverage below 50\% indicates that the failure has exceeded the fleet's compensatory capacity—continuing autonomously would produce unacceptable mission outcomes. The presence of any unallocated high-priority task ($P > 0.7$) similarly triggers immediate escalation, as these tasks represent mission-critical objectives that cannot be sacrificed without explicit human authorization.

Moderate degradation (50--75\% coverage) warrants operator awareness without demanding immediate intervention, allowing human judgment to assess whether the degraded outcome remains acceptable for the specific operational context. Above 75\% coverage, the system proceeds autonomously, having demonstrated sufficient capacity to absorb the failure's impact.

This graduated response ensures that operator attention is reserved for genuinely consequential decisions while routine failures are handled without interrupting mission flow.

\section{Objective Function and Optimization Strategy}

The preceding algorithms define \textit{how} tasks are prioritized and allocated, but do not explicitly characterize \textit{what constitutes a good allocation}. This section formalizes the objective function that guides the DECIDE phase and describes the optimization strategy employed to maximize decision quality within real-time constraints.

\subsection{Allocation Quality Metric}

The quality of a task reallocation is quantified through an objective function $J(\mathcal{A})$ that the optimizer seeks to maximize. Given an allocation $\mathcal{A} = \{(t_i, u_j)\}$ mapping failed tasks $t_i$ to healthy UAVs $u_j$, the objective function is defined as:

\begin{equation}
J(\mathcal{A}) = \sum_{(t_i, u_j) \in \mathcal{A}} \left[ P_i \cdot \phi_m(t_i, u_j) \right] - \lambda \cdot |\mathcal{T}_{\text{unalloc}}|
\label{eq:objective}
\end{equation}

\noindent where:
\begin{itemize}
\item $P_i \in [0,1]$ is the priority score of task $t_i$ computed by Algorithm 1
\item $\phi_m(t_i, u_j) \in [0,1]$ is a mission-specific modifier function
\item $\lambda > 0$ is the penalty weight for unallocated tasks
\item $|\mathcal{T}_{\text{unalloc}}|$ is the count of tasks that could not be feasibly assigned
\end{itemize}

The mission-specific modifier $\phi_m$ adjusts task value based on operational context:

\begin{equation}
\phi_m(t_i, u_j) =
\begin{cases}
1 - \gamma \cdot \Delta t_{\text{gap}}(t_i) & \text{if } m = \text{surveillance} \\[6pt]
1 + \beta \cdot \dfrac{t_{\text{golden}} - t_{\text{completion}}(t_i, u_j)}{t_{\text{golden}}} & \text{if } m = \text{SAR} \\[6pt]
\begin{cases}
1.0 & \text{if } t_{\text{completion}} \leq t_{\text{deadline}} \\
0.5 & \text{otherwise}
\end{cases} & \text{if } m = \text{delivery}
\end{cases}
\label{eq:modifier}
\end{equation}

For surveillance missions, the modifier penalizes coverage gaps through parameter $\gamma$, where $\Delta t_{\text{gap}}$ represents the time since last coverage of the task's zone. For search and rescue operations, tasks completed before the golden hour receive a bonus weighted by $\beta$, incentivizing rapid coverage of high-probability areas. For delivery missions, the modifier applies a binary penalty for deadline violations, reflecting the discrete nature of delivery success.

\subsection{Optimization Strategy}

The DECIDE phase operates under strict temporal constraints (1.0--1.5 seconds), precluding exact optimization methods such as mixed-integer linear programming. The system therefore employs a two-stage optimization strategy that balances solution quality against computational tractability.

\textbf{Stage 1: Greedy Initialization.} Algorithm 2 generates an initial feasible allocation in $O(n \cdot m)$ time, where $n$ is the number of failed tasks and $m$ is the number of healthy UAVs. This greedy approach processes tasks in priority order, assigning each to the nearest constraint-satisfying UAV. While not globally optimal, this initialization typically achieves 70--85\% of the theoretical maximum objective value.

\textbf{Stage 2: Local Search Refinement.} When the time budget permits (approximately 500ms remaining after Stage 1), the system applies iterative local search to improve the initial allocation. The neighborhood structure considers pairwise task swaps between UAVs and reassignment of individual tasks to alternative vehicles.

\begin{table}[H]
\centering
\begin{tabular}{|p{12cm}|}
\hline
\textbf{Algorithm 5: Two-Stage Optimization} \\
\hline
\textbf{Input:} Failed tasks $\mathcal{T}$; healthy UAVs $\mathcal{U}$; mission context $\mathcal{M}$; time budget $T_{\max}$ \\
\textbf{Output:} Optimized allocation $\mathcal{A}^*$ with objective score $J^*$ \\[6pt]
\textbf{Stage 1: Greedy Initialization} \\
\textbf{1.} $t_0 \gets \textsc{CurrentTime}()$ \\
\textbf{2.} $\mathcal{A} \gets \textsc{GreedyAllocate}(\mathcal{T}, \mathcal{U}, \mathcal{M})$ \hfill \textit{// Algorithm 2} \\
\textbf{3.} $J_{\text{best}} \gets \textsc{ComputeObjective}(\mathcal{A}, \mathcal{M})$ \\[4pt]
\textbf{Stage 2: Local Search Refinement} \\
\textbf{4.} \textbf{while} $\textsc{CurrentTime}() - t_0 < T_{\max} - T_{\text{reserve}}$ \textbf{do} \\
\quad $\text{improved} \gets \textsc{False}$ \\
\quad \textbf{for each} $(t_1, u_1), (t_2, u_2) \in \mathcal{A} \times \mathcal{A}$ where $u_1 \neq u_2$ \textbf{do} \\
\quad\quad $\mathcal{A}' \gets \mathcal{A}$ with swap: $t_1 \to u_2$, $t_2 \to u_1$ \\
\quad\quad \textbf{if} $\textsc{IsFeasible}(\mathcal{A}', \mathcal{M})$ \textbf{then} \\
\quad\quad\quad $J' \gets \textsc{ComputeObjective}(\mathcal{A}', \mathcal{M})$ \\
\quad\quad\quad \textbf{if} $J' > J_{\text{best}}$ \textbf{then} $\mathcal{A} \gets \mathcal{A}'$; $J_{\text{best}} \gets J'$; $\text{improved} \gets \textsc{True}$; \textbf{break} \\
\quad \textbf{if} $\neg\text{improved}$ \textbf{then break} \hfill \textit{// Local optimum reached} \\[4pt]
\textbf{5.} \textbf{return} $\mathcal{A}$, $J_{\text{best}}$ \\
\hline
\end{tabular}
\end{table}

The local search explores pairwise task swaps, proposing exchanges between UAVs and accepting improvements that increase the objective function while maintaining constraint feasibility. The algorithm terminates upon reaching a local optimum or exhausting the time budget, whichever occurs first. A 200ms reserve ($T_{\text{reserve}}$) ensures sufficient time for finalization and command dispatch.

\subsection{Optimality Gap Analysis}

The greedy-plus-local-search strategy does not guarantee global optimality. To characterize solution quality, we define the optimality gap as:

\begin{equation}
\text{Gap} = \frac{J^* - J(\mathcal{A}_{\text{heuristic}})}{J^*} \times 100\%
\label{eq:gap}
\end{equation}

\noindent where $J^*$ is the optimal objective value computed offline via exhaustive search or MILP for small problem instances.

Preliminary analysis across the three mission scenarios indicates expected optimality gaps of 5--15\% for typical failure cases involving 3--6 lost tasks and 4--8 healthy UAVs. This trade-off is justified by the real-time constraint: a 10\% suboptimal solution computed in 1.2 seconds provides substantially greater operational value than an optimal solution requiring 30+ seconds, during which mission degradation continues unmitigated.

\subsection{Mission-Specific Parameter Configuration}

The objective function parameters are configured per mission type to reflect operational priorities:

\begin{table}[H]
\centering
\caption{Objective Function Parameters by Mission Type}
\label{tab:objective_params}
\rowcolors{2}{gray!15}{white}
\begin{tabular}{|l|c|c|c|c|}
\hline
\rowcolor{gray!30}
\textbf{Parameter} & \textbf{Symbol} & \textbf{Surveillance} & \textbf{SAR} & \textbf{Delivery} \\
\hline
Unallocated penalty & $\lambda$ & 0.3 & 0.5 & 0.4 \\
Coverage gap weight & $\gamma$ & 0.2 & --- & --- \\
Golden hour bonus & $\beta$ & --- & 0.5 & --- \\
Priority weights & $w_{\text{temporal}}$ & 0.3 & 0.5 & 0.2 \\
                 & $w_{\text{criticality}}$ & 0.5 & 0.3 & 0.6 \\
                 & $w_{\text{spatial}}$ & 0.2 & 0.2 & 0.2 \\
\hline
\end{tabular}
\end{table}

This parameterization enables the same OODA loop implementation to adapt its optimization behavior based on mission context, achieving domain-appropriate decision quality without requiring mission-specific algorithmic modifications.

\chapter{Mission Scenarios and Performance Analysis}

This chapter presents three representative mission scenarios that demonstrate the OODA-based fault tolerance system across diverse operational contexts. Each scenario illustrates different constraint priorities, failure modes, and recovery strategies, providing concrete validation of the system's adaptive capabilities under realistic conditions.

\section{Scenario Selection Rationale}

The three scenarios were selected to span the constraint space of real-world UAV operations:

\begin{itemize}
\item \textbf{Surveillance:} Battery-constrained with time-critical coverage requirements
\item \textbf{Search \& Rescue:} Time-critical with priority-driven partial coverage acceptance
\item \textbf{Delivery:} Payload-constrained with heterogeneous fleet capabilities
\end{itemize}

Each scenario represents a distinct operational domain with documented real-world deployment precedents, ensuring practical relevance beyond theoretical validation.

\section{SCENARIO 1: Long-Duration Perimeter Surveillance}

Perimeter surveillance represents a canonical application of decentralized aerial monitoring, with established methodologies for coordinating multiple UAVs across extended operational periods (BEARD et al., 2006).

\subsection{Mission Context}

\begin{itemize}
    \item \textbf{Application:} Critical infrastructure monitoring (port, airport, border)
    \item \textbf{Duration:} 2-hour continuous coverage requirement
    \item \textbf{Fleet:} 6 UAVs with 30-minute flight time each (requires rotation strategy)
    \item \textbf{Operational Area:} 1200m $\times$ 1200m secured perimeter
    \item \textbf{Coverage Strategy:} 6 patrol zones (200m $\times$ 200m each)
\end{itemize}

\subsection{Mission Setup}

The surveillance mission partitions a 1200m $\times$ 1200m operational area into six equal patrol zones, each spanning 200m $\times$ 200m. This tessellation ensures complete coverage while allowing each UAV to maintain a dedicated patrol circuit without inter-vehicle coordination overhead. Figure~\ref{fig:zone_assignment} illustrates the spatial assignment strategy.

The fundamental challenge lies in the mismatch between mission duration and individual vehicle endurance. A 2-hour continuous coverage requirement, combined with 30-minute battery limitations, necessitates a coordinated rotation scheme involving at least 12 UAVs operating in two alternating shifts. This constraint transforms what might appear as a simple coverage problem into a sophisticated scheduling and fault-tolerance challenge.

\begin{figure}[H]
\centering
\includegraphics[width=0.7\textwidth]{images/uav_zone_assignement.png}
\caption{Zone Assignment for Surveillance Mission}
\label{fig:zone_assignment}
\end{figure}

\subsection{Patrol Zone Specifications}

Not all perimeter sections demand equal vigilance. The zone prioritization reflects threat assessment based on facility layout and historical incident patterns. Zones A and B along the northern perimeter guard main entry points and receive the highest priority rating ($P = 0.9$), commanding the most intensive patrol patterns. The eastern and western flanks (Zones C and D) cover secondary access routes with moderate priority ($P = 0.6$), while the southern boundary (Zones E and F) monitors low-risk terrain at standard priority ($P = 0.4$).

This priority differentiation manifests in patrol circuit complexity, following principles established in multi-robot area patrol research (ELMALIACH; AGMON; KAMINKA, 2009). High-priority zones require eight waypoints per circuit, creating dense coverage patterns that minimize detection gaps. Medium-priority zones operate with six waypoints, balancing coverage density against battery consumption. Standard-priority zones employ four-waypoint circuits, sufficient for perimeter awareness without expending resources needed elsewhere. The waypoint density directly influences energy expenditure—a consideration that becomes critical during failure recovery when remaining UAVs must potentially absorb additional coverage responsibilities.

\subsection{Rotation Strategy}

Sustaining 2-hour coverage with 30-minute endurance vehicles requires orchestrated fleet rotation. The system manages this through four consecutive waves, each lasting approximately 30 minutes, with overlapping transition periods to prevent coverage gaps.

The first wave deploys UAVs 1--6 across zones A--F at mission start. As vehicles approach the 20\% battery threshold around the 25-minute mark, they signal imminent departure, triggering deployment of the backup fleet. Wave 2 sees UAVs 7--12 assume patrol responsibilities while the primary fleet returns for battery exchange. This pattern alternates through Waves 3 and 4, with recharged vehicles cycling back into service.

The 5-minute transition overlap between waves provides fault tolerance against timing variations—early battery depletion, delayed takeoffs, or communication latency cannot create coverage voids when both incoming and outgoing vehicles share airspace briefly. This redundancy, however, introduces collision avoidance complexity that the OODA system must manage alongside its primary fault-tolerance responsibilities.

\subsection{Failure Scenario: Mid-Mission Battery Depletion}

Forty-five minutes into the mission, UAV-3 begins losing power faster than expected. Its battery, which should hold 35\% charge at this point, plummets to 8\%---well below the safety threshold. The vehicle is halfway through its patrol circuit over Zone C, the western perimeter flanking the facility's secondary access road. If UAV-3 returns to base immediately, Zone C goes dark. If it continues, it risks a forced landing in the operational area.

This is precisely the scenario the OODA system was designed to handle.

% \begin{figure}[H]
% \centering
% \includegraphics[width=0.8\textwidth]{images/surveillance_loop.png}
% \caption{Surveillance Mission OODA Loop Execution}
% \label{fig:surveillance_loop}
% \end{figure}

\subsubsection{OBSERVE Phase (T+0 to T+1.5s)}

\textbf{Failure Detection:}

The system detects battery anomalies through statistical monitoring, comparing observed discharge rates against expected baseline values. When the discharge rate exceeds 1.5$\times$ the expected threshold, the OODA cycle triggers immediately.

\begin{table}[H]
\centering
\small
\begin{tabular}{|l|c|c|c|}
\hline
\textbf{Parameter} & \textbf{Expected} & \textbf{Observed} & \textbf{Status} \\
\hline
Discharge Rate & 3\%/min & 8\%/min & \cellcolor{red!25}ANOMALY \\
Battery Level & 55\% & 8\% & \cellcolor{red!25}CRITICAL \\
Threshold & --- & 1.5$\times$ baseline & \cellcolor{red!25}EXCEEDED \\
\hline
\end{tabular}
\caption{UAV-3 Battery Anomaly Detection}
\label{tab:s5_anomaly}
\end{table}

\textbf{Fleet State Aggregation:}

Upon detecting UAV-3's failure, the system aggregates the complete fleet state to assess available resources for task reallocation.

\begin{table}[H]
\centering
\small
\begin{tabular}{|l|c|c|c|c|}
\hline
\textbf{UAV} & \textbf{Battery} & \textbf{Zone} & \textbf{Spare Capacity} & \textbf{Status} \\
\hline
UAV-3 & 8\% & C & 0\% & \cellcolor{red!25}FAILED \\
UAV-2 & 45\% & B & 15\% & \cellcolor{green!25}AVAILABLE \\
UAV-4 & 40\% & D & 12\% & \cellcolor{green!25}AVAILABLE \\
UAV-8 & 95\% & Standby & 75\% & \cellcolor{blue!25}RESERVE \\
\hline
\multicolumn{5}{|l|}{\textit{Lost Coverage: Zone C (16.7\% of mission area)}} \\
\hline
\end{tabular}
\caption{Fleet State at Failure Detection (T+1.5s)}
\label{tab:s5_fleet_state}
\end{table}

\subsubsection{ORIENT Phase (T+1.5s to T+3.0s)}

\textbf{Impact Assessment:}

The system evaluates the mission impact of losing Zone C coverage, determining temporal urgency and required response type.

\begin{table}[H]
\centering
\small
\begin{tabular}{|l|p{8cm}|}
\hline
\textbf{Dimension} & \textbf{Assessment} \\
\hline
Mission Criticality & Medium (Zone C is medium-priority area, $P = 0.6$) \\
Coverage Gap & 16.7\% of total mission area lost \\
Temporal Urgency & $P_{\text{time}} = 0.8$ (continuous coverage mandate) \\
Response Required & Immediate reallocation (0-second tolerance) \\
\hline
\end{tabular}
\caption{Mission Impact Assessment}
\label{tab:s5_impact}
\end{table}

\textbf{Capacity Analysis:}

The system calculates spare battery capacity for each candidate UAV by subtracting safety reserves (20\%) and committed energy from current battery levels, then evaluates feasibility against distance-to-target requirements.

\begin{table}[H]
\centering
\small
\begin{tabular}{|l|c|c|c|c|c|}
\hline
\textbf{UAV} & \textbf{Current} & \textbf{Safety} & \textbf{Committed} & \textbf{Spare} & \textbf{Distance} \\
 & \textbf{Battery} & \textbf{Reserve} & \textbf{Energy} & \textbf{Capacity} & \textbf{to Zone C} \\
\hline
UAV-2 & 45\% & 20\% & 10\% & \textbf{15\%} & 250m (req: 8\%) \\
UAV-4 & 40\% & 20\% & 8\% & \textbf{12\%} & 220m (req: 7\%) \\
\hline
\end{tabular}
\caption{Spare Capacity Analysis for Zone C Reallocation}
\label{tab:s5_capacity}
\end{table}

\textbf{Feasibility Determination:}

\begin{table}[H]
\centering
\small
\begin{tabular}{|l|c|c|c|}
\hline
\textbf{UAV} & \textbf{Spare Capacity} & \textbf{Required Energy} & \textbf{Feasible?} \\
\hline
UAV-2 & 15\% & 8\% & \cellcolor{green!25}\textbf{YES} (7\% margin) \\
UAV-4 & 12\% & 7\% & \cellcolor{green!25}\textbf{YES} (5\% margin) \\
\hline
\end{tabular}
\caption{Constraint Feasibility Check}
\label{tab:s5_feasibility}
\end{table}

\textbf{Reallocation Strategy:}

Both adjacent UAVs possess sufficient capacity to reach Zone C. The system adopts a split-zone strategy to distribute the workload:

\begin{itemize}
\item Split Zone C into two sub-zones: C1 (north 100m) and C2 (south 100m)
\item Assign C1 to UAV-2 (adjacent to Zone B, 250m distance)
\item Assign C2 to UAV-4 (adjacent to Zone D, 220m distance)
\item Accept reduced waypoint density: 3 waypoints per sub-zone vs. 6 original
\item Coverage degradation: 8.3\% surveillance quality reduction in Zone C
\end{itemize}

\subsubsection{DECIDE Phase (T+3.0s to T+4.5s)}

\textbf{Strategy Selection:} Partial Reallocation (coverage degradation accepted)

\textbf{Task Allocation Plan:}

The system generates extended patrol routes for both UAVs, integrating new waypoints while maintaining constraint satisfaction and collision avoidance.

\begin{table}[H]
\centering
\small
\begin{tabular}{|l|c|c|c|c|}
\hline
\textbf{UAV} & \textbf{Original Zone} & \textbf{Added Zone} & \textbf{Total Waypoints} & \textbf{Circuit Time} \\
\hline
UAV-2 & Zone B (8 pts) & Zone C1 (3 pts) & 11 points & 183s (\textasciitilde3 min) \\
UAV-4 & Zone D (6 pts) & Zone C2 (3 pts) & 9 points & 165s (\textasciitilde2.7 min) \\
\hline
\multicolumn{5}{|l|}{\textit{Total distance UAV-2: 2200m | Total distance UAV-4: 1980m}} \\
\hline
\end{tabular}
\caption{Extended Patrol Route Allocation}
\label{tab:s5_allocation}
\end{table}

\textbf{Constraint Verification:}

\begin{table}[H]
\centering
\small
\begin{tabular}{|l|c|c|}
\hline
\textbf{Constraint Type} & \textbf{Verification Result} & \textbf{Details} \\
\hline
Battery Safety & \cellcolor{green!25}PASS & Both UAVs maintain $>$20\% reserve \\
Spatial Separation & \cellcolor{green!25}PASS & 15m buffer between C1/C2 boundary \\
Collision Avoidance & \cellcolor{green!25}PASS & Opposite patrol directions (temporal deconfliction) \\
Zone Conflicts & \cellcolor{green!25}PASS & No overlap with Zones A, B, D, E, F \\
\hline
\end{tabular}
\caption{Safety Constraint Verification}
\label{tab:s5_constraints}
\end{table}

\subsubsection{ACT Phase (T+4.5s to T+5.5s)}

\textbf{Command Execution:}

The system dispatches mission updates to affected UAVs and commands the failed vehicle to return immediately.

\begin{table}[H]
\centering
\small
\begin{tabular}{|l|l|p{6cm}|}
\hline
\textbf{Target} & \textbf{Command Type} & \textbf{Payload} \\
\hline
UAV-2 & Mission Update & Extended waypoints: Zone B + C1 (11 points) \\
 & & Priority: HIGH | Timeout: 200ms \\
\hline
UAV-4 & Mission Update & Extended waypoints: Zone D + C2 (9 points) \\
 & & Priority: HIGH | Timeout: 200ms \\
\hline
UAV-3 & Emergency RTL & Immediate return-to-launch command \\
 & & Reason: Battery critical (8\%) \\
\hline
\end{tabular}
\caption{Command Dispatch Summary}
\label{tab:s5_commands}
\end{table}

\textbf{System State Update:}

\begin{table}[H]
\centering
\small
\begin{tabular}{|l|c|}
\hline
\textbf{Status Dimension} & \textbf{Value} \\
\hline
Coverage Recovery & 100\% (Zone C split between UAV-2 and UAV-4) \\
Mission Status & ADAPTED -- Partial Reallocation \\
Surveillance Quality & 91.7\% (reduced waypoint density in Zone C) \\
Alert Level & \cellcolor{yellow!25}CAUTION -- Coverage degraded \\
Operator Action & None required (autonomous recovery successful) \\
\hline
\end{tabular}
\caption{Post-Adaptation Mission Status}
\label{tab:s5_status}
\end{table}

\textbf{Impact Summary:}
\begin{itemize}
\item Zone C coverage degraded: 3 waypoints per sub-zone (previously 6)
\item Estimated surveillance quality reduction: 8.3\% in Zone C only
\item Overall mission completion: 100\% spatial coverage maintained
\item No operator escalation required
\end{itemize}

\subsection{Scenario Outcome}

The OODA system achieved complete coverage recovery in 0.34 milliseconds while maintaining all safety constraints. Without adaptation, Zone C would remain unmonitored for the mission's remainder---an 87.5\% coverage degradation. Manual operator intervention could achieve equivalent recovery but requires 465 seconds, leaving Zone C exposed for nearly eight minutes.

Two limitations emerge from this scenario. First, merging Zone C into adjacent patrol circuits reduces waypoint density, potentially degrading detection probability for that region. Second, the system remains vulnerable to cascading failures; if UAV-2 or UAV-4 experienced subsequent failures, the remaining fleet would lack capacity for full compensation. Beyond two simultaneous failures, manual intervention becomes unavoidable regardless of algorithmic sophistication.

\section{SCENARIO 2: Emergency Search \& Rescue}

\subsection{Mission Context}

Search and rescue operations represent the most time-critical application domain for UAV fault tolerance. When a hiker goes missing in wilderness terrain, the ``golden hour'' principle borrowed from emergency medicine applies: survival probability decreases precipitously with elapsed time, particularly in adverse weather or difficult terrain. Every minute of search delay translates directly into reduced likelihood of positive outcome.

This scenario models a missing person search across a 2000m $\times$ 2000m forested region—approximately 400 hectares of mixed terrain including dense canopy, streams, clearings, and steep ravines. A fleet of four UAVs equipped with thermal imaging cameras must systematically cover the area within a 3-hour search window, balancing thoroughness against the urgency imposed by deteriorating survival odds.

The search area decomposes into a 100m $\times$ 100m grid yielding 400 discrete cells, each representing a scan task. This resolution balances thermal camera field-of-view against the need for detailed coverage; coarser grids risk missing prone or sheltered subjects, while finer grids would exceed the fleet's coverage capacity within the time window.

\subsection{Mission Setup}

Unlike surveillance missions where all zones demand continuous attention, search and rescue prioritization follows probabilistic reasoning. The search grid receives priority assignments based on terrain analysis, the subject's last known position (LKP), and probability-of-area (POA) calculations derived from search theory. Figure~\ref{fig:priorities} illustrates the hierarchical priority structure guiding task allocation.

\begin{figure}[H]
\centering
\includegraphics[width=0.6\textwidth]{images/priorities.png}
\caption{Priority Tree for Search Zones}
\label{fig:priorities}
\end{figure}

The fundamental insight driving this prioritization is that lost persons do not distribute randomly across terrain. Behavioral studies indicate predictable patterns: subjects seek water sources, follow trails, and gravitate toward visible shelter. These behavioral priors, combined with distance-decay models centered on the LKP, yield probability distributions that concentrate search resources where success is most likely.

\subsection{Search Grid Prioritization}

Algorithm 1's priority scoring mechanism adapts to the SAR context by weighting temporal urgency heavily—the golden hour modifier amplifies scores for cells that can be reached quickly, incentivizing rapid coverage of high-probability areas before time runs out.

The highest-priority cells ($P > 0.7$) encompass three critical zone types. Zone 1, spanning 100 cells within a 500-meter radius of the last known position, receives maximum priority ($P = 0.9$) based on the statistical observation that most lost persons are found close to their last confirmed location. Zone 2 covers 40 cells containing water sources—streams and ponds within one kilometer of the LKP—reflecting the survival instinct that drives subjects toward hydration. Zone 3 identifies 20 cells containing potential shelters: cabins, rock overhangs, and clearings where a disoriented person might seek refuge.

Medium-priority zones ($0.4 < P < 0.7$) extend coverage to probable travel corridors. Zone 4 traces 80 cells along hiking trails and access routes that a mobile subject might follow, while Zone 5 marks 60 cells of open clearings visible from altitude—areas where a subject attempting to signal for help would logically position themselves.

Low-priority zones ($P < 0.4$) cover terrain where discovery is less likely but not impossible. Zone 6 encompasses 80 cells of dense forest canopy where thermal detection is compromised, and Zone 7 addresses 20 cells of steep terrain—cliffs and ravines where an injured subject might have fallen but which a mobile subject would typically avoid.

\subsection{Initial Task Allocation}

\textbf{Fleet Assignment:}

The four-UAV fleet divides the 400-cell search grid according to zone priorities, with higher-priority areas receiving dedicated coverage from individual UAVs to maximize early detection probability.

\begin{table}[H]
\centering
\small
\begin{tabular}{|l|c|c|c|c|}
\hline
\textbf{UAV} & \textbf{Assigned Zone} & \textbf{Priority} & \textbf{Cells} & \textbf{Est. Time} \\
\hline
UAV-1 & LKP Radius (Zone 1) & \cellcolor{red!25}HIGH (0.9) & 100 & 25 min \\
UAV-2 & Water + Shelters (Zones 2-3) & \cellcolor{red!25}HIGH (0.8) & 60 & 15 min \\
UAV-3 & Trails + Clearings (Zones 4-5) & \cellcolor{yellow!25}MEDIUM (0.6) & 140 & 35 min \\
UAV-4 & Dense/Steep (Zones 6-7) & \cellcolor{green!25}LOW (0.3) & 100 & 25 min \\
\hline
\multicolumn{5}{|l|}{\textit{Coverage rate: 4 cells/minute (thermal scan + image capture)}} \\
\hline
\end{tabular}
\caption{Initial Search Grid Allocation by Priority}
\label{tab:sar_initial_allocation}
\end{table}

% \begin{figure}[H]
% \centering
% \includegraphics[width=0.8\textwidth]{images/sar_loop.png}
% \caption{Search and Rescue Mission OODA Loop}
% \label{fig:sar_loop}
% \end{figure}

\subsection{Failure Scenario: Critical Zone UAV Loss}

Eight minutes into the search, UAV-2 drops off the network. The vehicle had been sweeping the creek drainage---one of the highest-probability zones based on the subject's likely path toward water. Its thermal camera was scanning the dense undergrowth where a hypothermic hiker might seek shelter.

The GPS signal vanished when the drone descended into a narrow ravine, blocked by the terrain that makes this area so dangerous for lost hikers. After 90 seconds without position updates, the failsafe activates: UAV-2 climbs to safe altitude and returns to base, abandoning 48 unsearched cells in the highest-priority zones.

With the golden hour already ticking, these cells represent the areas most likely to contain the missing person. The remaining fleet must absorb this loss immediately.

\subsubsection{OODA Cycle Execution}

The OODA cycle for SAR follows the same four-phase structure demonstrated in the surveillance scenario, adapted for time-critical search operations where the golden hour constraint dominates all decisions.

\textbf{OBSERVE (T+0 to T+1.5s):} GPS signal loss triggers UAV-2's autonomous failsafe after exceeding the 90-second timeout threshold. Fleet state aggregation reveals: UAV-1 at 75\% battery in Zone 1 (20\% spare capacity), UAV-3 at 80\% in Zone 4 (35\% spare), and UAV-4 at 78\% in Zone 7 (30\% spare). The failure leaves 48 high-priority cells unsearched---water sources and shelters representing the highest-probability areas for locating the missing person.

\textbf{ORIENT (T+1.5s to T+3.0s):} Impact assessment quantifies 30\% degradation in critical zone coverage with 52 minutes remaining in the golden hour. Capacity analysis confirms the fleet can absorb 76 cells total (20 + 32 + 24), exceeding the 48-cell requirement by 58\%. The system formulates a priority-based strategy: UAV-1 absorbs 20 water cells (already nearby, zero transit cost), UAV-3 redirects to 20 shelter cells before continuing reduced trail coverage, and UAV-4 maintains standby status as backup.

\textbf{DECIDE (T+3.0s to T+4.5s):} Strategy selection yields full reallocation with priority optimization. The reallocation plan assigns UAV-1 an expanded 105-cell coverage (original 85 plus 20 water), UAV-3 receives 80 cells (20 shelters plus reduced 60 trails), and UAV-4 continues its 100-cell assignment with 8-cell backup designation. Constraint verification confirms both active UAVs maintain battery reserves above 20\%: UAV-1 projects 23\% remaining (3\% margin), UAV-3 projects 37\% (17\% margin). Collision avoidance passes with 600-meter spatial separation and sequential zone entry.

\textbf{ACT (T+4.5s to T+6.0s):} Command dispatch prioritizes golden hour utilization with critical-priority mission updates. Post-adaptation status: 100\% high-priority coverage recovered, medium-priority trails reduced to 57\%, overall mission completion at 90\%---acceptable for SAR where high-probability areas take precedence over exhaustive coverage.

\subsection{Scenario Outcome}

The OODA system achieved 100\% high-priority zone coverage recovery in 0.50 milliseconds (R5) and 0.16 milliseconds (R6)---the fastest computation times measured across all scenarios. Both variants maintained zero constraint violations while consuming only 0.0000138\% of the golden hour. Without adaptation, coverage would degrade to 66.7\%, leaving critical water features and shelter areas unsearched during the most survivable window.

The golden hour calculus reveals the stakes: manual operator intervention requires 465 seconds (7.75 minutes), consuming 12.9\% of the critical window before any remedial action begins. In wilderness search operations where survival probability decays exponentially with time, this 7.7-minute advantage is not merely operational efficiency---it represents the difference between rescue and recovery.

Scenario R6 validates the permission system's flexibility. When the optimal reallocation requires UAV-4's out-of-grid capability, the system correctly identifies and leverages this permission, achieving the fastest computation time while respecting each vehicle's authorized operational envelope. This demonstrates that constraint awareness need not sacrifice performance when permissions are properly modeled.

\section{SCENARIO 3: Medical Supply Delivery}

\subsection{Mission Context}

\begin{itemize}
    \item \textbf{Application:} Emergency medical supply delivery to rural clinics
    \item \textbf{Duration:} 1-hour delivery window
    \item \textbf{Fleet:} 3 UAVs (heterogeneous payload capacities)
    \item \textbf{Delivery Area:} 5 clinics distributed over 3000m $\times$ 2000m region
\end{itemize}

\subsection{Mission Setup}

\textbf{Fleet Specifications:}

The heterogeneous three-UAV fleet comprises one heavy-lifter and two standard vehicles, each with distinct payload and endurance characteristics.

\begin{table}[H]
\centering
\small
\begin{tabular}{|l|c|c|c|c|c|}
\hline
\textbf{UAV} & \textbf{Type} & \textbf{Payload} & \textbf{Battery} & \textbf{Speed} & \textbf{Route} \\
\hline
UAV-1 & Heavy Lifter & 5.0 kg & 25 min & 12 m/s & Clinics 1-2 \\
UAV-2 & Standard & 2.5 kg & 30 min & 15 m/s & Clinics 3-4 \\
UAV-3 & Standard & 2.5 kg & 30 min & 15 m/s & Clinic 5 \\
\hline
\end{tabular}
\caption{Delivery Fleet Configuration}
\label{tab:delivery_fleet}
\end{table}

\textbf{Package Priorities:}

Algorithm 1 assigns priority scores based on medical urgency, temporal deadlines, and delivery distance, resulting in a stratified priority hierarchy.

\begin{table}[H]
\centering
\small
\begin{tabular}{|c|c|c|c|c|c|}
\hline
\textbf{Pkg} & \textbf{Contents} & \textbf{Weight} & \textbf{Destination} & \textbf{Deadline} & \textbf{Priority} \\
\hline
A & Insulin & 2.5 kg & Clinic 1 (800, 1200) & 30 min & \cellcolor{red!25}1.0 (CRITICAL) \\
B & Antibiotics & 2.0 kg & Clinic 2 (1500, 800) & 45 min & \cellcolor{yellow!25}0.7 (HIGH) \\
C & Bandages & 1.2 kg & Clinic 3 (2200, 1500) & 60 min & 0.4 (MEDIUM) \\
D & Gauze & 1.0 kg & Clinic 4 (2800, 600) & 60 min & 0.4 (MEDIUM) \\
E & Vitamins & 1.8 kg & Clinic 5 (1200, 300) & 90 min & 0.2 (LOW) \\
\hline
\end{tabular}
\caption{Package Prioritization by Medical Urgency}
\label{tab:delivery_packages}
\end{table}

\textbf{Delivery Route Map:}

\begin{figure}[H]
\centering
\includegraphics[width=0.85\textwidth]{images/delivery_route_map.png}
\caption{Delivery Route Map with Clinic Locations}
\label{fig:delivery_map}
\end{figure}

\subsection{Failure Scenario: Heavy Lifter Battery Anomaly}

Fifteen minutes into the delivery mission, UAV-1 is fighting an unexpected headwind. The heavy lifter is carrying 4.5 kilograms of medical supplies---insulin vials that will degrade if not delivered within the hour. The combination of payload mass and wind resistance is draining its battery nearly twice as fast as planned.

The numbers tell a sobering story: 40\% battery remaining instead of the projected 55\%. At this rate, UAV-1 cannot complete its route and return safely. It carries two critical packages---one destined for Clinic B (priority 0.95), another for Clinic E (priority 0.60). One of these deliveries will not happen as planned.

% \begin{figure}[H]
% \centering
% \includegraphics[width=0.8\textwidth]{images/delivery_loop.png}
% \caption{Delivery Mission OODA Loop}
% \label{fig:delivery_loop}
% \end{figure}

\subsubsection{OODA Cycle Execution}

The delivery scenario demonstrates the OODA system's intelligent escalation capability---recognizing when autonomous recovery is physically impossible and deferring appropriately to human operators.

\textbf{OBSERVE (T+0 to T+1.5s):} Battery anomaly detection triggers when UAV-1's discharge rate (5\%/min) exceeds baseline (3\%/min) by 1.5$\times$. Battery projection analysis reveals the current 40\% charge cannot support the full route: completing Package A delivery (200m) leaves 38.6\%, but continuing to Package B (1100m additional) and returning to depot (1500m) would deplete to 16.6\%---below the 20\% safety threshold. Package A (insulin, priority 1.0) remains deliverable; Package B (antibiotics, priority 0.7) becomes infeasible.

\textbf{ORIENT (T+1.5s to T+3.0s):} Fleet capacity analysis reveals no autonomous solution exists. UAV-2 carries 2.2 kg (packages C+D) against 2.5 kg capacity, leaving 0.3 kg spare. UAV-3 carries 1.8 kg (package E), leaving 0.7 kg spare. Neither can absorb Package B's 2.0 kg weight---a 1.3 kg (65\%) capacity deficit. The system evaluates alternatives: return-to-base swap (10--15 min, marginal), backup UAV-4 deployment (13 min, recommended), and ground vehicle (20--30 min, deadline violation).

\textbf{DECIDE (T+3.0s to T+4.5s):} Algorithm 4 (Operator Escalation Decision) determines autonomous recovery is infeasible. The decision matrix evaluates: coverage recovery at 80\% (passes >75\% threshold), but one critical task lost (Package B, $P=0.7$) and one constraint violation (payload). The system selects operator escalation with backup UAV recommendation, configuring a 30-second auto-failsafe timer. Autonomous actions proceed immediately: UAV-1 receives modified route (Package A only, then RTL), while UAV-2 and UAV-3 continue unchanged.

\textbf{ACT (T+4.5s to T+5.5s):} The system dispatches mission updates and presents the operator decision interface. The critical alert summarizes: Package B (2.0 kg, $P=0.7$) undeliverable due to payload constraints; recommended action is backup UAV-4 deployment (13 min, meets 45-minute deadline); backup option is ground vehicle (20--30 min, deadline risk). Final system state: operator escalation active, 80\% autonomous coverage preserved, zero constraint violations, safety maintained.

\begin{table}[H]
\centering
\small
\begin{tabular}{|l|c|c|}
\hline
\textbf{Decision Criterion} & \textbf{Threshold} & \textbf{Status} \\
\hline
Coverage Recovery & >75\% & \cellcolor{green!25}80\% (4/5 packages) \\
Critical Tasks Lost ($P > 0.7$) & 0 & \cellcolor{red!25}1 (Package B) \\
Constraint Violations & 0 & \cellcolor{red!25}1 (payload) \\
Autonomous Feasibility & TRUE & \cellcolor{red!25}\textbf{FALSE} \\
\hline
\multicolumn{3}{|l|}{\textbf{Decision:} Escalate to operator (critical task unrecoverable)} \\
\hline
\end{tabular}
\caption{Operator Escalation Decision Matrix}
\label{tab:delivery_escalation_decision}
\end{table}

\subsection{Scenario Outcome}

The delivery scenarios (D6 and D7) demonstrate what constraint-aware design looks like when physical reality defeats autonomous recovery. In D6, Package B (2.0 kg antibiotics) exceeds every available UAV's spare payload capacity---the maximum available is 0.7 kg. In D7, the destination coordinates (3500, 2500) lie outside the operational grid boundary, and no UAV holds out-of-grid permission. The OODA system correctly identified both situations as autonomously infeasible in 0.11 ms and 0.23 ms respectively, escalating to operators while preserving zero constraint violations.

This outcome inverts the usual success metric. A greedy baseline algorithm would report 100\% coverage by assigning Package B to UAV-2---overloading a 2.5 kg capacity vehicle to 4.2 kg. Such ``success'' would ground the vehicle or cause mid-flight failure. The OODA system's 0\% autonomous coverage in these scenarios is not failure; it is constraint-aware intelligence refusing to generate plans that physics cannot execute.

The system autonomously handles what it can: Package A (critical insulin, priority 1.0) proceeds to Clinic 1 via UAV-1's abbreviated route. The operator receives a structured escalation alert with recommended actions and a 30-second auto-failsafe timer. This hybrid approach---80\% autonomous plus 20\% supervised---proves more deployable than systems claiming 100\% autonomy while ignoring physical constraints. Honest capability assessment satisfies BVLOS regulatory requirements while avoiding mission-critical failures.

\section{Cross-Scenario Comparative Analysis}

\begin{table}[H]
\centering
\caption{Measured Performance by Scenario}
\label{tab:scenario_performance}
\begin{tabular}{|l|c|c|c|}
\hline
\textbf{Metric} & \textbf{Surveillance (S5)} & \textbf{SAR (R5/R6)} & \textbf{Delivery (D6/D7)} \\
\hline
Coverage Recovery & \textbf{100\%} & \textbf{100\%} & \textbf{0\% (escalated)} \\
OODA Computation & \textbf{0.34 ms} & \textbf{0.50/0.16 ms} & \textbf{0.11/0.23 ms} \\
Operator Escalation & 0\% (autonomous) & 0\% (autonomous) & 100\% (correct) \\
Constraint Violations & \textbf{0} & \textbf{0} & \textbf{0} \\
Dominant Constraint & Battery & Time (golden hour) & Payload \\
\hline
\end{tabular}
\end{table}

The three scenarios validate consistent OODA behavior across diverse operational contexts while revealing mission-specific constraint priorities. Surveillance and SAR achieved full autonomous recovery because their constraints (battery, time) permitted reallocation within fleet capacity. Delivery scenarios D6/D7 correctly escalated to operators when payload constraints (2.0 kg package vs. 0.7 kg maximum spare capacity) made autonomous recovery physically impossible---demonstrating that intelligent escalation is a feature, not a failure mode.

All scenarios completed OODA computation in under 0.5 milliseconds (range: 0.11--0.50 ms), six orders of magnitude faster than manual operator response. Zero constraint violations occurred across all five experimental runs. The 80--100\% autonomous recovery rate reduces operator workload significantly compared to fully manual intervention, while honest acknowledgment of physical limitations ensures deployable systems rather than theoretical claims that cannot survive contact with reality.

\chapter{Implementation and Validation Plan}

\section{Simulation Environment}

Validating fault-tolerant control systems presents a fundamental challenge: real-world UAV failures are dangerous, expensive, and difficult to reproduce systematically. This research therefore employs software-in-the-loop simulation (SOARES et al., 2025), implementing physically-grounded vehicle dynamics within a controlled computational environment where failures can be injected deterministically and experiments repeated indefinitely.

The simulation framework implements six-degree-of-freedom quaternion dynamics with cascaded PID control for attitude stabilization and waypoint tracking. Vehicle parameters represent a generic mid-size quadrotor platform (1.5 kg mass, 100 Wh battery capacity, 2.5 kg maximum payload) characteristic of commercial multi-rotor systems deployed in surveillance and delivery applications. The Quad SimCon codebase (QUAD SIMCON, 2020) serves as the foundational reference for dynamics modeling, adapted to support multi-vehicle coordination and centralized fault management.

Communication between simulated UAVs and the Ground Control Station employs TCP/IP transport with JSON-RPC 2.0 protocol, mirroring the request-response patterns of real telemetry systems. The GCS server (port 5555) aggregates fleet state at 2 Hz, while a Flask-based web dashboard (port 8085) provides real-time visualization through WebSocket streaming via SocketIO.

The implementation distributes functionality across purpose-specific modules. The OODA engine (\texttt{gcs/ooda\_engine.py}) orchestrates phase timing and state transitions. Fleet monitoring (\texttt{gcs/fleet\_monitor.py}) implements multi-modal failure detection. Constraint validation (\texttt{gcs/constraint\_validator.py}) enforces battery, payload, and collision safety margins. The mission manager (\texttt{gcs/mission\_manager.py}) maintains the task database with assignment tracking, while the objective function module (\texttt{gcs/objective\_function.py}) implements the two-stage optimization strategy. Vehicle dynamics reside in \texttt{uav/simulation.py}, with GCS communication handled by \texttt{uav/client.py}. The complete source is publicly available (EULÁLIO REIS, 2025).

\section{Validation Metrics}

Quantifying fault-tolerance effectiveness requires metrics that capture both recovery completeness and temporal responsiveness. Four primary metrics anchor the validation framework, each with established targets derived from operational requirements and baseline human performance.

\textbf{Coverage Recovery} measures the fraction of orphaned tasks successfully reassigned to healthy UAVs:

\begin{equation}
\rho = \frac{|\mathcal{T}_{\text{reallocated}}|}{|\mathcal{T}_{\text{failed}}|} \times 100\%
\label{eq:coverage}
\end{equation}

\noindent where $\mathcal{T}_{\text{failed}}$ denotes tasks originally assigned to the failed UAV and $\mathcal{T}_{\text{reallocated}}$ represents those successfully reassigned. The target threshold is $\rho > 65\%$ for single-UAV failures and $\rho > 50\%$ for simultaneous double failures, reflecting the diminishing fleet capacity available for absorption.

\textbf{Adaptation Time} captures end-to-end system responsiveness from failure detection to command dispatch:

\begin{equation}
T_{\text{adapt}} = T_{\text{ACT\_complete}} - T_{\text{failure\_detected}}
\label{eq:adaptation}
\end{equation}

This interval encompasses the complete OODA cycle plus any communication latency. The target $T_{\text{adapt}} < 5$ seconds ensures that mission disruption remains bounded; longer delays allow coverage gaps to propagate and time-critical tasks to expire.

\textbf{Battery Efficiency} evaluates how effectively the system exploits available fleet capacity:

\begin{equation}
\eta_{\text{battery}} = \frac{\sum_{u \in \mathcal{U}_{\text{healthy}}} \Delta B_u}{\sum_{u \in \mathcal{U}_{\text{healthy}}} B_{\text{spare},u}} \times 100\%
\label{eq:efficiency}
\end{equation}

\noindent where $\Delta B_u$ represents battery consumed by UAV $u$ for reallocation tasks and $B_{\text{spare},u}$ denotes available capacity above the safety reserve. Efficiency below 80\% suggests conservative allocation leaving recoverable tasks unassigned; efficiency approaching 100\% indicates the fleet operated near capacity limits.

\textbf{Mission Completion Rate} provides the aggregate success metric across experimental trials:

\begin{equation}
\xi = \frac{N_{\text{completed}}}{N_{\text{total}}} \times 100\%
\label{eq:completion}
\end{equation}

A mission counts as completed if primary objectives are achieved despite failures, with the target $\xi > 90\%$ for single-failure scenarios acknowledging that some failure combinations inevitably exceed compensatory capacity.

Secondary metrics supplement this primary framework: operator workload measured by escalation frequency per mission, communication bandwidth consumption in kilobytes per second, and decision quality assessed against optimal solutions computed offline via exhaustive search for small problem instances.

\section{Test Scenarios}

Comprehensive validation demands testing at multiple abstraction levels, from isolated algorithmic components to complete mission execution under failure conditions. The test suite comprises 169 automated cases organized into three hierarchical categories, executable in approximately 0.22 seconds via pytest—enabling continuous validation throughout development without impeding iteration velocity.

\textbf{Unit tests} (53 cases) validate individual algorithmic components in isolation. Battery management tests verify reserve calculations and discharge rate monitoring. Grid boundary tests confirm spatial constraint enforcement. State machine tests exercise all valid transitions and reject invalid state progressions. Constraint validation tests confirm that battery, payload, and collision checks correctly accept feasible allocations and reject violations. Priority scoring tests verify Algorithm 1's output against hand-calculated expected values across diverse task configurations.

\textbf{Integration tests} (81 cases) exercise subsystem interactions and mission-specific workflows. Surveillance mission tests validate continuous coverage maintenance with rotation scheduling across simulated 2-hour operations. Search and rescue tests verify grid-based coverage patterns with golden hour deadline enforcement, confirming that the system correctly prioritizes time-critical cells. Medical delivery tests exercise the two-phase pickup-dropoff workflow with payload constraint tracking. Cross-cutting integration tests validate multi-UAV coordination, collision avoidance during concurrent operations, and complete OODA cycle execution from failure injection through recovery confirmation.

\textbf{Regression tests} (15 cases) preserve fixes for defects discovered during development. Each regression test encodes a specific failure mode—edge cases in constraint boundary conditions, race conditions in concurrent state updates, or numerical precision issues in distance calculations—ensuring that resolved issues do not resurface as the codebase evolves.

Five experimental scenarios provide the primary validation evidence presented in Chapter~6. Scenario S5 exercises surveillance mission recovery from single-UAV battery depletion. Scenarios R5 and R6 test search and rescue operations under GPS loss, with R6 introducing an out-of-grid zone requiring operator permission. Scenarios D6 and D7 challenge the delivery mission with payload constraint violations and out-of-grid destinations respectively, validating the intelligent escalation behavior that distinguishes constraint-aware autonomy from naive allocation strategies.

\section{Baseline Comparisons}

Demonstrating the value of constraint-aware fault tolerance requires comparison against alternative approaches spanning the spectrum from no adaptation to optimal human performance. Four baseline strategies establish the comparative framework.

\textbf{No Adaptation} represents the null hypothesis: missions proceed with fixed assignments, and any UAV failure results in permanent loss of that vehicle's tasks. This baseline establishes the cost of ignoring failures entirely—expected coverage recovery of 0\% with guaranteed mission degradation proportional to the failed UAV's assignment load. While obviously suboptimal, this baseline quantifies the improvement attributable to any adaptive strategy.

\textbf{Manual Operator Replanning} represents the current operational standard in supervised UAV deployments. Upon failure notification, a human operator assesses fleet state, identifies available capacity, and manually reassigns tasks through the ground control interface. Skilled operators can achieve 80--95\% coverage recovery, but the process requires 5--10 minutes—an interval during which coverage gaps persist and time-critical tasks may expire. This baseline establishes the quality ceiling achievable without automation and the temporal cost that autonomous systems must improve upon.

\textbf{Greedy Nearest-Neighbor} implements the simplest autonomous reallocation: assign each orphaned task to the nearest UAV without constraint verification. This approach achieves rapid computation but ignores battery limitations, payload capacity, and collision risks. Expected coverage recovery ranges from 40--60\%, with the shortfall attributable to infeasible assignments that appear successful in the allocation but fail during execution when vehicles exhaust batteries or exceed payload limits.

\textbf{Hybrid OODA} (this work) combines priority-based allocation with comprehensive constraint validation. By integrating Algorithm 1's priority scoring with Algorithm 2's constraint-aware assignment, the system targets 65--95\% coverage recovery with sub-5-second adaptation time—matching or exceeding manual operator quality at three orders of magnitude faster response.

\section{Visualization and Logging}

Effective fault-tolerance development requires visibility into system behavior at multiple timescales: real-time monitoring during execution and post-hoc analysis for systematic evaluation. The implementation provides both through complementary visualization interfaces.

The real-time web dashboard, accessible at port 8085, renders fleet state through a Flask application with WebSocket streaming via SocketIO. A geographic map displays UAV positions with task coverage overlaid as a color-coded heatmap—completed tasks fade to green while pending assignments remain amber, and failed UAV positions flash red upon detection. Battery state appears as horizontal bar charts updated at 2 Hz, with the 20\% reserve threshold marked for immediate visual assessment of capacity margins. An OODA timeline panel tracks cycle execution, highlighting active phases and displaying computation times for each stage. The operator alert log scrolls escalation events with timestamps, urgency levels, and recommended actions, providing the human-in-the-loop interface required for supervised autonomy.

Post-mission analysis generates structured outputs for systematic evaluation. Coverage percentage traces over time reveal recovery dynamics—how quickly the system restores coverage after failure detection and whether the restored level remains stable. Adaptation time distributions across multiple trials expose variability in system responsiveness. Scatter plots correlating task priority with allocation success identify whether high-priority tasks receive preferential treatment as intended. Battery efficiency heatmaps by UAV reveal load balancing patterns, highlighting vehicles that absorbed disproportionate reallocation burden. All experimental results export to JSON format, enabling reproducible analysis and cross-study comparison with standardized data structures.

\chapter{Experimental Results and Validation}

\section{Quantitative Performance Results}

The system was validated through five experimental scenarios comparing OODA-based fault tolerance against three baseline strategies. Results demonstrate performance significantly exceeding initial expectations.

\subsection{Executive Summary}

Table~\ref{tab:executive_summary} provides a comprehensive overview of all experimental scenarios, enabling quick comparison across mission types.

\begin{table}[H]
\centering
\caption{Executive Summary of Experimental Results}
\label{tab:executive_summary}
\small
\begin{tabular}{|l|c|c|c|c|c|}
\hline
\textbf{Scenario} & \textbf{Mission} & \textbf{Coverage} & \textbf{Time} & \textbf{Safety} & \textbf{Outcome} \\
 & \textbf{Type} & \textbf{Recovery} & \textbf{(ms)} & \textbf{Violations} & \\
\hline
\textbf{S5} & Surveillance & 100\% & 0.34 & 0 & Full Autonomous \\
\hline
\textbf{R5} & SAR & 100\% & 0.50 & 0 & Full Autonomous \\
\hline
\textbf{R6} & SAR (OOG) & 100\% & 0.16 & 0 & Full Autonomous \\
\hline
\textbf{D6} & Delivery & 0\% (esc.) & 0.11 & 0 & Intelligent Escalation \\
\hline
\textbf{D7} & Delivery & 0\% (esc.) & 0.23 & 0 & Intelligent Escalation \\
\hline
\multicolumn{6}{|l|}{\textit{OOG = Out-of-Grid; esc. = escalated to operator}} \\
\hline
\end{tabular}
\end{table}

\begin{figure}[H]
\centering
\includegraphics[width=0.9\textwidth]{images/chapter6_decision_flow.png}
\caption{OODA Decision Flow in Experimental Scenarios}
\label{fig:ch6_decision_flow}
\end{figure}

\begin{figure}[H]
\centering
\includegraphics[width=0.9\textwidth]{images/chapter6_performance_radar.png}
\caption{Multi-Dimensional Performance Comparison}
\label{fig:ch6_radar}
\end{figure}

\subsection{Detailed Performance Metrics}

The measured OODA adaptation timings reveal performance characteristics that substantially exceed initial design expectations. The surveillance scenario (S5) completed its entire OODA cycle in 0.34 milliseconds, achieving a response 14,706 times faster than the conservative five-second target established during system design. Search and rescue operations demonstrated comparable efficiency, with scenario R5 completing in 0.50 milliseconds and the out-of-grid variant R6 achieving the fastest measured response at 0.16 milliseconds. Even the delivery scenarios requiring constraint violation detection and operator escalation (D6 and D7) completed their analysis in 0.11 and 0.23 milliseconds respectively, demonstrating that safety-critical constraint checking imposes negligible computational overhead.

Coverage recovery patterns across mission types illuminate the system's adaptive capabilities under realistic constraints. Both surveillance and search-and-rescue missions achieved complete 100\% task recovery following single-UAV failures, substantially exceeding the 75-95\% target range established during design. This outcome validates the priority-based reallocation strategy's effectiveness when sufficient spare capacity exists within the operational fleet. The delivery scenarios presented a contrasting yet equally important result: zero percent autonomous reallocation accompanied by intelligent escalation to human operators. This apparent absence of autonomous recovery represents not system failure but rather successful constraint violation detection---the system correctly identified physically infeasible reallocations and appropriately deferred to human judgment rather than compromising safety boundaries.

The speed advantage over manual operator intervention proves particularly striking when examined quantitatively. While the OODA system completes its full decision cycle in 0.16 to 0.50 milliseconds on average, manual operator replanning requires approximately 465 seconds (7.75 minutes) to achieve equivalent task reallocation. This represents a performance multiplication factor approaching 500,000, transforming response latency from minutes to sub-millisecond timescales. For time-critical search and rescue operations conducted under golden-hour constraints, this 7.7-minute advantage can directly influence mission success probability and, in life-threatening scenarios, determine whether intervention arrives within the window of survivability.

Safety validation across all experimental scenarios demonstrates perfect constraint adherence. The OODA system maintained zero constraint violations throughout all five test scenarios, respecting battery reserves, payload limits, and operational boundaries without exception. In stark contrast, the greedy baseline algorithm---which prioritizes coverage maximization without explicit constraint verification---produced two distinct safety violations: a payload overload in scenario D6 where package weight exceeded available capacity, and a boundary transgression in scenario D7 where destination coordinates fell outside permitted flight zones. Manual operator intervention similarly achieved zero violations but at the cost of unacceptable 7.75-minute response latency, reinforcing the value proposition of automated constraint-aware decision-making that combines both speed and safety.

\begin{figure}[H]
\centering
\includegraphics[width=0.9\textwidth]{images/chapter6_time_comparison.png}
\caption{Computation Time Comparison Across Approaches (Log Scale)}
\label{fig:ch6_time}
\end{figure}

\begin{figure}[H]
\centering
\includegraphics[width=0.85\textwidth]{images/chapter6_safety_violations.png}
\caption{Safety Validation: Constraint Violations by Approach}
\label{fig:ch6_safety}
\end{figure}

\section{Experimental Scenario Analysis}

\subsection{S5: Surveillance Mission Recovery}

The surveillance scenario (S5) evaluated system response to a single UAV failure during sustained perimeter monitoring operations. This experiment compared four distinct operational strategies to isolate the contribution of constraint-aware adaptive planning. The no-adaptation baseline, which simply aborted affected tasks without reallocation, achieved only 87.5\% coverage---effectively declaring mission failure upon detecting the fault. The greedy nearest-neighbor heuristic recovered full 100\% coverage in 0.18 milliseconds by assigning orphaned tasks to the spatially nearest available vehicles, though this approach succeeded only because the particular failure configuration happened not to violate capacity constraints. Manual operator intervention similarly achieved complete recovery but required 465 seconds of human cognitive processing to assess fleet state, evaluate alternatives, and formulate the reallocation plan.

The OODA system achieved 100\% coverage recovery in 0.34 milliseconds while maintaining perfect safety through explicit constraint verification at each allocation step. This represents a response 1,367,000 times faster than manual replanning---a performance advantage that transforms fault recovery from a multi-minute operational disruption into an essentially instantaneous adaptation invisible to mission progress. More significantly, the OODA approach guaranteed constraint satisfaction through sequential battery, payload, and collision verification, ensuring that the recovered mission plan remained executable rather than merely optimal on paper. All battery reserves remained above the 20\% safety threshold, spatial separation exceeded the 15-meter minimum, and no vehicle received task assignments beyond its physical capabilities.

\begin{figure}[H]
\centering
\includegraphics[width=0.95\textwidth]{images/surveillance_dashboard.png}
\caption{S5 Surveillance Mission: Real-time trajectory visualization showing UAV-3 failure (red marker) at T+45min and dynamic reallocation of Zone C coverage to UAV-2 and UAV-4. Dotted lines indicate pre-failure patrol circuits; solid lines show post-adaptation extended routes maintaining 100\% spatial coverage with reduced waypoint density.}
\label{fig:surveillance_trajectory}
\end{figure}

\subsection{R5 \& R6: Search \& Rescue with Time Criticality}

The search and rescue scenarios (R5 and R6) examined system performance under the most stringent temporal constraints encountered in civilian UAV operations: the sixty-minute golden hour during which victim survival probability remains highest. Scenario R5 simulated a standard UAV failure during systematic grid search operations, where the no-adaptation baseline abandoned 33.3\% of the search area---effectively leaving one-third of high-probability zones unsearched and potentially condemning an injured person to prolonged exposure. The OODA system recovered complete 100\% coverage in 0.50 milliseconds, consuming a negligible 0.0000138\% of the precious golden hour interval. Manual operator response achieved equivalent coverage but required 465 seconds (7.75 minutes), consuming 12.9\% of the time-critical window. In scenarios where every minute directly correlates with survival probability, this 7.7-minute differential represents not merely a performance metric but potentially the margin between life and death.

Scenario R6 introduced an additional complexity by positioning one high-priority search zone ten meters outside the nominal operational grid boundaries [0-1000, 0-1000]. This configuration tested the system's regulatory compliance mechanisms: could it correctly identify permitted vehicles for out-of-bounds operations while refusing allocation to unauthorized units? The experiment validated this capability comprehensively. With UAV-4 having received explicit out-of-grid authorization from the operator prior to mission commencement, the OODA system correctly reallocated the exterior zone to this permitted vehicle while excluding unauthorized units from consideration. The decision completed in 0.16 milliseconds---the fastest measured across all experimental scenarios---demonstrating that permission verification and regulatory constraint checking impose minimal computational burden. This result confirms that safety-first design principles need not compromise response speed when implemented through efficient data structures and sequential constraint evaluation.

\begin{figure}[H]
\centering
\includegraphics[width=0.95\textwidth]{images/sar_dashboard.png}
\caption{R5/R6 Search \& Rescue Mission: Grid-based search pattern showing UAV-2 failure during high-priority zone coverage (water sources, shelters). The visualization displays systematic cell-by-cell coverage with priority-based reallocation ensuring 100\% high-probability area completion within the golden hour constraint. Color-coded cells indicate completion status: green (completed), amber (pending), red (orphaned by failure, reassigned).}
\label{fig:sar_trajectory}
\end{figure}

\begin{figure}[H]
\centering
\includegraphics[width=\textwidth]{images/chapter6_golden_hour.png}
\caption{Search \& Rescue: Golden Hour Time Consumption}
\label{fig:ch6_golden_hour}
\end{figure}

\subsection{D6 \& D7: Delivery with Intelligent Escalation}

The delivery scenarios (D6 and D7) presented perhaps the most philosophically significant experimental results by validating the system's capacity to recognize its own limitations and appropriately defer to human judgment. Scenario D6 configured a deliberately infeasible task reallocation: following primary delivery vehicle failure, the stranded Package B weighed 2.0 kilograms while the maximum spare payload capacity across all operational vehicles reached only 0.7 kilograms. This represents a fundamental physical impossibility---no combination of available resources could legally transport the package without exceeding structural load limits.

The contrasting responses across different approaches illuminate critical distinctions in system design philosophy. The greedy baseline algorithm reported 100\% coverage recovery by simply assigning the package to the nearest available vehicle, producing a mission plan that would overload the selected UAV by nearly three times its spare capacity---a configuration guaranteed to cause either structural damage or flight instability. Manual operator analysis correctly identified the infeasibility, reporting zero percent autonomous coverage and acknowledging the need for alternative solutions such as backup vehicle deployment. The OODA system matched this correct assessment, completing constraint analysis in 0.11 milliseconds before escalating to operator intervention with explicit explanation: "Package B (2.0 kg) exceeds maximum available spare capacity (0.7 kg). Recommend backup UAV deployment or ground vehicle dispatch."

Scenario D7 examined boundary constraint violations by positioning a delivery destination at coordinates (3500, 2500) meters---well outside the authorized operational grid spanning [0-3000, 0-2000]. Unlike scenario R6 where a permitted vehicle existed for out-of-bounds operations, no UAV in scenario D7 carried the necessary authorization. The greedy algorithm again reported perfect coverage while violating the spatial boundary constraint. The OODA system refused this unsafe allocation, completing analysis in 0.23 milliseconds before escalating with regulatory justification: "Destination outside authorized flight zone. No vehicle possesses out-of-grid permission. Operator intervention required for boundary waiver or mission modification."

These results validate a crucial principle often overlooked in autonomous systems research: intelligent refusal represents successful operation, not system failure. The OODA system's zero percent autonomous reallocation in physically or regulatorily infeasible scenarios demonstrates correct constraint detection and appropriate escalation rather than algorithmic inadequacy. A system that claims 100\% autonomy by violating safety constraints provides less operational value than one that achieves 60\% autonomous coverage while correctly identifying when human judgment becomes necessary. This honest acknowledgment of limitations distinguishes deployable systems from theoretical frameworks that assume constraints away.

\begin{figure}[H]
\centering
\includegraphics[width=0.95\textwidth]{images/delivery_dashboard.png}
\caption{D6/D7 Medical Delivery Mission: Point-to-point delivery routes showing UAV-1 battery anomaly during critical insulin transport. The trajectory visualization illustrates the system's intelligent escalation behavior---Package A (priority 1.0, deliverable) proceeds via abbreviated route while Package B (priority 0.7, 2.0 kg payload exceeding 0.7 kg spare capacity) triggers operator escalation rather than generating an infeasible assignment. Dashboard displays real-time operator alert interface with recommended actions and 30-second auto-failsafe timer.}
\label{fig:delivery_trajectory}
\end{figure}

\begin{figure}[H]
\centering
\includegraphics[width=0.9\textwidth]{images/chapter6_constraint_space.png}
\caption{D6 Payload Constraint Violation: Geometric Illustration of Escalation Necessity}
\label{fig:ch6_constraint}
\end{figure}

\begin{figure}[H]
\centering
\includegraphics[width=0.9\textwidth]{images/chapter6_coverage_heatmap.png}
\caption{Coverage Recovery Matrix Across All Approaches and Scenarios}
\label{fig:ch6_coverage}
\end{figure}

\section{Synthesis and Contributions}

Table~\ref{tab:scenario_comparison} synthesizes performance characteristics across all experimental scenarios.

\begin{table}[H]
\centering
\caption{Measured Performance by Scenario (S5, R5/R6, D6/D7)}
\label{tab:scenario_comparison}
\begin{tabular}{|l|c|c|c|}
\hline
\textbf{Metric} & \textbf{Surveillance (S5)} & \textbf{SAR (R5/R6)} & \textbf{Delivery (D6/D7)} \\
\hline
Coverage Recovery & \textbf{100\%} & \textbf{100\%} & \textbf{0\% (escalated)} \\
\hline
OODA Computation & \textbf{0.34 ms} & \textbf{0.50/0.16 ms} & \textbf{0.11/0.23 ms} \\
\hline
Operator Escalation & 0\% (autonomous) & 0\% (autonomous) & 100\% (correct) \\
\hline
Constraint Violations & \textbf{0} & \textbf{0} & \textbf{0} \\
\hline
\end{tabular}
\end{table}

Four patterns emerge from this data. First, computational speed remains uniformly under half a millisecond across all mission types, validating the greedy heuristic approach over optimization algorithms with unpredictable execution times. Second, zero constraint violations occurred despite widely varying failure conditions---a perfect safety record that distinguishes this approach from opportunistic algorithms. Third, the system demonstrates selective autonomy: aggressive adaptation when feasible (surveillance, SAR), intelligent refusal when infeasible (delivery). Fourth, response times compress seven minutes of human cognitive processing into sub-millisecond algorithmic execution.

\subsection{Claims Validated}

The experimental program validated all primary thesis claims:

\begin{itemize}
\item \textbf{Rapid adaptation:} Measured times of 0.11--0.50 ms exceed the 5-second design target by four orders of magnitude
\item \textbf{Safety guarantee:} Zero constraint violations across all scenarios (battery, payload, spatial boundaries)
\item \textbf{Intelligent escalation:} D6 and D7 correctly refused infeasible reallocations rather than producing unsafe plans
\item \textbf{Time-critical advantage:} 7.7-minute advantage over manual operators in search-and-rescue scenarios
\item \textbf{Coverage recovery:} 100\% task recovery in surveillance and SAR missions, exceeding the 75--95\% target
\end{itemize}

\subsection{Research Contributions}

The work advances three contributions. First, a \textbf{unified constraint verification framework} that simultaneously enforces battery reserves, payload limits, and temporal deadlines through fail-fast sequential checking---the first multi-UAV fault tolerance system to address all three constraint categories together. Second, a \textbf{quantified degradation framework} with explicit decision rules governing when autonomous reallocation transitions from feasible to infeasible, enabling honest acknowledgment of system limitations rather than claiming universal fault tolerance. Third, a \textbf{hybrid autonomy architecture} that balances machine speed with human oversight, enabling deployment under current BVLOS regulations that require operator-in-the-loop supervision.

Beyond algorithmic innovation, the research delivers practical value: compatibility with commercial UAV platforms, 60\% reduction in operator intervention (S5, R5, R6 handled autonomously), and a comprehensive test suite (169 tests, 0.22s execution) enabling rapid validation. The open-source simulation platform provides infrastructure for future research without requiring physical hardware.

The philosophical contribution may prove most significant: demonstrating that realism-first design---explicitly modeling constraints rather than abstracting them away---produces systems that are both academically rigorous and operationally deployable.

\chapter{Limitations and Future Work}

This chapter presents a critical assessment of the proposed OODA-based fault-tolerant control system, examining current limitations and future research directions. Understanding these constraints is essential for establishing realistic performance expectations and identifying opportunities for system enhancement.

\section{Architectural Constraints}

\subsection{Centralized Control Architecture}

The system employs a centralized Ground Control Station for OODA loop execution, creating a single point of failure. Should the GCS experience hardware failure or communication loss, the fleet's adaptive capabilities are compromised. Individual UAVs implement autonomous Return-to-Launch protocols upon GCS timeout detection, preserving vehicle safety while sacrificing mission completion.

Future work could explore distributed OODA architectures using consensus algorithms for peer-to-peer coordination, building on hierarchical decentralized control approaches that handle communication failures (IZADI; GORDON; ZHANG, 2013). This approach would eliminate the centralization vulnerability while introducing new challenges including increased communication complexity, network partitioning risks, and Byzantine fault tolerance requirements. The transition to distributed coordination must carefully balance robustness against implementation complexity.

\subsection{Fleet Scalability}

The system supports fleets of three to twelve UAVs. Beyond this scale, two constraints become significant. Communication bandwidth scales linearly with fleet size, reaching 48 kilobytes per second for twelve UAVs at 2 Hz telemetry rates. Computational complexity for collision avoidance scales quadratically, as each reallocation requires pairwise verification among all vehicles. At twenty UAVs, collision checking workload increases 2.8-fold compared to twelve UAVs, potentially exceeding the six-second OODA cycle target.

Hierarchical architectures that partition large fleets into coordinated subgroups could address these limitations. Alternatively, computationally efficient approximate collision avoidance methods using spatial hashing could reduce verification complexity. These extensions would enable applications requiring coordination of dozens or hundreds of vehicles.

\subsection{Validation Methodology}

Current validation relies exclusively on software-in-the-loop simulation using physics-based models representative of mid-size commercial quadrotors. While appropriate for algorithm development, simulation abstracts real-world phenomena including GPS multipath errors, communication packet corruption, sensor noise, and environmental disturbances.

Hardware-in-the-loop testing with physical flight controllers would provide intermediate validation capturing timing constraints and communication latencies. Field trials with two to three physical UAVs would expose the system to genuine environmental challenges and enable parameter refinement. Future validation should prioritize systematic characterization of performance degradation under realistic operating conditions, establishing the operational envelope for reliable performance.

\section{Algorithmic Limitations}

\subsection{Greedy Task Reallocation}

The constraint-aware reallocation algorithm employs a greedy heuristic that assigns tasks to the nearest UAV satisfying capacity constraints. This achieves rapid execution compatible with real-time OODA requirements but does not guarantee globally optimal allocation. Early assignment decisions may consume capacity better reserved for higher-priority tasks, particularly when spare capacity is marginal and failed tasks are widely distributed.

Mixed-integer linear programming could guarantee optimal solutions for small to medium problems, though solution times may exceed OODA cycle budgets. Auction-based algorithms, particularly the Consensus-Based Bundle Algorithm, offer a promising middle ground achieving near-optimal solutions through distributed iterative bidding with polynomial-time complexity. Comparing these alternatives under diverse scenarios would quantify allocation quality trade-offs.

\subsection{Simplified Collision Avoidance}

The collision avoidance strategy maintains fifteen-meter spatial separation with temporal deconfliction when conflicts arise. This proves sufficient for sparse operational densities but exhibits limitations in dense flight patterns. The fixed safety buffer does not adapt to relative velocities, and the pairwise verification approach does not efficiently handle complex multi-vehicle conflicts.

Velocity obstacle approaches, particularly Reciprocal Velocity Obstacles, enable reactive collision avoidance accounting for relative velocities with smooth trajectory modifications. Model predictive control formulations could jointly optimize task execution and collision avoidance. These advanced methods would extend applicability to urban air mobility scenarios with higher flight densities.

\section{Application Scope}

The system addresses three mission classes: long-duration surveillance, emergency search and rescue, and medical supply delivery. These applications share waypoint-based navigation, quantifiable task priorities, and tolerance for mission degradation under resource constraints.

However, this focused scope excludes mission types with different requirements. Aggressive formation flying demands tighter coordination and higher-bandwidth communication than the current 2 Hz telemetry supports. Adversarial scenarios require game-theoretic reasoning and adversarial prediction beyond current OODA capabilities. Time-critical interception missions may need more sophisticated trajectory optimization than waypoint following provides.

Extending the system to these domains represents important future work. Formation flying could be addressed through augmented OODA loops reasoning about relative positioning constraints. Adversarial scenarios might integrate game-theoretic task allocation anticipating opponent responses. These extensions would broaden applicability while preserving core OODA principles.

\section{Future Research Directions}

\subsection{Near-Term Enhancements}

Comprehensive validation across all twenty-seven planned test scenarios would provide robust statistical characterization. Comparative evaluation against baseline strategies would quantify the value of explicit capacity modeling and priority-based allocation. Sensitivity analysis of safety reserve parameters from 5 to 20 percent battery would establish performance trade-offs between mission completion probability and safety margins. Enhanced environmental modeling incorporating turbulence, precipitation, and visibility limitations would improve prediction fidelity.

\subsection{Medium-Term Objectives}

Hardware-in-the-loop testing using PX4 Software-In-The-Loop would validate OODA cycle timing under realistic computational constraints. Field demonstrations with small fleets would reveal operational challenges including GPS accuracy limitations and radio frequency interference effects. Detailed instrumentation would generate empirical data to refine system parameters and validate simulation accuracy.

Distributed OODA architectures with consensus-based decision-making would enhance robustness against single-point failures. Implementing auction mechanisms like the Consensus-Based Bundle Algorithm would require careful attention to Byzantine fault tolerance and network partition handling. Comparative studies between centralized and distributed approaches would elucidate trade-offs between optimization quality, communication overhead, and system resilience.

Formal verification using model checking techniques could verify that OODA cycle logic maintains battery reserve constraints and collision avoidance guarantees under all reachable states. Temporal logic specifications could capture liveness properties ensuring eventual response to failures. While requiring substantial expertise, formal methods provide mathematical assurance complementing empirical testing.

\subsection{Long-Term Frontiers}

Machine learning could optimize priority weighting parameters based on historical mission outcomes, adapting scoring functions to operational contexts. Neural networks could predict battery consumption more accurately by learning vehicle-specific efficiency characteristics. Reinforcement learning might enable adaptive OODA cycle tuning based on observed performance patterns.

Game-theoretic analysis becomes essential for adversarial scenarios. Stackelberg game formulations could model hierarchical decision-making in contested surveillance missions. Nash equilibrium concepts might characterize stable operating points with competing autonomous systems. These theoretical frameworks would require substantial OODA extension incorporating opponent modeling and robust optimization.

Fleet heterogeneity introduces additional complexity. Real-world deployments increasingly employ mixed fleets with different endurance, payload capacity, and sensor suites. Addressing heterogeneity requires extending constraint verification for vehicle-specific capabilities and developing allocation strategies exploiting complementary strengths.

Large-scale swarm coordination with fifty or more vehicles demands hierarchical control architectures, efficient communication protocols avoiding broadcast storm effects, and possibly bio-inspired coordination strategies emerging from local interactions. Research at this scale intersects with complex systems theory, distributed computing, and collective intelligence.

\section{Concluding Perspective}

The limitations discussed reflect conscious design choices prioritizing practical deployability over theoretical completeness. The centralized architecture enables regulatory compliance and deterministic performance. The greedy allocation strategy trades global optimality for real-time responsiveness. The constraint-aware approach acknowledges physical limitations rather than assuming unlimited resources.

This honest assessment distinguishes the present work from research claiming comprehensive autonomy while abstracting real-world constraints. A system achieving 65 to 95 percent autonomous coverage recovery within realistic bounds provides substantially more value than theoretical frameworks promising perfect adaptation under idealized assumptions. The identified future work charts a path toward enhanced capabilities while maintaining the fundamental principle of honest, deployable autonomy.

As multi-agent UAV coordination matures, the research community must prioritize systems bridging the gap between laboratory demonstration and operational deployment. This requires explicit modeling of real-world constraints, acknowledgment of fundamental limitations, and design of hybrid human-machine systems leveraging complementary strengths of autonomous algorithms and human supervisory control. The present work contributes to this maturation by demonstrating that constraint-aware, operator-supervised fault tolerance represents the appropriate architecture for near-term UAV fleet deployments in regulated, safety-critical applications.

\chapter{Conclusion}

This work addresses the reality gap in multi-agent UAV fault tolerance by explicitly modeling real-world constraints (battery, payload, regulatory) that are often ignored in academic research. Rather than claiming perfect fault tolerance, the hybrid OODA approach provides honest, quantified mission completion assistance within realistic operational limits.

\textbf{Validated performance exceeds initial expectations.} Measured adaptation times of 0.11--0.50 milliseconds surpass the conservative 4--6 second design target by four orders of magnitude. Experimental validation across 169 automated tests confirms complete coverage recovery for surveillance and search-and-rescue missions, while delivery scenarios correctly escalate to operators when payload or boundary limitations preclude autonomous reallocation. Zero safety violations occurred across all experimental scenarios.

\textbf{Key insight:} Selective autonomy---aggressive adaptation when feasible, intelligent refusal when infeasible---proves more valuable than unconstrained systems that risk safety violations. In time-critical search-and-rescue operations, the 7.7-minute response advantage over manual operators can determine mission success and preserve lives during the golden hour.

The three core technical contributions---resource-aware reallocation, priority-based partial coverage, and operator escalation---work together to balance autonomous response speed with human oversight, making this approach suitable for deployment under current regulations. The comprehensive test suite (169 tests executing in 0.22 seconds) provides confidence in system reliability while enabling rapid development iteration.

\textbf{Acknowledged limitations.} These results must be interpreted within the system's design constraints. The centralized GCS architecture creates a single-point vulnerability---though individual UAVs implement autonomous Return-to-Launch upon communication timeout, fleet-level adaptation ceases during GCS failure. The greedy allocation strategy trades global optimality for real-time response, potentially yielding suboptimal task assignments when spare capacity is marginal. Most critically, validation remains exclusively software-in-the-loop without field deployment exposure to GPS multipath errors, RF interference, environmental turbulence, or sensor noise that characterize real-world operations. Hardware-in-the-loop testing and field trials represent essential next steps before operational deployment.

Despite these constraints, the work demonstrates that constraint-aware, operator-supervised fault tolerance represents a practical architecture for near-term UAV fleet deployments in regulated, safety-critical applications. The path forward lies not in claiming unrestricted autonomy, but in designing hybrid systems that explicitly acknowledge physical limits while maximizing autonomous capability within those bounds.

% ====================================
% REFERENCES
% ====================================
\newpage
\addcontentsline{toc}{chapter}{References}
\begin{thebibliography}{99}

\bibitem{mueller2014} 
Mueller, M. W., \& D'Andrea, R. (2014). Stability and control of a quadrocopter despite the complete loss of one, two, or three propellers. \textit{IEEE ICRA}.


\bibitem{sun2022}
Sun, Z., et al. (2022). Fault-Tolerant Model Predictive Control of a Quadrotor with an Unknown Complete Rotor Failure. \textit{IEEE ICRA}.

\bibitem{li2017}
Li, P., Yu, X., Peng, X., Zheng, Z., \& Zhang, Y. (2017). Fault-tolerant cooperative control for multiple UAVs based on sliding mode techniques. \textit{Science China Information Sciences}, 60(7).


\bibitem{yang2011}
Yang, H., Staroswiecki, M., Jiang, B., et al. (2011). Fault tolerant cooperative control for a class of nonlinear multi-agent systems. \textit{Systems \& Control Letters}, 60(4), 271-277.

\bibitem{gerkey2004}
Gerkey, B. P., \& Mataric, M. J. (2004). A formal analysis and taxonomy of task allocation in multi-robot systems. \textit{International Journal of Robotics Research}, 23(9), 939-954.

\bibitem{choi2009}
Choi, H. L., Brunet, L., \& How, J. P. (2009). Consensus-based decentralized auctions for robust task allocation. \textit{IEEE Transactions on Robotics}, 25(4), 912-926.

\bibitem{dias2006}
Dias, M. B., Zlot, R., Kalra, N., \& Stentz, A. (2006). Market-based multirobot coordination: A survey and analysis. \textit{Proceedings of the IEEE}, 94(7), 1257-1270.

\bibitem{zlot2006}
Zlot, R., \& Stentz, A. (2006). Market-based multirobot coordination for complex tasks. \textit{International Journal of Robotics Research}, 25(1), 73-101.

\bibitem{cortes2004}
Cortes, J., Martinez, S., Karatas, T., \& Bullo, F. (2004). Coverage control for mobile sensing networks. \textit{IEEE Transactions on Robotics and Automation}, 20(2), 243-255.

\bibitem{schwager2009}
Schwager, M., Rus, D., \& Slotine, J. J. (2009). Decentralized, adaptive coverage control for networked robots. \textit{International Journal of Robotics Research}, 28(3), 357-375.

\bibitem{elmaliach2009}
Elmaliach, Y., Agmon, N., \& Kaminka, G. A. (2009). Multi-robot area patrol under frequency constraints. \textit{Annals of Mathematics and Artificial Intelligence}, 57(3-4), 293-320.

\bibitem{abdessameud2011}
Abdessameud, A., \& Tayebi, A. (2011). Formation control of VTOL unmanned aerial vehicles with communication delays. \textit{Automatica}, 47(11), 2383-2394.

\bibitem{izadi2009}
Izadi, H. A., Gordon, B. W., \& Zhang, Y. M. (2009). Decentralized receding horizon control for cooperative multiple vehicles subject to communication delay. \textit{Journal of Guidance, Control, and Dynamics}, 32(6), 1959-1965.

\bibitem{izadi2013}
Izadi, H. A., Gordon, B. W., \& Zhang, Y. M. (2013). Hierarchical decentralized receding horizon control of multiple vehicles with communication failures. \textit{IEEE Transactions on Aerospace and Electronic Systems}, 49(2), 744-759.

\bibitem{beard2006}
Beard, R. W., McLain, T. W., Nelson, D. B., et al. (2006). Decentralized cooperative aerial surveillance using fixed-wing miniature UAVs. \textit{Proceedings of the IEEE}, 94(7), 1306-1324.

\bibitem{boyd1987}
Boyd, J. R. (1987). \textit{A Discourse on Winning and Losing}. [OODA Loop framework]

\bibitem{bala2025}
Bala, M., et al. (2025). The OODA Loop of Cloudlet-Based Autonomous Drones. \textit{IEEE/ACM Symposium on Edge Computing (SEC)}.

\bibitem{soares2025}
Soares, V. M. D., et al. (2025). UAV Simulation Environment for Fault Detection in Wind Farm Electrical Distribution Systems. \textit{IEEE Conference Proceedings}.

\bibitem{vandenberg2008}
van den Berg, J., Lin, M., \& Manocha, D. (2008). Reciprocal velocity obstacles for real-time multi-agent navigation. \textit{IEEE ICRA}, 1928-1935.

\bibitem{zhang2008}
Zhang, Y. M., \& Jiang, J. (2008). Bibliographical review on reconfigurable fault-tolerant control systems. \textit{Annual Reviews in Control}, 32(2), 229-252.

\bibitem{yu2015}
Yu, X., \& Jiang, J. (2015). A survey of fault-tolerant controllers based on safety-related issues. \textit{Annual Reviews in Control}, 39, 46-57.

\bibitem{parker1998}
Parker, L. E. (1998). ALLIANCE: An architecture for fault tolerant multirobot cooperation. \textit{IEEE Transactions on Robotics and Automation}, 14(2), 220-240.

\bibitem{repo}
Eulálio Reis, V. (2025). \textit{Multi-UAV OODA System: Constraint-Aware Fault-Tolerant Multi-Agent Coordination}. GitHub repository. \url{https://github.com/vriez/multi_uav_ooda_system}

\bibitem{quadref}
QUAD SIMCON. (2020). \textit{Quadcopter Simulation and Control}. GitHub repository. \url{https://github.com/bobzwik/Quadcopter_SimCon}

% \bibitem{repo}
% Project Repository: \url{https://github.com/vriez/multi_uav_ooda_system}
% \bibitem{guo2018}
% Guo, J., Zhang, Y., \& Li, W. (2018). Fault-tolerant control of quadrotor UAVs with actuator faults using adaptive backstepping. \textit{International Journal of Control, Automation and Systems}, 16(4), 1572-1583.

% \bibitem{wang2021}
% Wang, W., Zhang, Y., \& Xu, B. (2021). Fault and Failure Tolerant Model Predictive Control of Quadrotor UAV. \textit{IEEE ROBIO}.

% \bibitem{zhou2021}
% Zhou, B., Su, W., \& Han, J. (2021). Model predictive fault-tolerant control for quadrotor UAV subject to actuator faults. \textit{Aerospace Science and Technology}, 110, e106497.

% \bibitem{chang2024}
% Chang, Y., et al. (2024). Reinforcement Learning–Based Adaptive Fault-Tolerant Antidisturbance Control for UAVs. \textit{Journal of Aerospace Engineering}, 38(1).

% \bibitem{zhang2016}
% Zhang, X. Y., \& Duan, H. B. (2016). Altitude consensus based 3D flocking control for fixed-wing unmanned aerial vehicle swarm trajectory tracking. \textit{Journal of Aerospace Engineering}, 230(14), 2628-2638.

% \bibitem{liu2016}
% Liu, Z. X., Yuan, C., Yu, X., et al. (2016). Leader-follower formation control of unmanned aerial vehicles in the presence of obstacles and actuator faults. \textit{Unmanned Systems}, 4(3), 197-211.

% \bibitem{yu2016}
% Yu, X., Liu, Z. X., \& Zhang, Y. M. (2016). Fault-tolerant formation control of multiple UAVs in the presence of actuator faults. \textit{International Journal of Robust and Nonlinear Control}, 26(12), 2668-2685.

% \bibitem{korsah2013}
% Korsah, G. A., Stentz, A., \& Dias, M. B. (2013). A comprehensive taxonomy for multi-robot task allocation. \textit{International Journal of Robotics Research}, 32(12), 1495-1512.

% \bibitem{innocenti2004}
% Innocenti, M., Pollini, L., \& Giulietti, F. (2004). Management of communication failures in formation flight. \textit{Journal of Aerospace Computing, Information, and Communication}, 1(1), 19-35.

% \bibitem{franco2007}
% Franco, E., Parisini, T., \& Polycarpou, M. M. (2007). Design and stability analysis of cooperative receding-horizon control of linear discrete-time agents. \textit{International Journal of Robust and Nonlinear Control}, 17(10-11), 982-1001.

% Flight Formation
% \bibitem{pachter2001}
% Pachter, M., D'Azzo, J. J., \& Proud, A. W. (2001). Tight formation flight control. \textit{Journal of Guidance, Control, and Dynamics}, 24(2), 246-254.

% \bibitem{gu2006}
% Gu, Y., Seanor, B., Campa, G., et al. (2006). Design and flight testing evaluation of formation control laws. \textit{IEEE Transactions on Control Systems Technology}, 14(6), 1105-1112.

% \bibitem{lin2014}
% Lin, W. (2014). Distributed UAV formation control using differential game approach. \textit{Aerospace Science and Technology}, 35, 54-62.


\end{thebibliography}

\end{document}