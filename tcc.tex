\documentclass[12pt,a4paper,oneside]{book}

% ABNT formatting packages
\usepackage[utf8]{inputenc}
\usepackage[T1]{fontenc}
\usepackage[english]{babel}
\usepackage{indentfirst}
\usepackage{setspace}
\usepackage{graphicx}
\usepackage{float}
\usepackage[left=3cm,right=2cm,top=3cm,bottom=2cm]{geometry}
\usepackage{times}
\usepackage{caption}
\usepackage{subcaption}
\usepackage{amsmath}
\usepackage{amssymb}
\usepackage{listings}
\usepackage{xcolor}
\usepackage{url}
\usepackage{hyperref}
\usepackage{titlesec}
\usepackage{tocloft}
\usepackage{longtable}
\usepackage{array}
\usepackage{fancyhdr}

% ABNT-style formatting
\onehalfspacing
\setlength{\parindent}{1.25cm}

% Chapter and section formatting (ABNT style)
\titleformat{\chapter}[display]
  {\normalfont\bfseries\fontsize{12pt}{14pt}\selectfont}
  {\MakeUppercase{\chaptertitlename\ \thechapter}}{0pt}
  {\MakeUppercase}
\titlespacing*{\chapter}{0pt}{0pt}{12pt}

% Unnumbered chapter formatting (for Acknowledgments, Abstract, etc.)
\titleformat{name=\chapter,numberless}[display]
  {\normalfont\bfseries\fontsize{12pt}{14pt}\selectfont}
  {}{0pt}
  {\MakeUppercase}
\titlespacing*{name=\chapter,numberless}{0pt}{0pt}{12pt}

\titleformat{\section}
  {\normalfont\bfseries\fontsize{12pt}{14pt}\selectfont}
  {\thesection}{1em}{}
\titlespacing*{\section}{0pt}{12pt}{6pt}

\titleformat{\subsection}
  {\normalfont\bfseries\fontsize{12pt}{14pt}\selectfont}
  {\thesubsection}{1em}{}
\titlespacing*{\subsection}{0pt}{12pt}{6pt}

\titleformat{\subsubsection}
  {\normalfont\bfseries\fontsize{12pt}{14pt}\selectfont}
  {\thesubsubsection}{1em}{}
\titlespacing*{\subsubsection}{0pt}{12pt}{6pt}

% Caption formatting (ABNT style)
\captionsetup{
  format=plain,
  labelsep=endash,
  font=small,
  labelfont=bf,
  justification=justified,
  singlelinecheck=false
}

% Code listing style
\lstset{
  language=Python,
  basicstyle=\ttfamily\footnotesize,
  keywordstyle=\color{blue},
  commentstyle=\color{green!60!black},
  stringstyle=\color{red},
  numbers=left,
  numberstyle=\tiny\color{gray},
  stepnumber=1,
  numbersep=10pt,
  backgroundcolor=\color{white},
  showspaces=false,
  showstringspaces=false,
  showtabs=false,
  frame=single,
  tabsize=2,
  captionpos=b,
  breaklines=true,
  breakatwhitespace=false,
  escapeinside={(*@}{@*)},
  xleftmargin=2em,
  framexleftmargin=1.5em
}

% Hyperref configuration
\hypersetup{
    colorlinks=true,
    linkcolor=black,
    citecolor=black,
    filecolor=black,
    urlcolor=black,
    pdftitle={Constraint-Aware Fault-Tolerant Multi-Agent UAV System Using OODA Loop},
    pdfauthor={Vítor Eulálio Reis},
}

% Page numbering configuration (top right throughout document)
\pagestyle{fancy}
\fancyhf{} % Clear all header and footer fields
\fancyhead[R]{\thepage} % Page number in top right
\renewcommand{\headrulewidth}{0pt} % Remove header rule line
\renewcommand{\footrulewidth}{0pt} % Remove footer rule line

% Apply fancy style to plain pages (chapter opening pages)
\fancypagestyle{plain}{%
    \fancyhf{} % Clear all header and footer fields
    \fancyhead[R]{\thepage} % Page number in top right
    \renewcommand{\headrulewidth}{0pt} % Remove header rule line
    \renewcommand{\footrulewidth}{0pt} % Remove footer rule line
}

\begin{document}

% ====================================
% COVER PAGE
% ====================================
\begin{titlepage}
\begin{center}
\textbf{\uppercase{Universidade de São Paulo}}\\
\textbf{\uppercase{Escola de Engenharia de São Carlos}}

\vspace{8cm}

\textbf{\uppercase{Vítor Eulálio Reis}}

\vspace{4cm}

\textbf{Constraint-Aware Fault-Tolerant Multi-Agent UAV System Using OODA Loop: Realistic Mission Completion Assistance}

\vfill

São Carlos\\
2025
\end{center}
\end{titlepage}

% ====================================
% TITLE PAGE
% ====================================
\newpage
\thispagestyle{empty}
\begin{center}
\textbf{Vítor Eulálio Reis}

\vspace{8cm}

\textbf{Constraint-Aware Fault-Tolerant Multi-Agent UAV System Using OODA Loop}

\vspace{3cm}

\begin{minipage}{8cm}
\begin{flushleft}
Monograph presented to the Specialization Course in Aeronautical Systems, School of Engineering of São Carlos, University of São Paulo, as part of the requirements for obtaining the title of Specialist.

Advisor: Prof. João Paulo Eguea, PhD
\end{flushleft}
\end{minipage}

\vspace{2cm}

FINAL VERSION

\vfill

São Carlos\\
2025
\end{center}

% ====================================
% COPYRIGHT PAGE
% ====================================
\newpage
\thispagestyle{empty}
\vspace*{10cm}
\begin{center}
I AUTHORIZE THE TOTAL OR PARTIAL REPRODUCTION OF THIS WORK,\\
BY ANY CONVENTIONAL OR ELECTRONIC MEANS, FOR STUDY\\
AND RESEARCH PURPOSES, PROVIDED THE SOURCE IS CITED.
\end{center}

\vfill

\begin{center}
\begin{minipage}{12cm}
\begin{flushleft}
Cataloging card prepared by the Library Prof. Dr. Sérgio Rodrigues Fontes at EESC/USP with data provided by the author(s).

\vspace{0.5cm}

\noindent Eulálio Reis, Vítor

\hspace{0.5cm} Constraint-Aware Fault-Tolerant Multi-Agent UAV System Using OODA Loop: Realistic Mission Completion Assistance. / Vítor Eulálio Reis; advisor Prof. João Paulo Eguea, PhD. São Carlos, 2025.

\vspace{0.5cm}

\hspace{0.5cm} Specialization (Specialization in Aeronautical Systems) -- School of Engineering of São Carlos, University of São Paulo, 2025.

\vspace{0.5cm}

\hspace{0.5cm} 1. Drone. 2. OODA Loop. 3. Fault Tolerance. 4. Multi-Agent Systems. I. Title.
\end{flushleft}
\end{minipage}
\end{center}

% ====================================
% APPROVAL PAGE
% ====================================
\newpage
\thispagestyle{empty}
\begin{center}
\textbf{\uppercase{Approval Sheet}}

\vspace{2cm}

\textbf{APPROVAL SHEET}\\
\textit{Approval sheet}

\vspace{2cm}

\begin{tabular}{|p{13cm}|}
\hline
\textbf{Candidate / Student:} Vítor Eulálio Reis \\
\hline
\textbf{Title of TCC / Title:} Constraint-Aware Fault-Tolerant Multi-Agent UAV System Using OODA Loop \\
\hline
\textbf{Defense date / Date:} October 18, 2025 \\
\hline
\end{tabular}

\vspace{2cm}

\begin{tabular}{|p{10cm}|p{3cm}|}
\hline
\textbf{Examining Committee} & \textbf{Result} \\
\hline
{Prof° João Paulo Eguea, PhD} & submitted \\
\hline
\textbf{Affiliation:} School of Engineering of São Carlos / EESC-USP & \\
\hline
{Prof° Jorge Bidinotto, PhD} & submitted \\
\hline
\textbf{Affiliation:} School of Engineering of São Carlos / EESC-USP & \\
\hline
\end{tabular}

\vspace{2cm}

Chair of the Examining Committee:

\vspace{1cm}

\begin{center}
\rule{6cm}{0.4pt}\\
{Prof° João Paulo Eguea, PhD}\\
(Signature)
\end{center}

\end{center}

% ====================================
% DEDICATION (Optional)
% ====================================
\newpage
\thispagestyle{empty}
\vspace*{15cm}
\begin{flushright}
\begin{minipage}{8cm}
\textit{To my family, for their unconditional support throughout this endeavor.}
\end{minipage}
\end{flushright}

% ====================================
% ACKNOWLEDGMENTS
% ====================================
\newpage
\chapter*{Acknowledgments}
\addcontentsline{toc}{chapter}{Acknowledgments}

To the professors of the Especialização em Sistemas Aeronáuticos at the Escola de Engenharia de São Carlos, USP, for sharing their knowledge, expertise, craft, and for their dedication in supporting students throughout the course.

To Prof. Dr. João Paulo Eguea for his mentorship in guiding this research.

To Prof. Dr. Jorge Bidinotto for his leadership in managing the Specialization Program.

% ====================================
% ABSTRACT
% ====================================
\newpage
\chapter*{Abstract}
\addcontentsline{toc}{chapter}{Abstract}

\noindent EULÁLIO REIS, V. \textbf{Constraint-Aware Fault-Tolerant Multi-Agent UAV System Using OODA Loop: Realistic Mission Completion Assistance}. 2025. Monograph (Specialization) – School of Engineering of São Carlos, University of São Paulo, São Carlos, 2025.

\vspace{0.5cm}

This work proposes a constraint-aware fault-tolerant control system for multi-agent UAV operations using the OODA (Observe-Orient-Decide-Act) loop model. Unlike idealized approaches that assume unlimited resources, this work explicitly models real-world constraints (battery limits, payload depletion, regulatory restrictions) and provides Mission Completion Assistance through intelligent task reallocation and priority-based partial coverage strategies. The three core technical contributions are: (1) constraint-aware task reallocation algorithm that explicitly models battery/payload limits with collision avoidance, (2) priority-based partial coverage strategy with quantified performance degradation metrics, and (3) operator-escalation framework that identifies non-compensable failures and provides actionable recommendations. Rather than claiming perfect fault tolerance, this system minimizes mission failure impact within realistic constraints—a more honest and deployable contribution than purely autonomous approaches. Target applications include long-duration surveillance where coverage gaps create security risks, search and rescue where time-critical tasks must be prioritized, and package delivery where undelivered items can be reassigned.

\vspace{0.5cm}

\noindent \textbf{Keywords:} Drone. Fault Tolerance. OODA Loop. Multi-Agent Systems. Mission Planning.

% ====================================
% LIST OF FIGURES
% ====================================
\newpage
\listoffigures
\addcontentsline{toc}{chapter}{List of Figures}

% ====================================
% LIST OF ABBREVIATIONS
% ====================================
\newpage
\chapter*{List of Abbreviations and Acronyms}
\addcontentsline{toc}{chapter}{List of Abbreviations and Acronyms}

\begin{tabular}{ll}
BVLOS & Beyond Visual Line of Sight \\
FSM & Finite State Machine \\
GCS & Ground Control Station \\
GPS & Global Positioning System \\
MILP & Mixed Integer Linear Programming \\
OODA & Observe-Orient-Decide-Act \\
RF & Radio Frequency \\
RTL & Return-to-Launch \\
RVO & Reciprocal Velocity Obstacles \\
SAR & Search and Rescue \\
TSP & Traveling Salesman Problem \\
UAV & Unmanned Aerial Vehicle \\
\end{tabular}

% ====================================
% TABLE OF CONTENTS
% ====================================
\newpage
\tableofcontents

% ====================================
% MAIN CONTENT
% ====================================
\newpage
\setcounter{page}{1}
\pagenumbering{arabic}

\chapter{Introduction}

\section{The Reality Gap in Multi-Agent UAV Research}

Over the last decade, Unmanned Aerial Vehicles (UAVs) have progressed from experimental platforms to essential tools in logistics, environmental monitoring, and emergency response. Multi-UAV coordination, in particular, has emerged as a key enabler of scalable autonomy, allowing fleets to execute missions more efficiently and robustly than any single vehicle could achieve alone.

Despite this progress, much of the research in fault-tolerant multi-agent systems still operates within idealized boundaries. Academic frameworks often presume conditions that are far removed from operational reality: unlimited energy reserves, unrestricted payload capacity, instantaneous communication, and fully autonomous decision authority. These simplifying assumptions make theoretical analysis tractable, but they mask the complexities that dominate real-world deployments.

In practice, UAV operations are governed by strict physical and regulatory limits. Batteries must retain safety reserves of 10–20\% to satisfy aviation standards. Payload resources, whether spray tanks or delivery packages, deplete irreversibly during flight and cannot be replenished mid-mission. Communication links introduce latencies of up to two seconds, and mission execution beyond visual line of sight requires constant operator supervision.

The result is a persistent ``reality gap'' between simulated autonomy and deployable autonomy. Systems that appear reliable in laboratory conditions may struggle in the field when confronted with degraded batteries, intermittent links, or competing mission deadlines. Closing this gap requires architectures that are not only intelligent but also resilient—capable of adapting to uncertainty and resource constraints without losing situational awareness or regulatory compliance.

\section{The OODA Loop: From Combat Theory to Autonomous Resilience}

The concept of the OODA loop—Observe, Orient, Decide, Act—originated from U.S. Air Force Colonel John Boyd's studies of aerial combat dynamics. Boyd proposed that victory depends not merely on speed or strength, but on the ability to process information and adapt more quickly than the opponent. The OODA loop thus represents a continuous cycle of perception, reasoning, and action, where each iteration refines understanding and enhances responsiveness to change.

In the context of autonomous systems, the OODA loop provides a natural metaphor for adaptive control. The ``Observe'' phase involves collecting data from sensors and system telemetry; ``Orient'' interprets this data to establish situational awareness; ``Decide'' selects a course of action under the prevailing constraints; and ``Act'' executes that decision, feeding its outcomes back into the next cycle. Crucially, the loop is not linear but recursive: every action reshapes the environment that the next observation must interpret.

Applied to UAV fleets, this framework extends beyond traditional feedback control by integrating situational reasoning and contextual adaptation. Each vehicle, and the system as a whole, can maintain awareness not only of environmental conditions but also of internal resource states, communication health, and mission progress. The OODA architecture thus provides a foundation for resilient autonomy, where the fleet continuously senses its collective state, interprets disruptions, and reorganizes tasks in real time.

Most existing UAV coordination systems implement a subset of these principles, emphasizing either rapid reaction or long-term planning, but rarely both. What remains underexplored is the integration of the OODA loop into a real-time, resource-constrained, operator-supervised control system—one that can detect faults, reorient mission objectives, and adapt dynamically while remaining within the operational bounds of safety and regulation.

\section{Bridging the Gap: OODA-Based Fault-Tolerant Mission Control}

Despite its theoretical appeal, the OODA framework has rarely been implemented as a real-time operational control system for UAV fleets. Most prior work either focuses on single-vehicle fault recovery (Mueller \& D'Andrea, 2014) or treats multi-agent coordination under ideal conditions with unlimited resources (Li et al., 2017). Consequently, few systems integrate real-world resource constraints, probabilistic fault detection, and operator-in-the-loop supervision within a unified architecture.

This research addresses that gap through the development of a hybrid OODA-based fault-tolerant mission control system designed for real-time UAV fleet management. The system operates as a hybrid finite-state machine (FSM) that continuously cycles between monitoring and adaptation modes, combining deterministic state transitions with probabilistic failure identification to maintain robust mission execution under realistic constraints.

\section{System Overview and Contributions}

\subsection{Hybrid OODA–FSM Architecture}

At its core, the system functions as a hybrid finite-state controller implementing continuous monitoring at 2 Hz and invoking the OODA cycle upon failure detection. Deterministic state logic ensures predictable mission flow, while probabilistic reasoning handles uncertain or delayed telemetry data.

\subsection{Real-Time Failure Detection}

Effective fault tolerance begins with timely awareness. The system implements a multi-modal failure detection layer that continuously monitors fleet health through complementary sensing channels, enabling rapid identification of anomalies before they cascade into mission-critical failures.

At the foundation lies a 2 Hz telemetry polling mechanism that aggregates position estimates, battery state-of-charge readings, task completion progress, and communication heartbeat timestamps from each UAV. This continuous stream feeds three distinct detection pathways. The first monitors communication health: any telemetry gap exceeding 1.5 seconds triggers an immediate timeout alert, capturing link failures or vehicle loss-of-control events. The second pathway processes explicit fault codes propagated from onboard diagnostics—motor controller warnings, sensor degradation flags, or navigation system errors reported directly by the vehicle's flight stack.

The third and most sophisticated pathway employs statistical anomaly detection to identify subtle degradation patterns that precede catastrophic failure. Battery discharge rates exceeding 5\% per 30-second window suggest thermal runaway or cell damage; position discontinuities greater than 100 meters between consecutive updates indicate GPS spoofing, sensor fusion failures, or uncontrolled flight; altitude deviations beyond mission-defined safety envelopes reveal environmental disturbances or control system instability. By fusing these heterogeneous indicators, the detection layer achieves sub-second fault identification while minimizing false positives that would otherwise burden operators with spurious alerts.

\subsection{OODA Execution for Fault Recovery}

Upon fault detection, the system transitions from passive monitoring into active recovery through a structured OODA cycle. This process—completing in 4 to 5.5 seconds—transforms raw failure events into executable recovery plans while preserving mission continuity.

The cycle begins with the \textbf{Observe} phase (1.0–1.5 seconds), during which the system aggregates fresh telemetry from all operational vehicles, formally identifies which UAVs have failed, quantifies the resulting mission coverage loss, and establishes authoritative failure timestamps for logging and post-mission analysis. This snapshot provides the situational foundation upon which subsequent reasoning depends.

The \textbf{Orient} phase (1.0–1.5 seconds) translates observation into understanding. Here, the system evaluates the remaining fleet's collective capacity—factoring in current battery reserves, payload availability, and temporal constraints imposed by mission deadlines. Tasks orphaned by the failure undergo re-prioritization based on urgency weights and spatial proximity to healthy vehicles, establishing a ranked queue of recovery candidates.

Decision-making crystallizes in the \textbf{Decide} phase (1.0–1.5 seconds), where the system classifies recovery feasibility and selects an appropriate strategy. When 75\% or more of lost tasks can be absorbed by the remaining fleet, full autonomous reallocation proceeds. Partial reallocation addresses scenarios where 50–75\% recovery is achievable, focusing resources on the highest-priority tasks while flagging coverage gaps. Below the 50\% threshold, the system acknowledges its limitations and escalates to the human operator, presenting decision options rather than forcing suboptimal autonomous choices.

Finally, the \textbf{Act} phase (0.5–1.5 seconds) executes the selected strategy: dispatching updated waypoint sequences to reassigned vehicles, awaiting acknowledgment confirmations, and refreshing the operator dashboard with current mission state. The brevity of this phase reflects the computational efficiency of the preceding stages—by the time execution begins, all planning complexity has been resolved.

\subsection{Constraint-Aware Strategy Layer}

The three recovery strategies represent fundamentally different operational philosophies, each tailored to the severity and characteristics of the failure scenario.

\textbf{Full Reallocation} engages when the fleet retains sufficient capacity to absorb all orphaned tasks. The system executes a constraint-aware optimization routine (detailed in Algorithm 2) that integrates new waypoints into existing flight plans while minimizing total travel distance through heuristic TSP solutions. This strategy maximizes mission completion without operator intervention, treating the failure as a temporary disruption rather than a mission-altering event.

\textbf{Partial Reallocation} acknowledges resource limitations honestly. When complete recovery proves impossible, the system prioritizes tasks exceeding a 0.7 priority threshold—ensuring that mission-critical objectives receive available resources while lower-priority tasks are deferred or abandoned. Crucially, the system issues explicit coverage-gap alerts quantifying the impact: operators receive not merely notification that tasks were dropped, but precise metrics on what percentage of the mission objective remains achievable.

\textbf{Operator Escalation} activates when autonomous decision-making would produce unacceptable outcomes. Rather than forcing a poor solution, the system initiates a supervised decision protocol, presenting the operator with contextualized options: deploying reserve UAVs, extending mission timelines, accepting degraded coverage, or executing a controlled mission abort. A 30-second countdown ensures timely human response while providing an automatic failsafe—if no operator input arrives, the system defaults to the safest available action, preserving vehicle integrity over mission completion.

\subsection{Performance and Scalability}

The complete OODA cycle achieves reaction times between 4 and 5.5 seconds, representing a 75–150$\times$ speed improvement compared to manual operator intervention (typically 5–10 minutes). Computationally, the system scales linearly with fleet size for monitoring and resource evaluation, and quadratically for collision avoidance checks—supporting real-time performance for up to 12 UAVs without exceeding sub-6-second response thresholds.

\section{Summary}

By grounding UAV mission control in the OODA decision cycle, this research introduces a system capable of adaptive, explainable, and regulatorily compliant autonomy. The framework unites rapid machine-driven response with human-supervised oversight, offering a practical balance between safety and autonomy.

In doing so, it bridges the long-standing gap between theoretical multi-agent fault tolerance and real-world deployability—demonstrating that reactive, resource-aware autonomy is not merely a conceptual goal, but an achievable engineering reality for the next generation of UAV fleet operations.

\chapter{System Architecture}

\section{Centralized OODA Architecture with Distributed Execution}

The system implements a hierarchical control structure in which the Ground Control Station hosts the OODA Loop Engine for centralized decision-making while UAVs execute tasks with operational autonomy. This architecture balances the optimization advantages of global fleet visibility against the responsiveness requirements of distributed execution, a design choice grounded in both regulatory constraints and practical deployment considerations.

\begin{figure}[h!tbp]
    \centering
    \includegraphics[width=\textwidth, height=0.9\textheight, keepaspectratio]{images/architecture.png}
    \caption{Architecture Overview}
    \label{fig:architecture}
\end{figure}

\subsection{Design Rationale and Trade-offs}

The centralized OODA architecture derives from multiple convergent factors. Regulatory compliance for Beyond Visual Line of Sight operations mandates operator oversight, which centralized control naturally provides through clear authority hierarchies. Optimization quality benefits from global fleet state visibility, enabling superior task allocation compared to distributed consensus approaches. Implementation complexity reduces significantly with a single decision point, eliminating complex inter-UAV coordination protocols. This approach aligns with current commercial and military deployment practices.

The hybrid autonomy model acknowledges fundamental operational constraints. The system explicitly recognizes scenarios where complete failure compensation proves physically impossible, maintains safety through human-in-the-loop oversight for critical decisions, and enhances deployability by avoiding full BVLOS authority requirements. This design philosophy prioritizes honest performance assessment over aspirational capabilities.

Architectural trade-offs require careful mitigation. The centralized architecture introduces a single point of failure at the GCS, mitigated through autonomous Return-to-Launch behaviors triggered by communication timeout detection. Communication bandwidth scales linearly with fleet size, though the 2 Hz telemetry rate maintains manageable overhead for fleets not exceeding twelve UAVs. The 3-6 second response latency, while not instantaneous, remains acceptable for mission-level fault tolerance in applications where individual UAV aerodynamic stability persists throughout this interval.

\subsection{Communication System Design}

The bidirectional communication architecture operates at 2 Hz frequency with asymmetric packet sizes reflecting data flow characteristics. Communication utilizes \textbf{TCP/IP transport with JSON-RPC 2.0 protocol} for structured request-response messaging between UAVs and the GCS server (port 5555).

\textbf{Protocol structure:}

\begin{table}[H]
\centering
\small
\begin{tabular}{|l|p{9cm}|}
\hline
\textbf{Layer/Component} & \textbf{Specification} \\
\hline
\textbf{Transport} & TCP/IP for reliable delivery with connection-oriented communication \\
\hline
\textbf{Application Protocol} & JSON-RPC 2.0 for standardized remote procedure calls \\
\hline
\textbf{Message Format} & JSON-encoded request/response pairs with method invocation semantics \\
\hline
\textbf{Telemetry Methods} & UAVs invoke GCS methods to report status (position, battery, task progress) \\
\hline
\textbf{Command Methods} & GCS invokes UAV methods to dispatch waypoint updates and task reassignments \\
\hline
\end{tabular}
\caption{Communication Protocol Stack}
\label{tab:protocol_structure}
\end{table}

Telemetry uplink from UAVs to GCS transmits approximately 2 kilobyte JSON-RPC packets containing position coordinates, battery state of charge, operational status flags, and task completion progress. Command downlink from GCS to UAVs utilizes 1 kilobyte JSON-RPC packets conveying waypoint updates, task assignments, and collision avoidance parameters. One-way latency spans 0.5 to 1.0 seconds, typical of long-range RF links, yielding total system latency from failure detection to command execution of 3 to 6 seconds.

The 2 Hz telemetry rate selection reflects practical deployment constraints. Commercial UAV systems typically employ 1-2 Hz update rates, while research platforms with dedicated radio links may achieve 5-10 Hz though this remains uncommon. Higher rates increase bandwidth requirements and packet collision probability. Given that OODA loop execution requires 2-5 seconds, 2 Hz sampling provides adequate temporal resolution. This conservative choice ensures field deployability while maintaining responsive fault detection, contrasting with prior simulation work that often assumes instantaneous communication.

\section{OODA Loop Execution Flow}

The OODA loop implements continuous monitoring and reactive decision-making through four sequential phases. \textbf{Measured performance demonstrates sub-millisecond execution times (0.11-0.50 ms)} from failure detection to command dispatch, significantly exceeding initial design targets of 4-6 seconds. This represents approximately \textbf{500,000$\times$ faster response} than manual operator intervention (465 seconds average) and \textbf{10,000$\times$ faster than initial design expectations}.

\subsection{Computation Time vs. End-to-End System Latency}

An important distinction exists between \textbf{OODA algorithm computation time} and \textbf{total system response latency}, explaining the substantial performance improvement over initial estimates:

\textbf{OODA computation time (measured):} The core OODA loop algorithm---encompassing failure detection logic, capacity analysis, constraint validation, and reallocation optimization---executes in 0.11 to 0.50 milliseconds. This represents pure computational processing measured in software-in-the-loop simulation without network delays. The sub-millisecond execution validates that algorithmic complexity remains tractable for real-time deployment.

\textbf{End-to-end system latency (design target):} The original 4-6 second estimate accounts for complete system response including bidirectional RF communication delays (0.5-1.0 seconds uplink for failure notification, 0.5-1.0 seconds downlink for command dispatch), telemetry aggregation latency at 2 Hz sampling intervals (up to 0.5 seconds), and command acknowledgment verification (0.2 seconds). In field deployment with long-range radio links, total adaptation time spans:

\begin{equation}
T_{total} = T_{uplink} + T_{compute} + T_{downlink} + T_{ack} \approx 1.0 + 0.0005 + 1.0 + 0.2 = 2.2 \text{ seconds}
\end{equation}

This 2-3 second end-to-end response remains substantially faster than the 4-6 second initial conservative estimate, with the primary time consumption attributable to RF propagation delays rather than computational bottlenecks. The sub-millisecond computational performance ensures that algorithm execution never becomes the limiting factor, even for larger fleet sizes where complexity increases.

\textbf{Practical implication:} Real-world deployments will experience 2-3 second adaptation times dominated by communication latency, not the 0.3 ms algorithmic computation. This distinction matters for system design---network optimization provides greater latency reduction potential than further algorithmic speedup. Nonetheless, 2-3 seconds represents a 150-fold improvement over manual operator response (465 seconds), validating the OODA approach for time-critical applications such as search and rescue operations.

\begin{figure}[h!tbp]
    \centering
    \includegraphics[width=\textwidth, height=0.9\textheight, keepaspectratio]{images/ooda_loop.png}
    \caption{OODA Loop Execution Flow}
    \label{fig:ooda_loop}
\end{figure}

\subsection{OBSERVE Phase: Failure Detection and State Aggregation}

The OBSERVE phase (1.0-1.5s budget) establishes situational awareness through multi-modal failure detection: timeout detection for telemetry gaps >1.5s, explicit fault messages from UAV systems, and statistical anomaly detection (battery discharge >5\%/30s, position jumps >100m, altitude violations). \textit{Note: Computation executes in microseconds; time budget accommodates 2 Hz RF telemetry aggregation.} Upon failure confirmation, the system aggregates fleet state from all operational UAVs, identifies failed vehicle last-known state, enumerates lost tasks with priorities and deadlines, and calculates mission impact percentage.

\subsection{ORIENT Phase: Situation Assessment and Capacity Analysis}

The ORIENT phase (1.0-1.5s budget) transforms observations into actionable intelligence. Mission impact evaluation quantifies coverage loss, affected zones, and deadline pressure. Fleet capacity analysis inventories spare resources: battery capacity (subtracting committed energy and 15-20\% safety reserves, yielding ~3min flight per 10\% spare), payload capacity for delivery missions, and temporal margins relative to deadlines. Task prioritization applies Algorithm 1 to generate 0-1 priority scores from temporal urgency, mission criticality, and spatial cost. Feasibility classification: feasible (>75\% tasks reallocable, >90\% completion), partial (50-75\% reallocable, 75-90\% completion), or infeasible (<50\% reallocable, <75\% completion).

\subsection{DECIDE Phase: Strategic Planning and Algorithmic Optimization}

The DECIDE phase (1.0-1.5s budget) selects strategies via three-tier hierarchy. \textbf{Full Reallocation:} Executes Algorithm 2 for constraint-aware assignment to nearest UAVs with collision avoidance, optimizing path integration for 90-100\% completion. \textbf{Partial Reallocation:} Filters tasks with priority >0.7, allocates high-priority tasks first, generates coverage gap alerts with operator recommendations for 75-90\% completion. \textbf{Operator Escalation:} Generates critical alerts (urgency: high if coverage <50\% or critical tasks lost, medium if 50-75\%, low if >75\%), presents alternatives (backup UAV, mission abort, degraded coverage), implements 30s countdown timer with automatic safe-default execution.

\subsection{ACT Phase: Command Execution and System Update}

The ACT phase (0.5-1.5s budget) dispatches mission updates to affected UAVs (new waypoints, task assignments, collision parameters) via 2 Hz uplink with 200ms acknowledgment verification and 3-attempt retry logic. Dashboard updates display fleet status, mission progress, coverage heatmaps, and alerts. Performance logging captures failure details, phase durations, reallocation results, and escalation status. Following completion, the system returns to continuous 2 Hz monitoring, supporting cascading failure handling through iterative OODA cycles.

\section{Sequential Constraint Validation Process}

The constraint validation process implements fail-fast sequential checking that prioritizes constraints by criticality and commonality, enabling early termination when fundamental limitations preclude autonomous compensation while avoiding wasted computation on infeasible scenarios.

\begin{figure}[h!tbp]
    \centering
    \includegraphics[width=\textwidth, height=0.9\textheight, keepaspectratio]{images/constraints_checking.png}
    \caption{Constraint Checking Process}
    \label{fig:constraints}
\end{figure}


\subsection{Battery Constraint Verification}

Battery constraint evaluation serves as the primary validation gate due to its status as the most common limiting factor and safety-critical nature. For each lost task, the system calculates Euclidean distance from candidate UAVs, determines required battery percentage via vehicle-specific efficiency factors, and computes spare capacity as current charge minus committed energy and 15--20\% safety reserves. Pass: every lost task finds one candidate UAV within spare capacity. Fail: any task unreachable by all UAVs without violating safety margins, triggering immediate operator escalation. Prioritization derives from catastrophic failure risk and computational efficiency of distance-based evaluation.

\subsection{Payload Constraint Verification}

Payload constraint evaluation (conducted conditionally upon battery satisfaction) applies exclusively to cargo-carrying operations. The system calculates spare payload as maximum capacity minus current load, requiring task payload not exceed available spare capacity. Pass: all payload-requiring tasks find adequate capacity. Fail: removes payload-heavy tasks from consideration while reallocating feasible tasks, producing partial coverage. Payload constraints represent hard physical limits—mid-air transfers remain impossible, constraining reallocation to base station cargo swaps to avoid structural failure or flight instability.

\subsection{Time Constraint Verification}

Time constraint evaluation (positioned as final validation) verifies reallocated tasks achieve completion before deadline expiration via cumulative time calculation: transit time + task execution + current task remainder. Pass: yields feasible classification with 90--100\% completion plans. Fail: produces partial reallocation prioritizing by deadline urgency, accepting delays for low-priority tasks with 75--90\% completion projections. Final positioning reflects maximum flexibility—violations produce mission degradation rather than safety risks, unlike battery/payload constraints.

\subsection{Design Rationale}

Sequential checking provides: (1) early exit efficiency, saving ~60\% computation when battery constraints fail; (2) clear failure attribution for precise diagnostics and operator understanding; (3) prioritized ordering checking safety-critical battery bottlenecks first, physical payload limits second, flexible time constraints last.

\section{Communication Sequence and Information Flow}

The temporal communication sequence and information pipeline architecture demonstrate the system's end-to-end fault response capability, transforming raw sensor measurements into executable commands through systematic OODA processing.

\begin{figure}[h!tbp]
    \centering
    \includegraphics[width=\textwidth, height=0.9\textheight, keepaspectratio]{images/sequence_diagram.png}
    \caption{Mission Execution Sequence}
    \label{fig:sequence}
\end{figure}

\subsection{Temporal Execution Analysis}

The sequence diagram illustrates communication flow during failure events with precise timing. At reference time T, a UAV experiences failure, transmitting a high-priority fault message with 0.4-1.0 second latency. The OBSERVE phase confirms the fault and aggregates fleet state by T+1.5 seconds. The ORIENT phase completes impact assessment and capacity analysis by T+2.5 seconds. The DECIDE phase generates reallocation plans by T+3.5 seconds. The ACT phase dispatches commands and updates dashboards by T+4.0 seconds, establishing a 4-second response timeline from failure detection to command dispatch.

This temporal progression demonstrates responsiveness far exceeding human operator capabilities while accommodating realistic communication latencies inherent in long-range RF links. The explicit timing breakdown provides quantitative performance baselines for fault-tolerant mission adaptation.

\subsection{Information Pipeline Architecture}

\begin{figure}[H]
\centering
\includegraphics[width=0.9\textwidth]{images/processing_pipeline.png}
\caption{Data Flow Pipeline}
\label{fig:pipeline}
\end{figure}

The data flow architecture implements unidirectional information transformation with clear phase boundaries. The input layer ingests dual streams: high-frequency telemetry from UAV sensors updated at 2 Hz, and static mission definitions loaded at initialization. The processing layer transforms these inputs sequentially—OBSERVE produces fleet states and failure lists, ORIENT generates impact assessments and capacity analyses, DECIDE creates strategies and command packets, and ACT executes commands while logging performance metrics. The output layer distributes results through three channels: commands transmitted to UAVs, alerts displayed to operators, and logs persisted for post-mission analysis.

This unidirectional flow pattern provides clear data provenance enabling traceability, testability through phase isolation with mock input capability, and maintainability where phase modifications avoid backward ripple effects. The implicit feedback loop manifests as ACT phase commands update UAV operational states, with updated telemetry reflecting new states flowing back into OBSERVE phase in subsequent cycles.

\subsection{Scalability Considerations}

Computational complexity analysis reveals scaling characteristics across OODA phases. The OBSERVE phase scales linearly with fleet size through increased telemetry aggregation. The ORIENT phase maintains linear scaling for capacity calculations across vehicles. The DECIDE phase exhibits O(N$\times$M) complexity for task allocation across N UAVs and M tasks, with collision avoidance scaling quadratically O(N$^2$) for pairwise path verification. These complexity characteristics yield expected OODA execution times of 5-6 seconds for twelve-UAV fleets and 8-10 seconds for twenty-UAV fleets, potentially exceeding acceptable response thresholds for larger deployments.

\section{Scenario-Specific Adaptations}

The architecture demonstrates flexibility across diverse mission profiles through scenario-specific workflow patterns while maintaining consistent OODA loop mechanics. Three representative mission types illustrate architectural adaptation to varying operational requirements and constraint priorities.

\begin{figure}[h!tbp]
    \centering
    \includegraphics[width=\textwidth, height=0.9\textheight, keepaspectratio]{images/surveillance_loop.png}
    \caption{Surveillance Mission}
    \label{fig:surveillance}
\end{figure}

\textbf{Surveillance missions} emphasize continuous area coverage through patrol patterns with failures requiring coverage gap minimization and critical zone protection. Task characteristics include waypoint-based patrol routes, priority differentials between critical security zones and routine patrol areas, and deadline constraints for periodic revisit frequencies. Constraint considerations focus primarily on battery limitations affecting patrol duration and spatial coverage range, with time constraints enforcing revisit deadlines for critical areas. Target performance maintains 90-100 percent coverage through reallocation prioritizing critical zones.

\begin{figure}[H]
    \centering
    \includegraphics[width=\textwidth, height=0.9\textheight, keepaspectratio]{images/sar_loop.png}
    \caption{Search and Rescue Mission}
    \label{fig:sar}
\end{figure}


\textbf{Search and rescue missions} emphasize time-critical area coverage through systematic grid search patterns with failures requiring rapid reallocation to maintain search efficiency. Task characteristics include grid cell assignments with systematic coverage requirements and stringent deadline constraints reflecting survivor time sensitivity. Equal priority across search cells applies in unknown target location scenarios, though priority may concentrate in high-probability areas when available intelligence suggests target location. Battery and time constraints dominate feasibility assessment. Target performance accepts 75-90 percent coverage as acceptable degradation when full compensation proves infeasible, acknowledging that incomplete coverage in time-critical scenarios exceeds mission abortion alternatives.

\begin{figure}[h!tbp]
    \centering
    \includegraphics[width=\textwidth, height=0.9\textheight, keepaspectratio]{images/delivery_loop.png}
    \caption{Delivery Mission}
    \label{fig:delivery}
\end{figure}

\textbf{Delivery missions} emphasize reliable cargo transport to designated destinations with failures requiring payload-aware reallocation and potential return-to-base cargo swaps when reallocation proves infeasible. Task characteristics include point-to-point delivery routes with specific destination coordinates, payload weight requirements varying per package, and deadline constraints for time-sensitive deliveries. Payload constraints assume critical importance alongside battery limitations, with heavy cargo failures potentially requiring UAV return-to-launch for manual cargo transfer rather than mid-air reallocation. Priority schemes emphasize time-sensitive or high-value packages, accepting degradation in routine deliveries when capacity limitations preclude full compensation.

These scenario variations demonstrate architectural flexibility through consistent OODA mechanics with mission-specific constraint emphasis and priority schemes, validating the architecture's applicability across diverse operational domains while maintaining systematic fault response capabilities.

\chapter{Core Algorithms and Technical Contributions}

\section{Priority-Based Task Scoring Algorithm}

\textbf{ALGORITHM 1: Task Priority Calculation}

\begin{lstlisting}[language=Python, caption=Task Priority Calculation Algorithm]
def calculate_task_priority(task, fleet_state, mission_context):
    """
    Calculate priority score P_i in [0, 1] for task i
    
    Inputs:
        task: Task object with (position, type, deadline, criticality)
        fleet_state: Current state of all UAVs
        mission_context: Mission type and global parameters
    
    Output:
        P_i: Priority score (higher = more urgent/important)
    """
    
    # 1. TEMPORAL URGENCY COMPONENT (0-1, higher = more urgent)
    # Based on deadline proximity
    t_remaining = task.deadline - current_time
    t_total = task.deadline - task.start_time
    temporal_urgency = 1.0 - (t_remaining / t_total)
    temporal_urgency = max(0, min(1, temporal_urgency))
    
    # 2. MISSION CRITICALITY COMPONENT (mission-specific weights)
    criticality_weights = {
        # Surveillance mission
        'surveillance': {
            'security_critical_zone': 1.0,  # High-value areas
            'routine_patrol': 0.3,           # Standard coverage
            'redundant_coverage': 0.1        # Already covered recently
        },
        # Search & Rescue mission
        'sar': {
            'victim_likely_zone': 1.0,       # High probability areas
            'debris_search': 0.6,             # Secondary search
            'perimeter_patrol': 0.3           # Low priority
        },
        # Delivery mission
        'delivery': {
            'medical_emergency': 1.0,         # Life-critical
            'time_sensitive': 0.7,            # Commercial priority
            'standard_delivery': 0.4          # Regular packages
        }
    }
    
    mission_type = mission_context['type']
    task_criticality = criticality_weights[mission_type][task.type]
    
    # 3. SPATIAL REALLOCATION COST (battery required)
    # Find nearest available UAV
    min_distance = float('inf')
    for uav in fleet_state.healthy_uavs:
        distance = euclidean_distance(uav.position, task.position)
        if distance < min_distance:
            min_distance = distance
    
    # Normalize by max range
    max_range = mission_context['uav_max_range']
    spatial_cost = min_distance / max_range
    spatial_cost = max(0, min(1, spatial_cost))
    
    # 4. COMBINED PRIORITY SCORE
    # Weighted combination (weights sum to 1.0)
    w_temporal = 0.3
    w_criticality = 0.5
    w_spatial = 0.2
    
    P_i = (w_temporal * temporal_urgency + 
           w_criticality * task_criticality - 
           w_spatial * spatial_cost)
    
    # Normalize to [0, 1]
    P_i = max(0, min(1, P_i))
    
    return P_i

# Priority sorting for task allocation
def rank_tasks_by_priority(failed_tasks, fleet_state, mission_context):
    """
    Rank all failed UAV tasks by priority for reallocation
    """
    task_priority_list = []
    
    for task in failed_tasks:
        priority = calculate_task_priority(task, fleet_state, mission_context)
        task_priority_list.append((task, priority))
    
    # Sort in descending order of priority
    task_priority_list.sort(key=lambda x: x[1], reverse=True)
    
    return task_priority_list
\end{lstlisting}

\textbf{Justification:}
\begin{itemize}
\item \textbf{Temporal component:} Ensures deadline-driven tasks are prioritized
\item \textbf{Criticality component:} Mission-specific importance (medical > commercial)
\item \textbf{Spatial component:} Penalizes tasks far from available UAVs (battery cost)
\item \textbf{Weights:} Tunable based on mission requirements (can be adjusted)
\end{itemize}

\textbf{Related work:}
\begin{itemize}
\item Similar to auction-based task allocation (Choi et al., 2009)
\item Inspired by utility functions in market-based coordination (Dias et al., 2006)
\item Differs by explicitly modeling resource costs
\end{itemize}

\section{Constraint-Aware Task Reallocation with Collision Avoidance}

\textbf{ALGORITHM 2: Safe Task Reallocation}

\begin{lstlisting}[language=Python, caption=Constraint-Aware Task Reallocation Algorithm]
def reallocate_tasks_with_constraints(failed_uav, healthy_uavs, 
                                     mission_context):
    """
    Reallocate failed UAV tasks to healthy UAVs with:
    - Battery constraints
    - Payload constraints
    - Collision avoidance
    
    Returns:
        reallocation_plan: Dict mapping tasks to UAVs
        coverage_percentage: % of failed tasks successfully reallocated
        operator_alerts: List of non-compensable tasks
    """
    
    # 1. CALCULATE AVAILABLE CAPACITY
    available_capacity = {}
    for uav in healthy_uavs:
        # Battery constraint
        battery_reserve = uav.battery_remaining - SAFETY_RESERVE
        spare_battery = max(0, battery_reserve - uav.committed_battery)
        
        # Payload constraint (for delivery missions)
        spare_payload = uav.max_payload - uav.current_payload
        
        available_capacity[uav.id] = {
            'battery': spare_battery,
            'payload': spare_payload,
            'position': uav.current_position
        }
    
    # 2. RANK FAILED TASKS BY PRIORITY
    failed_tasks = failed_uav.task_queue
    ranked_tasks = rank_tasks_by_priority(failed_tasks, 
                                          healthy_uavs, 
                                          mission_context)
    
    # 3. GREEDY ALLOCATION WITH CONSTRAINTS
    reallocation_plan = {}
    unallocated_tasks = []
    
    for task, priority in ranked_tasks:
        allocated = False
        
        # Try to assign to nearest UAV with sufficient capacity
        candidate_uavs = sorted(
            healthy_uavs,
            key=lambda u: euclidean_distance(u.position, task.position)
        )
        
        for uav in candidate_uavs:
            # Check battery constraint
            distance_to_task = euclidean_distance(uav.position, 
                                                  task.position)
            battery_required = distance_to_task / uav.battery_efficiency
            
            if available_capacity[uav.id]['battery'] < battery_required:
                continue  # Insufficient battery
            
            # Check payload constraint (delivery missions only)
            if mission_context['type'] == 'delivery':
                if available_capacity[uav.id]['payload'] < task.payload:
                    continue  # Insufficient payload capacity
            
            # Check collision-free path
            if not is_collision_free_path(uav, task, reallocation_plan):
                continue  # Path conflict detected
            
            # ALLOCATION SUCCESSFUL
            reallocation_plan[task.id] = uav.id
            available_capacity[uav.id]['battery'] -= battery_required
            if mission_context['type'] == 'delivery':
                available_capacity[uav.id]['payload'] -= task.payload
            allocated = True
            break
        
        if not allocated:
            unallocated_tasks.append((task, priority))
    
    # 4. CALCULATE COVERAGE PERCENTAGE
    total_tasks = len(failed_tasks)
    reallocated_tasks = len(reallocation_plan)
    coverage_percentage = (reallocated_tasks / total_tasks) * 100
    
    # 5. GENERATE OPERATOR ALERTS
    operator_alerts = []
    for task, priority in unallocated_tasks:
        alert = {
            'task_id': task.id,
            'priority': priority,
            'reason': 'Insufficient fleet capacity',
            'recommendation': 'Manual intervention required' if priority > 0.7 
                            else 'Acceptable degradation'
        }
        operator_alerts.append(alert)
    
    return reallocation_plan, coverage_percentage, operator_alerts
\end{lstlisting}

\textbf{ALGORITHM 3: Collision Avoidance Check}

\begin{lstlisting}[language=Python, caption=Collision Avoidance Verification]
def is_collision_free_path(uav, new_task, current_plan):
    """
    Check if assigning new_task to uav creates collision risk
    with other UAVs in current_plan
    
    Uses spatial separation and temporal deconfliction
    """
    
    SAFETY_BUFFER = 15.0  # meters (3x typical UAV wingspan)
    TIME_BUFFER = 10.0    # seconds (safety margin)
    
    # Get UAV's new planned path
    new_path = generate_path(uav.position, new_task.position)
    new_timeline = estimate_arrival_times(new_path, uav.speed)
    
    # Check against all other UAVs' paths
    for other_uav_id, assigned_tasks in current_plan.items():
        if other_uav_id == uav.id:
            continue
        
        other_path = get_planned_path(other_uav_id, assigned_tasks)
        other_timeline = get_timeline(other_uav_id)
        
        # Spatial-temporal conflict detection
        for i, point1 in enumerate(new_path):
            t1 = new_timeline[i]
            
            for j, point2 in enumerate(other_path):
                t2 = other_timeline[j]
                
                # Check if UAVs are near same point at similar time
                spatial_distance = euclidean_distance(point1, point2)
                temporal_distance = abs(t1 - t2)
                
                if (spatial_distance < SAFETY_BUFFER and 
                    temporal_distance < TIME_BUFFER):
                    # COLLISION RISK DETECTED
                    return False
    
    # No conflicts found
    return True

def apply_temporal_separation(uav, conflicting_uav):
    """
    If spatial conflict exists, delay one UAV's task
    """
    # Delay the lower-priority UAV by TIME_BUFFER seconds
    if uav.task_priority < conflicting_uav.task_priority:
        uav.add_delay(TIME_BUFFER)
    else:
        conflicting_uav.add_delay(TIME_BUFFER)
\end{lstlisting}

\textbf{Collision avoidance approach:}
\begin{itemize}
\item \textbf{Spatial separation:} Maintain 15m minimum distance (3$\times$ wingspan)
\item \textbf{Temporal deconfliction:} If paths intersect, stagger arrival times
\item \textbf{Priority-based:} Lower-priority tasks delayed if conflict occurs
\end{itemize}

\textbf{Related work:}
\begin{itemize}
\item Based on velocity obstacles (van den Berg et al., 2008)
\item Simpler than distributed RVO but sufficient for centralized planning
\item Similar to air traffic control separation standards
\end{itemize}

\section{Operator Escalation Decision Rules}

\textbf{Decision tree for operator involvement:}

\begin{lstlisting}[language=Python, caption=Operator Escalation Decision Logic]
def determine_operator_escalation(coverage_percentage, unallocated_tasks):
    """
    Decide when to escalate to operator based on mission degradation
    """
    
    # ESCALATION TRIGGERS
    
    # 1. Critical coverage loss
    if coverage_percentage < 50:
        return {
            'escalate': True,
            'urgency': 'HIGH',
            'reason': 'Mission coverage below 50% threshold',
            'recommendation': 'Consider aborting mission or deploying backup UAV'
        }
    
    # 2. High-priority tasks unallocated
    critical_tasks_lost = [t for t in unallocated_tasks if t[1] > 0.7]
    if len(critical_tasks_lost) > 0:
        return {
            'escalate': True,
            'urgency': 'HIGH',
            'reason': f'{len(critical_tasks_lost)} critical tasks cannot be covered',
            'recommendation': 'Manual task prioritization required'
        }
    
    # 3. Moderate degradation
    if coverage_percentage < 75:
        return {
            'escalate': True,
            'urgency': 'MEDIUM',
            'reason': 'Moderate mission degradation (50-75% coverage)',
            'recommendation': 'Monitor closely, consider manual reallocation'
        }
    
    # 4. Minor degradation - autonomous handling
    return {
        'escalate': False,
        'urgency': 'LOW',
        'reason': 'Acceptable autonomous compensation (>75% coverage)',
        'recommendation': 'Continue autonomous operation'
    }
\end{lstlisting}

\section{Objective Function and Optimization Strategy}

The preceding algorithms define \textit{how} tasks are prioritized and allocated, but do not explicitly characterize \textit{what constitutes a good allocation}. This section formalizes the objective function that guides the DECIDE phase and describes the optimization strategy employed to maximize decision quality within real-time constraints.

\subsection{Allocation Quality Metric}

The quality of a task reallocation is quantified through an objective function $J(\mathcal{A})$ that the optimizer seeks to maximize. Given an allocation $\mathcal{A} = \{(t_i, u_j)\}$ mapping failed tasks $t_i$ to healthy UAVs $u_j$, the objective function is defined as:

\begin{equation}
J(\mathcal{A}) = \sum_{(t_i, u_j) \in \mathcal{A}} \left[ P_i \cdot \phi_m(t_i, u_j) \right] - \lambda \cdot |\mathcal{T}_{\text{unalloc}}|
\label{eq:objective}
\end{equation}

\noindent where:
\begin{itemize}
\item $P_i \in [0,1]$ is the priority score of task $t_i$ computed by Algorithm 1
\item $\phi_m(t_i, u_j) \in [0,1]$ is a mission-specific modifier function
\item $\lambda > 0$ is the penalty weight for unallocated tasks
\item $|\mathcal{T}_{\text{unalloc}}|$ is the count of tasks that could not be feasibly assigned
\end{itemize}

The mission-specific modifier $\phi_m$ adjusts task value based on operational context:

\begin{equation}
\phi_m(t_i, u_j) =
\begin{cases}
1 - \gamma \cdot \Delta t_{\text{gap}}(t_i) & \text{if } m = \text{surveillance} \\[6pt]
1 + \beta \cdot \dfrac{t_{\text{golden}} - t_{\text{completion}}(t_i, u_j)}{t_{\text{golden}}} & \text{if } m = \text{SAR} \\[6pt]
\begin{cases}
1.0 & \text{if } t_{\text{completion}} \leq t_{\text{deadline}} \\
0.5 & \text{otherwise}
\end{cases} & \text{if } m = \text{delivery}
\end{cases}
\label{eq:modifier}
\end{equation}

For surveillance missions, the modifier penalizes coverage gaps through parameter $\gamma$, where $\Delta t_{\text{gap}}$ represents the time since last coverage of the task's zone. For search and rescue operations, tasks completed before the golden hour receive a bonus weighted by $\beta$, incentivizing rapid coverage of high-probability areas. For delivery missions, the modifier applies a binary penalty for deadline violations, reflecting the discrete nature of delivery success.

\subsection{Optimization Strategy}

The DECIDE phase operates under strict temporal constraints (1.0--1.5 seconds), precluding exact optimization methods such as mixed-integer linear programming. The system therefore employs a two-stage optimization strategy that balances solution quality against computational tractability.

\textbf{Stage 1: Greedy Initialization.} Algorithm 2 generates an initial feasible allocation in $O(n \cdot m)$ time, where $n$ is the number of failed tasks and $m$ is the number of healthy UAVs. This greedy approach processes tasks in priority order, assigning each to the nearest constraint-satisfying UAV. While not globally optimal, this initialization typically achieves 70--85\% of the theoretical maximum objective value.

\textbf{Stage 2: Local Search Refinement.} When the time budget permits (approximately 500ms remaining after Stage 1), the system applies iterative local search to improve the initial allocation. The neighborhood structure considers pairwise task swaps between UAVs and reassignment of individual tasks to alternative vehicles:

\begin{lstlisting}[language=Python, caption=Two-Stage Optimization Strategy]
def optimize_allocation(failed_tasks, healthy_uavs, mission_context):
    """
    DECIDE phase optimization: maximize J(allocation) within time budget.
    """
    t_start = current_time_ms()
    TIME_BUDGET_MS = 1200  # Total DECIDE phase budget

    # Stage 1: Greedy initialization (Algorithm 2)
    allocation = greedy_allocate(failed_tasks, healthy_uavs, mission_context)
    best_score = compute_objective(allocation, mission_context)

    # Stage 2: Local search refinement (if time permits)
    elapsed = current_time_ms() - t_start
    if elapsed < TIME_BUDGET_MS - 200:  # Reserve 200ms for finalization
        allocation, best_score = local_search(
            allocation,
            failed_tasks,
            healthy_uavs,
            mission_context,
            max_time_ms = TIME_BUDGET_MS - elapsed - 200
        )

    return allocation, best_score

def local_search(allocation, tasks, uavs, context, max_time_ms):
    """
    Iterative improvement via task swaps and reassignments.
    """
    best = allocation.copy()
    best_score = compute_objective(best, context)
    t_start = current_time_ms()

    while (current_time_ms() - t_start) < max_time_ms:
        improved = False

        # Try pairwise swaps
        for t1, u1 in best.items():
            for t2, u2 in best.items():
                if u1 == u2:
                    continue

                # Propose swap: t1->u2, t2->u1
                candidate = best.copy()
                candidate[t1], candidate[t2] = u2, u1

                if is_feasible(candidate, context):
                    score = compute_objective(candidate, context)
                    if score > best_score:
                        best = candidate
                        best_score = score
                        improved = True
                        break
            if improved:
                break

        if not improved:
            break  # Local optimum reached

    return best, best_score
\end{lstlisting}

\subsection{Optimality Gap Analysis}

The greedy-plus-local-search strategy does not guarantee global optimality. To characterize solution quality, we define the optimality gap as:

\begin{equation}
\text{Gap} = \frac{J^* - J(\mathcal{A}_{\text{heuristic}})}{J^*} \times 100\%
\label{eq:gap}
\end{equation}

\noindent where $J^*$ is the optimal objective value computed offline via exhaustive search or MILP for small problem instances.

Preliminary analysis across the three mission scenarios indicates expected optimality gaps of 5--15\% for typical failure cases involving 3--6 lost tasks and 4--8 healthy UAVs. This trade-off is justified by the real-time constraint: a 10\% suboptimal solution computed in 1.2 seconds provides substantially greater operational value than an optimal solution requiring 30+ seconds, during which mission degradation continues unmitigated.

\subsection{Mission-Specific Parameter Configuration}

The objective function parameters are configured per mission type to reflect operational priorities:

\begin{table}[H]
\centering
\caption{Objective Function Parameters by Mission Type}
\label{tab:objective_params}
\begin{tabular}{|l|c|c|c|c|}
\hline
\textbf{Parameter} & \textbf{Symbol} & \textbf{Surveillance} & \textbf{SAR} & \textbf{Delivery} \\
\hline
Unallocated penalty & $\lambda$ & 0.3 & 0.5 & 0.4 \\
Coverage gap weight & $\gamma$ & 0.2 & --- & --- \\
Golden hour bonus & $\beta$ & --- & 0.5 & --- \\
Priority weights & $w_{\text{temporal}}$ & 0.3 & 0.5 & 0.2 \\
                 & $w_{\text{criticality}}$ & 0.5 & 0.3 & 0.6 \\
                 & $w_{\text{spatial}}$ & 0.2 & 0.2 & 0.2 \\
\hline
\end{tabular}
\end{table}

This parameterization enables the same OODA loop implementation to adapt its optimization behavior based on mission context, achieving domain-appropriate decision quality without requiring mission-specific algorithmic modifications.

\chapter{Mission Scenarios and Performance Analysis}

This chapter presents three representative mission scenarios that demonstrate the OODA-based fault tolerance system across diverse operational contexts. Each scenario illustrates different constraint priorities, failure modes, and recovery strategies, providing concrete validation of the system's adaptive capabilities under realistic conditions.

\section{Scenario Selection Rationale}

The three scenarios were selected to span the constraint space of real-world UAV operations:

\begin{itemize}
\item \textbf{Surveillance:} Battery-constrained with time-critical coverage requirements
\item \textbf{Search \& Rescue:} Time-critical with priority-driven partial coverage acceptance
\item \textbf{Delivery:} Payload-constrained with heterogeneous fleet capabilities
\end{itemize}

Each scenario represents a distinct operational domain with documented real-world deployment precedents, ensuring practical relevance beyond theoretical validation.

\section{SCENARIO 1: Long-Duration Perimeter Surveillance}

\subsection{Mission Context}

\begin{itemize}
    \item \textbf{Application:} Critical infrastructure monitoring (port, airport, border)
    \item \textbf{Duration:} 2-hour continuous coverage requirement
    \item \textbf{Fleet:} 6 UAVs with 30-minute flight time each (requires rotation strategy)
    \item \textbf{Operational Area:} 1200m $\times$ 1200m secured perimeter
    \item \textbf{Coverage Strategy:} 6 patrol zones (200m $\times$ 200m each)
\end{itemize}

\subsection{Mission Setup}

The surveillance mission divides the operational area into six equal patrol zones, with each UAV assigned continuous coverage of one zone. The 30-minute battery limitation relative to the 2-hour mission duration necessitates coordinated rotation scheduling to maintain uninterrupted coverage.

\begin{figure}[H]
\centering
\includegraphics[width=0.7\textwidth]{images/uav_zone_assignement.png}
\caption{Zone Assignment for Surveillance Mission}
\label{fig:zone_assignment}
\end{figure}

\subsection{Patrol Zone Specifications}

\textbf{Zone Characteristics:}
\begin{itemize}
\item \textbf{Zones A, B (North Perimeter):} High priority (P=0.9) - Main entry points
\item \textbf{Zones C, D (East/West):} Medium priority (P=0.6) - Secondary access
\item \textbf{Zones E, F (South Perimeter):} Standard priority (P=0.4) - Low-risk boundary
\end{itemize}

\textbf{Waypoint Density:}
\begin{itemize}
\item High-priority zones: 8 waypoints per patrol circuit
\item Medium-priority zones: 6 waypoints per patrol circuit
\item Standard-priority zones: 4 waypoints per patrol circuit
\end{itemize}

\subsection{Rotation Strategy}

The 2-hour mission duration with 30-minute battery life requires coordinated rotation:

\begin{itemize}
    \item \textbf{Wave 1 (0-30 min):} UAVs 1-6 patrol zones A-F
    \item \textbf{Transition 1 (25-30 min):} UAVs signal 20\% battery threshold
    \item \textbf{Wave 2 (30-60 min):} UAVs 1-6 return for battery swap, backup UAVs 7-12 deployed
    \item \textbf{Transition 2 (55-60 min):} Backup UAVs signal battery threshold
    \item \textbf{Wave 3 (60-90 min):} Recharged UAVs 1-6 resume patrol
    \item \textbf{Transition 3 (85-90 min):} Rotation cycle repeats
    \item \textbf{Wave 4 (90-120 min):} Final rotation with backup fleet 7-12
\end{itemize}

\subsection{Failure Scenario: Mid-Mission Battery Depletion}

\textbf{Failure Event:} UAV-3 experiences accelerated battery discharge at t=45 minutes during Wave 2, with battery dropping to 8\% (below 10\% safety threshold) while covering Zone C.

% \begin{figure}[H]
% \centering
% \includegraphics[width=0.8\textwidth]{images/surveillance_loop.png}
% \caption{Surveillance Mission OODA Loop Execution}
% \label{fig:surveillance_loop}
% \end{figure}

\subsubsection{OBSERVE Phase (T+0 to T+1.5s)}

\textbf{Failure Detection:}
\begin{lstlisting}[language=Python]
# Battery anomaly detected via statistical monitoring
battery_discharge_rate = (battery_previous - battery_current) / time_delta
# Expected: 3%/min, Observed: 8%/min
if battery_discharge_rate > ANOMALY_THRESHOLD:
    trigger_ooda_cycle()
\end{lstlisting}

\textbf{Fleet State Aggregation:}
\begin{itemize}
\item UAV-3: 8\% battery, Zone C (50\% patrol completion)
\item UAV-2 (Zone B): 45\% battery, 15\% spare capacity
\item UAV-4 (Zone D): 40\% battery, 12\% spare capacity
\item UAV-8 (standby): 95\% battery, available for emergency deployment
\item Lost coverage: Zone C, 16.7\% of total mission area
\end{itemize}

\subsubsection{ORIENT Phase (T+1.5s to T+3.0s)}

\textbf{Impact Assessment:}
\begin{itemize}
\item Mission criticality: Medium (Zone C is medium-priority area)
\item Coverage gap duration: 0 seconds (immediate reallocation required)
\item Temporal urgency: P\textsubscript{time} = 0.8 (continuous coverage mandate)
\end{itemize}

\textbf{Capacity Analysis:}
\begin{lstlisting}[language=Python]
# Calculate spare battery capacity
uav2_spare = 0.45 - 0.20 (safety) - 0.10 (committed) = 0.15 (15%)
uav4_spare = 0.40 - 0.20 (safety) - 0.08 (committed) = 0.12 (12%)

# Distance from adjacent UAVs to Zone C centroid
distance_uav2_to_zoneC = 250m  # requires ~8% battery
distance_uav4_to_zoneC = 220m  # requires ~7% battery

# Feasibility check
uav2_can_cover = (0.15 > 0.08)  # TRUE
uav4_can_cover = (0.12 > 0.07)  # TRUE
\end{lstlisting}

\textbf{Reallocation Strategy:}
\begin{itemize}
\item Split Zone C into two sub-zones: C1 (north 100m) and C2 (south 100m)
\item Assign C1 to UAV-2 (adjacent to Zone B)
\item Assign C2 to UAV-4 (adjacent to Zone D)
\item Accept reduced waypoint density: 3 waypoints per sub-zone vs. 6 original
\end{itemize}

\subsubsection{DECIDE Phase (T+3.0s to T+4.5s)}

\textbf{Strategy Selection:} Partial Reallocation (coverage degradation accepted)

\textbf{Waypoint Generation:}
\begin{lstlisting}[language=Python]
# UAV-2 extended patrol: Zone B + Zone C1
new_waypoints_uav2 = [
    zone_B_waypoints,  # Original 8 points
    zone_C1_waypoints   # Added 3 points
]
total_distance_uav2 = 1600m + 600m = 2200m
estimated_time = 2200m / 12m/s = 183s (~3 min per circuit)

# UAV-4 extended patrol: Zone D + Zone C2
new_waypoints_uav4 = [
    zone_D_waypoints,  # Original 6 points
    zone_C2_waypoints   # Added 3 points
]
\end{lstlisting}

\textbf{Collision Avoidance Check:}
\begin{itemize}
\item Zone C boundary between C1 and C2: 15m separation buffer
\item Temporal deconfliction: UAV-2 and UAV-4 patrol in opposite directions
\item No conflicts detected with other patrol zones
\end{itemize}

\subsubsection{ACT Phase (T+4.5s to T+5.5s)}

\textbf{Command Execution:}
\begin{lstlisting}[language=Python]
# Dispatch updated mission to UAV-2
send_mission_update(
    uav_id=2,
    new_waypoints=new_waypoints_uav2,
    priority=HIGH,
    acknowledge_timeout=200ms
)

# Dispatch updated mission to UAV-4
send_mission_update(
    uav_id=4,
    new_waypoints=new_waypoints_uav4,
    priority=HIGH
)

# Command UAV-3 immediate RTL
send_emergency_rtl(uav_id=3)
\end{lstlisting}

\textbf{Dashboard Update:}
\begin{itemize}
\item Coverage status: 91.7\% (slight degradation due to reduced waypoint density)
\item Mission status: ADAPTED - Partial Reallocation
\item Alert: ``Zone C coverage degraded to 3 waypoints per sub-zone (was 6)''
\item Estimated impact: 8.3\% surveillance quality reduction in Zone C
\end{itemize}

\subsection{Performance Analysis}

\textbf{Quantitative Results (Measured from Scenario S5):}
\begin{table}[H]
\centering
\caption{Surveillance Scenario Performance Metrics}
\label{tab:surveillance_performance}
\begin{tabular}{|l|c|c|}
\hline
\textbf{Metric} & \textbf{Target} & \textbf{Measured} \\
\hline
Coverage Recovery & >85\% & \textbf{100\%} \\
OODA Computation Time & <6s & \textbf{0.34 ms} \\
End-to-End Response Time & <6s & \textbf{2.2s (estimated with RF)} \\
Constraint Violations & 0 & \textbf{0 (verified)} \\
Mission Completion & >90\% & \textbf{100\% (mission continued)} \\
\hline
\end{tabular}
\end{table}

\textbf{Performance vs. Baselines:}
\begin{itemize}
\item \textbf{OODA:} 100\% recovery, 0.34ms computation, safe
\item \textbf{No Adaptation:} 87.5\% coverage (mission degraded)
\item \textbf{Greedy Baseline:} 100\% recovery, 0.18ms, safe
\item \textbf{Manual Operator:} 100\% recovery, 465s delay (unacceptable)
\end{itemize}

\textbf{Key Insights:}
\begin{itemize}
\item \textbf{Full Recovery Achieved:} System achieved 100\% coverage recovery (exceeded 85\% target)
\item \textbf{Sub-Millisecond Response:} 0.34ms computation validates real-time capability
\item \textbf{No Operator Intervention Required:} Autonomous reallocation sufficient
\item \textbf{Safety Preserved:} Zero constraint violations, all UAVs maintained 20\% battery reserve
\item \textbf{Comparison:} OODA 1,367$\times$ faster than manual operator (0.34ms vs 465s)
\end{itemize}

\textbf{Limitations Identified:}
\begin{itemize}
\item Reduced waypoint density in Zone C degrades detection probability
\item Strategy vulnerable to cascading failures (if UAV-2 or UAV-4 also fail)
\item Manual intervention required if >2 simultaneous failures occur
\end{itemize}

\section{SCENARIO 2: Emergency Search \& Rescue}

\subsection{Mission Context}

\begin{itemize}
    \item \textbf{Application:} Missing person in wilderness area
    \item \textbf{Duration:} 3-hour search window (golden hour + extended search)
    \item \textbf{Fleet:} 4 UAVs with thermal cameras
    \item \textbf{Search Area:} 2000m $\times$ 2000m forest region
    \item \textbf{Grid Resolution:} 100m $\times$ 100m cells (400 total cells)
\end{itemize}

\subsection{Mission Setup}

The search area is divided into a systematic grid with priority zones determined by terrain analysis, last known position (LKP), and probability of area (POA) calculations.

\begin{figure}[H]
\centering
\includegraphics[width=0.6\textwidth]{images/priorities.png}
\caption{Priority Tree for Search Zones}
\label{fig:priorities}
\end{figure}

\subsection{Search Grid Prioritization}

Using Algorithm 1 (calculate\_task\_priority), each grid cell receives a priority score:

\textbf{High Priority Zones (P > 0.7):}
\begin{itemize}
\item \textbf{Zone 1 (LKP radius):} 100 cells, P=0.9 - Last known position $\pm$ 500m
\item \textbf{Zone 2 (Water sources):} 40 cells, P=0.8 - Streams, ponds within 1km of LKP
\item \textbf{Zone 3 (Shelters):} 20 cells, P=0.75 - Cabins, overhangs, clearings
\end{itemize}

\textbf{Medium Priority Zones (0.4 < P < 0.7):}
\begin{itemize}
\item \textbf{Zone 4 (Trails):} 80 cells, P=0.6 - Hiking paths and access routes
\item \textbf{Zone 5 (Clearings):} 60 cells, P=0.5 - Open areas visible from air
\end{itemize}

\textbf{Low Priority Zones (P < 0.4):}
\begin{itemize}
\item \textbf{Zone 6 (Dense forest):} 80 cells, P=0.3 - Heavy canopy, low POA
\item \textbf{Zone 7 (Steep terrain):} 20 cells, P=0.2 - Cliffs, ravines (mobility limited)
\end{itemize}

\subsection{Initial Task Allocation}

\textbf{Fleet Assignment:}
\begin{lstlisting}[language=Python]
# UAV assignments by priority zones
uav1_assignment = zones[1:100]   # High priority LKP radius (100 cells)
uav2_assignment = zones[101:160] # High priority water + shelters (60 cells)
uav3_assignment = zones[161:300] # Medium priority trails + clearings (140 cells)
uav4_assignment = zones[301:400] # Low priority dense/steep (100 cells)

# Coverage rate: 4 cells/minute (thermal scan + image capture)
# Expected completion times:
# UAV-1: 25 minutes (critical zone)
# UAV-2: 15 minutes (critical zone)
# UAV-3: 35 minutes (medium zone)
# UAV-4: 25 minutes (low zone)
\end{lstlisting}

% \begin{figure}[H]
% \centering
% \includegraphics[width=0.8\textwidth]{images/sar_loop.png}
% \caption{Search and Rescue Mission OODA Loop}
% \label{fig:sar_loop}
% \end{figure}

\subsection{Failure Scenario: Critical Zone UAV Loss}

\textbf{Failure Event:} UAV-2 loses GPS signal at t=8 minutes due to terrain interference, triggering failsafe RTL after 90 seconds of signal loss. UAV-2 was 20\% through its high-priority assignment (12 of 60 cells completed).

\subsubsection{OBSERVE Phase (T+0 to T+1.5s)}

\textbf{Failure Detection:}
\begin{itemize}
\item GPS signal timeout detected at t=8:00
\item UAV-2 entered autonomous RTL mode at t=8:01:30
\item OODA cycle triggered at t=8:01:30 upon RTL confirmation
\item Lost coverage: 48 cells in high-priority zones (water sources + shelters)
\end{itemize}

\textbf{Fleet State at Failure:}
\begin{lstlisting}[language=Python]
fleet_state = {
    'uav1': {
        'progress': '15/100 cells (15%)',
        'battery': 75,
        'position': 'Zone 1 - LKP grid',
        'spare_capacity': '20% battery'
    },
    'uav2': {
        'status': 'RTL (GPS LOSS)',
        'progress': '12/60 cells (20%)',
        'lost_tasks': 48  # High priority cells
    },
    'uav3': {
        'progress': '10/140 cells (7%)',
        'battery': 80,
        'position': 'Zone 4 - Trails',
        'spare_capacity': '35% battery'
    },
    'uav4': {
        'progress': '8/100 cells (8%)',
        'battery': 78,
        'position': 'Zone 7 - Steep terrain',
        'spare_capacity': '30% battery'
    }
}
\end{lstlisting}

\subsubsection{ORIENT Phase (T+1.5s to T+3.0s)}

\textbf{Mission Impact Assessment:}
\begin{itemize}
\item Total mission degradation: 12\% (48 of 400 cells)
\item Critical zone degradation: 30\% (48 of 160 high-priority cells)
\item Temporal criticality: HIGH (golden hour: 60 minutes total, 52 minutes remaining)
\end{itemize}

\textbf{Capacity Analysis:}
\begin{lstlisting}[language=Python]
# Reallocation feasibility for 48 high-priority cells

# Option 1: UAV-1 (already in high-priority zone)
uav1_remaining_tasks = 85  # cells
uav1_spare_capacity = 20%  # battery (~5 minutes = 20 cells)
uav1_can_absorb = min(20, 48) = 20 cells

# Option 2: UAV-3 (medium priority, higher spare capacity)
distance_uav3_to_zone2 = 800m  # ~3% battery
uav3_spare_capacity = 35% - 3% = 32%  # (~8 minutes = 32 cells)
uav3_can_absorb = min(32, 48) = 32 cells

# Option 3: UAV-4 (low priority, redirect entirely)
distance_uav4_to_zone2 = 1200m  # ~5% battery
uav4_spare_capacity = 30% - 5% = 25%  # (~6 minutes = 24 cells)
uav4_can_absorb = min(24, 48) = 24 cells

# Total reallocation capacity: 20 + 32 + 24 = 76 cells (exceeds 48 needed)
# Feasibility: FULL REALLOCATION possible
\end{lstlisting}

\textbf{Priority-Based Allocation Strategy:}
\begin{enumerate}
\item Assign 20 water source cells to UAV-1 (already nearby, minimal diversion)
\item Assign 20 shelter cells to UAV-3 (highest spare capacity, redirect from trails)
\item Keep 8 remaining shelter cells for UAV-4 if UAV-3 runs low on battery
\item Defer UAV-4's low-priority zone 7 to secondary search phase
\end{enumerate}

\subsubsection{DECIDE Phase (T+3.0s to T+4.5s)}

\textbf{Strategy Selection:} Full Reallocation with Priority Optimization

\textbf{Waypoint Recalculation:}
\begin{lstlisting}[language=Python]
# UAV-1 new mission
uav1_new_waypoints = [
    remaining_zone1_cells,      # 85 cells (original)
    zone2_water_cells[0:20]     # 20 cells (added from UAV-2)
]
# Total: 105 cells, estimated time: 26 minutes
# Battery check: 75% - 26 min usage = 23% > 20% safety margin [PASS]

# UAV-3 new mission  
uav3_new_waypoints = [
    zone3_shelter_cells[0:20],  # 20 cells (from UAV-2, HIGH priority)
    remaining_zone4_cells[0:60] # 60 cells (original medium priority)
]
# Total: 80 cells (reduced from 140), estimated time: 20 minutes
# Battery check: 80% - 20 min - 3% transit = 37% remaining [PASS]

# UAV-4 contingency assignment (if UAV-3 needs battery margin)
uav4_backup = zone3_shelter_cells[20:28]  # 8 cells on standby
\end{lstlisting}

\textbf{Collision Avoidance Verification:}
\begin{itemize}
\item UAV-1 and UAV-3 zones separated by 600m (Zone 1 vs Zone 2/3)
\item Temporal deconfliction: UAV-1 completes Zone 1 before entering Zone 2
\item UAV-3 begins Zone 3 immediately, no overlap with UAV-1 timeline
\item No conflicts detected
\end{itemize}

\subsubsection{ACT Phase (T+4.5s to T+6.0s)}

\textbf{Command Dispatch:}
\begin{lstlisting}[language=Python]
# Update UAV-1 mission
send_mission_update(
    uav_id=1,
    new_waypoints=uav1_new_waypoints,
    priority=CRITICAL,
    reason="Absorbing UAV-2 water source cells"
)

# Update UAV-3 mission (major reroute)
send_mission_update(
    uav_id=3,
    new_waypoints=uav3_new_waypoints,
    priority=CRITICAL,
    reason="Redirect to high-priority shelter cells"
)

# Place UAV-4 on conditional standby
send_status_update(
    uav_id=4,
    message="Continue Zone 7. Standby for potential Zone 3 backup assignment."
)
\end{lstlisting}

\textbf{Dashboard Update:}
\begin{itemize}
\item Mission status: ADAPTED - Full Reallocation
\item Coverage recovery: 100\% of high-priority cells reallocated
\item Degraded coverage: Zone 4 (trails) reduced from 140 to 60 cells (57\% coverage)
\item Zone 7 (low priority) may be incomplete if UAV-4 redirected
\item Estimated mission completion: 90\% of total area (acceptable for SAR)
\end{itemize}

\subsection{Performance Analysis}

\textbf{Quantitative Results (Measured from Scenarios R5 \& R6):}
\begin{table}[H]
\centering
\caption{Search \& Rescue Scenario Performance Metrics}
\label{tab:sar_performance}
\begin{tabular}{|l|c|c|}
\hline
\textbf{Metric} & \textbf{Target} & \textbf{Measured} \\
\hline
Coverage Recovery & >90\% & \textbf{100\%} \\
OODA Computation Time (R5) & <6s & \textbf{0.50 ms} \\
OODA Computation Time (R6) & <6s & \textbf{0.16 ms (fastest)} \\
End-to-End Response Time & <6s & \textbf{2.3s (estimated with RF)} \\
Constraint Violations & 0 & \textbf{0 (verified)} \\
Golden Hour Impact & Minimize & \textbf{0.0000138\% consumed} \\
\hline
\end{tabular}
\end{table}

\textbf{Performance vs. Baselines:}
\begin{itemize}
\item \textbf{OODA (R5):} 100\% recovery, 0.50ms computation, safe
\item \textbf{OODA (R6):} 100\% recovery with out-of-grid permission, 0.16ms (fastest measured)
\item \textbf{No Adaptation:} 66.7\% coverage (33.3\% mission loss)
\item \textbf{Manual Operator:} 100\% recovery, 465s delay (12.9\% of golden hour)
\end{itemize}

\textbf{Key Insights:}
\begin{itemize}
\item \textbf{Full Recovery Achieved:} 100\% coverage recovery in both R5 and R6 scenarios
\item \textbf{Life-Saving Speed:} OODA saves 7.7 minutes vs manual in golden hour scenarios
\item \textbf{Permission System Validated:} R6 correctly used UAV-4 with out-of-grid permission
\item \textbf{Time-Critical Advantage:} 0.50ms computation vs 465s manual (930,000$\times$ faster)
\item \textbf{Safety Preserved:} Zero constraint violations across all SAR scenarios
\end{itemize}

\textbf{Critical Finding - Golden Hour Impact:}
\begin{itemize}
\item \textbf{OODA:} 0.50ms = 0.0000138\% of 60-minute golden hour
\item \textbf{Manual:} 465s = 12.9\% of golden hour (unacceptable delay)
\item \textbf{Value Proposition:} In life-or-death SAR missions, 7.7-minute advantage can determine victim survival
\end{itemize}

\section{SCENARIO 3: Medical Supply Delivery}

\subsection{Mission Context}

\begin{itemize}
    \item \textbf{Application:} Emergency medical supply delivery to rural clinics
    \item \textbf{Duration:} 1-hour delivery window
    \item \textbf{Fleet:} 3 UAVs (heterogeneous payload capacities)
    \item \textbf{Delivery Area:} 5 clinics distributed over 3000m $\times$ 2000m region
\end{itemize}

\subsection{Mission Setup}

\textbf{Fleet Specifications:}

\begin{lstlisting}[language=Python, caption=Delivery Fleet Configuration]
class DeliveryFleet:
    uav1 = {
        'type': 'Heavy Lifter',
        'payload_capacity': 5.0,  # kg
        'battery': 25,  # minutes
        'speed': 12,  # m/s
        'assigned_packages': ['A', 'B'],  # Total: 4.5 kg
        'route': [Clinic_1, Clinic_2]
    }
    uav2 = {
        'type': 'Standard',
        'payload_capacity': 2.5,  # kg
        'battery': 30,  # minutes
        'speed': 15,  # m/s  
        'assigned_packages': ['C', 'D'],  # Total: 2.2 kg
        'route': [Clinic_3, Clinic_4]
    }
    uav3 = {
        'type': 'Standard',
        'payload_capacity': 2.5,  # kg
        'battery': 30,  # minutes
        'speed': 15,  # m/s
        'assigned_packages': ['E'],  # Total: 1.8 kg
        'route': [Clinic_5]
    }
\end{lstlisting}

\textbf{Package Priorities Using Algorithm 1:}

\begin{lstlisting}[language=Python, caption=Package Priority Assignment]
packages = {
    'A': {
        'weight': 2.5,  # kg
        'destination': 'Clinic_1',
        'coordinates': (800, 1200),  # meters from depot
        'priority': 1.0, 
        'contents': 'Insulin (CRITICAL)',
        'deadline': 30,  # minutes
        'criticality': 'medical_emergency'
    },
    'B': {
        'weight': 2.0,  # kg
        'destination': 'Clinic_2', 
        'coordinates': (1500, 800),
        'priority': 0.7, 
        'contents': 'Antibiotics (HIGH)',
        'deadline': 45,  # minutes
        'criticality': 'time_sensitive'
    },
    'C': {
        'weight': 1.2,  # kg
        'destination': 'Clinic_3',
        'coordinates': (2200, 1500),
        'priority': 0.4, 
        'contents': 'Bandages (MEDIUM)',
        'deadline': 60,  # minutes
        'criticality': 'standard_delivery'
    },
    'D': {
        'weight': 1.0,  # kg
        'destination': 'Clinic_4',
        'coordinates': (2800, 600),
        'priority': 0.4, 
        'contents': 'Gauze (MEDIUM)',
        'deadline': 60,  # minutes
        'criticality': 'standard_delivery'
    },
    'E': {
        'weight': 1.8,  # kg
        'destination': 'Clinic_5',
        'coordinates': (1200, 300),
        'priority': 0.2, 
        'contents': 'Vitamins (LOW)',
        'deadline': 90,  # minutes
        'criticality': 'standard_delivery'
    },
}

# Priority calculation using calculate_task_priority():
# Package A: P = 1.0 (medical_emergency + temporal_urgency)
# Package B: P = 0.7 (time_sensitive)
# Packages C,D: P = 0.4 (standard_delivery)
# Package E: P = 0.2 (standard_delivery, low criticality)
\end{lstlisting}

\textbf{Delivery Route Map:}

\begin{figure}[H]
\centering
\includegraphics[width=0.85\textwidth]{images/delivery_route_map.png}
\caption{Delivery Route Map with Clinic Locations}
\label{fig:delivery_map}
\end{figure}

\subsection{Failure Scenario: Heavy Lifter Battery Anomaly}

\textbf{Failure Event:} UAV-1 experiences unexpected battery degradation at t=15 minutes due to headwind and payload weight. Battery discharge rate observed at 5\%/min vs. expected 3\%/min. Current battery: 40\% (expected 55\%).

% \begin{figure}[H]
% \centering
% \includegraphics[width=0.8\textwidth]{images/delivery_loop.png}
% \caption{Delivery Mission OODA Loop}
% \label{fig:delivery_loop}
% \end{figure}

\subsubsection{OBSERVE Phase (T+0 to T+1.5s)}

\textbf{Failure Detection:}
\begin{lstlisting}[language=Python]
# Battery anomaly detected via statistical monitoring
battery_discharge_rate = (55 - 40) / 5_minutes = 3%/min (expected)
observed_rate = 5%/min
anomaly_detected = (observed_rate > 1.5 * expected_rate)  # TRUE

# Current state
uav1_state = {
    'battery': 40,  # %
    'position': (600, 1000),  # 200m from Clinic_1
    'remaining_distance_to_clinic1': 200,  # m
    'remaining_distance_to_clinic2': 1100,  # m (after Clinic_1)
    'payload': 4.5,  # kg (both packages still loaded)
}

# Battery projection
battery_to_clinic1 = 40 - (200m / 12m/s) * 5%/min = 38.6%
battery_after_clinic1 = 38.6 - 2 (landing/takeoff) = 36.6%
battery_to_clinic2 = 36.6 - (1100m / 12m/s) * 5%/min = 28.9%
battery_after_clinic2 = 28.9 - 2 (landing) = 26.9%
battery_to_return = 26.9 - (1500m / 12m/s) * 5%/min = 16.6%

# Safety margin check
safety_threshold = 20%
if battery_to_return < safety_threshold:
    # FAILURE: Insufficient battery to complete mission safely
    trigger_ooda_cycle()
\end{lstlisting}

\textbf{Lost Packages:}
\begin{itemize}
\item Package A (2.5kg, P=1.0, Clinic\_1): Currently 200m away, deliverable
\item Package B (2.0kg, P=0.7, Clinic\_2): Cannot be delivered safely by UAV-1
\end{itemize}

\subsubsection{ORIENT Phase (T+1.5s to T+3.0s)}

\textbf{Impact Assessment:}
\begin{itemize}
\item Mission impact: 20\% cargo value lost (1 of 5 packages)
\item Critical package status: Package A (P=1.0) still deliverable
\item High-priority package at risk: Package B (P=0.7)
\end{itemize}

\textbf{Fleet Capacity Analysis:}
\begin{lstlisting}[language=Python]
# UAV-2 capacity evaluation
uav2_state = {
    'current_payload': 2.2,  # kg (Packages C + D)
    'max_payload': 2.5,  # kg
    'spare_payload': 0.3,  # kg
    'position': (1800, 1400),  # en route to Clinic_3
    'battery': 70  # %
}
# Package B weight: 2.0 kg
# UAV-2 cannot carry Package B (2.0 > 0.3 spare capacity) [FAIL]

# UAV-3 capacity evaluation
uav3_state = {
    'current_payload': 1.8,  # kg (Package E)
    'max_payload': 2.5,  # kg
    'spare_payload': 0.7,  # kg
    'position': (1000, 500),  # en route to Clinic_5
    'battery': 75,  # %
    'distance_to_clinic2': 900  # m
}
# Package B weight: 2.0 kg
# UAV-3 cannot carry Package B (2.0 > 0.7 spare capacity) [FAIL]

# Reallocation feasibility conclusion
feasible_reallocation = False
reason = "Package B (2.0kg) exceeds spare payload capacity of UAV-2 (0.3kg) and UAV-3 (0.7kg)"
\end{lstlisting}

\textbf{Alternative Strategies:}
\begin{enumerate}
\item \textbf{Return-to-Base Cargo Swap:} UAV-2 or UAV-3 returns to depot, drops current cargo, picks up Package B
    \begin{itemize}
    \item Time cost: 10-15 minutes (round trip + swap)
    \item Risk: Package B deadline violation (45 min deadline, 30 min elapsed + 15 min delay = 45 min total)
    \item Feasibility: MARGINAL
    \end{itemize}
\item \textbf{Backup UAV Deployment:} Deploy fresh UAV-4 from depot with Package B only
    \begin{itemize}
    \item Time cost: 5 minutes (preparation + launch)
    \item Delivery time: 8 minutes to Clinic\_2
    \item Total: 13 minutes (within deadline)
    \item Feasibility: RECOMMENDED
    \end{itemize}
\item \textbf{Ground Vehicle Escalation:} Dispatch emergency vehicle
    \begin{itemize}
    \item Time cost: 20-30 minutes (variable traffic)
    \item Feasibility: BACKUP OPTION
    \end{itemize}
\end{enumerate}

\subsubsection{DECIDE Phase (T+3.0s to T+4.5s)}

\textbf{Strategy Selection:} Operator Escalation with Backup UAV Recommendation

\textbf{Decision Logic:}
\begin{lstlisting}[language=Python]
def determine_operator_escalation(coverage_percentage, unallocated_tasks):
    # Package B: High-priority (P=0.7), cannot be autonomously reallocated
    critical_tasks_lost = [task for task in unallocated_tasks if task.priority > 0.7]
    
    if len(critical_tasks_lost) > 0:
        return {
            'escalate': True,
            'urgency': 'HIGH',
            'reason': f'Package B (Antibiotics, P=0.7) cannot be reallocated due to payload constraints',
            'recommendation': 'Deploy backup UAV-4 within 5 minutes to meet deadline',
            'backup_plan': 'Ground vehicle dispatch as fallback (20-30 min ETA)',
            'autonomous_action': 'UAV-1 will complete Package A delivery, then RTL'
        }
\end{lstlisting}

\textbf{Autonomous Actions (No Operator Input Required):}
\begin{enumerate}
\item Command UAV-1 to complete Package A delivery to Clinic\_1
\item Command UAV-1 to skip Clinic\_2 (Package B) and return directly to depot
\item Update UAV-2 and UAV-3 missions: No changes (continue as planned)
\end{enumerate}

\subsubsection{ACT Phase (T+4.5s to T+5.5s)}

\textbf{Command Dispatch:}
\begin{lstlisting}[language=Python]
# UAV-1: Modified mission (deliver A only, then RTL)
send_mission_update(
    uav_id=1,
    new_waypoints=[Clinic_1, Depot],
    packages=['A'],  # Remove Package B
    priority=CRITICAL,
    reason="Battery insufficient for Clinic_2. Package A delivery only."
)

# Operator Dashboard Alert
send_operator_alert(
    urgency='HIGH',
    title='Package B Reallocation Required',
    message='''
    Package B (Antibiotics, 2.0kg, P=0.7) cannot be delivered by UAV-1 
    due to battery constraints. Autonomous reallocation not possible 
    (exceeds spare payload capacity of UAV-2 and UAV-3).
    
    RECOMMENDED ACTION:
    Deploy backup UAV-4 within 5 minutes to meet 45-minute deadline.
    
    BACKUP OPTION:
    Ground vehicle dispatch (20-30 min ETA).
    
    COUNTDOWN TO AUTO-FAILSAFE: 30 seconds
    (If no response, system will automatically deploy UAV-4)
    ''',
    options=['Deploy UAV-4', 'Dispatch Ground Vehicle', 'Accept Delay'],
    auto_action='Deploy UAV-4',
    countdown=30  # seconds
)

# Auto-failsafe preparation (if operator doesn't respond)
prepare_backup_uav(
    uav_id=4,
    payload=['B'],
    destination='Clinic_2',
    launch_authorization='PENDING_OPERATOR_APPROVAL'
)
\end{lstlisting}

\textbf{Dashboard Update:}
\begin{itemize}
\item Mission status: OPERATOR ESCALATION - High Priority Package Reallocation
\item Coverage: 80\% (4 of 5 packages deliverable autonomously)
\item Alert status: CRITICAL - Operator response required within 30 seconds
\item Autonomous action: Package A delivery proceeding, UAV-1 RTL commanded
\item Backup UAV-4: STANDBY (awaiting operator confirmation)
\end{itemize}

\subsection{Performance Analysis}

\textbf{Quantitative Results (Measured from Scenarios D6 \& D7):}
\begin{table}[H]
\centering
\caption{Delivery Scenario Performance Metrics}
\label{tab:delivery_performance}
\begin{tabular}{|l|c|c|}
\hline
\textbf{Metric} & \textbf{Target} & \textbf{Measured} \\
\hline
Autonomous Coverage (D6) & >75\% & \textbf{0\% (escalated)} \\
Autonomous Coverage (D7) & >75\% & \textbf{0\% (escalated)} \\
OODA Computation Time (D6) & <6s & \textbf{0.11 ms} \\
OODA Computation Time (D7) & <6s & \textbf{0.23 ms} \\
Operator Escalation & Appropriate & \textbf{YES (correct)} \\
Constraint Violations & 0 & \textbf{0 (safe)} \\
Safety Preserved & No violations & \textbf{YES (verified)} \\
\hline
\end{tabular}
\end{table}

\textbf{Performance vs. Baselines:}
\begin{itemize}
\item \textbf{OODA (D6):} 0\% autonomous (payload violation), correctly escalated, 0.11ms
\item \textbf{OODA (D7):} 0\% autonomous (grid boundary), correctly escalated, 0.23ms
\item \textbf{Greedy Baseline (D6):} 100\% coverage but \textbf{UNSAFE} (payload violation)
\item \textbf{Greedy Baseline (D7):} 100\% coverage but \textbf{UNSAFE} (boundary violation)
\item \textbf{Manual Operator:} 0\% autonomous (acknowledges infeasibility), 465s
\end{itemize}

\textbf{Key Insights:}
\begin{itemize}
\item \textbf{Intelligent Escalation Validated:} 0\% autonomous reallocation is NOT a failure---demonstrates correct constraint detection
\item \textbf{Safety-First Design:} OODA correctly refused unsafe operations; Greedy violated constraints
\item \textbf{D6 Finding:} Package B (2.0 kg) exceeds all UAV spare capacity (max 0.7 kg)
\item \textbf{D7 Finding:} Destination (3500, 2500) outside grid [0-3000, 0-2000]; no UAV has permission
\item \textbf{Critical Result:} Zero constraint violations vs. Greedy's 2 violations proves constraint-aware design
\end{itemize}

\textbf{Comparison to Idealized Systems:}

\begin{table}[H]
\centering
\caption{Comparison with Alternative Approaches}
\label{tab:delivery_comparison}
\begin{tabular}{|p{4cm}|p{3cm}|p{3cm}|p{3cm}|}
\hline
\textbf{Approach} & \textbf{Coverage} & \textbf{Safety} & \textbf{Deployability} \\
\hline
No Adaptation (Abort) & 60\% (3/5) & Safe & Simple \\
\hline
Greedy Nearest-Neighbor & 100\% (claims) & \textbf{UNSAFE} (UAV-2 overloaded to 4.2kg / 2.5kg max) & Not deployable \\
\hline
Idealized Full Autonomy & 100\% (claims) & Assumes unlimited resources & Not deployable \\
\hline
\textbf{Hybrid OODA (This Work)} & \textbf{80\% autonomous + operator support for remaining 20\%} & \textbf{Safe (no overloads)} & \textbf{Deployable with operator oversight} \\
\hline
\end{tabular}
\end{table}

\textbf{Value Proposition:}
\begin{itemize}
\item \textbf{Avoided Mission Failure:} Package A (critical insulin) delivered successfully
\item \textbf{Operator Workload Reduction:} System autonomously handled 80\%, requiring intervention only for payload-constrained Package B
\item \textbf{Regulatory Compliance:} Operator-in-loop design satisfies BVLOS supervision requirements
\item \textbf{Honest Performance:} 80\% autonomous + 20\% supervised is more valuable than 100\% claims that cannot be deployed
\end{itemize}

\section{Cross-Scenario Comparative Analysis}

\subsection{Constraint Prioritization by Mission Type}

\begin{table}[H]
\centering
\caption{Constraint Priority by Mission Type}
\label{tab:constraint_priority}
\begin{tabular}{|l|c|c|c|c|}
\hline
\textbf{Mission Type} & \textbf{Battery} & \textbf{Payload} & \textbf{Time} & \textbf{Dominant Failure Mode} \\
\hline
Surveillance & Critical & N/A & High & Battery depletion during rotation \\
Search \& Rescue & Critical & N/A & Critical & GPS loss / Communication timeout \\
Delivery & High & Critical & High & Payload constraint violations \\
\hline
\end{tabular}
\end{table}

\subsection{Performance Comparison Across Scenarios}

\begin{table}[H]
\centering
\caption{Measured Performance by Scenario (S5, R5/R6, D6/D7)}
\label{tab:scenario_performance}
\begin{tabular}{|l|c|c|c|}
\hline
\textbf{Metric} & \textbf{Surveillance (S5)} & \textbf{SAR (R5/R6)} & \textbf{Delivery (D6/D7)} \\
\hline
Coverage Recovery & \textbf{100\%} & \textbf{100\%} & \textbf{0\% (escalated)} \\
OODA Computation & \textbf{0.34 ms} & \textbf{0.50/0.16 ms} & \textbf{0.11/0.23 ms} \\
Operator Escalation & 0\% (autonomous) & 0\% (autonomous) & 100\% (correct) \\
Constraint Violations & \textbf{0} & \textbf{0} & \textbf{0} \\
Mission Outcome & 100\% complete & 100\% complete & Escalated (safe) \\
\hline
\end{tabular}
\end{table}

\textbf{Key Findings Across All Scenarios:}
\begin{itemize}
\item \textbf{Speed:} All scenarios demonstrate sub-millisecond computation (0.11-0.50 ms)
\item \textbf{Safety:} Zero constraint violations across all 5 experimental scenarios
\item \textbf{Coverage:} 100\% recovery for S5, R5, R6 where physically feasible
\item \textbf{Escalation:} D6/D7 correctly escalated when constraints prevent safe reallocation
\item \textbf{Advantage:} 500,000$\times$ faster than manual operator (465s) across all scenarios
\end{itemize}

\subsection{Failure Mode Taxonomy}

\begin{table}[H]
\centering
\caption{Failure Modes and System Response}
\label{tab:failure_taxonomy}
\begin{tabular}{|p{3.5cm}|p{3cm}|p{3.5cm}|p{3.5cm}|}
\hline
\textbf{Failure Mode} & \textbf{Scenario} & \textbf{Autonomous Recovery} & \textbf{Operator Escalation} \\
\hline
Battery depletion & Surveillance & Partial reallocation (91.7\%) & Not required \\
\hline
GPS signal loss & SAR & Full reallocation (100\% high-priority) & Not required \\
\hline
Payload constraint violation & Delivery & Partial (80\%) & Required for 20\% \\
\hline
Cascading failures (hypothetical) & All & Degraded (<50\%) & Required \\
\hline
\end{tabular}
\end{table}

\subsection{Key Takeaways}

\begin{enumerate}
\item \textbf{No One-Size-Fits-All Solution:} Each mission type requires different constraint prioritization and recovery strategies

\item \textbf{Partial Autonomy is Valuable:} 80-91\% autonomous recovery reduces operator workload significantly compared to 0\% (no adaptation) or 100\% manual intervention

\item \textbf{Honest Performance Metrics:} Reporting 80\% autonomous + 20\% supervised is more valuable than claiming 100\% autonomy that cannot be deployed

\item \textbf{Operator Escalation is a Feature, Not a Bug:} Recognizing physical limitations and involving operators for non-compensable failures is essential for safe, deployable systems

\item \textbf{Real-World Constraints Matter:} Battery, payload, and time constraints dominate failure recovery—ignoring them (as many academic works do) produces non-deployable systems
\end{enumerate}

These three scenarios validate the OODA-based hybrid autonomy architecture across diverse operational contexts, demonstrating both its capabilities and its honest acknowledgment of limitations—a critical requirement for real-world deployment.

\chapter{Implementation and Validation Plan}

\section{Simulation Environment}

\textbf{Software stack:}
\begin{itemize}
\item \textbf{Physics simulation:} Python (6-DOF quaternion dynamics with PID cascade control)
\item \textbf{Visualization:} Flask web dashboard (port 8085) with WebSocket real-time updates via SocketIO
\item \textbf{Control:} PID controllers for waypoint following with position/velocity cascade
\item \textbf{Communication:} TCP/IP with JSON-RPC 2.0 protocol (GCS server on port 5555)
\item \textbf{Package management:} UV for dependency management and virtual environment
\end{itemize}

\textbf{Based on:}
\begin{itemize}
\item Quad SimCon codebase as reference for UAV dynamics modeling
\item Project repository: \url{https://github.com/vriesz/fuse}
\item Reference model: DJI Matrice 350 RTK parameters
\end{itemize}

\textbf{Implementation modules:}
\begin{itemize}
\item \texttt{gcs/ooda\_engine.py} -- Core OODA loop implementation with phase timing
\item \texttt{gcs/fleet\_monitor.py} -- Telemetry collection and multi-modal failure detection
\item \texttt{gcs/constraint\_validator.py} -- Battery/payload/collision constraint checking
\item \texttt{gcs/mission\_manager.py} -- Task database with assignment tracking
\item \texttt{gcs/objective\_function.py} -- Two-stage optimization (greedy + local search)
\item \texttt{uav/simulation.py} -- 6-DOF dynamics with PID cascade control
\item \texttt{uav/client.py} -- GCS communication over TCP/IP with JSON-RPC 2.0
\item \texttt{visualization/web\_dashboard.py} -- Flask app with WebSocket updates
\end{itemize}

\section{Validation Metrics}

\textbf{Primary metrics:}

\begin{enumerate}
\item \textbf{Coverage Percentage:} \% of original mission tasks completed after failure
\begin{equation}
\text{Coverage\%} = \frac{\text{Reallocated tasks}}{\text{Total failed tasks}} \times 100
\end{equation}
Target: >65\% for single failure, >50\% for double failure

\item \textbf{Adaptation Time:} Time from failure detection to new commands issued
\begin{equation}
T_{adapt} = T_{ACT} - T_{failure\_event}
\end{equation}
Target: <5 seconds

\item \textbf{Battery Efficiency:} \% of available spare capacity utilized
\begin{equation}
\text{Efficiency\%} = \frac{\text{Spare capacity used}}{\text{Total spare capacity}} \times 100
\end{equation}
Target: >80\% utilization

\item \textbf{Mission Completion Rate:} \% of missions completed despite failures
\begin{equation}
\text{Completion\%} = \frac{\text{Missions finished}}{\text{Total missions}} \times 100
\end{equation}
Target: >90\% for single failure scenarios
\end{enumerate}

\textbf{Secondary metrics:}
\begin{itemize}
\item Operator workload (number of escalations per mission)
\item Communication bandwidth usage (kB/s)
\item Decision quality (compared to optimal offline solution)
\end{itemize}

\section{Test Scenarios}

The system validation comprises \textbf{169 automated tests} organized into three categories:

\textbf{Unit tests (53):}
\begin{itemize}
\item Battery management and reserve calculations
\item Grid boundary validation
\item State machine transitions
\item Constraint validation logic
\item Priority scoring algorithm
\end{itemize}

\textbf{Integration tests (81):}
\begin{itemize}
\item \textbf{Surveillance missions:} Continuous coverage with rotation strategy
\item \textbf{Search \& Rescue:} Grid-based coverage with golden hour deadline
\item \textbf{Medical Delivery:} Priority-based two-phase delivery (pickup → dropoff)
\item Multi-UAV coordination and collision avoidance
\item OODA loop execution across failure scenarios
\end{itemize}

\textbf{Regression tests (15):}
\begin{itemize}
\item Bug fix validation to prevent regression
\item Edge cases discovered during development
\item Constraint violation prevention
\end{itemize}

\textbf{Experimental validation scenarios (5):}
\begin{itemize}
\item S5: Surveillance with single UAV failure
\item R5: Search \& Rescue with single UAV failure
\item R6: SAR with out-of-grid zone requiring permission
\item D6: Delivery with payload constraint violation
\item D7: Delivery with out-of-grid destination
\end{itemize}

All tests execute in approximately 0.22 seconds with pytest, enabling rapid validation during development.

\section{Baseline Comparisons}

\textbf{Comparison strategies:}

\begin{enumerate}
\item \textbf{No Adaptation (Baseline):} Fixed mission, aborts on UAV failure
\begin{itemize}
\item Expected: 0\% coverage recovery, mission failure
\end{itemize}

\item \textbf{Manual Operator Replanning:} Human operator manually reassigns tasks
\begin{itemize}
\item Expected: 80-95\% coverage recovery, 5-10 minute delay
\end{itemize}

\item \textbf{Greedy Nearest-Neighbor:} Simple algorithm assigns to nearest UAV
\begin{itemize}
\item Expected: 40-60\% coverage (ignores battery constraints)
\end{itemize}

\item \textbf{Hybrid OODA (This Work):} Priority-based with constraints
\begin{itemize}
\item Expected: 65-95\% coverage recovery, <5 second adaptation
\end{itemize}
\end{enumerate}

\section{Visualization and Logging}

\textbf{Real-time web dashboard (port 8085):}
\begin{itemize}
\item Flask application with WebSocket updates via SocketIO
\item Fleet position map with task coverage heatmap
\item Battery utilization per UAV (bar charts)
\item OODA loop execution timeline with phase indicators
\item Operator alerts and escalation log
\item Live telemetry streaming at 2 Hz update rate
\end{itemize}

\textbf{Post-mission analysis:}
\begin{itemize}
\item Coverage percentage over time
\item Adaptation time distribution (histogram)
\item Task priority vs. allocation success (scatter plot)
\item Battery efficiency heatmap by UAV
\item JSON-formatted experiment results for reproducibility
\end{itemize}

\chapter{Experimental Results and Validation}

\section{Quantitative Performance Results}

The system was validated through five experimental scenarios comparing OODA-based fault tolerance against three baseline strategies. Results demonstrate performance significantly exceeding initial expectations.

\subsection{Executive Summary}

Table~\ref{tab:executive_summary} provides a comprehensive overview of all experimental scenarios, enabling quick comparison across mission types.

\begin{table}[H]
\centering
\caption{Executive Summary of Experimental Results}
\label{tab:executive_summary}
\small
\begin{tabular}{|l|c|c|c|c|c|}
\hline
\textbf{Scenario} & \textbf{Mission} & \textbf{Coverage} & \textbf{Time} & \textbf{Safety} & \textbf{Outcome} \\
 & \textbf{Type} & \textbf{Recovery} & \textbf{(ms)} & \textbf{Violations} & \\
\hline
\textbf{S5} & Surveillance & 100\% & 0.34 & 0 & Full Autonomous \\
\hline
\textbf{R5} & SAR & 100\% & 0.50 & 0 & Full Autonomous \\
\hline
\textbf{R6} & SAR (OOG) & 100\% & 0.16 & 0 & Full Autonomous \\
\hline
\textbf{D6} & Delivery & 0\% (esc.) & 0.11 & 0 & Intelligent Escalation \\
\hline
\textbf{D7} & Delivery & 0\% (esc.) & 0.23 & 0 & Intelligent Escalation \\
\hline
\multicolumn{6}{|l|}{\textit{OOG = Out-of-Grid; esc. = escalated to operator}} \\
\hline
\end{tabular}
\end{table}

\begin{figure}[H]
\centering
\includegraphics[width=0.9\textwidth]{images/chapter6_decision_flow.png}
\caption{OODA Decision Flow in Experimental Scenarios}
\label{fig:ch6_decision_flow}
\end{figure}

\begin{figure}[H]
\centering
\includegraphics[width=0.9\textwidth]{images/chapter6_performance_radar.png}
\caption{Multi-Dimensional Performance Comparison}
\label{fig:ch6_radar}
\end{figure}

\subsection{Detailed Performance Metrics}

The measured OODA adaptation timings reveal performance characteristics that substantially exceed initial design expectations. The surveillance scenario (S5) completed its entire OODA cycle in 0.34 milliseconds, achieving a response 14,706 times faster than the conservative five-second target established during system design. Search and rescue operations demonstrated comparable efficiency, with scenario R5 completing in 0.50 milliseconds and the out-of-grid variant R6 achieving the fastest measured response at 0.16 milliseconds. Even the delivery scenarios requiring constraint violation detection and operator escalation (D6 and D7) completed their analysis in 0.11 and 0.23 milliseconds respectively, demonstrating that safety-critical constraint checking imposes negligible computational overhead.

Coverage recovery patterns across mission types illuminate the system's adaptive capabilities under realistic constraints. Both surveillance and search-and-rescue missions achieved complete 100\% task recovery following single-UAV failures, substantially exceeding the 75-95\% target range established during design. This outcome validates the priority-based reallocation strategy's effectiveness when sufficient spare capacity exists within the operational fleet. The delivery scenarios presented a contrasting yet equally important result: zero percent autonomous reallocation accompanied by intelligent escalation to human operators. This apparent absence of autonomous recovery represents not system failure but rather successful constraint violation detection---the system correctly identified physically infeasible reallocations and appropriately deferred to human judgment rather than compromising safety boundaries.

The speed advantage over manual operator intervention proves particularly striking when examined quantitatively. While the OODA system completes its full decision cycle in 0.16 to 0.50 milliseconds on average, manual operator replanning requires approximately 465 seconds (7.75 minutes) to achieve equivalent task reallocation. This represents a performance multiplication factor approaching 500,000, transforming response latency from minutes to sub-millisecond timescales. For time-critical search and rescue operations conducted under golden-hour constraints, this 7.7-minute advantage can directly influence mission success probability and, in life-threatening scenarios, determine whether intervention arrives within the window of survivability.

Safety validation across all experimental scenarios demonstrates perfect constraint adherence. The OODA system maintained zero constraint violations throughout all five test scenarios, respecting battery reserves, payload limits, and operational boundaries without exception. In stark contrast, the greedy baseline algorithm---which prioritizes coverage maximization without explicit constraint verification---produced two distinct safety violations: a payload overload in scenario D6 where package weight exceeded available capacity, and a boundary transgression in scenario D7 where destination coordinates fell outside permitted flight zones. Manual operator intervention similarly achieved zero violations but at the cost of unacceptable 7.75-minute response latency, reinforcing the value proposition of automated constraint-aware decision-making that combines both speed and safety.

\begin{figure}[H]
\centering
\includegraphics[width=0.9\textwidth]{images/chapter6_time_comparison.png}
\caption{Computation Time Comparison Across Approaches (Log Scale)}
\label{fig:ch6_time}
\end{figure}

\begin{figure}[H]
\centering
\includegraphics[width=0.85\textwidth]{images/chapter6_safety_violations.png}
\caption{Safety Validation: Constraint Violations by Approach}
\label{fig:ch6_safety}
\end{figure}

\section{Experimental Scenario Analysis}

\subsection{S5: Surveillance Mission Recovery}

The surveillance scenario (S5) evaluated system response to a single UAV failure during sustained perimeter monitoring operations. This experiment compared four distinct operational strategies to isolate the contribution of constraint-aware adaptive planning. The no-adaptation baseline, which simply aborted affected tasks without reallocation, achieved only 87.5\% coverage---effectively declaring mission failure upon detecting the fault. The greedy nearest-neighbor heuristic recovered full 100\% coverage in 0.18 milliseconds by assigning orphaned tasks to the spatially nearest available vehicles, though this approach succeeded only because the particular failure configuration happened not to violate capacity constraints. Manual operator intervention similarly achieved complete recovery but required 465 seconds of human cognitive processing to assess fleet state, evaluate alternatives, and formulate the reallocation plan.

The OODA system achieved 100\% coverage recovery in 0.34 milliseconds while maintaining perfect safety through explicit constraint verification at each allocation step. This represents a response 1,367,000 times faster than manual replanning---a performance advantage that transforms fault recovery from a multi-minute operational disruption into an essentially instantaneous adaptation invisible to mission progress. More significantly, the OODA approach guaranteed constraint satisfaction through sequential battery, payload, and collision verification, ensuring that the recovered mission plan remained executable rather than merely optimal on paper. All battery reserves remained above the 20\% safety threshold, spatial separation exceeded the 15-meter minimum, and no vehicle received task assignments beyond its physical capabilities.

\subsection{R5 \& R6: Search \& Rescue with Time Criticality}

The search and rescue scenarios (R5 and R6) examined system performance under the most stringent temporal constraints encountered in civilian UAV operations: the sixty-minute golden hour during which victim survival probability remains highest. Scenario R5 simulated a standard UAV failure during systematic grid search operations, where the no-adaptation baseline abandoned 33.3\% of the search area---effectively leaving one-third of high-probability zones unsearched and potentially condemning an injured person to prolonged exposure. The OODA system recovered complete 100\% coverage in 0.50 milliseconds, consuming a negligible 0.0000138\% of the precious golden hour interval. Manual operator response achieved equivalent coverage but required 465 seconds (7.75 minutes), consuming 12.9\% of the time-critical window. In scenarios where every minute directly correlates with survival probability, this 7.7-minute differential represents not merely a performance metric but potentially the margin between life and death.

Scenario R6 introduced an additional complexity by positioning one high-priority search zone ten meters outside the nominal operational grid boundaries [0-1000, 0-1000]. This configuration tested the system's regulatory compliance mechanisms: could it correctly identify permitted vehicles for out-of-bounds operations while refusing allocation to unauthorized units? The experiment validated this capability comprehensively. With UAV-4 having received explicit out-of-grid authorization from the operator prior to mission commencement, the OODA system correctly reallocated the exterior zone to this permitted vehicle while excluding unauthorized units from consideration. The decision completed in 0.16 milliseconds---the fastest measured across all experimental scenarios---demonstrating that permission verification and regulatory constraint checking impose minimal computational burden. This result confirms that safety-first design principles need not compromise response speed when implemented through efficient data structures and sequential constraint evaluation.

\begin{figure}[H]
\centering
\includegraphics[width=\textwidth]{images/chapter6_golden_hour.png}
\caption{Search \& Rescue: Golden Hour Time Consumption}
\label{fig:ch6_golden_hour}
\end{figure}

\subsection{D6 \& D7: Delivery with Intelligent Escalation}

The delivery scenarios (D6 and D7) presented perhaps the most philosophically significant experimental results by validating the system's capacity to recognize its own limitations and appropriately defer to human judgment. Scenario D6 configured a deliberately infeasible task reallocation: following primary delivery vehicle failure, the stranded Package B weighed 2.0 kilograms while the maximum spare payload capacity across all operational vehicles reached only 0.7 kilograms. This represents a fundamental physical impossibility---no combination of available resources could legally transport the package without exceeding structural load limits.

The contrasting responses across different approaches illuminate critical distinctions in system design philosophy. The greedy baseline algorithm reported 100\% coverage recovery by simply assigning the package to the nearest available vehicle, producing a mission plan that would overload the selected UAV by nearly three times its spare capacity---a configuration guaranteed to cause either structural damage or flight instability. Manual operator analysis correctly identified the infeasibility, reporting zero percent autonomous coverage and acknowledging the need for alternative solutions such as backup vehicle deployment. The OODA system matched this correct assessment, completing constraint analysis in 0.11 milliseconds before escalating to operator intervention with explicit explanation: "Package B (2.0 kg) exceeds maximum available spare capacity (0.7 kg). Recommend backup UAV deployment or ground vehicle dispatch."

Scenario D7 examined boundary constraint violations by positioning a delivery destination at coordinates (3500, 2500) meters---well outside the authorized operational grid spanning [0-3000, 0-2000]. Unlike scenario R6 where a permitted vehicle existed for out-of-bounds operations, no UAV in scenario D7 carried the necessary authorization. The greedy algorithm again reported perfect coverage while violating the spatial boundary constraint. The OODA system refused this unsafe allocation, completing analysis in 0.23 milliseconds before escalating with regulatory justification: "Destination outside authorized flight zone. No vehicle possesses out-of-grid permission. Operator intervention required for boundary waiver or mission modification."

These results validate a crucial principle often overlooked in autonomous systems research: intelligent refusal represents successful operation, not system failure. The OODA system's zero percent autonomous reallocation in physically or regulatorily infeasible scenarios demonstrates correct constraint detection and appropriate escalation rather than algorithmic inadequacy. A system that claims 100\% autonomy by violating safety constraints provides less operational value than one that achieves 60\% autonomous coverage while correctly identifying when human judgment becomes necessary. This honest acknowledgment of limitations distinguishes deployable systems from theoretical frameworks that assume constraints away.

\begin{figure}[H]
\centering
\includegraphics[width=0.9\textwidth]{images/chapter6_constraint_space.png}
\caption{D6 Payload Constraint Violation: Geometric Illustration of Escalation Necessity}
\label{fig:ch6_constraint}
\end{figure}

\begin{figure}[H]
\centering
\includegraphics[width=0.9\textwidth]{images/chapter6_coverage_heatmap.png}
\caption{Coverage Recovery Matrix Across All Approaches and Scenarios}
\label{fig:ch6_coverage}
\end{figure}

\section{Comparative Analysis}

Table~\ref{tab:scenario_comparison} synthesizes performance characteristics across all experimental scenarios, revealing patterns that transcend individual mission contexts. The data demonstrates remarkable consistency in computational efficiency despite substantial variation in problem complexity: all five scenarios completed their OODA cycles within the sub-millisecond regime spanning 0.11 to 0.50 milliseconds. This narrow performance band confirms that algorithmic complexity scales gracefully with problem size within the tested operational envelope of three to four vehicles and five to eight tasks.

\begin{table}[H]
\centering
\caption{Measured Performance by Scenario (S5, R5/R6, D6/D7)}
\label{tab:scenario_comparison}
\begin{tabular}{|l|c|c|c|}
\hline
\textbf{Metric} & \textbf{Surveillance (S5)} & \textbf{SAR (R5/R6)} & \textbf{Delivery (D6/D7)} \\
\hline
Coverage Recovery & \textbf{100\%} & \textbf{100\%} & \textbf{0\% (escalated)} \\
\hline
OODA Computation & \textbf{0.34 ms} & \textbf{0.50/0.16 ms} & \textbf{0.11/0.23 ms} \\
\hline
Operator Escalation & 0\% (autonomous) & 0\% (autonomous) & 100\% (correct) \\
\hline
Constraint Violations & \textbf{0} & \textbf{0} & \textbf{0} \\
\hline
Mission Outcome & 100\% complete & 100\% complete & Escalated (safe) \\
\hline
\end{tabular}
\end{table}

Several cross-scenario patterns emerge from this data that illuminate fundamental system characteristics. First, computational speed remains uniformly sub-millisecond across all mission types, with the full range from fastest (0.11 ms) to slowest (0.50 ms) spanning less than half a millisecond. This consistency validates the architectural decision to employ greedy heuristics with linear-time constraint checking rather than optimization algorithms whose execution time might vary unpredictably with problem structure. Second, safety performance proves absolute rather than statistical: zero constraint violations occurred across all five scenarios despite widely varying constraint types (battery, payload, spatial boundaries) and failure conditions. This perfect safety record distinguishes the constraint-aware approach from opportunistic algorithms that might occasionally produce valid solutions but lack systematic safety guarantees.

Third, the system demonstrates appropriate domain-specific adaptation through its coverage recovery patterns: complete 100\% recovery in surveillance and search-and-rescue scenarios where sufficient spare capacity existed, versus correct escalation in delivery scenarios where physical or regulatory constraints precluded safe autonomous reallocation. This selective autonomy---aggressive adaptation when feasible, intelligent refusal when infeasible---represents the hybrid human-machine operational paradigm essential for deploying autonomous systems within contemporary regulatory frameworks. Finally, the performance advantage over manual operator intervention remains consistent across mission types at approximately 500,000-fold, transforming multi-minute human cognitive processing into sub-millisecond machine response regardless of whether the decision involves surveillance zone reassignment, search grid reallocation, or constraint violation detection.

\section{Thesis Claims Validated}

The experimental program comprehensively validated all primary thesis claims advanced in the introduction, transforming theoretical propositions into empirically demonstrated capabilities. The claim of sub-second adaptation response received dramatic confirmation through measured execution times spanning 0.11 to 0.50 milliseconds---performance approximately 10,000 times faster than the conservative five-second design target and 500,000 times faster than manual operator baseline. This result validates both the algorithmic efficiency of the greedy constraint-aware reallocation strategy and the architectural decision to implement centralized rather than consensus-based coordination, which eliminates multi-round negotiation overhead.

The safety guarantee claim achieved perfect experimental confirmation: zero constraint violations occurred across all five scenarios encompassing battery reserve verification, payload capacity checking, and spatial boundary enforcement. This absolute safety record distinguishes the present work from approaches that treat constraints as soft preferences rather than hard limits. The intelligent escalation claim received validation through scenarios D6 and D7, where the system correctly refused physically or regulatorily infeasible reallocations rather than producing superficially complete but unexecutable mission plans. This capability to recognize operational boundaries and appropriately defer to human judgment represents a critical requirement for regulatory approval of autonomous systems in civilian airspace.

The time-critical advantage claim gained empirical support through search-and-rescue scenario analysis demonstrating 7.7-minute response advantage over manual operator intervention---a temporal margin that directly impacts victim survival probability during golden hour operations. Finally, the coverage recovery claim achieved validation through 100\% task recovery in surveillance and search-and-rescue missions, substantially exceeding the 75-95\% target range established during system design. These results collectively confirm that constraint-aware OODA-based fault tolerance delivers the rapid, safe, and intelligent adaptation essential for real-world UAV fleet operations.

\section{Technical Contributions}

The research advances three distinct technical contributions that collectively enable constraint-aware fault tolerance for multi-UAV systems. First, the constraint-aware task reallocation algorithm explicitly models the triad of operational constraints---battery reserves, payload capacity, and temporal deadlines---within a unified sequential verification framework augmented with geometric collision avoidance. While prior work addresses individual constraint types in isolation, this represents the first multi-UAV fault tolerance system to simultaneously enforce all three constraint categories through fail-fast sequential checking. The priority-based partial coverage strategy extends this foundation by providing graceful degradation mechanisms when complete recovery proves infeasible, enabling quantified mission continuation rather than binary success-failure outcomes.

Second, the quantified performance degradation framework establishes rigorous metrics for evaluating fault tolerance under realistic resource limitations. This framework calculates precise coverage percentages following vehicle failures, determines when autonomous reallocation transitions from feasible to infeasible regimes, and formulates explicit decision rules governing operator escalation. The novelty lies not in mathematical sophistication but rather in philosophical approach: honest acknowledgment that some failure configurations exceed autonomous compensation capabilities, accompanied by systematic methods for identifying such scenarios and formulating appropriate human-machine collaboration strategies. This contrasts sharply with idealized approaches that claim universal fault tolerance by implicitly assuming unlimited spare capacity.

Third, the hybrid autonomy architecture balances aggressive machine adaptation against human supervisory oversight through a clear authority hierarchy: autonomous response when constraint satisfaction permits, mandatory escalation when physical or regulatory boundaries are approached. This architectural pattern enables deployment within contemporary regulatory frameworks requiring operator-in-the-loop supervision for beyond-visual-line-of-sight operations. The practical contribution extends beyond algorithmic innovation to address the institutional and regulatory barriers that often prevent academic fault tolerance systems from transitioning to operational deployment.

\section{Practical Contributions}

Beyond theoretical advances, the research delivers tangible practical contributions that address real-world deployment requirements. The system achieves deployability through compatibility with commercial UAV platforms exemplified by the DJI Matrice 350 RTK, whose specifications informed the simulation parameters and constraint models. This design choice ensures that the developed algorithms can transition directly to physical hardware without requiring custom vehicle development---a critical pathway for technology transfer to operational users.

Regulatory compliance emerges naturally from the operator-in-the-loop architecture with intelligent escalation. By maintaining human supervisory authority over all safety-critical decisions while automating routine adaptation, the system satisfies beyond-visual-line-of-sight operational requirements under current aviation regulations. This hybrid authority model bridges the regulatory gap between fully manual control and aspirational full autonomy, providing a pragmatic intermediate point compatible with near-term certification frameworks.

The measured economic impact demonstrates substantial operational value through multiple quantified dimensions. Response speed improves by a factor of 500,000 compared to manual replanning, compressing seven-minute human cognitive processing into sub-millisecond machine execution. Operator workload experiences measurable reduction through automatic handling of sixty percent of failure scenarios (S5, R5, R6), with escalation reserved exclusively for constrained cases where human judgment provides essential value (D6, D7). Mission failure prevention achieves quantifiable benefit: whereas the no-adaptation baseline abandons 12.5 to 33.3 percent of tasks following vehicle failures, the OODA system recovers these assignments when physically feasible or escalates appropriately when infeasible. In time-critical search-and-rescue operations, the 7.7-minute response advantage over manual intervention translates directly to life-saving potential through preservation of the golden hour window.

The comprehensive test coverage comprising 169 automated tests---53 unit tests verifying individual components, 81 integration tests validating end-to-end workflows, and 15 regression tests preventing historical bugs---executes in 0.22 seconds, enabling rapid validation during development and deployment. This test infrastructure provides confidence in system reliability while facilitating future extensions and modifications.

\section{Academic Contributions}

The research contributes to academic discourse through its realism-first design philosophy that explicitly bridges the theory-practice gap plaguing much multi-agent systems research. By foregrounding rather than abstracting real-world constraints, the work demonstrates that practical deployability and theoretical rigor need not conflict. The application of OODA loop principles to constrained multi-agent coordination extends Boyd's original military framework into civilian autonomous systems, establishing a conceptual bridge between human decision-making theory and machine adaptation algorithms. Finally, the open-source simulation platform developed during this research provides future investigators with validated infrastructure for exploring fault tolerance, task allocation, and human-machine collaboration in multi-UAV systems without requiring physical hardware or flight testing facilities.

\chapter{Limitations and Future Work}

This chapter presents a critical assessment of the proposed OODA-based fault-tolerant control system, examining current limitations and future research directions. Understanding these constraints is essential for establishing realistic performance expectations and identifying opportunities for system enhancement.

\section{Architectural Constraints}

\subsection{Centralized Control Architecture}

The system employs a centralized Ground Control Station for OODA loop execution, creating a single point of failure. Should the GCS experience hardware failure or communication loss, the fleet's adaptive capabilities are compromised. Individual UAVs implement autonomous Return-to-Launch protocols upon GCS timeout detection, preserving vehicle safety while sacrificing mission completion.

Future work could explore distributed OODA architectures using consensus algorithms for peer-to-peer coordination. This approach would eliminate the centralization vulnerability while introducing new challenges including increased communication complexity, network partitioning risks, and Byzantine fault tolerance requirements. The transition to distributed coordination must carefully balance robustness against implementation complexity.

\subsection{Fleet Scalability}

The system supports fleets of three to twelve UAVs. Beyond this scale, two constraints become significant. Communication bandwidth scales linearly with fleet size, reaching 48 kilobytes per second for twelve UAVs at 2 Hz telemetry rates. Computational complexity for collision avoidance scales quadratically, as each reallocation requires pairwise verification among all vehicles. At twenty UAVs, collision checking workload increases 2.8-fold compared to twelve UAVs, potentially exceeding the six-second OODA cycle target.

Hierarchical architectures that partition large fleets into coordinated subgroups could address these limitations. Alternatively, computationally efficient approximate collision avoidance methods using spatial hashing could reduce verification complexity. These extensions would enable applications requiring coordination of dozens or hundreds of vehicles.

\subsection{Validation Methodology}

Current validation relies exclusively on software-in-the-loop simulation using physics-based models validated against DJI Matrice 350 RTK specifications. While appropriate for algorithm development, simulation abstracts real-world phenomena including GPS multipath errors, communication packet corruption, sensor noise, and environmental disturbances.

Hardware-in-the-loop testing with physical flight controllers would provide intermediate validation capturing timing constraints and communication latencies. Field trials with two to three physical UAVs would expose the system to genuine environmental challenges and enable parameter refinement. Future validation should prioritize systematic characterization of performance degradation under realistic operating conditions, establishing the operational envelope for reliable performance.

\section{Algorithmic Limitations}

\subsection{Greedy Task Reallocation}

The constraint-aware reallocation algorithm employs a greedy heuristic that assigns tasks to the nearest UAV satisfying capacity constraints. This achieves rapid execution compatible with real-time OODA requirements but does not guarantee globally optimal allocation. Early assignment decisions may consume capacity better reserved for higher-priority tasks, particularly when spare capacity is marginal and failed tasks are widely distributed.

Mixed-integer linear programming could guarantee optimal solutions for small to medium problems, though solution times may exceed OODA cycle budgets. Auction-based algorithms, particularly the Consensus-Based Bundle Algorithm, offer a promising middle ground achieving near-optimal solutions through distributed iterative bidding with polynomial-time complexity. Comparing these alternatives under diverse scenarios would quantify allocation quality trade-offs.

\subsection{Simplified Collision Avoidance}

The collision avoidance strategy maintains fifteen-meter spatial separation with temporal deconfliction when conflicts arise. This proves sufficient for sparse operational densities but exhibits limitations in dense flight patterns. The fixed safety buffer does not adapt to relative velocities, and the pairwise verification approach does not efficiently handle complex multi-vehicle conflicts.

Velocity obstacle approaches, particularly Reciprocal Velocity Obstacles, enable reactive collision avoidance accounting for relative velocities with smooth trajectory modifications. Model predictive control formulations could jointly optimize task execution and collision avoidance. These advanced methods would extend applicability to urban air mobility scenarios with higher flight densities.

\section{Application Scope}

The system addresses three mission classes: long-duration surveillance, emergency search and rescue, and medical supply delivery. These applications share waypoint-based navigation, quantifiable task priorities, and tolerance for mission degradation under resource constraints.

However, this focused scope excludes mission types with different requirements. Aggressive formation flying demands tighter coordination and higher-bandwidth communication than the current 2 Hz telemetry supports. Adversarial scenarios require game-theoretic reasoning and adversarial prediction beyond current OODA capabilities. Time-critical interception missions may need more sophisticated trajectory optimization than waypoint following provides.

Extending the system to these domains represents important future work. Formation flying could be addressed through augmented OODA loops reasoning about relative positioning constraints. Adversarial scenarios might integrate game-theoretic task allocation anticipating opponent responses. These extensions would broaden applicability while preserving core OODA principles.

\section{Future Research Directions}

\subsection{Near-Term Enhancements}

Comprehensive validation across all twenty-seven planned test scenarios would provide robust statistical characterization. Comparative evaluation against baseline strategies would quantify the value of explicit capacity modeling and priority-based allocation. Sensitivity analysis of safety reserve parameters from 5 to 20 percent battery would establish performance trade-offs between mission completion probability and safety margins. Enhanced environmental modeling incorporating turbulence, precipitation, and visibility limitations would improve prediction fidelity.

\subsection{Medium-Term Objectives}

Hardware-in-the-loop testing using PX4 Software-In-The-Loop would validate OODA cycle timing under realistic computational constraints. Field demonstrations with small fleets would reveal operational challenges including GPS accuracy limitations and radio frequency interference effects. Detailed instrumentation would generate empirical data to refine system parameters and validate simulation accuracy.

Distributed OODA architectures with consensus-based decision-making would enhance robustness against single-point failures. Implementing auction mechanisms like the Consensus-Based Bundle Algorithm would require careful attention to Byzantine fault tolerance and network partition handling. Comparative studies between centralized and distributed approaches would elucidate trade-offs between optimization quality, communication overhead, and system resilience.

Formal verification using model checking techniques could verify that OODA cycle logic maintains battery reserve constraints and collision avoidance guarantees under all reachable states. Temporal logic specifications could capture liveness properties ensuring eventual response to failures. While requiring substantial expertise, formal methods provide mathematical assurance complementing empirical testing.

\subsection{Long-Term Frontiers}

Machine learning could optimize priority weighting parameters based on historical mission outcomes, adapting scoring functions to operational contexts. Neural networks could predict battery consumption more accurately by learning vehicle-specific efficiency characteristics. Reinforcement learning might enable adaptive OODA cycle tuning based on observed performance patterns.

Game-theoretic analysis becomes essential for adversarial scenarios. Stackelberg game formulations could model hierarchical decision-making in contested surveillance missions. Nash equilibrium concepts might characterize stable operating points with competing autonomous systems. These theoretical frameworks would require substantial OODA extension incorporating opponent modeling and robust optimization.

Fleet heterogeneity introduces additional complexity. Real-world deployments increasingly employ mixed fleets with different endurance, payload capacity, and sensor suites. Addressing heterogeneity requires extending constraint verification for vehicle-specific capabilities and developing allocation strategies exploiting complementary strengths.

Large-scale swarm coordination with fifty or more vehicles demands hierarchical control architectures, efficient communication protocols avoiding broadcast storm effects, and possibly bio-inspired coordination strategies emerging from local interactions. Research at this scale intersects with complex systems theory, distributed computing, and collective intelligence.

\section{Concluding Perspective}

The limitations discussed reflect conscious design choices prioritizing practical deployability over theoretical completeness. The centralized architecture enables regulatory compliance and deterministic performance. The greedy allocation strategy trades global optimality for real-time responsiveness. The constraint-aware approach acknowledges physical limitations rather than assuming unlimited resources.

This honest assessment distinguishes the present work from research claiming comprehensive autonomy while abstracting real-world constraints. A system achieving 65 to 95 percent autonomous coverage recovery within realistic bounds provides substantially more value than theoretical frameworks promising perfect adaptation under idealized assumptions. The identified future work charts a path toward enhanced capabilities while maintaining the fundamental principle of honest, deployable autonomy.

As multi-agent UAV coordination matures, the research community must prioritize systems bridging the gap between laboratory demonstration and operational deployment. This requires explicit modeling of real-world constraints, acknowledgment of fundamental limitations, and design of hybrid human-machine systems leveraging complementary strengths of autonomous algorithms and human supervisory control. The present work contributes to this maturation by demonstrating that constraint-aware, operator-supervised fault tolerance represents the appropriate architecture for near-term UAV fleet deployments in regulated, safety-critical applications.

\chapter{Conclusion}

This work addresses the reality gap in multi-agent UAV fault tolerance by explicitly modeling real-world constraints (battery, payload, regulatory) that are often ignored in academic research. Rather than claiming perfect fault tolerance, the hybrid OODA approach provides honest, quantified mission completion assistance within realistic operational limits.

\textbf{Validated performance exceeds initial expectations.} The implemented system demonstrates sub-millisecond adaptation times (0.11--0.50 ms), approximately 10,000$\times$ faster than the initial 4--6 second design target and 500,000$\times$ faster than manual operator response (465 seconds). Experimental validation across 169 automated tests spanning unit, integration, and regression scenarios confirms 100\% coverage recovery for surveillance and search-and-rescue missions, while delivery scenarios correctly escalate to operators when payload or boundary constraints preclude autonomous reallocation. The system achieved zero constraint violations across all experimental scenarios, validating the constraint-aware design philosophy.

\textbf{Key insight:} The validated system achieves 100\% coverage recovery for surveillance and search-and-rescue missions while correctly escalating infeasible delivery scenarios to operators, demonstrating that constraint-aware autonomy with intelligent escalation is more valuable than unconstrained systems that violate safety boundaries. In time-critical search-and-rescue operations, the 7.7-minute response advantage over manual operators can determine mission success and save lives.

The three core technical contributions—constraint-aware reallocation, priority-based partial coverage, and operator escalation—work together to balance autonomous response speed with human oversight, making this approach suitable for real-world deployment under current regulations. The sub-millisecond computational performance ensures algorithmic complexity never becomes a bottleneck, while the comprehensive test suite (53 unit, 81 integration, 15 regression tests executing in 0.22 seconds) provides confidence in system reliability and correctness.

% ====================================
% REFERENCES
% ====================================
\newpage
\addcontentsline{toc}{chapter}{References}
\begin{thebibliography}{99}

\bibitem{mueller2014} 
Mueller, M. W., \& D'Andrea, R. (2014). Stability and control of a quadrocopter despite the complete loss of one, two, or three propellers. \textit{IEEE ICRA}.


\bibitem{sun2022}
Sun, Z., et al. (2022). Fault-Tolerant Model Predictive Control of a Quadrotor with an Unknown Complete Rotor Failure. \textit{IEEE ICRA}.

\bibitem{li2017}
Li, P., Yu, X., Peng, X., Zheng, Z., \& Zhang, Y. (2017). Fault-tolerant cooperative control for multiple UAVs based on sliding mode techniques. \textit{Science China Information Sciences}, 60(7).


\bibitem{yang2011}
Yang, H., Staroswiecki, M., Jiang, B., et al. (2011). Fault tolerant cooperative control for a class of nonlinear multi-agent systems. \textit{Systems \& Control Letters}, 60(4), 271-277.

\bibitem{gerkey2004}
Gerkey, B. P., \& Mataric, M. J. (2004). A formal analysis and taxonomy of task allocation in multi-robot systems. \textit{International Journal of Robotics Research}, 23(9), 939-954.

\bibitem{choi2009}
Choi, H. L., Brunet, L., \& How, J. P. (2009). Consensus-based decentralized auctions for robust task allocation. \textit{IEEE Transactions on Robotics}, 25(4), 912-926.

\bibitem{dias2006}
Dias, M. B., Zlot, R., Kalra, N., \& Stentz, A. (2006). Market-based multirobot coordination: A survey and analysis. \textit{Proceedings of the IEEE}, 94(7), 1257-1270.

\bibitem{zlot2006}
Zlot, R., \& Stentz, A. (2006). Market-based multirobot coordination for complex tasks. \textit{International Journal of Robotics Research}, 25(1), 73-101.

\bibitem{cortes2004}
Cortes, J., Martinez, S., Karatas, T., \& Bullo, F. (2004). Coverage control for mobile sensing networks. \textit{IEEE Transactions on Robotics and Automation}, 20(2), 243-255.

\bibitem{schwager2009}
Schwager, M., Rus, D., \& Slotine, J. J. (2009). Decentralized, adaptive coverage control for networked robots. \textit{International Journal of Robotics Research}, 28(3), 357-375.

\bibitem{elmaliach2009}
Elmaliach, Y., Agmon, N., \& Kaminka, G. A. (2009). Multi-robot area patrol under frequency constraints. \textit{Annals of Mathematics and Artificial Intelligence}, 57(3-4), 293-320.

\bibitem{abdessameud2011}
Abdessameud, A., \& Tayebi, A. (2011). Formation control of VTOL unmanned aerial vehicles with communication delays. \textit{Automatica}, 47(11), 2383-2394.

\bibitem{izadi2009}
Izadi, H. A., Gordon, B. W., \& Zhang, Y. M. (2009). Decentralized receding horizon control for cooperative multiple vehicles subject to communication delay. \textit{Journal of Guidance, Control, and Dynamics}, 32(6), 1959-1965.

\bibitem{izadi2013}
Izadi, H. A., Gordon, B. W., \& Zhang, Y. M. (2013). Hierarchical decentralized receding horizon control of multiple vehicles with communication failures. \textit{IEEE Transactions on Aerospace and Electronic Systems}, 49(2), 744-759.

\bibitem{beard2006}
Beard, R. W., McLain, T. W., Nelson, D. B., et al. (2006). Decentralized cooperative aerial surveillance using fixed-wing miniature UAVs. \textit{Proceedings of the IEEE}, 94(7), 1306-1324.

\bibitem{boyd1987}
Boyd, J. R. (1987). \textit{A Discourse on Winning and Losing}. [OODA Loop framework]

\bibitem{bala2025}
Bala, M., et al. (2025). The OODA Loop of Cloudlet-Based Autonomous Drones. \textit{IEEE/ACM Symposium on Edge Computing (SEC)}.

\bibitem{soares2025}
Soares, V. M. D., et al. (2025). UAV Simulation Environment for Fault Detection in Wind Farm Electrical Distribution Systems. \textit{IEEE Conference Proceedings}.

\bibitem{vandenberg2008}
van den Berg, J., Lin, M., \& Manocha, D. (2008). Reciprocal velocity obstacles for real-time multi-agent navigation. \textit{IEEE ICRA}, 1928-1935.

\bibitem{zhang2008}
Zhang, Y. M., \& Jiang, J. (2008). Bibliographical review on reconfigurable fault-tolerant control systems. \textit{Annual Reviews in Control}, 32(2), 229-252.

\bibitem{yu2015}
Yu, X., \& Jiang, J. (2015). A survey of fault-tolerant controllers based on safety-related issues. \textit{Annual Reviews in Control}, 39, 46-57.

\bibitem{parker1998}
Parker, L. E. (1998). ALLIANCE: An architecture for fault tolerant multirobot cooperation. \textit{IEEE Transactions on Robotics and Automation}, 14(2), 220-240.

\bibitem{repo}
Project Repository: \url{https://github.com/vriesz/fuse}

\bibitem{quadref}
Quad SimCon Reference: \url{https://github.com/bobzwik/Quadcopter_SimCon}

% \bibitem{guo2018}
% Guo, J., Zhang, Y., \& Li, W. (2018). Fault-tolerant control of quadrotor UAVs with actuator faults using adaptive backstepping. \textit{International Journal of Control, Automation and Systems}, 16(4), 1572-1583.

% \bibitem{wang2021}
% Wang, W., Zhang, Y., \& Xu, B. (2021). Fault and Failure Tolerant Model Predictive Control of Quadrotor UAV. \textit{IEEE ROBIO}.

% \bibitem{zhou2021}
% Zhou, B., Su, W., \& Han, J. (2021). Model predictive fault-tolerant control for quadrotor UAV subject to actuator faults. \textit{Aerospace Science and Technology}, 110, e106497.

% \bibitem{chang2024}
% Chang, Y., et al. (2024). Reinforcement Learning–Based Adaptive Fault-Tolerant Antidisturbance Control for UAVs. \textit{Journal of Aerospace Engineering}, 38(1).

% \bibitem{zhang2016}
% Zhang, X. Y., \& Duan, H. B. (2016). Altitude consensus based 3D flocking control for fixed-wing unmanned aerial vehicle swarm trajectory tracking. \textit{Journal of Aerospace Engineering}, 230(14), 2628-2638.

% \bibitem{liu2016}
% Liu, Z. X., Yuan, C., Yu, X., et al. (2016). Leader-follower formation control of unmanned aerial vehicles in the presence of obstacles and actuator faults. \textit{Unmanned Systems}, 4(3), 197-211.

% \bibitem{yu2016}
% Yu, X., Liu, Z. X., \& Zhang, Y. M. (2016). Fault-tolerant formation control of multiple UAVs in the presence of actuator faults. \textit{International Journal of Robust and Nonlinear Control}, 26(12), 2668-2685.

% \bibitem{korsah2013}
% Korsah, G. A., Stentz, A., \& Dias, M. B. (2013). A comprehensive taxonomy for multi-robot task allocation. \textit{International Journal of Robotics Research}, 32(12), 1495-1512.

% \bibitem{innocenti2004}
% Innocenti, M., Pollini, L., \& Giulietti, F. (2004). Management of communication failures in formation flight. \textit{Journal of Aerospace Computing, Information, and Communication}, 1(1), 19-35.

% \bibitem{franco2007}
% Franco, E., Parisini, T., \& Polycarpou, M. M. (2007). Design and stability analysis of cooperative receding-horizon control of linear discrete-time agents. \textit{International Journal of Robust and Nonlinear Control}, 17(10-11), 982-1001.

% Flight Formation
% \bibitem{pachter2001}
% Pachter, M., D'Azzo, J. J., \& Proud, A. W. (2001). Tight formation flight control. \textit{Journal of Guidance, Control, and Dynamics}, 24(2), 246-254.

% \bibitem{gu2006}
% Gu, Y., Seanor, B., Campa, G., et al. (2006). Design and flight testing evaluation of formation control laws. \textit{IEEE Transactions on Control Systems Technology}, 14(6), 1105-1112.

% \bibitem{lin2014}
% Lin, W. (2014). Distributed UAV formation control using differential game approach. \textit{Aerospace Science and Technology}, 35, 54-62.


\end{thebibliography}

\end{document}