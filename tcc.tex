\documentclass[12pt,a4paper,oneside]{book}

% ABNT formatting packages
\usepackage[utf8]{inputenc}
\usepackage[T1]{fontenc}
\usepackage[english]{babel}
\usepackage{indentfirst}
\usepackage{setspace}
\usepackage{graphicx}
\usepackage{float}
\usepackage[left=3cm,right=2cm,top=3cm,bottom=2cm]{geometry}
\usepackage{times}
\usepackage{caption}
\usepackage{subcaption}
\usepackage{amsmath}
\usepackage{amssymb}
\usepackage{listings}
\usepackage[table]{xcolor}
\usepackage{url}
\usepackage{hyperref}
\usepackage{titlesec}
\usepackage{tocloft}
\usepackage{longtable}
\usepackage{array}
\usepackage{booktabs}
\usepackage{fancyhdr}

% ABNT-style formatting
\onehalfspacing
\setlength{\parindent}{1.25cm}

% Chapter and section formatting (ABNT style)
\titleformat{\chapter}[display]
  {\normalfont\bfseries\fontsize{12pt}{14pt}\selectfont}
  {\MakeUppercase{\chaptertitlename\ \thechapter}}{0pt}
  {\MakeUppercase}
\titlespacing*{\chapter}{0pt}{0pt}{12pt}

% Unnumbered chapter formatting (for Acknowledgments, Abstract, etc.)
\titleformat{name=\chapter,numberless}[display]
  {\normalfont\bfseries\fontsize{12pt}{14pt}\selectfont}
  {}{0pt}
  {\MakeUppercase}
\titlespacing*{name=\chapter,numberless}{0pt}{0pt}{12pt}

\titleformat{\section}
  {\normalfont\bfseries\fontsize{12pt}{14pt}\selectfont}
  {\thesection}{1em}{}
\titlespacing*{\section}{0pt}{12pt}{6pt}

\titleformat{\subsection}
  {\normalfont\bfseries\fontsize{12pt}{14pt}\selectfont}
  {\thesubsection}{1em}{}
\titlespacing*{\subsection}{0pt}{12pt}{6pt}

\titleformat{\subsubsection}
  {\normalfont\bfseries\fontsize{12pt}{14pt}\selectfont}
  {\thesubsubsection}{1em}{}
\titlespacing*{\subsubsection}{0pt}{12pt}{6pt}

% Caption formatting (ABNT style)
\captionsetup{
  format=plain,
  labelsep=endash,
  font=small,
  labelfont=bf,
  justification=justified,
  singlelinecheck=false
}

% Code listing style
\lstset{
  language=Python,
  basicstyle=\ttfamily\footnotesize,
  keywordstyle=\color{blue},
  commentstyle=\color{green!60!black},
  stringstyle=\color{red},
  numbers=left,
  numberstyle=\tiny\color{gray},
  stepnumber=1,
  numbersep=10pt,
  backgroundcolor=\color{white},
  showspaces=false,
  showstringspaces=false,
  showtabs=false,
  frame=single,
  tabsize=2,
  captionpos=b,
  breaklines=true,
  breakatwhitespace=false,
  escapeinside={(*@}{@*)},
  xleftmargin=2em,
  framexleftmargin=1.5em
}

% Hyperref configuration
\hypersetup{
    colorlinks=true,
    linkcolor=black,
    citecolor=black,
    filecolor=black,
    urlcolor=black,
    pdftitle={Constraint-Aware Fault-Tolerant Multi-Agent UAV System Using OODA Loop},
    pdfauthor={Vítor Eulálio Reis},
}

% Page numbering configuration (top right throughout document)
\pagestyle{fancy}
\fancyhf{} % Clear all header and footer fields
\fancyhead[R]{\thepage} % Page number in top right
\renewcommand{\headrulewidth}{0pt} % Remove header rule line
\renewcommand{\footrulewidth}{0pt} % Remove footer rule line

% Apply fancy style to plain pages (chapter opening pages)
\fancypagestyle{plain}{%
    \fancyhf{} % Clear all header and footer fields
    \fancyhead[R]{\thepage} % Page number in top right
    \renewcommand{\headrulewidth}{0pt} % Remove header rule line
    \renewcommand{\footrulewidth}{0pt} % Remove footer rule line
}

\begin{document}

% ====================================
% COVER PAGE
% ====================================
\begin{titlepage}
\begin{center}
\textbf{\uppercase{Universidade de São Paulo}}\\
\textbf{\uppercase{Escola de Engenharia de São Carlos}}

\vspace{8cm}

\textbf{\uppercase{Vítor Eulálio Reis}}

\vspace{4cm}

\textbf{Constraint-Aware Fault-Tolerant Multi-Agent UAV System Using OODA Loop: Realistic Mission Completion Assistance}

\vfill

São Carlos\\
2025
\end{center}
\end{titlepage}

% ====================================
% TITLE PAGE
% ====================================
\newpage
\thispagestyle{empty}
\begin{center}
\textbf{Vítor Eulálio Reis}

\vspace{8cm}

\textbf{Constraint-Aware Fault-Tolerant Multi-Agent UAV System Using OODA Loop}

\vspace{3cm}

\begin{minipage}{8cm}
\begin{flushleft}
Monograph presented to the Specialization Course in Aeronautical Systems, School of Engineering of São Carlos, University of São Paulo, as part of the requirements for obtaining the title of Specialist.

Advisor: Prof° João Paulo Eguea, PhD
\end{flushleft}
\end{minipage}

\vspace{2cm}

FINAL VERSION

\vfill

São Carlos\\
2025
\end{center}

% ====================================
% COPYRIGHT PAGE
% ====================================
\newpage
\thispagestyle{empty}
\vspace*{10cm}
\begin{center}
I AUTHORIZE THE TOTAL OR PARTIAL REPRODUCTION OF THIS WORK,\\
BY ANY CONVENTIONAL OR ELECTRONIC MEANS, FOR STUDY\\
AND RESEARCH PURPOSES, PROVIDED THE SOURCE IS CITED.
\end{center}

\vfill

\begin{center}
\begin{minipage}{12cm}
\begin{flushleft}
Cataloging card prepared by the Library Prof° Dr. Sérgio Rodrigues Fontes at EESC/USP with data provided by the author(s).

\vspace{0.5cm}

\noindent Eulálio Reis, Vítor

\hspace{0.5cm} Constraint-Aware Fault-Tolerant Multi-Agent UAV System Using OODA Loop: Realistic Mission Completion Assistance. / Vítor Eulálio Reis; advisor Prof° João Paulo Eguea, PhD. São Carlos, 2025.

\vspace{0.5cm}

\hspace{0.5cm} Specialization (Specialization in Aeronautical Systems) -- School of Engineering of São Carlos, University of São Paulo, 2025.

\vspace{0.5cm}

\hspace{0.5cm} 1. Drone. 2. OODA Loop. 3. Fault Tolerance. 4. Multi-Agent Systems. I. Title.
\end{flushleft}
\end{minipage}
\end{center}

% ====================================
% APPROVAL PAGE
% ====================================
\newpage
\thispagestyle{empty}
\begin{center}
\textbf{\uppercase{Approval Sheet}}

\vspace{2cm}



\vspace{2cm}

\begin{tabular}{|p{13cm}|}
\hline
\textbf{Candidate / Student:} Vítor Eulálio Reis \\
\hline
\textbf{Title of TCC / Title:} Constraint-Aware Fault-Tolerant Multi-Agent UAV System Using OODA Loop \\
\hline
\textbf{Defense date / Date:} October 18, 2025 \\
\hline
\end{tabular}

\vspace{2cm}

\begin{tabular}{|p{10cm}|p{3cm}|}
\hline
\textbf{Examining Committee} & \textbf{Result} \\
\hline
{Prof° João Paulo Eguea, PhD} &  \\
\hline
\textbf{Affiliation:} School of Engineering of São Carlos / EESC-USP & \\
\hline
{Prof° Jorge Bidinotto, PhD} &  \\
\hline
\textbf{Affiliation:} School of Engineering of São Carlos / EESC-USP & \\
\hline
\end{tabular}

\vspace{2cm}

Chair of the Examining Committee:

\vspace{1cm}

\begin{center}
\rule{6cm}{0.4pt}\\
{Prof° João Paulo Eguea, PhD}\\
(Signature)
\end{center}

\end{center}

% ====================================
% DEDICATION (Optional)
% ====================================
\newpage
\thispagestyle{empty}
\vspace*{15cm}
\begin{flushright}
\begin{minipage}{8cm}
\textit{To my family, for their unconditional support throughout this endeavor.}
\end{minipage}
\end{flushright}

% ====================================
% ACKNOWLEDGMENTS
% ====================================
\newpage
\chapter*{Acknowledgments}

To the professors of the Especialização em Sistemas Aeronáuticos at the Escola de Engenharia de São Carlos, USP, for sharing their knowledge, expertise, craft, and for their dedication in supporting students throughout the course.

To Prof° Dr. João Paulo Eguea for his mentorship in guiding this research.

To Prof° Dr. Jorge Bidinotto for his leadership in managing the Specialization Program.

\vspace{0.5cm}

\noindent\textbf{AI Disclosure:} This work was developed with assistance from large language models. \textbf{Claude} (Anthropic) was used via Claude Code with both Sonnet (claude-sonnet-4-20250514) and Opus (claude-opus-4-20250514) models to support software development, experimentation and user interface design. Claude Sonnet was also used in playground mode for literature survey, reference verification, formatting, comparative analysis, and standardization. \textbf{Gemini} (Google) was used in playground mode for literature survey, linguistic refinement, and as a simulated peer reviewer. \textbf{ChatGPT} (OpenAI) was used in playground mode as a simulated peer reviewer and evaluator. All technical decisions, system architecture, algorithm design, experimental validation, and intellectual contributions remain solely the author's responsibility.

% ====================================
% ABSTRACT
% ====================================
\newpage
\chapter*{Abstract}

\noindent EULÁLIO REIS, V. \textbf{Constraint-Aware Fault-Tolerant Multi-Agent UAV System Using OODA Loop: Realistic Mission Completion Assistance}. 2025. Monograph (Specialization) – School of Engineering of São Carlos, University of São Paulo, São Carlos, 2025.

\vspace{0.5cm}

The failure of a UAV during mission execution creates an immediate resource allocation challenge: reassigning orphaned tasks to operational vehicles while respecting battery limitations, payload capacity, and regulatory constraints that academic research often ignores. This work develops a fault-tolerant control system for multi-UAV fleets that addresses this challenge under realistic operational conditions.

The system applies Boyd's OODA loop (Observe-Orient-Decide-Act) to transform vehicle failures into recoverable events. Upon detecting a fault, the system evaluates remaining fleet capacity, prioritizes orphaned tasks, and reallocates them to healthy vehicles while enforcing safety margins. When physical or regulatory limitations make full recovery impossible, the system escalates to human operators with quantified impact assessments rather than forcing unsafe autonomous decisions.

Three technical contributions emerge from this work: a resource-aware reallocation algorithm that jointly considers battery reserves, payload limits, and collision avoidance; a priority-based partial coverage strategy that gracefully degrades mission objectives when complete recovery is infeasible; and an intelligent escalation framework that distinguishes compensable failures from those requiring human judgment.

Experimental validation demonstrates adaptation times of 0.2-1.2 milliseconds for the complete OODA cycle---nearly four orders of magnitude faster than the design target---with complete task recovery in surveillance and search-and-rescue scenarios. In time-critical applications, this speed advantage translates to preserved lives during the golden hour. The approach prioritizes honest performance over inflated claims: acknowledging what autonomous systems cannot do proves as valuable as demonstrating what they can.

\vspace{0.5cm}

\noindent \textbf{Keywords:} Drone. Fault Tolerance. OODA Loop. Multi-Agent Systems. Mission Planning.

% ====================================
% LIST OF FIGURES
% ====================================
\newpage
\listoffigures

% ====================================
% LIST OF TABLES
% ====================================
\newpage
\listoftables

% ====================================
% LIST OF ABBREVIATIONS
% ====================================
\newpage
\chapter*{List of Abbreviations and Acronyms}

\begin{tabular}{ll}
6-DOF & Six Degrees of Freedom \\
BVLOS & Beyond Visual Line of Sight \\
FSM & Finite State Machine \\
GCS & Ground Control Station \\
GPS & Global Positioning System \\
JSON-RPC & JavaScript Object Notation -- Remote Procedure Call \\
LKP & Last Known Position \\
MILP & Mixed Integer Linear Programming \\
OODA & Observe-Orient-Decide-Act \\
OOG & Out-of-Grid \\
PID & Proportional-Integral-Derivative \\
POA & Probability of Area \\
PX4 SITL & PX4 Software-In-The-Loop \\
RF & Radio Frequency \\
RTL & Return-to-Launch \\
RVO & Reciprocal Velocity Obstacles \\
SAR & Search and Rescue \\
TCP/IP & Transmission Control Protocol/Internet Protocol \\
TSP & Traveling Salesman Problem \\
UAV & Unmanned Aerial Vehicle \\
\end{tabular}

% ====================================
% TABLE OF CONTENTS
% ====================================
\newpage
\tableofcontents

% ====================================
% MAIN CONTENT
% ====================================
\newpage
\setcounter{page}{1}
\pagenumbering{arabic}

\chapter{Introduction}

\section{Motivation and Problem Statement}

Multi-UAV coordination has emerged as a key enabler of scalable autonomy in logistics, environmental monitoring, and emergency response. Yet much fault-tolerant multi-agent research still operates within idealized boundaries (ZHANG; JIANG, 2008; YU; JIANG, 2015), presuming unlimited energy, unrestricted payload capacity, instantaneous communication, and fully autonomous decision authority—assumptions that mask real-world complexities.

In practice, UAV operations face strict physical and regulatory limits: batteries must retain 10–20\% safety reserves, payloads deplete irreversibly mid-mission, communication links introduce latencies up to two seconds, and beyond visual line of sight operations require constant operator supervision. The resulting ``reality gap'' means systems reliable in laboratory conditions may struggle in the field. Closing this gap requires architectures that adapt to uncertainty and resource constraints while maintaining situational awareness and regulatory compliance.

The OODA loop—Observe, Orient, Decide, Act—originated from Colonel John Boyd's studies of aerial combat dynamics (BOYD, 1987), proposing that success depends on processing information and adapting faster than adversaries. For autonomous systems, it provides a recursive cycle: ``Observe'' collects sensor and telemetry data; ``Orient'' establishes situational awareness; ``Decide'' selects actions under constraints; and ``Act'' executes decisions while feeding outcomes back into subsequent observations. Applied to UAV fleets, OODA extends beyond traditional feedback control by integrating situational reasoning with awareness of resource states, communication health, and mission progress. Most existing coordination systems emphasize either rapid reaction or long-term planning, but rarely both (BALA et al., 2025).

Despite its theoretical appeal, the OODA framework has rarely been implemented as a real-time operational control system for UAV fleets. Most prior work either focuses on single-vehicle fault recovery (MUELLER; D'ANDREA, 2014; SUN et al., 2022) or treats multi-agent coordination under ideal conditions with unlimited resources (LI et al., 2017; YANG et al., 2011). Architectures for fault-tolerant multi-robot cooperation such as ALLIANCE (PARKER, 1998) have demonstrated resilience principles, but few systems integrate real-world resource constraints, probabilistic fault detection, and operator-in-the-loop supervision within a unified architecture.

\section{Proposed Solution and Contributions}

This research addresses that gap through the development of a hybrid OODA-based fault-tolerant mission control system designed for real-time UAV fleet management. The system operates as a hybrid finite-state machine (FSM) that continuously cycles between monitoring and adaptation modes, combining deterministic state transitions with probabilistic failure identification to maintain robust mission execution under realistic constraints. The principal contributions are:

\textbf{Hybrid OODA–FSM Architecture.} At its core, the system functions as a hybrid finite-state controller implementing continuous monitoring at 2 Hz and invoking the OODA cycle upon failure detection. Deterministic state logic ensures predictable mission flow, while probabilistic reasoning handles uncertain or delayed telemetry data.

\textbf{Real-Time Failure Detection.} The system implements multi-modal failure detection through 2 Hz telemetry polling, feeding three detection pathways: communication timeout alerts for telemetry gaps exceeding 1.5 seconds, explicit fault codes from onboard diagnostics, and statistical anomaly detection for subtle degradation patterns (battery discharge >5\%/30s, position jumps >100m, altitude violations). This fusion achieves sub-second fault identification while minimizing false positives.

\textbf{OODA Execution for Fault Recovery.} Upon fault detection, the system executes a structured OODA cycle completing in 4 to 5.5 seconds. The Observe phase (1.0–1.5s) aggregates fleet telemetry and quantifies mission impact. The Orient phase (1.0–1.5s) evaluates remaining capacity—battery reserves, payload availability, temporal constraints—and re-prioritizes orphaned tasks. The Decide phase (1.0–1.5s) classifies feasibility: full reallocation when >75\% of tasks are recoverable, partial reallocation for 50–75\%, or operator escalation below 50\%. The Act phase (0.5–1.5s) dispatches commands and updates the operator dashboard.

\textbf{Constraint-Aware Strategy Layer.} Three recovery strategies address different failure severities. \emph{Full Reallocation} engages when the fleet retains sufficient capacity to absorb all orphaned tasks, executing a constraint-aware optimization routine that integrates new waypoints while minimizing travel distance through heuristic TSP solutions. \emph{Partial Reallocation} prioritizes tasks exceeding a 0.7 priority threshold while issuing coverage-gap alerts quantifying the impact. \emph{Operator Escalation} activates when autonomous decision-making would produce unacceptable outcomes, initiating a supervised decision protocol with a 30-second countdown that defaults to the safest available action if no operator input arrives.

\textbf{Performance and Scalability.} The complete OODA cycle achieves reaction times between 4 and 5.5 seconds under realistic communication latency conditions. Computationally, the system scales linearly with fleet size for monitoring and resource evaluation, and quadratically for collision avoidance checks—supporting real-time performance for up to 12 UAVs without exceeding sub-6-second response thresholds.

\section{Research Objectives}

The general objective is to design, implement, and validate a constraint-aware fault-tolerant multi-agent coordination framework for UAV systems, employing the OODA loop paradigm to enable mission continuity under operational constraints. This addresses the methodological gap between theoretical fault tolerance and deployable autonomous systems through explicit modeling of energy limitations, payload capacity, regulatory boundaries, and human-in-the-loop decision authority. The specific objectives are:

\textbf{SO1: Hybrid OODA-FSM Architecture Design.}
To design and implement a centralized OODA Loop Engine integrated with a deterministic finite-state machine for continuous fleet monitoring, enabling formally verifiable state transitions with probabilistic failure identification while maintaining BVLOS regulatory compliance.

\textbf{SO2: Multi-Modal Real-Time Failure Detection.}
To develop fault detection mechanisms at 2 Hz telemetry frequency with three complementary pathways: (i) communication timeout detection (>1.5s threshold), (ii) explicit fault code recognition from onboard diagnostics, and (iii) statistical anomaly detection for gradual degradation (battery discharge >5\%/30s, position discontinuities >100m, altitude violations).

\textbf{SO3: Constraint-Aware Task Reallocation.}
To formalize priority-based allocation algorithms optimizing across energy reserves (20\% margin), payload capacity, temporal deadlines, and collision avoidance (15m buffer), employing sequential fail-fast constraint verification to guarantee physically realizable mission plans.

\textbf{SO4: Operator Escalation Framework.}
To establish formal decision rules distinguishing autonomously compensable failures from those requiring human judgment: operator escalation for coverage recovery below 50\%, partial reallocation for 50--75\% recovery, and autonomous full reallocation above 75\%, preserving operator authority over safety-critical determinations.

\textbf{SO5: Statistical Validation Across Mission Typologies.}
To conduct statistical validation (N=30 trials per condition) across three mission types---surveillance, search-and-rescue, and delivery---quantifying coverage recovery, adaptation latency, constraint violations, and escalation appropriateness relative to baseline and greedy strategies.

\textbf{SO6: Real-Time Response Performance.}
To demonstrate OODA cycle execution within 6-second latency (OBSERVE <500ms, ORIENT <500ms, DECIDE <1200ms, ACT <300ms), ensuring negligible computational overhead in time-critical contexts such as golden-hour search operations.

\textbf{SO7: Safety-Critical Constraint Satisfaction.}
To validate zero safety constraint violations across all trials regarding battery reserves, payload limits, spatial separation, and regulatory compliance, demonstrating that the architecture maintains safety invariants without compromising coverage objectives.

\section{Document Organization}

This monograph is organized as follows. Chapter~\ref{ch:literature} reviews theoretical foundations and prior work, identifying the research gap. Chapter~3 presents the system architecture, detailing the centralized OODA engine and distributed execution model. Chapter~4 describes the core algorithms, including constraint-aware reallocation. Chapter~5 introduces the three mission scenarios used for validation. Chapter~6 details experimental methodology. Chapter~7 presents results and statistical validation. Chapter~8 discusses limitations and future work. Chapter~9 concludes with a summary of contributions.

By grounding UAV mission control in the OODA decision cycle, this research introduces a system capable of adaptive, explainable, and regulatorily compliant autonomy—bridging the gap between theoretical multi-agent fault tolerance and real-world deployability.

\chapter{Literature Review}
\label{ch:literature}

This chapter examines prior work relevant to fault-tolerant multi-UAV coordination under operational constraints. The review covers fault tolerance at the vehicle level (Section~\ref{sec:lit-single-uav}), multi-agent coordination architectures (Section~\ref{sec:lit-multi-agent}), task allocation algorithms (Section~\ref{sec:lit-task-allocation}), coverage control (Section~\ref{sec:lit-coverage}), communication-aware coordination (Section~\ref{sec:lit-communication}), and the OODA loop in autonomous systems (Section~\ref{sec:lit-ooda}). Section~\ref{sec:lit-gap} synthesizes these threads to identify the research gap.

\section{Fault Tolerance in Single Unmanned Aerial Vehicles}
\label{sec:lit-single-uav}

Vehicle-level fault tolerance has achieved substantial maturity through control-theoretic approaches. Mueller and D'Andrea \cite{mueller2014} demonstrated that quadrotors can maintain controlled flight despite loss of one to three propellers by exploiting remaining actuator authority through control reallocation, achieving stable hovering with a single propeller albeit with reduced maneuverability.

Sun et al. \cite{sun2022} extended this to scenarios where the failed rotor is unknown a priori, using model predictive control (MPC) to estimate failure configurations online and adapt allocation accordingly, achieving stable recovery within 2.3 seconds.

\textbf{Critical Assessment:} These contributions establish that individual vehicles can survive actuator failures. However, both assume (i) an operator or system to issue high-level commands post-recovery, (ii) no coordination with other vehicles, and (iii) unlimited recovery time. In fleet operations where failed vehicles hold assigned tasks and battery constraints limit duration, vehicle-level recovery alone proves insufficient.

\section{Multi-Agent Coordination Architectures}
\label{sec:lit-multi-agent}

Parker's \cite{parker1998} ALLIANCE architecture represents foundational work in fault-tolerant multi-robot cooperation, employing behavior-based control with motivational dynamics: each robot maintains motivation levels for available tasks, with impatience parameters causing robots to assume tasks when teammates fail. Experimental validation demonstrated graceful degradation under robot failures.

\textbf{Limitations:} ALLIANCE assumes tasks are interchangeable among robots and does not model consumable resources. The motivational dynamics operate on implicit timeouts rather than explicit failure detection, introducing response latency.

Li et al. \cite{li2017} developed sliding-mode cooperative control for multiple UAVs, proving Lyapunov stability for the fleet under bounded actuator faults. Yang et al. \cite{yang2011} generalized this to nonlinear multi-agent systems, establishing theoretical fault tolerance through adaptive control laws that compensate for unknown fault magnitudes.

\textbf{Critical Assessment:} Both provide rigorous stability proofs but assume continuous communication, known dynamics, and no resource constraints. Battery depletion, payload consumption, and regulatory boundaries do not appear in their formulations.

\section{Task Allocation Under Resource Constraints}
\label{sec:lit-task-allocation}

Gerkey and Matarić \cite{gerkey2004} established the canonical taxonomy for multi-robot task allocation (MRTA), classifying problems along three dimensions: single-task versus multi-task robots, single-robot versus multi-robot tasks, and instantaneous versus time-extended assignment. Their complexity analysis proved that most MRTA variants are NP-hard, motivating heuristic approaches.

The Consensus-Based Bundle Algorithm (CBBA) of Choi et al. \cite{choi2009} enables decentralized task allocation through iterated auctions, with robots bidding on task bundles and a consensus phase resolving conflicts. CBBA guarantees convergence within $O(NM)$ communication rounds for $N$ robots and $M$ tasks.

Market-based coordination, surveyed by Dias et al. \cite{dias2006}, offers scalable allocation through economic mechanisms. Zlot and Stentz \cite{zlot2006} extended market approaches to complex tasks, demonstrating allocation among 80 simulated robots.

\textbf{Critical Assessment:} These algorithms optimize allocation efficiency but model robot capabilities as static. Dias et al. note that ``incorporating resource consumption and task deadlines remains an open challenge.'' CBBA can theoretically encode battery constraints, but most applications assume fixed capabilities throughout the mission.

\section{Coverage Control and Spatial Coordination}
\label{sec:lit-coverage}

Voronoi-based coverage control, formalized by Cortés et al. \cite{cortes2004}, partitions the operational area among robots such that each is responsible for the region closest to its position, with gradient-descent convergence to locally optimal configurations.

Schwager et al. \cite{schwager2009} extended this to adaptive coverage, learning density functions online from sensor measurements with a decentralized algorithm requiring only local neighbor information.

Elmaliach et al. \cite{elmaliach2009} addressed multi-robot patrol under frequency constraints with cyclic strategies guaranteeing visit intervals for convex environments.

\textbf{Critical Assessment:} Coverage algorithms assume continuous operation without resource depletion. When robots return to base for battery or payload replenishment, coverage guarantees break. None address mid-mission robot loss or coverage gap quantification.

\section{Communication-Aware Multi-Vehicle Control}
\label{sec:lit-communication}

Communication constraints fundamentally affect coordination feasibility. Abdessameud and Tayebi \cite{abdessameud2011} analyzed formation control for VTOL vehicles under communication delays, proving stability for delays below thresholds dependent on formation geometry.

Izadi et al. \cite{izadi2009} developed decentralized receding horizon control for cooperative vehicles with communication delays, partitioning fleets into subgroups with designated leaders. Izadi et al. \cite{izadi2013} extended this to communication failures, showing hierarchical architectures maintain stability despite intermittent link losses.

Beard et al. \cite{beard2006} presented one of the few complete systems integrating cooperative surveillance with communication constraints, deploying fixed-wing UAVs for decentralized target tracking with prioritized information sharing.

\textbf{Critical Assessment:} These works establish that coordination remains possible under degraded communication but assume eventual recovery or tolerance for permanent link losses. The interaction between communication failures and resource constraints---where delayed coordination may exhaust battery before updated commands arrive---remains unaddressed.

\section{The OODA Loop in Autonomous Systems}
\label{sec:lit-ooda}

Boyd's \cite{boyd1987} OODA loop---Observe, Orient, Decide, Act---originated from analysis of aerial combat, arguing that faster cycling through this loop yields decisive advantage. The framework emphasizes ``Orient'' as the critical phase where observations are filtered through experience to create situational awareness.

Bala et al. \cite{bala2025} applied OODA to cloudlet-based autonomous drones, mapping phases to edge computing primitives and demonstrating 340ms average decision latency for obstacle avoidance.

\textbf{Critical Assessment:} Bala et al. demonstrate OODA's applicability to single-drone decisions but do not address fleet-level coordination, failure recovery across vehicles, or constraint satisfaction. The phases map to individual operations rather than fleet-wide situational awareness.

\section{Research Gap Analysis}
\label{sec:lit-gap}

Table~\ref{tab:literature-gap} synthesizes the coverage of key concerns across the reviewed literature.

\begin{table}[htbp]
\centering
\caption{Coverage of operational concerns across reviewed literature.}
\label{tab:literature-gap}
\small
\begin{tabular}{lcccccc}
\toprule
\textbf{Work} & \textbf{Fleet} & \textbf{Battery} & \textbf{Payload} & \textbf{Comm.} & \textbf{Failure} & \textbf{Operator} \\
 & \textbf{Coord.} & & & \textbf{Delay} & \textbf{Recov.} & \textbf{Loop} \\
\midrule
Mueller \& D'Andrea (2014) & -- & -- & -- & -- & Vehicle & -- \\
Sun et al. (2022) & -- & -- & -- & -- & Vehicle & -- \\
Parker (1998) & \checkmark & -- & -- & -- & Implicit & -- \\
Li et al. (2017) & \checkmark & -- & -- & -- & Theoretical & -- \\
Gerkey \& Matarić (2004) & \checkmark & -- & -- & -- & -- & -- \\
Choi et al. (2009) & \checkmark & -- & -- & \checkmark & -- & -- \\
Cortés et al. (2004) & \checkmark & -- & -- & -- & -- & -- \\
Izadi et al. (2013) & \checkmark & -- & -- & \checkmark & Comm. & -- \\
Bala et al. (2025) & -- & -- & -- & \checkmark & -- & -- \\
\textbf{This Work} & \checkmark & \checkmark & \checkmark & \checkmark & \checkmark & \checkmark \\
\bottomrule
\end{tabular}
\end{table}

The reviewed literature reveals three distinct research trajectories that have not converged:

\begin{enumerate}
    \item \textbf{Vehicle-level fault tolerance} achieves robust recovery for individual UAVs but ignores fleet-level mission impact.
    \item \textbf{Multi-agent coordination} demonstrates scalable allocation and coverage but assumes static capabilities and unlimited resources.
    \item \textbf{Communication-aware control} addresses delays and link failures but does not integrate resource depletion or mission-level recovery.
\end{enumerate}

No prior work integrates:
\begin{itemize}
    \item Real-time failure detection with explicit latency bounds
    \item Constraint-aware task reallocation respecting battery reserves, payload capacity, and regulatory boundaries
    \item Graduated operator escalation distinguishing autonomously compensable failures from those requiring human judgment
    \item Coverage gap quantification enabling informed operator decisions
\end{itemize}

This research addresses that gap through a hybrid OODA-based architecture that unifies rapid failure detection, constraint-aware reallocation, and operator-in-the-loop supervision within a formally structured decision cycle.

\chapter{System Architecture}

\section{Centralized OODA Architecture with Distributed Execution}

The system implements a hierarchical control structure: the Ground Control Station hosts the OODA Loop Engine for centralized decision-making while UAVs execute tasks autonomously. This architecture balances global fleet visibility for optimization against distributed execution for responsiveness. Figure~\ref{fig:architecture} presents this layered architecture, illustrating information flow between the GCS decision engine, communication infrastructure, and UAV fleet---separating centralized planning (OODA loop processing fleet-wide telemetry for task allocation) from decentralized execution (vehicles maintaining local autonomy for navigation and obstacle avoidance).

\begin{figure}[h!tbp]
    \centering
    \includegraphics[width=\textwidth, height=0.9\textheight, keepaspectratio]{images/architecture.png}
    \caption{Architecture Overview}
    \label{fig:architecture}
\end{figure}

\subsection{Design Rationale and Trade-offs}

The centralized architecture derives from convergent factors: BVLOS regulatory compliance mandates operator oversight, global fleet visibility enables superior optimization versus distributed consensus, and single-point decision-making eliminates complex inter-UAV coordination protocols. The hybrid autonomy model recognizes scenarios where complete failure compensation proves physically impossible, maintaining safety through human-in-the-loop oversight.

Architectural trade-offs require mitigation. The GCS single point of failure is addressed through autonomous Return-to-Launch triggered by communication timeout. Communication bandwidth scales linearly with fleet size, though 2 Hz telemetry maintains manageable overhead for up to twelve UAVs. The bidirectional architecture exhibits asymmetric data flow: telemetry uplink (~2 KB) exceeds command downlink (~1 KB).

The 2 Hz telemetry rate reflects practical deployment constraints. Commercial systems typically employ 1-2 Hz; higher frequencies increase bandwidth and packet collision probability without proportional benefit given 3-6 second OODA response latency. This ensures field deployability while maintaining responsive fault detection.

\section{OODA Loop Execution Flow}

The OODA loop implements continuous monitoring and reactive decision-making through four sequential phases. \textbf{Statistical validation (N=30 runs per scenario) demonstrates algorithmic computation times of 0.1--0.9 ms (mean)}, with 95\% confidence intervals confirming sub-millisecond performance. This ensures algorithmic processing never becomes the bottleneck, leaving the 4-6 second end-to-end budget for communication latency.

\subsection{Computation Time vs. End-to-End System Latency}

A key distinction exists between \textbf{OODA algorithm computation time} and \textbf{total system response latency}. The 4-6 second design target specifies end-to-end response including communication; sub-millisecond measurements represent pure algorithmic computation. These are complementary metrics:

\textbf{OODA computation time (measured):} The complete OODA cycle---encompassing all four phases with failure detection, capacity analysis, constraint validation, and reallocation optimization---executes in 0.2 to 1.2 milliseconds. The core greedy allocation (Algorithm 2) executes in 0.1-0.3 ms. This represents pure computation in software-in-the-loop simulation without network delays, validating tractable algorithmic complexity for real-time deployment.

\textbf{End-to-end system latency (design target):} The 4-6 second estimate accounts for bidirectional RF communication delays (ABDESSAMEUD; TAYEBI, 2011; IZADI; GORDON; ZHANG, 2009): 0.5-1.0s uplink for failure notification, 0.5-1.0s downlink for command dispatch, telemetry aggregation at 2 Hz (up to 0.5s), and acknowledgment verification (0.2s). In field deployment:

\begin{equation}
T_{total} = T_{uplink} + T_{compute} + T_{downlink} + T_{ack} \approx 1.0 + 0.001 + 1.0 + 0.2 = 2.2 \text{ seconds}
\end{equation}

This 2-3 second response is faster than the conservative 4-6 second estimate, with time consumption dominated by RF propagation rather than computation. Millisecond-scale algorithmic performance ensures computation never becomes the limiting factor.

\textbf{Practical implication:} Real-world deployments experience 2-3 second adaptation times dominated by communication latency. Network optimization provides greater latency reduction potential than algorithmic speedup. The 2-3 second response validates OODA for time-critical applications such as search and rescue.

Figure~\ref{fig:ooda_loop} illustrates the OODA loop execution flow. Each Act phase feeds telemetry back into Observe, creating continuous feedback that maintains situational awareness. Failure events trigger immediate cycle activation.

\begin{figure}[h!tbp]
    \centering
    \includegraphics[width=\textwidth, height=0.9\textheight, keepaspectratio]{images/ooda_loop.png}
    \caption{OODA Loop Execution Flow}
    \label{fig:ooda_loop}
\end{figure}

\subsection{OBSERVE Phase: Failure Detection and State Aggregation}

The OBSERVE phase (1.0-1.5s budget) establishes situational awareness through multi-modal failure detection: timeout detection for telemetry gaps >1.5s, explicit fault messages from UAV systems, and statistical anomaly detection (battery discharge >5\%/30s, position jumps >100m, altitude violations). Upon failure confirmation, the system aggregates fleet state from all operational UAVs, identifies failed vehicle last-known state, enumerates lost tasks with priorities and deadlines, and calculates mission impact percentage.

\subsection{ORIENT Phase: Situation Assessment and Capacity Analysis}

The ORIENT phase (1.0-1.5s budget) transforms observations into actionable intelligence, drawing on coverage control theory for mobile sensing networks (CORTÉS et al., 2004; SCHWAGER; RUS; SLOTINE, 2009). Mission impact evaluation quantifies coverage loss, affected zones, and deadline pressure. Fleet capacity analysis inventories spare resources: battery capacity (subtracting committed energy and 15-20\% safety reserves, yielding ~3min flight per 10\% spare), payload capacity for delivery missions, and temporal margins relative to deadlines. Task prioritization applies Algorithm 1 to generate 0-1 priority scores from temporal urgency, mission criticality, and spatial cost. Feasibility classification: feasible (>75\% tasks reallocable, >90\% completion), partial (50-75\% reallocable, 75-90\% completion), or infeasible (<50\% reallocable, <75\% completion).

\subsection{DECIDE Phase: Strategic Planning and Algorithmic Optimization}

The DECIDE phase (1.0-1.5s budget) selects strategies via three-tier hierarchy. \textbf{Full Reallocation:} Executes Algorithm 2 for constraint-aware assignment to nearest UAVs with collision avoidance, optimizing path integration for 90-100\% completion. \textbf{Partial Reallocation:} Filters tasks with priority >0.7, allocates high-priority tasks first, generates coverage gap alerts with operator recommendations for 75-90\% completion. \textbf{Operator Escalation:} Generates critical alerts (urgency: high if coverage <50\% or critical tasks lost, medium if 50-75\%, low if >75\%), presents alternatives (backup UAV, mission abort, degraded coverage), implements 30s countdown timer with automatic safe-default execution.

\subsection{ACT Phase: Command Execution and System Update}

The ACT phase (0.5-1.5s budget) dispatches mission updates to affected UAVs (new waypoints, task assignments, collision parameters) via 2 Hz uplink with 200ms acknowledgment verification and 3-attempt retry logic. Dashboard updates display fleet status, mission progress, coverage heatmaps, and alerts. Performance logging captures failure details, phase durations, reallocation results, and escalation status. Following completion, the system returns to continuous 2 Hz monitoring, supporting cascading failure handling through iterative OODA cycles.

\section{Sequential Constraint Validation Process}

The constraint validation process implements fail-fast sequential checking, enabling early termination when fundamental limitations preclude autonomous compensation. Figure~\ref{fig:constraints} presents the decision flowchart: each constraint category serves as a gate before subsequent checks proceed. The ordering---battery first, payload second, temporal last---reflects criticality and computational efficiency of early rejection.

\begin{figure}[h!tbp]
    \centering
    \includegraphics[width=\textwidth, height=0.9\textheight, keepaspectratio]{images/constraints_checking.png}
    \caption{Constraint Checking Process}
    \label{fig:constraints}
\end{figure}


\subsection{Battery Constraint Verification}

Battery constraint evaluation serves as the primary gate due to safety-criticality. For each lost task, the system calculates Euclidean distance from candidate UAVs, determines required battery via efficiency factors, and computes spare capacity as current charge minus committed energy and 15--20\% reserves. Pass: every lost task finds a candidate UAV within spare capacity. Fail: any task unreachable without violating safety margins triggers operator escalation.

\subsection{Payload Constraint Verification}

Payload constraint evaluation (conditional on battery satisfaction) applies to cargo operations. Spare payload equals maximum capacity minus current load. Pass: all payload-requiring tasks find adequate capacity. Fail: removes payload-heavy tasks while reallocating feasible ones. Payload constraints are hard physical limits---mid-air transfers are impossible, constraining reallocation to base station cargo swaps.

\subsection{Time Constraint Verification}

Time constraint evaluation (final validation) verifies tasks complete before deadlines via cumulative calculation: transit + execution + current task remainder. Pass: 90--100\% completion plans. Fail: partial reallocation prioritizing deadline urgency, accepting delays for low-priority tasks (75--90\% completion). Final positioning reflects flexibility---violations produce mission degradation rather than safety risks.

\subsection{Design Rationale}

Sequential checking provides: (1) early exit efficiency, saving ~60\% computation when battery constraints fail; (2) clear failure attribution for diagnostics; (3) prioritized ordering---safety-critical battery first, physical payload second, flexible time last.

\section{Communication Sequence and Information Flow}

The communication sequence and pipeline architecture demonstrate end-to-end fault response, transforming sensor measurements into executable commands through OODA processing.

\begin{figure}[h!tbp]
    \centering
    \includegraphics[width=\textwidth, height=0.9\textheight, keepaspectratio]{images/sequence_diagram.png}
    \caption{Mission Execution Sequence}
    \label{fig:sequence}
\end{figure}

\subsection{Temporal Execution Analysis}

Figure~\ref{fig:sequence} illustrates communication flow during failure events. At time T, a UAV fails, transmitting a fault message (0.4-1.0s latency). OBSERVE confirms and aggregates fleet state by T+1.5s. ORIENT completes impact assessment by T+2.5s. DECIDE generates reallocation plans by T+3.5s. ACT dispatches commands by T+4.0s, establishing a 4-second response timeline.

\subsection{Information Pipeline Architecture}

\begin{figure}[H]
\centering
\includegraphics[width=0.9\textwidth]{images/processing_pipeline.png}
\caption{Data Flow Pipeline}
\label{fig:pipeline}
\end{figure}

The data flow architecture (Figure~\ref{fig:pipeline}) implements unidirectional transformation with clear phase boundaries. The input layer ingests telemetry at 2 Hz and static mission definitions. The processing layer transforms sequentially: OBSERVE produces fleet states and failure lists, ORIENT generates impact assessments, DECIDE creates strategies and commands, ACT executes and logs. The output layer distributes commands to UAVs, alerts to operators, and logs for analysis.

This unidirectional pattern provides traceability, testability through phase isolation, and maintainability. The feedback loop manifests as ACT commands update UAV states, with telemetry flowing back into OBSERVE in subsequent cycles.

\subsection{Scalability Considerations}

Complexity analysis: OBSERVE and ORIENT scale linearly with fleet size. DECIDE exhibits O(N$\times$M) for task allocation across N UAVs and M tasks, with collision avoidance scaling O(N$^2$) for pairwise verification. Expected execution times: 5-6 seconds for twelve-UAV fleets, 8-10 seconds for twenty-UAV fleets, potentially exceeding acceptable thresholds for larger deployments.

\section{Scenario-Specific Adaptations}

The architecture demonstrates flexibility through scenario-specific workflow patterns while maintaining consistent OODA mechanics. Three mission types illustrate adaptation to varying requirements and constraint priorities (Figures~\ref{fig:surveillance}, \ref{fig:sar}, \ref{fig:delivery}).

\begin{figure}[h!tbp]
    \centering
    \includegraphics[width=\textwidth, height=0.9\textheight, keepaspectratio]{images/surveillance_loop.png}
    \caption{Surveillance Mission OODA Workflow}
    \label{fig:surveillance}
\end{figure}

\textbf{Surveillance missions} (Figure~\ref{fig:surveillance}) emphasize continuous area coverage with failures requiring coverage gap minimization. The workflow depicts a 120m $\times$ 120m perimeter divided into 9 zones (40m $\times$ 40m) monitored by 6 UAVs with 30-minute endurance. Priorities differentiate critical zones ($P=0.9$ for top row) from routine areas ($P=0.4$ for bottom row). Battery limitations dominate constraint considerations. When UAV-3 experiences battery anomaly (8\% vs.\ expected 55\%), the OODA cycle evaluates spare capacity (UAV-2: 15\%, UAV-4: 12\%) and executes split-zone reallocation---Zone 5 divides between UAV-2 and UAV-4, achieving 100\% coverage in 0.7 milliseconds.

\begin{figure}[H]
    \centering
    \includegraphics[width=\textwidth, height=0.9\textheight, keepaspectratio]{images/sar_loop.png}
    \caption{Search and Rescue Mission OODA Workflow}
    \label{fig:sar}
\end{figure}

\textbf{Search and rescue missions} (Figure~\ref{fig:sar}) emphasize time-critical coverage through systematic grid search under golden hour constraints. The 120m $\times$ 120m area with 9 zones is searched by 4 thermal-equipped UAVs. Zone priorities reflect distance from last known position: $P=0.9$ (closest), $P=0.6$ (corridors), $P=0.4$ (remote). Battery and time constraints dominate. When UAV-2 loses GPS in a ravine, the system reallocates Zone 3 to UAV-1 and Zone 4 to UAV-3, achieving 100\% coverage in 1.2 milliseconds. Target: 75-90\% coverage acceptable when full compensation proves infeasible.

\begin{figure}[h!tbp]
    \centering
    \includegraphics[width=\textwidth, height=0.9\textheight, keepaspectratio]{images/delivery_loop.png}
    \caption{Delivery Mission OODA Workflow}
    \label{fig:delivery}
\end{figure}

\textbf{Delivery missions} (Figure~\ref{fig:delivery}) emphasize reliable cargo transport with payload-aware reallocation. The heterogeneous 3-UAV fleet---UAV-1 (1.0 kg capacity), UAV-2 and UAV-3 (0.5 kg each)---delivers 5 packages (0.2-0.5 kg). Payload and battery constraints dominate. When UAV-1 experiences battery anomaly while carrying 0.9 kg, Package B (0.4 kg) exceeds spare capacity (UAV-2: 0.05 kg, UAV-3: 0.15 kg), triggering operator escalation. The OODA cycle completes in 0.2 milliseconds, preserving 80\% autonomous coverage while deferring the infeasible task to human judgment.

These variations demonstrate architectural flexibility through consistent OODA mechanics with mission-specific constraint emphasis.

\chapter{Core Algorithms and Technical Contributions}

\section{Priority-Based Task Scoring Algorithm}

Task reallocation following UAV failure constitutes a resource-constrained scheduling problem under uncertainty, wherein orphaned tasks must be systematically evaluated to determine allocation precedence. This evaluation reconciles three potentially competing objectives: temporal urgency imposed by mission deadlines, intrinsic task criticality derived from domain-specific requirements, and the resource expenditure necessary to execute reallocation. Algorithm~1 formalizes this multi-objective prioritization through a weighted scoring function that synthesizes these dimensions into a unified, comparable metric.

Before presenting the algorithm, we formally define the key data structures that encode the system state and operational context.

\textbf{Definition 1 (Fleet State).} The fleet state $\mathcal{F}$ is a tuple $\mathcal{F} = (\mathcal{U}, \mathcal{U}_{\text{healthy}}, \mathcal{U}_{\text{failed}})$ where:
\begin{itemize}
\item $\mathcal{U} = \{u_1, u_2, \ldots, u_n\}$ is the set of all UAVs in the fleet
\item $\mathcal{U}_{\text{healthy}} \subseteq \mathcal{U}$ is the subset of operational UAVs available for task assignment
\item $\mathcal{U}_{\text{failed}} \subseteq \mathcal{U}$ is the subset of UAVs that have experienced failures
\end{itemize}
Each UAV $u \in \mathcal{U}$ is characterized by a state vector:
\begin{equation}
u = (\text{pos}, \text{battery}, \text{committed}, \text{payload}, \text{speed}, \eta, \text{status}, \text{permissions})
\end{equation}
where $\text{pos} \in \mathbb{R}^3$ is the current position, $\text{battery} \in [0, 100]$ is the remaining charge percentage, $\text{committed} \in [0, 100]$ is the battery percentage already allocated to pending tasks, $\text{payload} \in \mathbb{R}_{\geq 0}$ is the current cargo weight, $\text{speed} \in \mathbb{R}_{> 0}$ is the cruise velocity, $\eta \in \mathbb{R}_{> 0}$ is the energy efficiency (meters per percent battery), $\text{status} \in \{\text{idle}, \text{active}, \text{failed}, \text{returning}\}$ is the operational state, and $\text{permissions}$ encodes regulatory authorizations (e.g., out-of-grid flight).

\textbf{Definition 2 (Mission Context).} The mission context $\mathcal{M}$ encapsulates the operational parameters and constraints:
\begin{equation}
\mathcal{M} = (\text{type}, \mathcal{T}, \mathcal{G}, W, \Theta)
\end{equation}
where:
\begin{itemize}
\item $\text{type} \in \{\text{surveillance}, \text{SAR}, \text{delivery}\}$ specifies the mission category
\item $\mathcal{T} = \{t_1, t_2, \ldots, t_k\}$ is the set of tasks, where each task $t_i = (\text{pos}, \text{type}, t_i^{\text{start}}, t_i^{\text{deadline}}, \text{priority}_{\text{base}}, \text{payload})$ contains position, type, start time, deadline, base priority, and payload weight
\item $\mathcal{G} \subset \mathbb{R}^2$ defines the authorized operational grid boundaries, with $R_{\max} = \text{diag}(\mathcal{G})$ denoting the maximum operational range (grid diagonal, e.g., $\sqrt{120^2 + 120^2} \approx 170$m)
\item $W: \text{type} \times \text{task\_type} \to [0, 1]$ is the criticality weight lookup table
\item $\Theta = (B_{\text{reserve}}, D_{\text{safe}}, T_{\text{safe}})$ contains safety thresholds: battery reserve (20\%), spatial separation (15m), and temporal buffer (10s)
\end{itemize}

These formal definitions establish the mathematical foundation for the algorithms that follow, ensuring precise specification of inputs and enabling rigorous analysis of algorithmic behavior.

Additionally, we define the clamping function used throughout the algorithms:
\begin{equation}
\text{clamp}(x, a, b) = \max(a, \min(b, x))
\end{equation}
which constrains $x$ to the interval $[a, b]$. When written as $\text{clamp}(x)$, the default interval $[0, 1]$ is assumed.

\begin{table}[H]
\centering
\begin{tabular}{|p{12cm}|}
\hline
\textbf{Algorithm 1: Task Priority Calculation} \\
\hline
\textbf{Input:} Task $t_i$ with position, type, and deadline; fleet state $\mathcal{F}$; mission context $\mathcal{M}$ \\
\textbf{Output:} Priority score $P_i \in [0, 1]$ \\[6pt]
\textbf{1.} Compute temporal urgency: $\tau \gets 1 - \dfrac{t_i^{\text{deadline}} - t_{\text{now}}}{t_i^{\text{deadline}} - t_i^{\text{start}}}$ \\[6pt]
\textbf{2.} Look up mission criticality weight: $c \gets W_{\mathcal{M}.\text{type}}[t_i.\text{type}]$ \\[6pt]
\textbf{3.} Find minimum distance to healthy UAV: $d_{\min} \gets \min_{u \in \mathcal{F}.\text{healthy}} \| u.\text{pos} - t_i.\text{pos} \|$ \\[6pt]
\textbf{4.} Normalize spatial cost: $\sigma \gets d_{\min} \,/\, R_{\max}$ \\[6pt]
\textbf{5.} Combine components: $P_i \gets w_\tau \cdot \text{clamp}(\tau) + w_c \cdot c - w_\sigma \cdot \text{clamp}(\sigma)$ \\[6pt]
\textbf{6.} \textbf{return} $\text{clamp}(P_i, 0, 1)$ \\
\hline
\end{tabular}
\end{table}

Each component of the scoring function addresses a distinct aspect of the prioritization problem. The temporal urgency term $\tau$ increases monotonically as deadlines approach, ensuring time-critical objectives receive elevated priority. The mission criticality weight $c$ encodes domain-specific importance hierarchies---critical packages receive maximum weighting (1.0) in delivery missions while routine packages score 0.4; in search and rescue, high-probability victim zones supersede peripheral patrols. The spatial cost term $\sigma$ penalizes tasks distant from available UAVs, implicitly incorporating battery expenditure into the priority calculus.

The default weight configuration ($w_\tau = 0.3$, $w_c = 0.5$, $w_\sigma = 0.2$) emphasizes mission criticality as the primary determinant while maintaining responsiveness to temporal constraints; these parameters admit mission-specific customization as detailed in Section~4.4. This formulation extends auction-based allocation methods (CHOI; BRUNET; HOW, 2009) and utility-theoretic coordination (DIAS et al., 2006; ZLOT; STENTZ, 2006), with the principal distinction being the explicit integration of resource costs within the priority calculus rather than their treatment as post-hoc constraints. The theoretical foundations derive from the ST-SR-IA problem class in Gerkey and Matarić's (2004) taxonomy of multi-robot task allocation.

\section{Constraint-Aware Task Reallocation with Collision Avoidance}

With tasks ranked by Algorithm~1, the subsequent challenge involves constructing feasible assignments to the operational fleet while satisfying interdependent constraints. This task allocation problem constitutes an instance of the generalized assignment problem, known to be NP-hard, rendering exact optimization computationally intractable for real-time fault recovery where decision latency directly impacts mission outcomes.

Algorithm~2 resolves this complexity-feasibility tension through a greedy strategy that processes tasks in descending priority order, assigning each to the nearest constraint-satisfying UAV. This approach guarantees constraint satisfaction by construction while achieving deterministic polynomial-time execution---a trade-off justified by the temporal criticality of fault recovery, where a promptly executed suboptimal allocation provides greater operational value than an optimal solution computed after deadlines have elapsed.

\begin{table}[H]
\centering
\begin{tabular}{|p{12cm}|}
\hline
\textbf{Algorithm 2: Constraint-Aware Task Reallocation} \\
\hline
\textbf{Input:} Failed UAV's task queue $\mathcal{T}$; healthy UAVs $\mathcal{U}$; mission context $\mathcal{M}$ \\
\textbf{Output:} Allocation $\mathcal{A}$; coverage percentage; operator alerts \\[6pt]
\textbf{1.} \textbf{for each} $u \in \mathcal{U}$ \textbf{do} \\
\quad Compute spare battery: $b_u \gets u.\text{battery} - B_{\text{reserve}} - u.\text{committed}$ \\
\quad Compute spare payload: $p_u \gets u.\text{max\_payload} - u.\text{current\_payload}$ \\[4pt]
\textbf{2.} $\mathcal{T}_{\text{ranked}} \gets \textsc{RankByPriority}(\mathcal{T}, \mathcal{U}, \mathcal{M})$ \hfill \textit{// Algorithm 1} \\[4pt]
\textbf{3.} $\mathcal{A} \gets \emptyset$; \quad $\mathcal{T}_{\text{unalloc}} \gets \emptyset$ \\[4pt]
\textbf{4.} \textbf{for each} $(t_i, P_i) \in \mathcal{T}_{\text{ranked}}$ \textbf{do} \\
\quad Sort candidates by distance: $\mathcal{U}_{\text{sorted}} \gets \text{sort}(\mathcal{U}, \text{key}=\|u.\text{pos} - t_i.\text{pos}\|)$ \\
\quad \textbf{for each} $u \in \mathcal{U}_{\text{sorted}}$ \textbf{do} \\
\quad\quad $d \gets \|u.\text{pos} - t_i.\text{pos}\|$; \quad $b_{\text{req}} \gets d \,/\, \eta_u$ \\
\quad\quad \textbf{if} $b_u < b_{\text{req}}$ \textbf{then continue} \hfill \textit{// Battery constraint} \\
\quad\quad \textbf{if} $\mathcal{M}.\text{type} = \text{delivery} \land p_u < t_i.\text{payload}$ \textbf{then continue} \hfill \textit{// Payload constraint} \\
\quad\quad \textbf{if} $\neg\textsc{CollisionFree}(u, t_i, \mathcal{A})$ \textbf{then continue} \hfill \textit{// Algorithm 3} \\
\quad\quad $\mathcal{A} \gets \mathcal{A} \cup \{(t_i, u)\}$; \quad $b_u \gets b_u - b_{\text{req}}$; \quad \textbf{break} \\
\quad \textbf{if} $t_i \notin \mathcal{A}$ \textbf{then} $\mathcal{T}_{\text{unalloc}} \gets \mathcal{T}_{\text{unalloc}} \cup \{(t_i, P_i)\}$ \\[4pt]
\textbf{5.} coverage $\gets |\mathcal{A}| \,/\, |\mathcal{T}| \times 100\%$ \\[4pt]
\textbf{6.} \textbf{return} $\mathcal{A}$, coverage, $\textsc{GenerateAlerts}(\mathcal{T}_{\text{unalloc}})$ \\
\hline
\end{tabular}
\end{table}

The algorithm enforces constraints through a hierarchical validation pipeline. Battery constraints constitute the primary filter, verifying that candidates retain sufficient energy to reach the task location while maintaining the 20\% safety margin that accounts for environmental uncertainties such as headwinds and navigation deviations. Payload constraints, activated exclusively in delivery contexts, ensure adequate lift capacity for specified cargo. Collision avoidance, delegated to Algorithm~3, evaluates spatiotemporal compatibility between proposed and committed trajectories.

\begin{table}[H]
\centering
\begin{tabular}{|p{12cm}|}
\hline
\textbf{Algorithm 3: Collision-Free Path Verification} \\
\hline
\textbf{Input:} UAV $u$; candidate task $t$; current allocation $\mathcal{A}$ \\
\textbf{Output:} Boolean indicating path safety \\[6pt]
\textbf{Constants:} $D_{\text{safe}} = 15\text{m}$ (spatial buffer); $T_{\text{safe}} = 10\text{s}$ (temporal buffer) \\[6pt]
\textbf{1.} $\pi_{\text{new}} \gets \textsc{GeneratePath}(u.\text{pos}, t.\text{pos})$ \\
\textbf{2.} $\tau_{\text{new}} \gets \textsc{EstimateTimeline}(\pi_{\text{new}}, u.\text{speed})$ \\[4pt]
\textbf{3.} \textbf{for each} $(t', u') \in \mathcal{A}$ where $u' \neq u$ \textbf{do} \\
\quad $\pi_{\text{other}} \gets \textsc{GetPlannedPath}(u')$ \\
\quad $\tau_{\text{other}} \gets \textsc{GetTimeline}(u')$ \\
\quad \textbf{for each} $(p_1, t_1) \in (\pi_{\text{new}}, \tau_{\text{new}})$ \textbf{do} \\
\quad\quad \textbf{for each} $(p_2, t_2) \in (\pi_{\text{other}}, \tau_{\text{other}})$ \textbf{do} \\
\quad\quad\quad \textbf{if} $\|p_1 - p_2\| < D_{\text{safe}} \land |t_1 - t_2| < T_{\text{safe}}$ \textbf{then return} \textsc{False} \\[4pt]
\textbf{4.} \textbf{return} \textsc{True} \\
\hline
\end{tabular}
\end{table}

Algorithm~3 implements spatiotemporal conflict detection derived from velocity obstacle methods (VAN DEN BERG; LIN; MANOCHA, 2008), simplified for centralized planning where global path information obviates distributed negotiation. Conflicts arise when trajectories bring UAVs within 15 meters---approximately three wingspans---during overlapping time windows, reflecting conservative separation standards for low-altitude operations. Upon conflict detection, the candidate assignment is rejected and the next-nearest UAV evaluated; when no conflict-free assignment exists, temporal deconfliction through departure delay scheduling provides an alternative resolution.

\section{Operator Escalation Decision Rules}

A distinguishing feature of this architecture is its capacity for honest self-assessment. Rather than forcing autonomous solutions in all circumstances, the system explicitly recognizes when human judgment is required. Algorithm~4 encodes the escalation logic as a hierarchical decision tree that evaluates mission degradation severity.

\begin{table}[H]
\centering
\begin{tabular}{|p{12cm}|}
\hline
\textbf{Algorithm 4: Operator Escalation Decision} \\
\hline
\textbf{Input:} Coverage percentage $\rho$; unallocated tasks $\mathcal{T}_{\text{unalloc}}$ with priorities \\
\textbf{Output:} Escalation decision with urgency level and recommendation \\[6pt]
\textbf{1.} \textbf{if} $\rho < 50\%$ \textbf{then} \\
\quad \textbf{return} (escalate=\textsc{True}, urgency=\textsc{High}, \\
\quad\quad reason=``Coverage below 50\% threshold'', \\
\quad\quad recommendation=``Deploy backup UAV or abort mission'') \\[4pt]
\textbf{2.} $n_{\text{critical}} \gets |\{t \in \mathcal{T}_{\text{unalloc}} : P_t > 0.7\}|$ \\
\quad \textbf{if} $n_{\text{critical}} > 0$ \textbf{then} \\
\quad \textbf{return} (escalate=\textsc{True}, urgency=\textsc{High}, \\
\quad\quad reason=``$n_{\text{critical}}$ critical tasks unassignable'', \\
\quad\quad recommendation=``Manual prioritization required'') \\[4pt]
\textbf{3.} \textbf{if} $\rho < 75\%$ \textbf{then} \\
\quad \textbf{return} (escalate=\textsc{True}, urgency=\textsc{Medium}, \\
\quad\quad reason=``Moderate degradation (50--75\% coverage)'', \\
\quad\quad recommendation=``Monitor; consider manual reallocation'') \\[4pt]
\textbf{4.} \textbf{return} (escalate=\textsc{False}, urgency=\textsc{Low}, \\
\quad reason=``Acceptable autonomous compensation ($>$75\%)'', \\
\quad recommendation=``Continue autonomous operation'') \\
\hline
\end{tabular}
\end{table}

The escalation hierarchy reflects operational risk tolerance. Coverage below 50\% indicates that the failure has exceeded the fleet's compensatory capacity—continuing autonomously would produce unacceptable mission outcomes. The presence of any unallocated high-priority task ($P > 0.7$) similarly triggers immediate escalation, as these tasks represent mission-critical objectives that cannot be sacrificed without explicit human authorization.

Moderate degradation (50--75\% coverage) warrants operator awareness without demanding immediate intervention, allowing human judgment to assess whether the degraded outcome remains acceptable for the specific operational context. Above 75\% coverage, the system proceeds autonomously, having demonstrated sufficient capacity to absorb the failure's impact.

This graduated response ensures that operator attention is reserved for genuinely consequential decisions while routine failures are handled without interrupting mission flow.

\section{Objective Function and Optimization Strategy}

The preceding algorithms define \textit{how} tasks are prioritized and allocated, but do not explicitly characterize \textit{what constitutes a good allocation}. This section formalizes the objective function that guides the DECIDE phase and describes the optimization strategy employed to maximize decision quality within real-time constraints.

\subsection{Allocation Quality Metric}

The quality of a task reallocation is quantified through an objective function $J(\mathcal{A})$ that the optimizer seeks to maximize. The optimization problem can be formally stated as:

\textbf{Problem Statement (Task Reallocation Optimization).} Given a set of orphaned tasks $\mathcal{T}_{\text{failed}} = \{t_1, \ldots, t_k\}$ from failed UAVs and a set of healthy UAVs $\mathcal{U}_{\text{healthy}} = \{u_1, \ldots, u_m\}$, find an allocation $\mathcal{A}^* \subseteq \mathcal{T}_{\text{failed}} \times \mathcal{U}_{\text{healthy}}$ that maximizes mission value while satisfying operational constraints.

\textbf{Decision Variable.} The optimization variable is the allocation $\mathcal{A} = \{(t_i, u_j)\}$, a set of task-UAV pairs indicating that task $t_i$ is assigned to UAV $u_j$. Each task may be assigned to at most one UAV (or remain unallocated if no feasible assignment exists).

\textbf{Objective Function.} The allocation quality is quantified as:

\begin{equation}
\mathcal{A}^* = \arg\max_{\mathcal{A}} J(\mathcal{A}) = \arg\max_{\mathcal{A}} \left\{ \sum_{(t_i, u_j) \in \mathcal{A}} \left[ P_i \cdot \phi_m(t_i, u_j) \right] - \lambda \cdot |\mathcal{T}_{\text{unalloc}}| \right\}
\label{eq:objective}
\end{equation}

\noindent where:
\begin{itemize}
\item $P_i \in [0,1]$ is the priority score of task $t_i$ computed by Algorithm 1
\item $\phi_m(t_i, u_j) \in [0,1]$ is a mission-specific modifier function (defined in Equation~\ref{eq:modifier})
\item $\lambda > 0$ is the penalty weight for unallocated tasks
\item $\mathcal{T}_{\text{unalloc}} = \mathcal{T}_{\text{failed}} \setminus \{t_i : (t_i, u_j) \in \mathcal{A}\}$ is the set of tasks that could not be feasibly assigned
\end{itemize}

\textbf{Constraint Set.} The allocation $\mathcal{A}$ must satisfy the following constraints for each assignment $(t_i, u_j) \in \mathcal{A}$:

\begin{enumerate}
\item \textbf{Battery constraint:} The UAV must retain sufficient energy to reach the task and return safely:
\begin{equation}
b_{u_j} - \frac{\|u_j.\text{pos} - t_i.\text{pos}\|}{\eta_{u_j}} \geq B_{\text{reserve}}
\end{equation}
where $b_{u_j}$ is the current battery level, $\eta_{u_j}$ is energy efficiency, and $B_{\text{reserve}} = 20\%$ is the safety margin.

\item \textbf{Payload constraint} (delivery missions only): The UAV must have sufficient spare capacity:
\begin{equation}
p_{u_j}^{\text{max}} - p_{u_j}^{\text{current}} \geq t_i.\text{payload}
\end{equation}

\item \textbf{Spatial constraint:} The task location must be within authorized boundaries or the UAV must hold appropriate permissions:
\begin{equation}
t_i.\text{pos} \in \mathcal{G} \quad \lor \quad \text{``out-of-grid''} \in u_j.\text{permissions}
\end{equation}

\item \textbf{Collision avoidance constraint:} The proposed trajectory must maintain safe separation from all other UAVs (verified by Algorithm 3).

\item \textbf{Uniqueness constraint:} Each task is assigned to at most one UAV:
\begin{equation}
\forall t_i \in \mathcal{T}_{\text{failed}}: |\{u_j : (t_i, u_j) \in \mathcal{A}\}| \leq 1
\end{equation}
\end{enumerate}

\textbf{Solution Method.} This constrained optimization problem is an instance of the generalized assignment problem, which is NP-hard. Given the real-time requirement (solution within 1.0--1.5 seconds), exact methods such as mixed-integer linear programming (MILP) or branch-and-bound are computationally infeasible. The system instead employs a \textbf{greedy heuristic with local search refinement} (Algorithm 5), which guarantees constraint satisfaction by construction and achieves 70--85\% of optimal objective value in sub-millisecond computation time. This trade-off prioritizes operational responsiveness over theoretical optimality---a suboptimal solution computed immediately provides greater mission value than an optimal solution computed after deadlines have elapsed.

The mission-specific modifier $\phi_m$ adjusts task value based on operational context. We define the following temporal variables:
\begin{itemize}
\item $\Delta t_{\text{gap}}(t_i)$ --- time elapsed since the last coverage of task $t_i$'s zone (normalized to $[0, 1]$)
\item $t_{\text{golden}}$ --- the golden hour duration for SAR missions (typically 60 minutes)
\item $t_{\text{completion}}(t_i, u_j)$ --- estimated completion time if task $t_i$ is assigned to UAV $u_j$
\item $t_{\text{deadline}}$ --- the delivery deadline for task $t_i$
\end{itemize}

\begin{equation}
\phi_m(t_i, u_j) =
\begin{cases}
1 - \gamma \cdot \Delta t_{\text{gap}}(t_i) & \text{if } m = \text{surveillance} \\[6pt]
1 + \beta \cdot \dfrac{t_{\text{golden}} - t_{\text{completion}}(t_i, u_j)}{t_{\text{golden}}} & \text{if } m = \text{SAR} \\[6pt]
\begin{cases}
1.0 & \text{if } t_{\text{completion}} \leq t_{\text{deadline}} \\
0.5 & \text{otherwise}
\end{cases} & \text{if } m = \text{delivery}
\end{cases}
\label{eq:modifier}
\end{equation}

For surveillance missions, the modifier penalizes coverage gaps through parameter $\gamma$, incentivizing prompt re-coverage of zones that have gone unmonitored. For search and rescue operations, tasks completed before the golden hour receive a bonus weighted by $\beta$, incentivizing rapid coverage of high-probability areas. For delivery missions, the modifier applies a binary penalty for deadline violations, reflecting the discrete nature of delivery success.

\subsection{Optimization Strategy}

The DECIDE phase operates under strict temporal constraints (1.0--1.5 seconds), precluding exact optimization methods such as mixed-integer linear programming. The system therefore employs a two-stage optimization strategy that balances solution quality against computational tractability.

\textbf{Stage 1: Greedy Initialization.} Algorithm 2 generates an initial feasible allocation in $O(n \cdot m)$ time, where $n$ is the number of failed tasks and $m$ is the number of healthy UAVs. This greedy approach processes tasks in priority order, assigning each to the nearest constraint-satisfying UAV. While not globally optimal, this initialization typically achieves 70--85\% of the theoretical maximum objective value.

\textbf{Stage 2: Local Search Refinement.} When the time budget permits (approximately 500ms remaining after Stage 1), the system applies iterative local search to improve the initial allocation. The neighborhood structure considers pairwise task swaps between UAVs and reassignment of individual tasks to alternative vehicles.

\begin{table}[H]
\centering
\begin{tabular}{|p{12cm}|}
\hline
\textbf{Algorithm 5: Two-Stage Optimization} \\
\hline
\textbf{Input:} Failed tasks $\mathcal{T}$; healthy UAVs $\mathcal{U}$; mission context $\mathcal{M}$; time budget $T_{\max}$ \\
\textbf{Output:} Optimized allocation $\mathcal{A}^*$ with objective score $J^*$ \\[6pt]
\textbf{Stage 1: Greedy Initialization} \\
\textbf{1.} $t_0 \gets \textsc{CurrentTime}()$ \\
\textbf{2.} $\mathcal{A} \gets \textsc{GreedyAllocate}(\mathcal{T}, \mathcal{U}, \mathcal{M})$ \hfill \textit{// Algorithm 2} \\
\textbf{3.} $J_{\text{best}} \gets \textsc{ComputeObjective}(\mathcal{A}, \mathcal{M})$ \\[4pt]
\textbf{Stage 2: Local Search Refinement} \\
\textbf{4.} \textbf{while} $\textsc{CurrentTime}() - t_0 < T_{\max} - T_{\text{reserve}}$ \textbf{do} \\
\quad $\text{improved} \gets \textsc{False}$ \\
\quad \textbf{for each} $(t_1, u_1), (t_2, u_2) \in \mathcal{A} \times \mathcal{A}$ where $u_1 \neq u_2$ \textbf{do} \\
\quad\quad $\mathcal{A}' \gets \mathcal{A}$ with swap: $t_1 \to u_2$, $t_2 \to u_1$ \\
\quad\quad \textbf{if} $\textsc{IsFeasible}(\mathcal{A}', \mathcal{M})$ \textbf{then} \\
\quad\quad\quad $J' \gets \textsc{ComputeObjective}(\mathcal{A}', \mathcal{M})$ \\
\quad\quad\quad \textbf{if} $J' > J_{\text{best}}$ \textbf{then} $\mathcal{A} \gets \mathcal{A}'$; $J_{\text{best}} \gets J'$; $\text{improved} \gets \textsc{True}$; \textbf{break} \\
\quad \textbf{if} $\neg\text{improved}$ \textbf{then break} \hfill \textit{// Local optimum reached} \\[4pt]
\textbf{5.} \textbf{return} $\mathcal{A}$, $J_{\text{best}}$ \\
\hline
\end{tabular}
\end{table}

The local search explores pairwise task swaps, proposing exchanges between UAVs and accepting improvements that increase the objective function while maintaining constraint feasibility. The algorithm terminates upon reaching a local optimum or exhausting the time budget, whichever occurs first. A 200ms reserve ($T_{\text{reserve}}$) ensures sufficient time for finalization and command dispatch.

\subsection{Optimality Gap Analysis}

The greedy-plus-local-search strategy does not guarantee global optimality. To characterize solution quality, we define the optimality gap as:

\begin{equation}
\text{Gap} = \frac{J^* - J(\mathcal{A}_{\text{heuristic}})}{J^*} \times 100\%
\label{eq:gap}
\end{equation}

\noindent where $J^*$ is the optimal objective value computed offline via exhaustive search or MILP for small problem instances.

Preliminary analysis across the three mission scenarios indicates expected optimality gaps of 5--15\% for typical failure cases involving 3--6 lost tasks and 4--8 healthy UAVs. This trade-off is justified by the real-time constraint: a 10\% suboptimal solution computed in 1.2 seconds provides substantially greater operational value than an optimal solution requiring 30+ seconds, during which mission degradation continues unmitigated.

\subsection{Mission-Specific Parameter Configuration}

The objective function parameters are configured per mission type to reflect operational priorities, as shown in Table~\ref{tab:objective_params}:

\begin{table}[H]
\centering
\caption{Objective Function Parameters by Mission Type}
\label{tab:objective_params}
\begin{tabular}{|l|c|c|c|c|}
\hline
\rowcolor{gray!30}
\textbf{Parameter} & \textbf{Symbol} & \textbf{Surveillance} & \textbf{SAR} & \textbf{Delivery} \\
\hline
Unallocated penalty & $\lambda$ & 0.3 & 0.5 & 0.4 \\
\rowcolor{gray!15}
Coverage gap weight & $\gamma$ & 0.2 & --- & --- \\
Golden hour bonus & $\beta$ & --- & 0.5 & --- \\
\rowcolor{gray!15}
Priority weight (temporal) & $w_{\text{temporal}}$ & 0.3 & 0.5 & 0.2 \\
Priority weight (criticality) & $w_{\text{criticality}}$ & 0.5 & 0.3 & 0.6 \\
\rowcolor{gray!15}
Priority weight (spatial) & $w_{\text{spatial}}$ & 0.2 & 0.2 & 0.2 \\
\hline
\end{tabular}
\end{table}

This parameterization enables the same OODA loop implementation to adapt its optimization behavior based on mission context, achieving domain-appropriate decision quality without requiring mission-specific algorithmic modifications.

\chapter{Mission Scenarios and Performance Analysis}

This chapter presents three mission scenarios demonstrating OODA-based fault tolerance across diverse operational contexts, each illustrating different constraint priorities, failure modes, and recovery strategies.

\section{Scenario Selection Rationale}

The three scenarios were selected to span the constraint space of real-world UAV operations:

\begin{itemize}
\item \textbf{Surveillance:} Battery-constrained with time-critical coverage requirements
\item \textbf{Search \& Rescue:} Time-critical with priority-driven partial coverage acceptance
\item \textbf{Delivery:} Payload-constrained with heterogeneous fleet capabilities
\end{itemize}

Each scenario represents a distinct operational domain with documented real-world deployment precedents.

\section{SCENARIO 1: Long-Duration Perimeter Surveillance}

Perimeter surveillance represents a canonical application of decentralized aerial monitoring, with established methodologies for coordinating multiple UAVs across extended operational periods (BEARD et al., 2006).

\subsection{Mission Context}

\begin{itemize}
    \item \textbf{Application:} Critical infrastructure monitoring (port, airport, border)
    \item \textbf{Duration:} 2-hour continuous coverage requirement
    \item \textbf{Fleet:} 6 UAVs with 30-minute flight time each (requires rotation strategy)
    \item \textbf{Operational Area:} 120m $\times$ 120m secured perimeter
    \item \textbf{Coverage Strategy:} 9 patrol zones (40m $\times$ 40m each) in a 3$\times$3 grid
\end{itemize}

\subsection{Mission Setup}

The surveillance mission partitions the 120m $\times$ 120m area into nine patrol zones (3$\times$3 grid, 40m $\times$ 40m each), ensuring complete coverage while allowing dedicated patrol circuits without inter-vehicle coordination overhead (Figure~\ref{fig:operational_environment}).

The fundamental challenge lies in the mismatch between mission duration (2 hours) and vehicle endurance (30 minutes), necessitating coordinated rotation with at least 12 UAVs in alternating shifts---transforming a simple coverage problem into a sophisticated scheduling and fault-tolerance challenge.

\begin{figure}[H]
\centering
\includegraphics[width=0.7\textwidth]{images/uav_zone_assignement.png}
\caption{Operational Environment for Experimental Scenarios}
\label{fig:operational_environment}
\end{figure}

\subsection{Patrol Zone Specifications}

Not all perimeter sections demand equal vigilance. The zone prioritization reflects threat assessment based on facility layout and historical incident patterns. The top row (Zones 1--3) along the northern perimeter guards main entry points and receives the highest priority rating ($P = 0.9$), commanding the most intensive patrol patterns. The middle row (Zones 4--6) covers central access routes with moderate priority ($P = 0.6$), while the southern boundary (Zones 7--9) monitors low-risk terrain at standard priority ($P = 0.4$).

This priority differentiation manifests in patrol circuit complexity (ELMALIACH; AGMON; KAMINKA, 2009): high-priority zones require eight waypoints per circuit, medium-priority zones operate with six, and standard-priority zones employ four. Waypoint density directly influences energy expenditure---critical during failure recovery when remaining UAVs must absorb additional coverage responsibilities.

\subsection{Rotation Strategy}

Sustaining 2-hour coverage with 30-minute endurance vehicles requires orchestrated fleet rotation. The system manages this through four consecutive waves, each lasting approximately 30 minutes, with overlapping transition periods to prevent coverage gaps.

The first wave deploys UAVs 1--6 across zones A--F at mission start. As vehicles approach the 20\% battery threshold around the 25-minute mark, they signal imminent departure, triggering deployment of the backup fleet. Wave 2 sees UAVs 7--12 assume patrol responsibilities while the primary fleet returns for battery exchange. This pattern alternates through Waves 3 and 4, with recharged vehicles cycling back into service.

The 5-minute transition overlap provides fault tolerance against timing variations, though this redundancy introduces collision avoidance complexity that the OODA system must manage alongside its primary responsibilities.

\subsection{Failure Scenario: Mid-Mission Battery Depletion}

Forty-five minutes into the mission, UAV-3's battery plummets to 8\% (expected: 35\%)---well below the safety threshold. The vehicle is halfway through Zone 5; returning immediately leaves the zone dark, while continuing risks a forced landing. This is precisely the scenario the OODA system was designed to handle.

% \begin{figure}[H]
% \centering
% \includegraphics[width=0.8\textwidth]{images/surveillance_loop.png}
% \caption{Surveillance Mission OODA Loop Execution}
% \label{fig:surveillance_loop}
% \end{figure}

\subsubsection{OBSERVE Phase (T+0 to T+1.5s)}

\textbf{Failure Detection:}

The system detects battery anomalies through statistical monitoring, comparing observed discharge rates against expected baseline values. When the discharge rate exceeds 1.5$\times$ the expected threshold, the OODA cycle triggers immediately, as quantified in Table~\ref{tab:s5_anomaly}.

\begin{table}[H]
\centering
\small
\begin{tabular}{|l|c|c|c|}
\hline
\textbf{Parameter} & \textbf{Expected} & \textbf{Observed} & \textbf{Status} \\
\hline
Discharge Rate & 3\%/min & 8\%/min & \cellcolor{red!25}ANOMALY \\
Battery Level & 55\% & 8\% & \cellcolor{red!25}CRITICAL \\
Threshold & --- & 1.5$\times$ baseline & \cellcolor{red!25}EXCEEDED \\
\hline
\end{tabular}
\caption{UAV-3 Battery Anomaly Detection}
\label{tab:s5_anomaly}
\end{table}

\textbf{Fleet State Aggregation:}

Upon detecting UAV-3's failure, the system aggregates the complete fleet state to assess available resources for task reallocation, as shown in Table~\ref{tab:s5_fleet_state}.

\begin{table}[H]
\centering
\small
\begin{tabular}{|l|c|c|c|c|}
\hline
\textbf{UAV} & \textbf{Battery} & \textbf{Zone} & \textbf{Spare Capacity} & \textbf{Status} \\
\hline
UAV-3 & 8\% & 5 & 0\% & \cellcolor{red!25}FAILED \\
UAV-2 & 45\% & 2 & 15\% & \cellcolor{green!25}AVAILABLE \\
UAV-4 & 40\% & 6 & 12\% & \cellcolor{green!25}AVAILABLE \\
UAV-5 & 80\% & 8 & 35\% & \cellcolor{blue!25}AVAILABLE \\
\hline
\multicolumn{5}{|l|}{\textit{Lost Coverage: Zone 5 (11.1\% of mission area)}} \\
\hline
\end{tabular}
\caption{Fleet State at Failure Detection (T+1.5s)}
\label{tab:s5_fleet_state}
\end{table}

\subsubsection{ORIENT Phase (T+1.5s to T+3.0s)}

\textbf{Impact Assessment:}

The system evaluates the mission impact of losing Zone 5 coverage, determining temporal urgency and required response type (Table~\ref{tab:s5_impact}).

\begin{table}[H]
\centering
\small
\begin{tabular}{|l|p{8cm}|}
\hline
\textbf{Dimension} & \textbf{Assessment} \\
\hline
Mission Criticality & Medium (Zone 5 is medium-priority area, $P = 0.6$) \\
Coverage Gap & 11.1\% of total mission area lost \\
Temporal Urgency & $P_{\text{time}} = 0.8$ (continuous coverage mandate) \\
Response Required & Immediate reallocation (0-second tolerance) \\
\hline
\end{tabular}
\caption{Mission Impact Assessment}
\label{tab:s5_impact}
\end{table}

\textbf{Capacity Analysis:}

The system calculates spare battery capacity for each candidate UAV by subtracting safety reserves (20\%) and committed energy from current battery levels, then evaluates feasibility against distance-to-target requirements (Table~\ref{tab:s5_capacity}).

\begin{table}[H]
\centering
\small
\begin{tabular}{|l|c|c|c|c|c|}
\hline
\textbf{UAV} & \textbf{Current} & \textbf{Safety} & \textbf{Committed} & \textbf{Spare} & \textbf{Distance} \\
 & \textbf{Battery} & \textbf{Reserve} & \textbf{Energy} & \textbf{Capacity} & \textbf{to Zone 5} \\
\hline
UAV-2 & 45\% & 20\% & 10\% & \textbf{15\%} & 40m (req: 3\%) \\
UAV-4 & 40\% & 20\% & 8\% & \textbf{12\%} & 40m (req: 3\%) \\
\hline
\end{tabular}
\caption{Spare Capacity Analysis for Zone 5 Reallocation}
\label{tab:s5_capacity}
\end{table}

\textbf{Feasibility Determination:}

Table~\ref{tab:s5_feasibility} confirms that both candidate UAVs satisfy the energy requirements for Zone 5 coverage.

\begin{table}[H]
\centering
\small
\begin{tabular}{|l|c|c|c|}
\hline
\textbf{UAV} & \textbf{Spare Capacity} & \textbf{Required Energy} & \textbf{Feasible?} \\
\hline
UAV-2 & 15\% & 3\% & \cellcolor{green!25}\textbf{YES} (12\% margin) \\
UAV-4 & 12\% & 3\% & \cellcolor{green!25}\textbf{YES} (9\% margin) \\
\hline
\end{tabular}
\caption{Constraint Feasibility Check}
\label{tab:s5_feasibility}
\end{table}

\textbf{Reallocation Strategy:}

Both adjacent UAVs possess sufficient capacity to reach Zone 5. The system adopts a split-zone strategy to distribute the workload:

\begin{itemize}
\item Split Zone 5 into two sub-zones: 5A (north 20m) and 5B (south 20m)
\item Assign 5A to UAV-2 (adjacent from Zone 2, 40m distance)
\item Assign 5B to UAV-4 (adjacent from Zone 6, 40m distance)
\item Accept reduced waypoint density: 3 waypoints per sub-zone vs. 6 original
\item Coverage degradation: 8.3\% surveillance quality reduction in Zone 5
\end{itemize}

\subsubsection{DECIDE Phase (T+3.0s to T+4.5s)}

\textbf{Strategy Selection:} Partial Reallocation (coverage degradation accepted)

\textbf{Task Allocation Plan:}

The system generates extended patrol routes for both UAVs, integrating new waypoints while maintaining constraint satisfaction and collision avoidance (Table~\ref{tab:s5_allocation}).

\begin{table}[H]
\centering
\small
\begin{tabular}{|l|c|c|c|c|}
\hline
\textbf{UAV} & \textbf{Original Zone} & \textbf{Added Zone} & \textbf{Total Waypoints} & \textbf{Circuit Time} \\
\hline
UAV-2 & Zone 2 (8 pts) & Zone 5A (3 pts) & 11 points & 120s (\textasciitilde2 min) \\
UAV-4 & Zone 6 (6 pts) & Zone 5B (3 pts) & 9 points & 105s (\textasciitilde1.7 min) \\
\hline
\multicolumn{5}{|l|}{\textit{Total distance UAV-2: 100m | Total distance UAV-4: 100m}} \\
\hline
\end{tabular}
\caption{Extended Patrol Route Allocation}
\label{tab:s5_allocation}
\end{table}

\textbf{Constraint Verification:}

Before finalizing the reallocation, the system verifies all safety constraints are satisfied, as detailed in Table~\ref{tab:s5_constraints}.

\begin{table}[H]
\centering
\small
\begin{tabular}{|l|c|c|}
\hline
\textbf{Constraint Type} & \textbf{Verification Result} & \textbf{Details} \\
\hline
Battery Safety & \cellcolor{green!25}PASS & Both UAVs maintain $>$20\% reserve \\
Spatial Separation & \cellcolor{green!25}PASS & 15m buffer between 5A/5B boundary \\
Collision Avoidance & \cellcolor{green!25}PASS & Opposite patrol directions (temporal deconfliction) \\
Zone Conflicts & \cellcolor{green!25}PASS & No overlap with Zones 1, 2, 3, 4, 6, 7, 8, 9 \\
\hline
\end{tabular}
\caption{Safety Constraint Verification}
\label{tab:s5_constraints}
\end{table}

\subsubsection{ACT Phase (T+4.5s to T+5.5s)}

\textbf{Command Execution:}

The system dispatches mission updates to affected UAVs and commands the failed vehicle to return immediately, as summarized in Table~\ref{tab:s5_commands}.

\begin{table}[H]
\centering
\small
\begin{tabular}{|l|l|p{6cm}|}
\hline
\textbf{Target} & \textbf{Command Type} & \textbf{Payload} \\
\hline
UAV-2 & Mission Update & Extended waypoints: Zone 2 + 5A (11 points) \\
 & & Priority: HIGH | Timeout: 200ms \\
\hline
UAV-4 & Mission Update & Extended waypoints: Zone 6 + 5B (9 points) \\
 & & Priority: HIGH | Timeout: 200ms \\
\hline
UAV-3 & Emergency RTL & Immediate return-to-launch command \\
 & & Reason: Battery critical (8\%) \\
\hline
\end{tabular}
\caption{Command Dispatch Summary}
\label{tab:s5_commands}
\end{table}

\textbf{System State Update:}

Table~\ref{tab:s5_status} presents the post-adaptation mission status, confirming successful autonomous recovery.

\begin{table}[H]
\centering
\small
\begin{tabular}{|l|c|}
\hline
\textbf{Status Dimension} & \textbf{Value} \\
\hline
Coverage Recovery & 100\% (Zone 5 split between UAV-2 and UAV-4) \\
Mission Status & ADAPTED -- Partial Reallocation \\
Surveillance Quality & 91.7\% (reduced waypoint density in Zone 5) \\
Alert Level & \cellcolor{yellow!25}CAUTION -- Coverage degraded \\
Operator Action & None required (autonomous recovery successful) \\
\hline
\end{tabular}
\caption{Post-Adaptation Mission Status}
\label{tab:s5_status}
\end{table}

\textbf{Impact Summary:}
\begin{itemize}
\item Zone 5 coverage degraded: 3 waypoints per sub-zone (previously 6)
\item Estimated surveillance quality reduction: 8.3\% in Zone 5 only
\item Overall mission completion: 100\% spatial coverage maintained
\item No operator escalation required
\end{itemize}

\subsection{Scenario Outcome}

The OODA system achieved complete coverage recovery in 0.7 milliseconds of computation time while maintaining all safety constraints. Without adaptation, Zone 5 would remain unmonitored for the mission's remainder---an 88.9\% coverage degradation. The sub-millisecond computation ensures that algorithmic processing contributes negligibly to the 2-3 second total system response time (dominated by communication latency).

Two limitations emerge from this scenario. First, merging Zone 5 into adjacent patrol circuits reduces waypoint density, potentially degrading detection probability for that region. Second, the system remains vulnerable to cascading failures; if UAV-2 or UAV-4 experienced subsequent failures, the remaining fleet would lack capacity for full compensation. Beyond two simultaneous failures, manual intervention becomes unavoidable regardless of algorithmic sophistication.

\section{SCENARIO 2: Emergency Search \& Rescue}

\subsection{Mission Context}

Search and rescue operations represent the most time-critical application domain for UAV fault tolerance. The ``golden hour'' principle applies: survival probability decreases precipitously with elapsed time. Every minute of search delay translates directly into reduced likelihood of positive outcome.

This scenario models a missing person search across the same 120m $\times$ 120m area (3$\times$3 grid). A fleet of four UAVs with thermal imaging must cover the area within a 3-hour search window. The zone-based decomposition matches the surveillance scenario's spatial model, enabling consistent constraint validation across mission types.

\subsection{Mission Setup}

Unlike surveillance missions where all zones demand continuous attention, search and rescue prioritization follows probabilistic reasoning. The search grid receives priority assignments based on terrain analysis, the subject's last known position (LKP), and probability-of-area (POA) calculations derived from search theory.

Lost persons do not distribute randomly; behavioral studies indicate predictable patterns (seeking water, following trails, gravitating toward shelter). These priors, combined with distance-decay models from the LKP, concentrate search resources where success is most likely.

\subsection{Search Grid Prioritization}

Algorithm 1's priority scoring adapts to SAR by weighting temporal urgency heavily---the golden hour modifier amplifies scores for zones reachable quickly. The 9-zone grid follows the same priority structure, with zones in three tiers based on distance from the LKP:

\begin{itemize}
\item \textbf{High Priority ($P = 0.9$)}: Zones 1, 2, and 3 (top row, closest to LKP) receive maximum priority based on the statistical observation that most lost persons are found close to their last confirmed location.
\item \textbf{Medium Priority ($P = 0.6$)}: Zones 4, 5, and 6 (middle row) represent probable travel corridors where a mobile subject might move while seeking help or shelter.
\item \textbf{Low Priority ($P = 0.4$)}: Zones 7, 8, and 9 (bottom row, farthest from LKP) cover areas where discovery is less likely but not impossible.
\end{itemize}

\subsection{Initial Task Allocation}

\textbf{Fleet Assignment:}

The four-UAV fleet divides the 9-zone search grid according to zone priorities, with higher-priority areas receiving dedicated coverage from individual UAVs to maximize early detection probability (Table~\ref{tab:sar_initial_allocation}).

\begin{table}[H]
\centering
\small
\begin{tabular}{|l|c|c|c|}
\hline
\textbf{UAV} & \textbf{Assigned Zones} & \textbf{Priority} & \textbf{Est. Time} \\
\hline
UAV-1 & Zones 1, 2 (top-left, top-center) & \cellcolor{red!25}HIGH (0.9) & 8 min \\
UAV-2 & Zones 3, 4 (top-right, mid-left) & \cellcolor{red!25}HIGH/MED (0.9/0.6) & 8 min \\
UAV-3 & Zones 5, 6 (mid-center, mid-right) & \cellcolor{yellow!25}MEDIUM (0.6) & 8 min \\
UAV-4 & Zones 7, 8, 9 (bottom row) & \cellcolor{green!25}LOW (0.4) & 12 min \\
\hline
\multicolumn{4}{|l|}{\textit{Coverage rate: 1 zone per 4 minutes (thermal scan + image capture)}} \\
\hline
\end{tabular}
\caption{Initial Search Grid Allocation by Priority}
\label{tab:sar_initial_allocation}
\end{table}

% \begin{figure}[H]
% \centering
% \includegraphics[width=0.8\textwidth]{images/sar_loop.png}
% \caption{Search and Rescue Mission OODA Loop}
% \label{fig:sar_loop}
% \end{figure}

\subsection{Failure Scenario: Critical Zone UAV Loss}

Eight minutes into the search, UAV-2 drops off the network. The vehicle had been sweeping Zone 3 (top-right, high priority) and Zone 4 (mid-left, medium priority)---critical areas based on the subject's likely travel patterns from the LKP. Its thermal camera was scanning the zone boundaries where a disoriented hiker might have wandered.

The GPS signal vanished when the drone descended into a narrow ravine, blocked by the terrain that makes this area so dangerous for lost hikers. After 90 seconds without position updates, the failsafe activates: UAV-2 climbs to safe altitude and returns to base, abandoning 2 unsearched zones (Zones 3 and 4) including one high-priority area.

With the golden hour already ticking, these zones represent areas most likely to contain the missing person. The remaining fleet must absorb this loss immediately.

\subsubsection{OODA Cycle Execution}

The OODA cycle for SAR follows the same four-phase structure demonstrated in the surveillance scenario, adapted for time-critical search operations where the golden hour constraint dominates all decisions.

\textbf{OBSERVE (T+0 to T+1.5s):} GPS signal loss triggers UAV-2's autonomous failsafe after exceeding the 90-second timeout threshold. Fleet state aggregation reveals: UAV-1 at 75\% battery completing Zone 2 (20\% spare capacity), UAV-3 at 80\% in Zone 5 (35\% spare), and UAV-4 at 78\% in Zone 7 (30\% spare). The failure leaves 2 zones unsearched (Zones 3 and 4)---one high-priority and one medium-priority area representing critical search regions.

\textbf{ORIENT (T+1.5s to T+3.0s):} Impact assessment quantifies 22\% coverage degradation (2 of 9 zones) with 52 minutes remaining in the golden hour. Capacity analysis confirms the fleet can absorb 2 additional zones: UAV-1 has capacity for 1 more zone, UAV-3 can handle 1 more, and UAV-4 maintains its 3-zone assignment. The system formulates a priority-based strategy: UAV-1 absorbs Zone 3 (high priority, adjacent to current position), UAV-3 absorbs Zone 4 (medium priority, maintains row coverage).

\textbf{DECIDE (T+3.0s to T+4.5s):} Strategy selection yields full reallocation with priority optimization. The reallocation plan expands UAV-1 to 3 zones (Zones 1, 2, 3) and UAV-3 to 3 zones (Zones 4, 5, 6), while UAV-4 continues its 3-zone assignment (Zones 7, 8, 9). Constraint verification confirms all UAVs maintain battery reserves above 20\%: UAV-1 projects 23\% remaining (3\% margin), UAV-3 projects 37\% (17\% margin). Collision avoidance passes with sequential zone entry and 15m safety buffer.

\textbf{ACT (T+4.5s to T+6.0s):} Command dispatch prioritizes golden hour utilization with critical-priority mission updates. Post-adaptation status: 100\% zone coverage recovered (9/9 zones assigned), full spatial coverage maintained---demonstrating that the OODA system achieves complete recovery even in time-critical SAR operations.

\subsection{Scenario Outcome}

The OODA system achieved 100\% zone coverage recovery in 1.2 milliseconds (R5) and 0.3 milliseconds (R6). Both variants maintained zero constraint violations while consuming only 0.00003\% of the golden hour. Without adaptation, coverage would degrade to 77.8\% (7 of 9 zones), leaving 2 critical zones (including one high-priority area) unsearched during the most survivable window.

The millisecond-scale OODA computation ensures fault recovery consumes a negligible fraction of the golden hour, preserving maximum time for actual search operations.

Scenario R6 validates the permission system's flexibility: when optimal reallocation requires UAV-4's out-of-grid capability, the system correctly leverages this permission while respecting each vehicle's authorized operational envelope.

\section{SCENARIO 3: Package Delivery}

\subsection{Mission Context}

\begin{itemize}
    \item \textbf{Application:} Package delivery to distributed destinations with payload constraints
    \item \textbf{Duration:} 1-hour delivery window
    \item \textbf{Fleet:} 3 UAVs (heterogeneous payload capacities)
    \item \textbf{Operational Area:} 120m $\times$ 120m (3$\times$3 grid, 40m $\times$ 40m zones)
    \item \textbf{Delivery Points:} 5 destinations distributed across grid zones
\end{itemize}

\subsection{Mission Setup}

\textbf{Fleet Specifications:}

The heterogeneous three-UAV fleet comprises one heavy-lifter and two standard vehicles, each with distinct payload and endurance characteristics, as detailed in Table~\ref{tab:delivery_fleet}.

\begin{table}[H]
\centering
\small
\begin{tabular}{|l|c|c|c|c|c|}
\hline
\textbf{UAV} & \textbf{Type} & \textbf{Payload} & \textbf{Battery} & \textbf{Speed} & \textbf{Route} \\
\hline
UAV-1 & Heavy Lifter & 1.0 kg & 25 min & 12 m/s & Dest. 1-2 \\
UAV-2 & Standard & 0.5 kg & 30 min & 15 m/s & Dest. 3-4 \\
UAV-3 & Standard & 0.5 kg & 30 min & 15 m/s & Dest. 5 \\
\hline
\end{tabular}
\caption{Delivery Fleet Configuration}
\label{tab:delivery_fleet}
\end{table}

\textbf{Package Priorities:}

Algorithm 1 assigns priority scores based on package urgency, temporal deadlines, and delivery distance, resulting in the stratified priority hierarchy shown in Table~\ref{tab:delivery_packages}.

\begin{table}[H]
\centering
\small
\begin{tabular}{|c|c|c|c|c|c|}
\hline
\textbf{Pkg} & \textbf{Contents} & \textbf{Weight} & \textbf{Destination} & \textbf{Deadline} & \textbf{Priority} \\
\hline
A & Electronics & 0.5 kg & Dest. 1, Zone 1 (20, 100) & 30 min & \cellcolor{red!25}1.0 (CRITICAL) \\
B & Components & 0.4 kg & Dest. 2, Zone 3 (100, 100) & 45 min & \cellcolor{yellow!25}0.7 (HIGH) \\
C & Tools & 0.25 kg & Dest. 3, Zone 2 (60, 100) & 60 min & 0.4 (MEDIUM) \\
D & Supplies & 0.2 kg & Dest. 4, Zone 6 (100, 60) & 60 min & 0.4 (MEDIUM) \\
E & Parts & 0.35 kg & Dest. 5, Zone 8 (60, 20) & 90 min & 0.2 (LOW) \\
\hline
\end{tabular}
\caption{Package Prioritization by Urgency Level}
\label{tab:delivery_packages}
\end{table}

\subsection{Failure Scenario: Heavy Lifter Battery Anomaly}

Fifteen minutes into the delivery mission, UAV-1 is fighting an unexpected headwind. The heavy lifter is carrying 0.9 kilograms of time-sensitive cargo that must be delivered within the hour. The combination of payload mass and wind resistance is draining its battery nearly twice as fast as planned.

The numbers tell a sobering story: 40\% battery remaining instead of the projected 55\%. At this rate, UAV-1 cannot complete its route and return safely. It carries two critical packages---one destined for Destination 1 (priority 1.0), another for Destination 2 (priority 0.7). One of these deliveries will not happen as planned.

% \begin{figure}[H]
% \centering
% \includegraphics[width=0.8\textwidth]{images/delivery_loop.png}
% \caption{Delivery Mission OODA Loop}
% \label{fig:delivery_loop}
% \end{figure}

\subsubsection{OODA Cycle Execution}

The delivery scenario demonstrates the OODA system's intelligent escalation capability---recognizing when autonomous recovery is physically impossible and deferring appropriately to human operators.

\textbf{OBSERVE (T+0 to T+1.5s):} Battery anomaly detection triggers when UAV-1's discharge rate (5\%/min) exceeds baseline (3\%/min) by 1.5$\times$. Battery projection analysis reveals the current 40\% charge cannot support the full route: completing Package A delivery to Zone 1 (20m) leaves 38.6\%, but continuing to Package B at Zone 3 (80m additional) and returning to depot (100m) would deplete to 16.6\%---below the 20\% safety threshold. Package A (electronics, priority 1.0) remains deliverable; Package B (components, priority 0.7) becomes infeasible.

\textbf{ORIENT (T+1.5s to T+3.0s):} Fleet capacity analysis reveals no autonomous solution exists. UAV-2 carries 0.45 kg (packages C+D) against 0.5 kg capacity, leaving 0.05 kg spare. UAV-3 carries 0.35 kg (package E), leaving 0.15 kg spare. Neither can absorb Package B's 0.4 kg weight---a 0.25 kg (62.5\%) capacity deficit. The system evaluates alternatives: return-to-base swap (10--15 min, marginal), backup UAV-4 deployment (13 min, recommended), and ground vehicle (20--30 min, deadline violation).

\textbf{DECIDE (T+3.0s to T+4.5s):} Algorithm 4 (Operator Escalation Decision) determines autonomous recovery is infeasible. The decision matrix evaluates: coverage recovery at 80\% (passes >75\% threshold), but one critical task lost (Package B, $P=0.7$) and one constraint violation (payload), as summarized in Table~\ref{tab:delivery_escalation_decision}. The system selects operator escalation with backup UAV recommendation, configuring a 30-second auto-failsafe timer. Autonomous actions proceed immediately: UAV-1 receives modified route (Package A only, then RTL), while UAV-2 and UAV-3 continue unchanged.

\textbf{ACT (T+4.5s to T+5.5s):} The system dispatches mission updates and presents the operator decision interface. The critical alert summarizes: Package B (0.4 kg, $P=0.7$) undeliverable due to payload constraints; recommended action is backup UAV-4 deployment (13 min, meets 45-minute deadline); backup option is ground vehicle (20--30 min, deadline risk). Final system state: operator escalation active, 80\% autonomous coverage preserved, zero constraint violations, safety maintained.

\begin{table}[H]
\centering
\small
\begin{tabular}{|l|c|c|}
\hline
\textbf{Decision Criterion} & \textbf{Threshold} & \textbf{Status} \\
\hline
Coverage Recovery & >75\% & \cellcolor{green!25}80\% (4/5 packages) \\
Critical Tasks Lost ($P > 0.7$) & 0 & \cellcolor{red!25}1 (Package B) \\
Constraint Violations & 0 & \cellcolor{red!25}1 (payload) \\
Autonomous Feasibility & TRUE & \cellcolor{red!25}\textbf{FALSE} \\
\hline
\multicolumn{3}{|l|}{\textbf{Decision:} Escalate to operator (critical task unrecoverable)} \\
\hline
\end{tabular}
\caption{Operator Escalation Decision Matrix}
\label{tab:delivery_escalation_decision}
\end{table}

\subsection{Scenario Outcome}

The delivery scenarios demonstrate constraint-aware design when physical reality defeats autonomous recovery. In D6, Package B (0.4 kg) exceeds all available spare payload capacity (maximum 0.15 kg); in D7, destination coordinates lie outside the operational grid with no UAV holding out-of-grid permission. The OODA system identified both as autonomously infeasible in 0.2 ms and 0.4 ms respectively, escalating to operators while preserving zero constraint violations.

This outcome inverts the usual success metric. A greedy baseline would report 100\% coverage by overloading UAV-2 to 0.85 kg against 0.5 kg capacity---``success'' that would cause mid-flight failure. The OODA system's 0\% autonomous coverage represents constraint-aware intelligence refusing plans that physics cannot execute. The hybrid approach (80\% autonomous, 20\% supervised) proves more deployable than systems claiming full autonomy while ignoring physical constraints.

\section{Cross-Scenario Comparative Analysis}

Table~\ref{tab:scenario_performance} synthesizes performance metrics across all three mission types, enabling direct comparison of OODA behavior under different operational constraints.

\begin{table}[H]
\centering
\small
\caption{Measured Performance by Scenario}
\label{tab:scenario_performance}
\begin{tabular}{|l|c|c|c|}
\hline
\textbf{Metric} & \textbf{Surveillance (S5)} & \textbf{SAR (R5/R6)} & \textbf{Delivery (D6/D7)} \\
\hline
Coverage Recovery & \textbf{100\%} & \textbf{100\%} & \textbf{0\% (escalated)} \\
OODA Time R5/D6 (mean $\pm$ std) & \textbf{0.42 $\pm$ 0.15 ms} & \textbf{0.88 $\pm$ 0.56 ms} & \textbf{0.18 $\pm$ 0.21 ms} \\
OODA Time R6/D7 (mean $\pm$ std) & --- & \textbf{0.24 $\pm$ 0.15 ms} & \textbf{0.10 $\pm$ 0.03 ms} \\
Operator Escalation & 0\% (autonomous) & 0\% (autonomous) & 100\% (correct) \\
Constraint Violations & \textbf{0} (N=30) & \textbf{0} (N=30) & \textbf{0} (N=30) \\
Dominant Constraint & Battery & Time (golden hour) & Payload \\
\hline
\end{tabular}
\end{table}

The three scenarios validate consistent OODA behavior across diverse contexts. Surveillance and SAR achieved full autonomous recovery because constraints permitted reallocation within fleet capacity. Delivery scenarios correctly escalated when payload constraints made autonomous recovery physically impossible---demonstrating that intelligent escalation is a feature, not a failure mode.

Statistical validation (N=30 runs per scenario) confirms consistent sub-millisecond computation across all experiments. Mean OODA computation times ranged from 0.10 ms (D7) to 0.88 ms (R5), with 95\% confidence intervals confirming stable performance (e.g., R5: 0.68--1.08 ms CI). Combined with projected communication latency, total end-to-end response of ~2.2 seconds remains well within the 4-6 second design target. Zero constraint violations occurred across all 150 experimental runs (30 runs $\times$ 5 scenarios). The combination of full autonomous recovery in feasible scenarios with appropriate escalation in infeasible cases demonstrates that honest acknowledgment of physical limitations produces deployable systems.

\chapter{Methodology and Experimental Design}

This chapter describes the methodological approach employed to validate the OODA-based fault-tolerant system. Validation employs software-in-the-loop simulation with deterministic failure injection, enabling controlled and reproducible experiments. Performance is assessed through comparative evaluation against baseline strategies across three mission types, with statistical validation (N=30 runs per scenario) ensuring result reliability.

\section{Simulation Environment}

Validating fault-tolerant control systems presents a fundamental challenge: real-world UAV failures are dangerous, expensive, and difficult to reproduce. This research employs software-in-the-loop simulation (SOARES et al., 2025), implementing physically-grounded vehicle dynamics where failures can be injected deterministically and experiments repeated indefinitely.

The simulation framework implements six-degree-of-freedom quaternion dynamics with cascaded PID control. Vehicle parameters represent a generic mid-size quadrotor (1.5 kg mass, 100 Wh battery, 0.5--1.0 kg payload capacity) characteristic of commercial systems. The Quad SimCon codebase (QUAD SIMCON, 2020) serves as the foundational reference, adapted to support multi-vehicle coordination and centralized fault management.

Communication between simulated UAVs and the Ground Control Station employs TCP/IP transport with JSON-RPC 2.0 protocol, mirroring the request-response patterns of real telemetry systems. The GCS server (port 5555) aggregates fleet state at 2 Hz, while a Flask-based web dashboard (port 8085) provides real-time visualization through WebSocket streaming via SocketIO.

The implementation distributes functionality across purpose-specific modules:

\begin{itemize}
    \item \texttt{gcs/ooda\_engine.py} --- orchestrates phase timing and state transitions
    \item \texttt{gcs/fleet\_monitor.py} --- implements multi-modal failure detection
    \item \texttt{gcs/constraint\_validator.py} --- enforces battery, payload, and collision safety margins
    \item \texttt{gcs/mission\_manager.py} --- maintains the task database with assignment tracking
    \item \texttt{gcs/objective\_function.py} --- implements the two-stage optimization strategy
    \item \texttt{uav/simulation.py} --- vehicle dynamics
    \item \texttt{uav/client.py} --- GCS communication
\end{itemize}

\noindent The complete source is publicly available (EULÁLIO REIS, 2025).

\section{Validation Metrics}

Four primary metrics anchor the validation framework, each with targets derived from operational requirements and baseline human performance.

\textbf{Coverage Recovery} measures the fraction of orphaned tasks successfully reassigned to healthy UAVs:

\begin{equation}
\rho = \frac{|\mathcal{T}_{\text{reallocated}}|}{|\mathcal{T}_{\text{failed}}|} \times 100\%
\label{eq:coverage}
\end{equation}

\noindent where $\mathcal{T}_{\text{failed}}$ denotes tasks originally assigned to the failed UAV and $\mathcal{T}_{\text{reallocated}}$ represents those successfully reassigned. The target threshold is $\rho > 65\%$ for single-UAV failures and $\rho > 50\%$ for simultaneous double failures, reflecting the diminishing fleet capacity available for absorption.

\textbf{Adaptation Time} captures end-to-end system responsiveness from failure detection to command dispatch:

\begin{equation}
T_{\text{adapt}} = T_{\text{ACT\_complete}} - T_{\text{failure\_detected}}
\label{eq:adaptation}
\end{equation}

This interval encompasses the complete OODA cycle plus communication latency. The target $T_{\text{adapt}} < 5$ seconds ensures bounded mission disruption.

\textbf{Battery Efficiency} evaluates how effectively the system exploits available fleet capacity:

\begin{equation}
\eta_{\text{battery}} = \frac{\sum_{u \in \mathcal{U}_{\text{healthy}}} \Delta B_u}{\sum_{u \in \mathcal{U}_{\text{healthy}}} B_{\text{spare},u}} \times 100\%
\label{eq:efficiency}
\end{equation}

\noindent where $\Delta B_u$ represents battery consumed by UAV $u$ for reallocation tasks and $B_{\text{spare},u}$ denotes available capacity above the safety reserve. Efficiency below 80\% suggests conservative allocation leaving recoverable tasks unassigned; efficiency approaching 100\% indicates the fleet operated near capacity limits.

\textbf{Mission Completion Rate} provides the aggregate success metric across experimental trials:

\begin{equation}
\xi = \frac{N_{\text{completed}}}{N_{\text{total}}} \times 100\%
\label{eq:completion}
\end{equation}

A mission counts as completed if primary objectives are achieved despite failures, with target $\xi > 90\%$ for single-failure scenarios.

Secondary metrics supplement this primary framework: operator workload measured by escalation frequency per mission, communication bandwidth consumption in kilobytes per second, and decision quality assessed against optimal solutions computed offline via exhaustive search for small problem instances.

\section{Test Scenarios}

The test suite comprises 169 automated cases organized into three hierarchical categories, executable in approximately 0.22 seconds via pytest.

\textbf{Unit tests} (53 cases) validate individual algorithmic components in isolation. Battery management tests verify reserve calculations and discharge rate monitoring. Grid boundary tests confirm spatial constraint enforcement. State machine tests exercise all valid transitions and reject invalid state progressions. Constraint validation tests confirm that battery, payload, and collision checks correctly accept feasible allocations and reject violations. Priority scoring tests verify Algorithm 1's output against hand-calculated expected values across diverse task configurations.

\textbf{Integration tests} (81 cases) exercise subsystem interactions and mission-specific workflows. Surveillance mission tests validate continuous coverage maintenance with rotation scheduling across simulated 2-hour operations. Search and rescue tests verify grid-based coverage patterns with golden hour deadline enforcement, confirming that the system correctly prioritizes time-critical cells. Delivery tests exercise the two-phase pickup-dropoff workflow with payload constraint tracking. Cross-cutting integration tests validate multi-UAV coordination, collision avoidance during concurrent operations, and complete OODA cycle execution from failure injection through recovery confirmation.

\textbf{Regression tests} (15 cases) preserve fixes for defects discovered during development, encoding specific failure modes (constraint boundary edge cases, race conditions, numerical precision issues) to prevent regression.

Five experimental scenarios provide the primary validation evidence presented in Chapter~7. The scenario codes use letter prefixes to indicate mission type: \textbf{S} for Surveillance, \textbf{R} for Rescue (SAR), and \textbf{D} for Delivery. Scenario S5 exercises surveillance mission recovery from single-UAV battery depletion. Scenarios R5 and R6 test search and rescue operations under GPS loss, with R6 introducing an out-of-grid zone requiring operator permission. Scenarios D6 and D7 challenge the delivery mission with payload constraint violations and out-of-grid destinations respectively, validating the intelligent escalation behavior that distinguishes constraint-aware autonomy from naive allocation strategies.

\section{Baseline Comparisons}

Three baseline strategies establish the comparative framework.

\textbf{No Adaptation} represents the null hypothesis: missions proceed with fixed assignments, and UAV failure results in permanent task loss. This baseline establishes the cost of ignoring failures---expected coverage recovery of 0\% with mission degradation proportional to failed assignment load.

\textbf{Greedy Nearest-Neighbor} implements the simplest autonomous reallocation: assign each orphaned task to the nearest UAV without constraint verification. Expected coverage may reach 100\% in unconstrained scenarios, but risks safety violations when assignments exceed vehicle capabilities---plans that appear successful but fail during execution.

\textbf{Hybrid OODA} (this work) combines priority-based allocation with comprehensive constraint validation. The key differentiator is not speed---both approaches complete in milliseconds---but safety: OODA refuses infeasible allocations and escalates appropriately, whereas greedy approaches blindly assign tasks regardless of physical constraints.

\section{Visualization and Logging}

The web dashboard renders fleet state via Flask with WebSocket streaming: UAV positions with color-coded coverage, battery bars at 2 Hz with 20\% reserve marked, OODA timeline with phase timing, and operator alert log. Post-mission analysis generates coverage traces, adaptation time distributions, and battery efficiency heatmaps, all JSON exportable.

\chapter{Experimental Results and Validation}

\section{Quantitative Performance Results}

The system was validated through five experimental scenarios comparing OODA-based fault tolerance against three baseline strategies. Results demonstrate performance significantly exceeding initial expectations.

\subsection{Executive Summary}

Table~\ref{tab:executive_summary} provides a comprehensive overview of all experimental scenarios, enabling quick comparison across mission types. Table~\ref{tab:targets_achieved} summarizes achievement against design targets, confirming that all performance objectives were met or exceeded. The decision pathways underlying these outcomes are visualized in Figure~\ref{fig:ch6_decision_flow}, which traces the OODA cycle through each scenario type, while Figure~\ref{fig:ch6_radar} presents a multi-dimensional performance comparison across coverage, speed, safety, and autonomy metrics.

\begin{table}[H]
\centering
\caption{Executive Summary of Experimental Results}
\label{tab:executive_summary}
\small
\begin{tabular}{|l|c|c|c|c|c|}
\hline
\textbf{Scenario} & \textbf{Mission} & \textbf{Coverage} & \textbf{Time} & \textbf{Safety} & \textbf{Outcome} \\
 & \textbf{Type} & \textbf{Recovery} & \textbf{(ms)} & \textbf{Violations} & \\
\hline
\textbf{S5} & Surveillance & 100\% & 0.34 & 0 & Full Autonomous \\
\hline
\textbf{R5} & SAR & 100\% & 0.50 & 0 & Full Autonomous \\
\hline
\textbf{R6} & SAR (OOG) & 100\% & 0.16 & 0 & Full Autonomous \\
\hline
\textbf{D6} & Delivery & 0\% (esc.) & 0.11 & 0 & Intelligent Escalation \\
\hline
\textbf{D7} & Delivery & 0\% (esc.) & 0.23 & 0 & Intelligent Escalation \\
\hline
\multicolumn{6}{|l|}{\textit{OOG = Out-of-Grid; esc. = escalated to operator}} \\
\hline
\end{tabular}
\end{table}

\begin{table}[H]
\centering
\caption{Target Achievement Summary}
\label{tab:targets_achieved}
\begin{tabular}{|p{5cm}|c|c|c|}
\hline
\textbf{Metric} & \textbf{Target} & \textbf{Achieved} & \textbf{Improvement} \\
\hline
OODA Computation Time & --- & 0.10--0.88 ms (N=30) & Sub-ms achieved \\
\hline
End-to-End Response (projected) & <6000 ms & $\sim$2200 ms & 2.7$\times$ faster \\
\hline
Coverage Recovery (S5, R5/R6) & >75\% & 100\% (N=30) & Exceeded by 25\% \\
\hline
Constraint Violations & 0 & 0 (N=150) & Perfect \\
\hline
Golden Hour Impact (SAR) & Minimize & 0.00003\% & Negligible \\
\hline
Safe Escalation Rate & >90\% & 100\% & Perfect \\
\hline
\end{tabular}
\end{table}

\begin{figure}[H]
\centering
\includegraphics[width=0.9\textwidth]{images/chapter6_decision_flow.png}
\caption{OODA Decision Flow in Experimental Scenarios}
\label{fig:ch6_decision_flow}
\end{figure}

\begin{figure}[H]
\centering
\includegraphics[width=0.9\textwidth]{images/chapter6_performance_radar.png}
\caption{Multi-Dimensional Performance Comparison}
\label{fig:ch6_radar}
\end{figure}

\subsection{Detailed Performance Metrics}

The measured OODA adaptation timings reveal performance characteristics that substantially exceed initial design expectations, as quantified in Figure~\ref{fig:ch6_time}. The surveillance scenario (S5) completed its entire OODA cycle in 0.7 milliseconds, achieving a response 7,000 times faster than the conservative five-second target established during system design. Search and rescue operations demonstrated comparable efficiency, with scenario R5 completing in 1.2 milliseconds and the out-of-grid variant R6 achieving the fastest measured response at 0.3 milliseconds. Even the delivery scenarios requiring constraint violation detection and operator escalation (D6 and D7) completed their analysis in 0.2 and 0.4 milliseconds respectively, demonstrating that safety-critical constraint checking imposes negligible computational overhead.

Coverage recovery patterns across mission types illuminate the system's adaptive capabilities under realistic constraints. Both surveillance and search-and-rescue missions achieved complete 100\% task recovery following single-UAV failures, substantially exceeding the 75-95\% target range established during design. This outcome validates the priority-based reallocation strategy's effectiveness when sufficient spare capacity exists within the operational fleet. The delivery scenarios presented a contrasting yet equally important result: zero percent autonomous reallocation accompanied by intelligent escalation to human operators. This apparent absence of autonomous recovery represents not system failure but rather successful constraint violation detection---the system correctly identified physically infeasible reallocations and appropriately deferred to human judgment rather than compromising safety boundaries.

The computational performance proves particularly striking when examined against design targets. While the initial system specification anticipated 4-6 second OODA cycle times, measured performance of 0.2 to 1.2 milliseconds exceeds this target by nearly four orders of magnitude. This margin ensures that algorithmic computation never becomes the system bottleneck, even under adverse conditions or with larger fleet sizes. For time-critical search and rescue operations conducted under golden-hour constraints, sub-millisecond fault recovery preserves maximum mission time for actual search operations.

Safety validation across all experimental scenarios demonstrates perfect constraint adherence, as summarized in Figure~\ref{fig:ch6_safety} and detailed in Tables~\ref{tab:coverage_matrix} and~\ref{tab:safety_violations}. The OODA system maintained zero constraint violations throughout all five test scenarios, respecting battery reserves, payload limits, and operational boundaries without exception. In stark contrast, the greedy baseline algorithm---which prioritizes coverage maximization without explicit constraint verification---produced two distinct safety violations: a payload overload in scenario D6 where package weight exceeded available capacity, and a boundary transgression in scenario D7 where destination coordinates fell outside permitted flight zones. This comparison reinforces the core value proposition: OODA achieves equivalent speed to greedy approaches while guaranteeing constraint satisfaction.

\begin{table}[H]
\centering
\caption{Coverage Recovery Matrix (\%)}
\label{tab:coverage_matrix}
\begin{tabular}{|l|c|c|c|c|}
\hline
\textbf{Approach} & \textbf{S5} & \textbf{R5} & \textbf{R6} & \textbf{D6/D7} \\
 & \textbf{Surveillance} & \textbf{SAR} & \textbf{SAR-OOG} & \textbf{Delivery} \\
\hline
\textbf{OODA} & \textbf{100} & \textbf{100} & \textbf{100} & \textbf{0 (esc.)}* \\
\hline
Greedy & 100 & 100 & 100 & 100\textsuperscript{\textdagger} \\
\hline
No Adaptation & 88.9 & 77.8 & 88.9 & 80 \\
\hline
\multicolumn{5}{|l|}{* Intelligent escalation is correct behavior; \textsuperscript{\textdagger} Unsafe (constraint violations)} \\
\hline
\end{tabular}
\end{table}

\begin{table}[H]
\centering
\caption{Constraint Violations by Approach}
\label{tab:safety_violations}
\begin{tabular}{|l|c|c|c|c|}
\hline
\textbf{Approach} & \textbf{Battery} & \textbf{Payload} & \textbf{Boundary} & \textbf{Total} \\
\hline
\textbf{OODA} & \textbf{0} & \textbf{0} & \textbf{0} & \textbf{0} \\
\hline
Greedy & 0 & 1 (D6) & 1 (D7) & \textbf{2} \\
\hline
No Adaptation & 0 & 0 & 0 & 0 \\
\hline
\end{tabular}
\end{table}

\begin{figure}[H]
\centering
\includegraphics[width=0.9\textwidth]{images/chapter6_time_comparison.png}
\caption{Computation Time Comparison Across Approaches (Log Scale)}
\label{fig:ch6_time}
\end{figure}

\begin{figure}[H]
\centering
\includegraphics[width=0.85\textwidth]{images/chapter6_safety_violations.png}
\caption{Safety Validation: Constraint Violations by Approach}
\label{fig:ch6_safety}
\end{figure}

\section{Experimental Scenario Analysis}

\subsection{S5: Surveillance Mission Recovery}

The surveillance scenario (S5) evaluated system response to a single UAV failure during sustained perimeter monitoring operations. This experiment compared three distinct operational strategies to isolate the contribution of constraint-aware adaptive planning. The no-adaptation baseline, which simply aborted affected tasks without reallocation, achieved only 88.9\% coverage---effectively declaring mission failure upon detecting the fault. The greedy nearest-neighbor heuristic recovered full 100\% coverage in 1.5 milliseconds by assigning orphaned tasks to the spatially nearest available vehicles, though this approach succeeded only because the particular failure configuration happened not to violate capacity constraints.

The OODA system achieved 100\% coverage recovery in 0.7 milliseconds while maintaining perfect safety through explicit constraint verification at each allocation step. This computation time, nearly four orders of magnitude faster than the 4-6 second design target, transforms fault recovery into an essentially instantaneous adaptation invisible to mission progress. More significantly, the OODA approach guaranteed constraint satisfaction through sequential battery, payload, and collision verification, ensuring that the recovered mission plan remained executable rather than merely optimal on paper. All battery reserves remained above the 20\% safety threshold, spatial separation exceeded the 15-meter minimum, and no vehicle received task assignments beyond its physical capabilities. Figure~\ref{fig:surveillance_trajectory} visualizes the pre- and post-failure trajectories, illustrating how the system seamlessly redistributes patrol responsibilities while preserving continuous area coverage.

\begin{figure}[H]
\centering
\includegraphics[width=0.95\textwidth]{images/surveillance_dashboard.png}
\caption{Surveillance Mission Dashboard (S5)}
\label{fig:surveillance_trajectory}
\end{figure}

As shown in Figure~\ref{fig:surveillance_trajectory}, the dashboard provides a real-time visualization of the five-UAV fleet conducting zone-bounded perimeter monitoring. The interface displays individual UAV trajectories, current positions, and zone assignments following dynamic workload reallocation after vehicle failure detection.

\subsection{R5 \& R6: Search \& Rescue with Time Criticality}

The search and rescue scenarios (R5 and R6) examined system performance under the most stringent temporal constraints encountered in civilian UAV operations: the sixty-minute golden hour during which victim survival probability remains highest. Scenario R5 simulated a standard UAV failure during systematic grid search operations, where the no-adaptation baseline abandoned 22.2\% of the search area (2 of 9 zones)---effectively leaving critical zones including one high-priority area unsearched and potentially condemning an injured person to prolonged exposure. The OODA system recovered complete 100\% coverage in 1.2 milliseconds, consuming a negligible 0.00003\% of the precious golden hour interval, as illustrated in Figure~\ref{fig:ch6_golden_hour}. In scenarios where every second directly correlates with survival probability, minimizing fault recovery overhead ensures maximum time remains available for actual search operations.

Scenario R6 introduced an additional complexity by positioning one high-priority search zone ten meters outside the nominal 120m $\times$ 120m operational grid boundaries. This configuration tested the system's regulatory compliance mechanisms: could it correctly identify permitted vehicles for out-of-bounds operations while refusing allocation to unauthorized units? The experiment validated this capability comprehensively. With UAV-4 having received explicit out-of-grid authorization from the operator prior to mission commencement, the OODA system correctly reallocated the exterior zone to this permitted vehicle while excluding unauthorized units from consideration. The decision completed in 0.3 milliseconds---the fastest measured across all experimental scenarios---demonstrating that permission verification and regulatory constraint checking impose minimal computational burden. This result confirms that safety-first design principles need not compromise response speed when implemented through efficient data structures and sequential constraint evaluation. Figure~\ref{fig:sar_trajectory} depicts the zone-based search pattern and priority-driven reallocation that ensures complete coverage of high-probability areas within the golden hour window.

\begin{figure}[H]
\centering
\includegraphics[width=0.95\textwidth]{images/sar_dashboard.png}
\caption{Search and Rescue Mission Dashboard (R5/R6)}
\label{fig:sar_trajectory}
\end{figure}

Figure~\ref{fig:sar_trajectory} illustrates the lawnmower search pattern execution across priority-weighted zones targeting water sources and shelters. The visualization demonstrates systematic zone-by-zone coverage with priority-based task reallocation. Asset detection states are color-coded: yellow denotes active detection by UAV-4, green indicates confirmed identification by UAV-3, and red marks an unconfirmed asset located outside the defined search boundaries. UAV-5 remains positioned at the grid perimeter, awaiting operator authorization to extend beyond the initially specified operational area.

\begin{figure}[H]
\centering
\includegraphics[width=\textwidth]{images/chapter6_golden_hour.png}
\caption{Search \& Rescue: Golden Hour Time Consumption}
\label{fig:ch6_golden_hour}
\end{figure}

\subsection{D6 \& D7: Delivery with Intelligent Escalation}

The delivery scenarios (D6 and D7) presented perhaps the most philosophically significant experimental results by validating the system's capacity to recognize its own limitations and appropriately defer to human judgment. Scenario D6 configured a deliberately infeasible task reallocation: following primary delivery vehicle failure, the stranded Package B weighed 0.4 kilograms while the maximum spare payload capacity across all operational vehicles reached only 0.15 kilograms. This represents a fundamental physical impossibility---no combination of available resources could legally transport the package without exceeding structural load limits, as geometrically illustrated in Figure~\ref{fig:ch6_constraint}.

The contrasting responses across different approaches illuminate critical distinctions in system design philosophy. The greedy baseline algorithm reported 100\% coverage recovery by simply assigning the package to the nearest available vehicle, producing a mission plan that would overload the selected UAV by nearly three times its spare capacity---a configuration guaranteed to cause either structural damage or flight instability. The OODA system correctly identified the infeasibility, completing constraint analysis in 0.2 milliseconds before escalating to operator intervention with explicit explanation: "Package B (0.4 kg) exceeds maximum available spare capacity (0.15 kg). Recommend backup UAV deployment or ground vehicle dispatch."

Scenario D7 examined boundary constraint violations by positioning a delivery destination at coordinates (150, 100) meters---well outside the authorized 120m $\times$ 120m operational grid. Unlike scenario R6 where a permitted vehicle existed for out-of-bounds operations, no UAV in scenario D7 carried the necessary authorization. The greedy algorithm again reported perfect coverage while violating the spatial boundary constraint. The OODA system refused this unsafe allocation, completing analysis in 0.4 milliseconds before escalating with regulatory justification: "Destination outside authorized flight zone. No vehicle possesses out-of-grid permission. Operator intervention required for boundary waiver or mission modification."

These results validate a crucial principle often overlooked in autonomous systems research: intelligent refusal represents successful operation, not system failure. The OODA system's zero percent autonomous reallocation in physically or regulatorily infeasible scenarios demonstrates correct constraint detection and appropriate escalation rather than algorithmic inadequacy. A system that claims 100\% autonomy by violating safety constraints provides less operational value than one that achieves 60\% autonomous coverage while correctly identifying when human judgment becomes necessary. This honest acknowledgment of limitations distinguishes deployable systems from theoretical frameworks that assume constraints away. Figure~\ref{fig:delivery_trajectory} illustrates this intelligent escalation behavior, showing how deliverable tasks proceed while physically infeasible assignments trigger operator alerts with recommended actions.

\begin{figure}[H]
\centering
\includegraphics[width=0.95\textwidth]{images/delivery_dashboard.png}
\caption{Delivery Mission Dashboard (D6/D7)}
\label{fig:delivery_trajectory}
\end{figure}

Figure~\ref{fig:delivery_trajectory} presents the point-to-point delivery route visualization. UAV-2 (blue, loaded) approaches the designated drop-off location, while UAV-3 proceeds toward an out-of-bounds pick-up position following explicit operator authorization. UAV-4 has reached the operational boundary, triggering an escalation request for operator permission. The dashboard interface displays real-time fleet status alongside the active operator alert requiring human intervention.

\begin{figure}[H]
\centering
\includegraphics[width=0.9\textwidth]{images/chapter6_constraint_space.png}
\caption{D6 Payload Constraint Violation: Geometric Illustration of Escalation Necessity}
\label{fig:ch6_constraint}
\end{figure}

\begin{figure}[H]
\centering
\includegraphics[width=0.9\textwidth]{images/chapter6_coverage_heatmap.png}
\caption{Coverage Recovery Matrix Across All Approaches and Scenarios}
\label{fig:ch6_coverage}
\end{figure}

\section{Synthesis and Contributions}

Table~\ref{tab:scenario_comparison} synthesizes performance characteristics across all experimental scenarios, with Figure~\ref{fig:ch6_coverage} providing a visual coverage recovery matrix across all approaches and scenarios.

\begin{table}[H]
\centering
\small
\caption{Statistical Performance by Scenario (N=30 runs each)}
\label{tab:scenario_comparison}
\begin{tabular}{|l|c|c|c|}
\hline
\textbf{Metric} & \textbf{Surveillance (S5)} & \textbf{SAR (R5/R6)} & \textbf{Delivery (D6/D7)} \\
\hline
Coverage Recovery & \textbf{100\%} & \textbf{100\%} & \textbf{0\% (escalated)} \\
\hline
OODA Time R5/D6 (mean $\pm$ std) & \textbf{0.42 $\pm$ 0.15 ms} & \textbf{0.88 $\pm$ 0.56 ms} & \textbf{0.18 $\pm$ 0.21 ms} \\
\hline
OODA Time R6/D7 (mean $\pm$ std) & --- & \textbf{0.24 $\pm$ 0.15 ms} & \textbf{0.10 $\pm$ 0.03 ms} \\
\hline
Operator Escalation & 0\% (autonomous) & 0\% (autonomous) & 100\% (correct) \\
\hline
Constraint Violations & \textbf{0} (all runs) & \textbf{0} (all runs) & \textbf{0} (all runs) \\
\hline
\end{tabular}
\end{table}

\begin{table}[H]
\centering
\caption{Comprehensive Approach Comparison}
\label{tab:approach_comparison}
\small
\begin{tabular}{|p{3.5cm}|p{3.5cm}|p{3.5cm}|p{3.5cm}|}
\hline
\textbf{Criterion} & \textbf{OODA} & \textbf{Greedy} & \textbf{No Adapt} \\
\hline
\textbf{Coverage} & 100\% (S5, R5/R6), Escalate (D6/D7) & 100\% all scenarios & 77.8--88.9\% \\
\hline
\textbf{Speed} & 0.2--1.2 ms & 0.1--1.5 ms & 0 ms \\
\hline
\textbf{Safety} & \textbf{0 violations} & \textbf{2 violations} & 0 violations \\
\hline
\textbf{Constraint Awareness} & \textbf{Full} & None & N/A \\
\hline
\textbf{Deployability} & \textbf{YES} & \textbf{NO (unsafe)} & NO (degrades) \\
\hline
\textbf{Regulatory Compliance} & YES & NO & NO \\
\hline
\textbf{Scalability} & Good & Good & N/A \\
\hline
\end{tabular}
\end{table}

\begin{table}[H]
\centering
\caption{Speed vs. Safety vs. Coverage Trade-offs}
\label{tab:tradeoffs}
\begin{tabular}{|l|c|c|c|c|}
\hline
\textbf{Approach} & \textbf{Speed} & \textbf{Safety} & \textbf{Coverage} & \textbf{Overall Score} \\
 & \textbf{(0--10)} & \textbf{(0--10)} & \textbf{(0--10)} & \textbf{(0--30)} \\
\hline
\textbf{OODA} & 10 & \textbf{10} & 8 & \textbf{28} \\
\hline
Greedy & 10 & \textbf{2} & 10 & 22 \\
\hline
No Adaptation & 10 & 10 & \textbf{3} & 23 \\
\hline
\multicolumn{5}{|l|}{\textit{Speed: 10 = <10ms; Safety: 10 = zero violations; Coverage: 10 = 100\% recovery}} \\
\hline
\end{tabular}
\end{table}

Tables~\ref{tab:approach_comparison} and~\ref{tab:tradeoffs} synthesize these findings into a comprehensive comparison framework. Four patterns emerge from this data. First, computational speed remains uniformly under 1.5 milliseconds across all mission types, validating the greedy heuristic approach over optimization algorithms with unpredictable execution times. Second, zero constraint violations occurred despite widely varying failure conditions---a perfect safety record that distinguishes this approach from opportunistic algorithms. Third, the system demonstrates selective autonomy: aggressive adaptation when feasible (surveillance, SAR), intelligent refusal when infeasible (delivery). Fourth, sub-millisecond response times ensure fault recovery never becomes a mission bottleneck.

\subsection{Claims Validated}

The experimental program validated all primary thesis claims:

\begin{itemize}
\item \textbf{Rapid adaptation:} Statistical validation (N=30) confirms mean computation times of 0.10--0.88 ms for OODA cycles, ensuring algorithmic processing contributes negligibly to the 2-3 second projected end-to-end response (well within the 4-6 second design target)
\item \textbf{Safety guarantee:} Zero constraint violations across all 150 experimental runs (battery, payload, spatial boundaries)
\item \textbf{Intelligent escalation:} D6 and D7 correctly refused infeasible reallocations rather than producing unsafe plans
\item \textbf{Coverage recovery:} 100\% task recovery in surveillance and SAR missions, matching greedy baseline while maintaining safety
\end{itemize}

\subsection{Research Contributions}

The work advances three contributions. First, a \textbf{unified constraint verification framework} that simultaneously enforces battery reserves, payload limits, and temporal deadlines through fail-fast sequential checking---the first multi-UAV fault tolerance system to address all three constraint categories together. Second, a \textbf{quantified degradation framework} with explicit decision rules governing when autonomous reallocation transitions from feasible to infeasible, enabling honest acknowledgment of system limitations rather than claiming universal fault tolerance. Third, a \textbf{hybrid autonomy architecture} that balances machine speed with human oversight, enabling deployment under current BVLOS regulations that require operator-in-the-loop supervision.

Beyond algorithmic innovation, the research delivers practical value: compatibility with commercial UAV platforms, autonomous recovery in three of five scenarios (S5, R5, R6) with appropriate escalation in the remaining two, and a comprehensive test suite (169 tests, 0.22s execution) enabling rapid validation. The open-source simulation platform provides infrastructure for future research without requiring physical hardware.

The philosophical contribution may prove most significant: demonstrating that realism-first design---explicitly modeling constraints rather than abstracting them away---produces systems that are both academically rigorous and operationally deployable.

\chapter{Limitations and Future Work}

This chapter examines current limitations and future research directions for the OODA-based fault-tolerant control system.

\section{Architectural Constraints}

\subsection{Centralized Control Architecture}

The system employs a centralized Ground Control Station for OODA loop execution, creating a single point of failure. Should the GCS experience hardware failure or communication loss, the fleet's adaptive capabilities are compromised. Individual UAVs implement autonomous Return-to-Launch protocols upon GCS timeout detection, preserving vehicle safety while sacrificing mission completion.

Distributed OODA architectures using consensus algorithms (IZADI; GORDON; ZHANG, 2013) could eliminate this vulnerability while introducing communication complexity, network partitioning risks, and Byzantine fault tolerance requirements.

\subsection{Fleet Scalability}

The system supports fleets of three to twelve UAVs. Beyond this scale, two constraints become significant. Communication bandwidth scales linearly with fleet size, reaching 48 kilobytes per second for twelve UAVs at 2 Hz telemetry rates. Computational complexity for collision avoidance scales quadratically, as each reallocation requires pairwise verification among all vehicles. At twenty UAVs, collision checking workload increases 2.8-fold compared to twelve UAVs, potentially exceeding the six-second OODA cycle target.

Hierarchical architectures partitioning large fleets into coordinated subgroups, or spatial hashing for approximate collision avoidance, could enable coordination of dozens or hundreds of vehicles.

\subsection{Validation Approach and Limitations}

\subsubsection{Experimental Methodology}

Validation employs a software-in-the-loop simulation framework with deterministic failure injection, enabling controlled and reproducible experiments. The methodology proceeds in four stages:

\begin{enumerate}
\item \textbf{Scenario Definition:} Each experiment specifies mission type (surveillance, SAR, delivery), fleet size, task distribution, and failure events with precise timing and affected UAVs.

\item \textbf{Baseline Comparison:} Three strategies are evaluated under identical conditions: (a) \textit{No Adaptation}---tasks remain assigned to failed UAVs, establishing coverage loss baseline; (b) \textit{Greedy Nearest}---assigns orphaned tasks to nearest capable UAV without pre-validation, measuring constraint violation frequency; (c) \textit{OODA}---the proposed constraint-aware approach with pre-allocation validation.

\item \textbf{Statistical Validation:} Each scenario executes N=30 independent runs to establish statistical significance. Results report mean $\pm$ standard deviation with 95\% confidence intervals, ensuring reproducibility claims are quantified rather than anecdotal.

\item \textbf{Metrics Collection:} Coverage recovery percentage, OODA computation time, constraint violations, and escalation events are recorded per run and aggregated across the statistical ensemble.
\end{enumerate}

This methodology isolates the contribution of constraint-aware decision-making by holding environmental variables constant while varying only the reallocation strategy.

\subsubsection{Simulation Limitations}

Current validation relies exclusively on software-in-the-loop simulation using physics-based models representative of mid-size commercial quadrotors. While appropriate for algorithm development, simulation abstracts real-world phenomena including GPS multipath errors, communication packet corruption, sensor noise, and environmental disturbances.

Hardware-in-the-loop testing with physical flight controllers would capture timing constraints and communication latencies. Field trials with two to three UAVs would expose the system to genuine environmental challenges, establishing the operational envelope for reliable performance.

\section{Algorithmic Limitations}

\subsection{Greedy Task Reallocation}

The constraint-aware reallocation algorithm employs a greedy heuristic that assigns tasks to the nearest UAV satisfying capacity constraints. This achieves rapid execution compatible with real-time OODA requirements but does not guarantee globally optimal allocation. Early assignment decisions may consume capacity better reserved for higher-priority tasks, particularly when spare capacity is marginal and failed tasks are widely distributed.

Mixed-integer linear programming could guarantee optimal solutions though solution times may exceed OODA cycle budgets. Auction-based algorithms, particularly the Consensus-Based Bundle Algorithm, offer near-optimal solutions through distributed iterative bidding with polynomial-time complexity.

\subsection{Simplified Collision Avoidance}

The collision avoidance strategy maintains fifteen-meter spatial separation with temporal deconfliction when conflicts arise. This proves sufficient for sparse operational densities but exhibits limitations in dense flight patterns. The fixed safety buffer does not adapt to relative velocities, and the pairwise verification approach does not efficiently handle complex multi-vehicle conflicts.

Velocity obstacle approaches (Reciprocal Velocity Obstacles) and model predictive control formulations could extend applicability to urban air mobility scenarios with higher flight densities.

\section{Application Scope}

The system addresses three mission classes: long-duration surveillance, emergency search and rescue, and priority-based delivery. These applications share waypoint-based navigation, quantifiable task priorities, and tolerance for mission degradation under resource constraints.

However, this focused scope excludes mission types with different requirements. Aggressive formation flying demands tighter coordination and higher-bandwidth communication than the current 2 Hz telemetry supports. Adversarial scenarios require game-theoretic reasoning and adversarial prediction beyond current OODA capabilities. Time-critical interception missions may need more sophisticated trajectory optimization than waypoint following provides.

Extending to these domains could involve augmented OODA loops for formation flying or game-theoretic task allocation for adversarial scenarios, broadening applicability while preserving core OODA principles.

\section{Future Research Directions}

\textbf{Near-term enhancements} include validation across all twenty-seven planned scenarios, sensitivity analysis of safety reserves (5--20\% battery), and environmental modeling incorporating turbulence effects.

\textbf{Medium-term objectives} encompass hardware-in-the-loop testing via PX4 SITL, field demonstrations revealing GPS and RF interference challenges, and distributed OODA architectures with consensus-based decision-making. Auction mechanisms such as the Consensus-Based Bundle Algorithm require attention to Byzantine fault tolerance. Formal verification using model checking and temporal logic could guarantee constraint satisfaction and liveness properties.

\textbf{Long-term frontiers} involve machine learning for priority optimization---neural networks for battery prediction, reinforcement learning for adaptive OODA tuning---and game-theoretic analysis for adversarial scenarios (Stackelberg formulations, Nash equilibria). Fleet heterogeneity with mixed vehicle capabilities and swarm coordination for fleets exceeding fifty vehicles demand hierarchical architectures and allocation strategies exploiting complementary strengths.

\section{Concluding Perspective}

The limitations discussed reflect conscious design choices prioritizing practical deployability over theoretical completeness. The centralized architecture enables regulatory compliance and deterministic performance. The greedy allocation strategy trades global optimality for real-time responsiveness. The constraint-aware approach acknowledges physical limitations rather than assuming unlimited resources.

A system achieving 65--95\% autonomous coverage recovery within realistic bounds provides more value than theoretical frameworks promising perfect adaptation under idealized assumptions. The identified future work charts a path toward enhanced capabilities while maintaining honest, deployable autonomy.

As multi-agent UAV coordination matures, the research community must prioritize systems bridging the gap between laboratory demonstration and operational deployment. This requires explicit modeling of real-world constraints, acknowledgment of fundamental limitations, and design of hybrid human-machine systems leveraging complementary strengths of autonomous algorithms and human supervisory control. The present work contributes to this maturation by demonstrating that constraint-aware, operator-supervised fault tolerance represents the appropriate architecture for near-term UAV fleet deployments in regulated, safety-critical applications.

\chapter{Conclusion}

This research addresses the methodological gap between theoretical fault tolerance mechanisms and deployable autonomous UAV systems through the design, implementation, and validation of a constraint-aware multi-agent coordination framework. By explicitly modeling operational constraints---energy limitations, payload capacity, and regulatory airspace boundaries---the proposed hybrid OODA architecture provides quantified mission completion assistance within realistic operational bounds, advancing beyond idealized fault tolerance assumptions prevalent in existing literature.

\section{Summary of Contributions}

The research objectives established in Chapter 1 have been systematically addressed through the development and empirical validation of the proposed framework.

\textbf{SO1 (Hybrid OODA-FSM Architecture):} The centralized OODA Loop Engine integrated with a deterministic finite-state machine was successfully implemented, enabling formally verifiable state transitions with probabilistic failure identification. The architecture maintains compliance with Beyond Visual Line of Sight (BVLOS) operational requirements through explicit regulatory boundary enforcement.

\textbf{SO2 (Multi-Modal Failure Detection):} Fault detection mechanisms operating at 2 Hz telemetry acquisition frequency were validated across three complementary pathways: communication timeout detection (1.5-second threshold), explicit fault code recognition, and statistical anomaly detection. Empirical results confirm reliable detection of abnormal battery discharge rates (>5\% per 30-second window), position discontinuities (>100m), and altitude constraint violations.

\textbf{SO3 (Constraint-Aware Task Reallocation):} The priority-based allocation algorithms demonstrated effective joint optimization across energy reserves (20\% safety margin), payload capacity limitations, temporal deadlines, and collision avoidance requirements (15m spatial separation). Sequential fail-fast constraint verification ensures physically realizable mission plans, with delivery scenarios correctly identifying infeasible reallocations and escalating to operator intervention.

\textbf{SO4 (Operator Escalation Framework):} Graduated escalation policies were validated through experimental scenarios: operator escalation for coverage recovery below 50\%, partial reallocation for degradation within 50--75\%, and autonomous full reallocation above 75\% recovery. This framework preserves human authority over safety-critical determinations while maximizing autonomous response capability.

\textbf{SO5 (Statistical Validation):} Rigorous validation (N=30 independent trials per experimental condition) across surveillance, search-and-rescue, and delivery mission typologies yielded quantified metrics for coverage recovery percentage, adaptation latency, and escalation appropriateness. Results demonstrate complete coverage recovery for surveillance and search-and-rescue missions, with delivery scenarios exhibiting appropriate constraint-driven escalation behavior.

\textbf{SO6 (Real-Time Performance):} Statistical validation demonstrates mean computation times of 0.10--0.88 milliseconds for OODA cycle algorithmic processing, with 95\% confidence intervals confirming consistent sub-millisecond performance. Combined with projected communication latency (~2 seconds), end-to-end system response of approximately 2.2 seconds satisfies the 6-second design target (OBSERVE <500ms, ORIENT <500ms, DECIDE <1200ms, ACT <300ms).

\textbf{SO7 (Safety Constraint Satisfaction):} Zero safety violations were recorded across all 150 statistical runs and 169 automated test cases, empirically validating that the constraint-aware architecture maintains safety invariants---battery reserves, payload limits, spatial separation, and regulatory boundaries---without compromising mission coverage objectives.

\section{Principal Findings}

The experimental results yield a central insight: \textit{selective autonomy}---characterized by aggressive adaptation when operationally feasible and intelligent refusal when constraints preclude safe execution---demonstrates superior practical value compared to unconstrained systems that risk safety violations. In time-critical search-and-rescue operations, sub-millisecond fault recovery ensures adaptation overhead remains negligible relative to mission duration, preserving maximum operational time during golden-hour scenarios.

The three principal technical contributions---resource-aware reallocation, priority-based partial coverage, and graduated operator escalation---function synergistically to balance autonomous response latency with human oversight requirements, establishing suitability for deployment under current regulatory frameworks.

\section{Limitations and Threats to Validity}

These findings must be interpreted within acknowledged methodological constraints. The centralized Ground Control Station architecture introduces a single-point-of-failure vulnerability; while individual UAVs implement autonomous Return-to-Launch protocols upon communication timeout, fleet-level adaptive coordination ceases during GCS failure. The greedy allocation strategy prioritizes real-time responsiveness over global optimality, potentially yielding suboptimal task assignments under marginal spare capacity conditions.

Most significantly, validation remains exclusively software-in-the-loop, without exposure to field deployment conditions including GPS multipath errors, radio frequency interference, environmental turbulence, and sensor noise characteristic of operational environments. Hardware-in-the-loop testing and controlled field trials constitute essential validation steps prior to operational deployment.

\section{Future Research Directions}

Several research trajectories emerge from this work: (i) distributed OODA architectures eliminating single-point-of-failure vulnerabilities, (ii) machine learning approaches for predictive failure detection based on historical telemetry patterns, (iii) formal verification of safety constraint satisfaction guarantees, and (iv) field validation under realistic environmental conditions.

\section{Concluding Remarks}

This research demonstrates that constraint-aware, operator-supervised fault tolerance constitutes a viable architectural paradigm for near-term UAV fleet deployments in regulated, safety-critical applications. The methodological contribution lies not in claiming unrestricted autonomy, but in designing hybrid human-machine systems that explicitly acknowledge physical and regulatory limitations while maximizing autonomous capability within those bounds. The proposed framework bridges theoretical fault tolerance research and practical deployment requirements, contributing toward the broader objective of dependable autonomous aerial systems.

% ====================================
% REFERENCES
% ====================================
\newpage
\addcontentsline{toc}{chapter}{References}
\begin{thebibliography}{99}

\bibitem{mueller2014} 
Mueller, M. W., \& D'Andrea, R. (2014). Stability and control of a quadrocopter despite the complete loss of one, two, or three propellers. \textit{IEEE ICRA}.


\bibitem{sun2022}
Sun, Z., et al. (2022). Fault-Tolerant Model Predictive Control of a Quadrotor with an Unknown Complete Rotor Failure. \textit{IEEE ICRA}.

\bibitem{li2017}
Li, P., Yu, X., Peng, X., Zheng, Z., \& Zhang, Y. (2017). Fault-tolerant cooperative control for multiple UAVs based on sliding mode techniques. \textit{Science China Information Sciences}, 60(7).


\bibitem{yang2011}
Yang, H., Staroswiecki, M., Jiang, B., et al. (2011). Fault tolerant cooperative control for a class of nonlinear multi-agent systems. \textit{Systems \& Control Letters}, 60(4), 271-277.

\bibitem{gerkey2004}
Gerkey, B. P., \& Mataric, M. J. (2004). A formal analysis and taxonomy of task allocation in multi-robot systems. \textit{International Journal of Robotics Research}, 23(9), 939-954.

\bibitem{choi2009}
Choi, H. L., Brunet, L., \& How, J. P. (2009). Consensus-based decentralized auctions for robust task allocation. \textit{IEEE Transactions on Robotics}, 25(4), 912-926.

\bibitem{dias2006}
Dias, M. B., Zlot, R., Kalra, N., \& Stentz, A. (2006). Market-based multirobot coordination: A survey and analysis. \textit{Proceedings of the IEEE}, 94(7), 1257-1270.

\bibitem{zlot2006}
Zlot, R., \& Stentz, A. (2006). Market-based multirobot coordination for complex tasks. \textit{International Journal of Robotics Research}, 25(1), 73-101.

\bibitem{cortes2004}
Cortes, J., Martinez, S., Karatas, T., \& Bullo, F. (2004). Coverage control for mobile sensing networks. \textit{IEEE Transactions on Robotics and Automation}, 20(2), 243-255.

\bibitem{schwager2009}
Schwager, M., Rus, D., \& Slotine, J. J. (2009). Decentralized, adaptive coverage control for networked robots. \textit{International Journal of Robotics Research}, 28(3), 357-375.

\bibitem{elmaliach2009}
Elmaliach, Y., Agmon, N., \& Kaminka, G. A. (2009). Multi-robot area patrol under frequency constraints. \textit{Annals of Mathematics and Artificial Intelligence}, 57(3-4), 293-320.

\bibitem{abdessameud2011}
Abdessameud, A., \& Tayebi, A. (2011). Formation control of VTOL unmanned aerial vehicles with communication delays. \textit{Automatica}, 47(11), 2383-2394.

\bibitem{izadi2009}
Izadi, H. A., Gordon, B. W., \& Zhang, Y. M. (2009). Decentralized receding horizon control for cooperative multiple vehicles subject to communication delay. \textit{Journal of Guidance, Control, and Dynamics}, 32(6), 1959-1965.

\bibitem{izadi2013}
Izadi, H. A., Gordon, B. W., \& Zhang, Y. M. (2013). Hierarchical decentralized receding horizon control of multiple vehicles with communication failures. \textit{IEEE Transactions on Aerospace and Electronic Systems}, 49(2), 744-759.

\bibitem{beard2006}
Beard, R. W., McLain, T. W., Nelson, D. B., et al. (2006). Decentralized cooperative aerial surveillance using fixed-wing miniature UAVs. \textit{Proceedings of the IEEE}, 94(7), 1306-1324.

\bibitem{boyd1987}
Boyd, J. R. (1987). \textit{A Discourse on Winning and Losing}. [OODA Loop framework]

\bibitem{bala2025}
Bala, M., et al. (2025). The OODA Loop of Cloudlet-Based Autonomous Drones. \textit{IEEE/ACM Symposium on Edge Computing (SEC)}.

\bibitem{soares2025}
Soares, V. M. D., et al. (2025). UAV Simulation Environment for Fault Detection in Wind Farm Electrical Distribution Systems. \textit{IEEE Conference Proceedings}.

\bibitem{vandenberg2008}
van den Berg, J., Lin, M., \& Manocha, D. (2008). Reciprocal velocity obstacles for real-time multi-agent navigation. \textit{IEEE ICRA}, 1928-1935.

\bibitem{zhang2008}
Zhang, Y. M., \& Jiang, J. (2008). Bibliographical review on reconfigurable fault-tolerant control systems. \textit{Annual Reviews in Control}, 32(2), 229-252.

\bibitem{yu2015}
Yu, X., \& Jiang, J. (2015). A survey of fault-tolerant controllers based on safety-related issues. \textit{Annual Reviews in Control}, 39, 46-57.

\bibitem{parker1998}
Parker, L. E. (1998). ALLIANCE: An architecture for fault tolerant multirobot cooperation. \textit{IEEE Transactions on Robotics and Automation}, 14(2), 220-240.

\bibitem{repo}
Eulálio Reis, V. (2025). \textit{Multi-UAV OODA System: Constraint-Aware Fault-Tolerant Multi-Agent Coordination}. GitHub repository. \url{https://github.com/vriez/multi_uav_ooda_system}

\bibitem{quadref}
QUAD SIMCON. (2020). \textit{Quadcopter Simulation and Control}. GitHub repository. \url{https://github.com/bobzwik/Quadcopter_SimCon}

% \bibitem{repo}
% Project Repository: \url{https://github.com/vriez/multi_uav_ooda_system}
% \bibitem{guo2018}
% Guo, J., Zhang, Y., \& Li, W. (2018). Fault-tolerant control of quadrotor UAVs with actuator faults using adaptive backstepping. \textit{International Journal of Control, Automation and Systems}, 16(4), 1572-1583.

% \bibitem{wang2021}
% Wang, W., Zhang, Y., \& Xu, B. (2021). Fault and Failure Tolerant Model Predictive Control of Quadrotor UAV. \textit{IEEE ROBIO}.

% \bibitem{zhou2021}
% Zhou, B., Su, W., \& Han, J. (2021). Model predictive fault-tolerant control for quadrotor UAV subject to actuator faults. \textit{Aerospace Science and Technology}, 110, e106497.

% \bibitem{chang2024}
% Chang, Y., et al. (2024). Reinforcement Learning–Based Adaptive Fault-Tolerant Antidisturbance Control for UAVs. \textit{Journal of Aerospace Engineering}, 38(1).

% \bibitem{zhang2016}
% Zhang, X. Y., \& Duan, H. B. (2016). Altitude consensus based 3D flocking control for fixed-wing unmanned aerial vehicle swarm trajectory tracking. \textit{Journal of Aerospace Engineering}, 230(14), 2628-2638.

% \bibitem{liu2016}
% Liu, Z. X., Yuan, C., Yu, X., et al. (2016). Leader-follower formation control of unmanned aerial vehicles in the presence of obstacles and actuator faults. \textit{Unmanned Systems}, 4(3), 197-211.

% \bibitem{yu2016}
% Yu, X., Liu, Z. X., \& Zhang, Y. M. (2016). Fault-tolerant formation control of multiple UAVs in the presence of actuator faults. \textit{International Journal of Robust and Nonlinear Control}, 26(12), 2668-2685.

% \bibitem{korsah2013}
% Korsah, G. A., Stentz, A., \& Dias, M. B. (2013). A comprehensive taxonomy for multi-robot task allocation. \textit{International Journal of Robotics Research}, 32(12), 1495-1512.

% \bibitem{innocenti2004}
% Innocenti, M., Pollini, L., \& Giulietti, F. (2004). Management of communication failures in formation flight. \textit{Journal of Aerospace Computing, Information, and Communication}, 1(1), 19-35.

% \bibitem{franco2007}
% Franco, E., Parisini, T., \& Polycarpou, M. M. (2007). Design and stability analysis of cooperative receding-horizon control of linear discrete-time agents. \textit{International Journal of Robust and Nonlinear Control}, 17(10-11), 982-1001.

% Flight Formation
% \bibitem{pachter2001}
% Pachter, M., D'Azzo, J. J., \& Proud, A. W. (2001). Tight formation flight control. \textit{Journal of Guidance, Control, and Dynamics}, 24(2), 246-254.

% \bibitem{gu2006}
% Gu, Y., Seanor, B., Campa, G., et al. (2006). Design and flight testing evaluation of formation control laws. \textit{IEEE Transactions on Control Systems Technology}, 14(6), 1105-1112.

% \bibitem{lin2014}
% Lin, W. (2014). Distributed UAV formation control using differential game approach. \textit{Aerospace Science and Technology}, 35, 54-62.


\end{thebibliography}

\end{document}