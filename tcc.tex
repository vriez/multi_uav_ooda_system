\documentclass[12pt,a4paper,oneside]{book}

% ABNT formatting packages
\usepackage[utf8]{inputenc}
\usepackage[T1]{fontenc}
\usepackage[english]{babel}
\usepackage{indentfirst}
\usepackage{setspace}
\usepackage{graphicx}
\usepackage{float}
\usepackage[left=3cm,right=2cm,top=3cm,bottom=2cm]{geometry}
\usepackage{times}
\usepackage{caption}
\usepackage{subcaption}
\usepackage{amsmath}
\usepackage{amssymb}
\usepackage{listings}
\usepackage[table]{xcolor}
\usepackage{url}
\usepackage{hyperref}
\usepackage{titlesec}
\usepackage{tocloft}
\usepackage{longtable}
\usepackage{array}
\usepackage{booktabs}
\usepackage{fancyhdr}

% ABNT-style formatting
\onehalfspacing
\setlength{\parindent}{1.25cm}

% Chapter and section formatting (ABNT style)
\titleformat{\chapter}[display]
  {\normalfont\bfseries\fontsize{12pt}{14pt}\selectfont}
  {\MakeUppercase{\chaptertitlename\ \thechapter}}{0pt}
  {\MakeUppercase}
\titlespacing*{\chapter}{0pt}{0pt}{12pt}

% Unnumbered chapter formatting (for Acknowledgments, Abstract, etc.)
\titleformat{name=\chapter,numberless}[display]
  {\normalfont\bfseries\fontsize{12pt}{14pt}\selectfont}
  {}{0pt}
  {\MakeUppercase}
\titlespacing*{name=\chapter,numberless}{0pt}{0pt}{12pt}

\titleformat{\section}
  {\normalfont\bfseries\fontsize{12pt}{14pt}\selectfont}
  {\thesection}{1em}{}
\titlespacing*{\section}{0pt}{12pt}{6pt}

\titleformat{\subsection}
  {\normalfont\bfseries\fontsize{12pt}{14pt}\selectfont}
  {\thesubsection}{1em}{}
\titlespacing*{\subsection}{0pt}{12pt}{6pt}

\titleformat{\subsubsection}
  {\normalfont\bfseries\fontsize{12pt}{14pt}\selectfont}
  {\thesubsubsection}{1em}{}
\titlespacing*{\subsubsection}{0pt}{12pt}{6pt}

% Caption formatting (ABNT style)
\captionsetup{
  format=plain,
  labelsep=endash,
  font=small,
  labelfont=bf,
  justification=justified,
  singlelinecheck=false
}

% Code listing style
\lstset{
  language=Python,
  basicstyle=\ttfamily\footnotesize,
  keywordstyle=\color{blue},
  commentstyle=\color{green!60!black},
  stringstyle=\color{red},
  numbers=left,
  numberstyle=\tiny\color{gray},
  stepnumber=1,
  numbersep=10pt,
  backgroundcolor=\color{white},
  showspaces=false,
  showstringspaces=false,
  showtabs=false,
  frame=single,
  tabsize=2,
  captionpos=b,
  breaklines=true,
  breakatwhitespace=false,
  escapeinside={(*@}{@*)},
  xleftmargin=2em,
  framexleftmargin=1.5em
}

% Hyperref configuration
\hypersetup{
    colorlinks=true,
    linkcolor=black,
    citecolor=black,
    filecolor=black,
    urlcolor=black,
    pdftitle={Constraint-Aware Fault-Tolerant Multi-Agent UAV System Using OODA Loop},
    pdfauthor={Vítor Eulálio Reis},
}

% Page numbering configuration (top right throughout document)
\pagestyle{fancy}
\fancyhf{} % Clear all header and footer fields
\fancyhead[R]{\thepage} % Page number in top right
\renewcommand{\headrulewidth}{0pt} % Remove header rule line
\renewcommand{\footrulewidth}{0pt} % Remove footer rule line

% Apply fancy style to plain pages (chapter opening pages)
\fancypagestyle{plain}{%
    \fancyhf{} % Clear all header and footer fields
    \fancyhead[R]{\thepage} % Page number in top right
    \renewcommand{\headrulewidth}{0pt} % Remove header rule line
    \renewcommand{\footrulewidth}{0pt} % Remove footer rule line
}

\begin{document}

% ====================================
% COVER PAGE
% ====================================
\begin{titlepage}
\begin{center}
\textbf{\uppercase{Universidade de São Paulo}}\\
\textbf{\uppercase{Escola de Engenharia de São Carlos}}

\vspace{8cm}

\textbf{\uppercase{Vítor Eulálio Reis}}

\vspace{4cm}

\textbf{Constraint-Aware Fault-Tolerant Multi-Agent UAV System Using OODA Loop: Realistic Mission Completion Assistance}

\vfill

São Carlos\\
2025
\end{center}
\end{titlepage}

% ====================================
% TITLE PAGE
% ====================================
\newpage
\thispagestyle{empty}
\begin{center}
\textbf{Vítor Eulálio Reis}

\vspace{8cm}

\textbf{Constraint-Aware Fault-Tolerant Multi-Agent UAV System Using OODA Loop}

\vspace{3cm}

\begin{minipage}{8cm}
\begin{flushleft}
Monograph presented to the Specialization Course in Aeronautical Systems, School of Engineering of São Carlos, University of São Paulo, as part of the requirements for obtaining the title of Specialist.

Advisor: Prof° João Paulo Eguea, PhD
\end{flushleft}
\end{minipage}

\vspace{2cm}

FINAL VERSION

\vfill

São Carlos\\
2025
\end{center}

% ====================================
% COPYRIGHT PAGE
% ====================================
\newpage
\thispagestyle{empty}
\vspace*{10cm}
\begin{center}
I AUTHORIZE THE TOTAL OR PARTIAL REPRODUCTION OF THIS WORK,\\
BY ANY CONVENTIONAL OR ELECTRONIC MEANS, FOR STUDY\\
AND RESEARCH PURPOSES, PROVIDED THE SOURCE IS CITED.
\end{center}

\vfill

\begin{center}
\begin{minipage}{12cm}
\begin{flushleft}
Cataloging card prepared by the Library Prof° Dr. Sérgio Rodrigues Fontes at EESC/USP with data provided by the author(s).

\vspace{0.5cm}

\noindent Eulálio Reis, Vítor

\hspace{0.5cm} Constraint-Aware Fault-Tolerant Multi-Agent UAV System Using OODA Loop: Realistic Mission Completion Assistance. / Vítor Eulálio Reis; advisor Prof° João Paulo Eguea, PhD. São Carlos, 2025.

\vspace{0.5cm}

\hspace{0.5cm} Specialization (Specialization in Aeronautical Systems) -- School of Engineering of São Carlos, University of São Paulo, 2025.

\vspace{0.5cm}

\hspace{0.5cm} 1. Drone. 2. OODA Loop. 3. Fault Tolerance. 4. Multi-Agent Systems. I. Title.
\end{flushleft}
\end{minipage}
\end{center}

% ====================================
% APPROVAL PAGE
% ====================================
\newpage
\thispagestyle{empty}
\begin{center}
\textbf{\uppercase{Approval Sheet}}

\vspace{2cm}



\vspace{2cm}

\begin{tabular}{|p{13cm}|}
\hline
\textbf{Candidate / Student:} Vítor Eulálio Reis \\
\hline
\textbf{Title of TCC / Title:} Constraint-Aware Fault-Tolerant Multi-Agent UAV System Using OODA Loop \\
\hline
\textbf{Defense date / Date:} October 18, 2025 \\
\hline
\end{tabular}

\vspace{2cm}

\begin{tabular}{|p{10cm}|p{3cm}|}
\hline
\textbf{Examining Committee} & \textbf{Result} \\
\hline
{Prof° João Paulo Eguea, PhD} &  \\
\hline
\textbf{Affiliation:} School of Engineering of São Carlos / EESC-USP & \\
\hline
{Prof° Jorge Bidinotto, PhD} &  \\
\hline
\textbf{Affiliation:} School of Engineering of São Carlos / EESC-USP & \\
\hline
\end{tabular}

\vspace{2cm}

Chair of the Examining Committee:

\vspace{1cm}

\begin{center}
\rule{6cm}{0.4pt}\\
{Prof° João Paulo Eguea, PhD}\\
(Signature)
\end{center}

\end{center}

% ====================================
% DEDICATION (Optional)
% ====================================
\newpage
\thispagestyle{empty}
\vspace*{15cm}
\begin{flushright}
\begin{minipage}{8cm}
\textit{To my family, for their unconditional support throughout this endeavor.}
\end{minipage}
\end{flushright}

% ====================================
% ACKNOWLEDGMENTS
% ====================================
\newpage
\chapter*{Acknowledgments}

To the professors of the Especialização em Sistemas Aeronáuticos at the Escola de Engenharia de São Carlos, USP, for sharing their knowledge, expertise, craft, and for their dedication in supporting students throughout the course.

To Prof° Dr. João Paulo Eguea for his mentorship in guiding this research.

To Prof° Dr. Jorge Bidinotto for his leadership in managing the Specialization Program.

\vspace{0.5cm}

\noindent\textbf{AI Disclosure:} This work was developed with assistance from large language models. \textbf{Claude} (Anthropic) was used via Claude Code with both Sonnet (claude-sonnet-4-20250514) and Opus (claude-opus-4-20250514) models to support software development, experimentation and user interface design. Claude Sonnet was also used in playground mode for literature survey, reference verification, formatting, comparative analysis, and standardization. \textbf{Gemini} (Google) was used in playground mode for literature survey, linguistic refinement, and as a simulated peer reviewer. \textbf{ChatGPT} (OpenAI) was used in playground mode as a simulated peer reviewer and evaluator. All technical decisions, system architecture, algorithm design, experimental validation, and intellectual contributions remain solely the author's responsibility.

% ====================================
% ABSTRACT
% ====================================
\newpage
\chapter*{Abstract}

\noindent EULÁLIO REIS, V. \textbf{Constraint-Aware Fault-Tolerant Multi-Agent UAV System Using OODA Loop: Realistic Mission Completion Assistance}. 2025. Monograph (Specialization) – School of Engineering of São Carlos, University of São Paulo, São Carlos, 2025.

\vspace{0.5cm}

The failure of a UAV during mission execution creates an immediate resource allocation challenge: reassigning orphaned tasks to operational vehicles while respecting battery limitations, payload capacity, and regulatory constraints that academic research often ignores. This work develops a fault-tolerant control system for multi-UAV fleets that addresses this challenge under realistic operational conditions.

The system applies Boyd's OODA loop (Observe-Orient-Decide-Act) to transform vehicle failures into recoverable events. Upon detecting a fault, the system evaluates remaining fleet capacity, prioritizes orphaned tasks, and reallocates them to healthy vehicles while enforcing safety margins. When physical or regulatory limitations make full recovery impossible, the system escalates to human operators with quantified impact assessments rather than forcing unsafe autonomous decisions.

Three technical contributions emerge from this work: a resource-aware reallocation algorithm that jointly considers battery reserves, payload limits, and collision avoidance; a priority-based partial coverage strategy that gracefully degrades mission objectives when complete recovery is infeasible; and an intelligent escalation framework that distinguishes compensable failures from those requiring human judgment.

Experimental validation demonstrates adaptation times of 0.2-1.2 milliseconds for the complete OODA cycle---nearly four orders of magnitude faster than the design target---with complete task recovery in surveillance and search-and-rescue scenarios. In time-critical applications, this speed advantage translates to preserved lives during the golden hour. The approach prioritizes honest performance over inflated claims: acknowledging what autonomous systems cannot do proves as valuable as demonstrating what they can.

\vspace{0.5cm}

\noindent \textbf{Keywords:} Drone. Fault Tolerance. OODA Loop. Multi-Agent Systems. Mission Planning.

% ====================================
% LIST OF FIGURES
% ====================================
\newpage
\listoffigures

% ====================================
% LIST OF TABLES
% ====================================
\newpage
\listoftables

% ====================================
% LIST OF ABBREVIATIONS
% ====================================
\newpage
\chapter*{List of Abbreviations and Acronyms}

\begin{tabular}{ll}
6-DOF & Six Degrees of Freedom \\
BVLOS & Beyond Visual Line of Sight \\
FSM & Finite State Machine \\
GCS & Ground Control Station \\
GPS & Global Positioning System \\
JSON-RPC & JavaScript Object Notation -- Remote Procedure Call \\
LKP & Last Known Position \\
MILP & Mixed Integer Linear Programming \\
OODA & Observe-Orient-Decide-Act \\
OOG & Out-of-Grid \\
PID & Proportional-Integral-Derivative \\
POA & Probability of Area \\
PX4 SITL & PX4 Software-In-The-Loop \\
RF & Radio Frequency \\
RTL & Return-to-Launch \\
RVO & Reciprocal Velocity Obstacles \\
SAR & Search and Rescue \\
TCP/IP & Transmission Control Protocol/Internet Protocol \\
TSP & Traveling Salesman Problem \\
UAV & Unmanned Aerial Vehicle \\
\end{tabular}

% ====================================
% TABLE OF CONTENTS
% ====================================
\newpage
\tableofcontents

% ====================================
% MAIN CONTENT
% ====================================
\newpage
\setcounter{page}{1}
\pagenumbering{arabic}

\chapter{Introduction}

\section{Motivation and Problem Statement}

Multi-UAV coordination has emerged as a key enabler of scalable autonomy in logistics, environmental monitoring, and emergency response. Yet much fault-tolerant multi-agent research still operates within idealized boundaries (ZHANG; JIANG, 2008; YU; JIANG, 2015), presuming unlimited energy, unrestricted payload capacity, instantaneous communication, and fully autonomous decision authority—assumptions that mask real-world complexities.

In practice, UAV operations face strict physical and regulatory limits: batteries must retain 10–20\% safety reserves, payloads deplete irreversibly mid-mission, communication links introduce latencies up to two seconds, and beyond visual line of sight operations require constant operator supervision. The resulting ``reality gap'' means systems reliable in laboratory conditions may struggle in the field. Closing this gap requires architectures that adapt to uncertainty and resource constraints while maintaining situational awareness and regulatory compliance.

The OODA loop—Observe, Orient, Decide, Act—originated from Colonel John Boyd's studies of aerial combat dynamics (BOYD, 1987), proposing that success depends on processing information and adapting faster than adversaries. For autonomous systems, it provides a recursive cycle: ``Observe'' collects sensor and telemetry data; ``Orient'' establishes situational awareness; ``Decide'' selects actions under constraints; and ``Act'' executes decisions while feeding outcomes back into subsequent observations. Applied to UAV fleets, OODA extends beyond traditional feedback control by integrating situational reasoning with awareness of resource states, communication health, and mission progress. Most existing coordination systems emphasize either rapid reaction or long-term planning, but rarely both (BALA et al., 2025).

Despite its theoretical appeal, the OODA framework has rarely been implemented as a real-time operational control system for UAV fleets. Most prior work either focuses on single-vehicle fault recovery (MUELLER; D'ANDREA, 2014; SUN et al., 2022) or treats multi-agent coordination under ideal conditions with unlimited resources (LI et al., 2017; YANG et al., 2011). Architectures for fault-tolerant multi-robot cooperation such as ALLIANCE (PARKER, 1998) have demonstrated resilience principles, but few systems integrate real-world resource constraints, probabilistic fault detection, and operator-in-the-loop supervision within a unified architecture.

\section{Proposed Solution and Contributions}

This research addresses that gap through the development of a hybrid OODA-based fault-tolerant mission control system designed for real-time UAV fleet management. The system operates as a hybrid finite-state machine (FSM) that continuously cycles between monitoring and adaptation modes, combining deterministic state transitions with probabilistic failure identification to maintain robust mission execution under realistic constraints. The principal contributions are:

\textbf{Hybrid OODA–FSM Architecture.} At its core, the system functions as a hybrid finite-state controller implementing continuous monitoring at 2 Hz and invoking the OODA cycle upon failure detection. Deterministic state logic ensures predictable mission flow, while probabilistic reasoning handles uncertain or delayed telemetry data.

\textbf{Real-Time Failure Detection.} The system implements multi-modal failure detection through 2 Hz telemetry polling, feeding three detection pathways: communication timeout alerts for telemetry gaps exceeding 1.5 seconds, explicit fault codes from onboard diagnostics, and statistical anomaly detection for subtle degradation patterns (battery discharge >5\%/30s, position jumps >100m, altitude violations). This fusion achieves sub-second fault identification while minimizing false positives.

\textbf{OODA Execution for Fault Recovery.} Upon fault detection, the system executes a structured OODA cycle completing in 4 to 5.5 seconds. The Observe phase (1.0–1.5s) aggregates fleet telemetry and quantifies mission impact. The Orient phase (1.0–1.5s) evaluates remaining capacity—battery reserves, payload availability, temporal constraints—and re-prioritizes orphaned tasks. The Decide phase (1.0–1.5s) classifies feasibility: full reallocation when >75\% of tasks are recoverable, partial reallocation for 50–75\%, or operator escalation below 50\%. The Act phase (0.5–1.5s) dispatches commands and updates the operator dashboard.

\textbf{Constraint-Aware Strategy Layer.} Three recovery strategies address different failure severities. \emph{Full Reallocation} engages when the fleet retains sufficient capacity to absorb all orphaned tasks, executing a constraint-aware optimization routine that integrates new waypoints while minimizing travel distance through heuristic TSP solutions. \emph{Partial Reallocation} prioritizes tasks exceeding a 0.7 priority threshold while issuing coverage-gap alerts quantifying the impact. \emph{Operator Escalation} activates when autonomous decision-making would produce unacceptable outcomes, initiating a supervised decision protocol with a 30-second countdown that defaults to the safest available action if no operator input arrives.

\textbf{Performance and Scalability.} The complete OODA cycle achieves reaction times between 4 and 5.5 seconds under realistic communication latency conditions. Computationally, the system scales linearly with fleet size for monitoring and resource evaluation, and quadratically for collision avoidance checks—supporting real-time performance for up to 12 UAVs without exceeding sub-6-second response thresholds.

\section{Research Objectives}

The general objective is to design, implement, and validate a constraint-aware fault-tolerant multi-agent coordination framework for UAV systems, employing the OODA loop paradigm to enable mission continuity under operational constraints. This addresses the methodological gap between theoretical fault tolerance and deployable autonomous systems through explicit modeling of energy limitations, payload capacity, regulatory boundaries, and human-in-the-loop decision authority. The specific objectives are:

\textbf{SO1: Hybrid OODA-FSM Architecture Design.}
To design and implement a centralized OODA Loop Engine integrated with a deterministic finite-state machine for continuous fleet monitoring, enabling formally verifiable state transitions with probabilistic failure identification while maintaining BVLOS regulatory compliance.

\textbf{SO2: Multi-Modal Real-Time Failure Detection.}
To develop fault detection mechanisms at 2 Hz telemetry frequency with three complementary pathways: (i) communication timeout detection (>1.5s threshold), (ii) explicit fault code recognition from onboard diagnostics, and (iii) statistical anomaly detection for gradual degradation (battery discharge >5\%/30s, position discontinuities >100m, altitude violations).

\textbf{SO3: Constraint-Aware Task Reallocation.}
To formalize priority-based allocation algorithms optimizing across energy reserves (20\% margin), payload capacity, temporal deadlines, and collision avoidance (15m buffer), employing sequential fail-fast constraint verification to guarantee physically realizable mission plans.

\textbf{SO4: Operator Escalation Framework.}
To establish formal decision rules distinguishing autonomously compensable failures from those requiring human judgment: operator escalation for coverage recovery below 50\%, partial reallocation for 50--75\% recovery, and autonomous full reallocation above 75\%, preserving operator authority over safety-critical determinations.

\textbf{SO5: Statistical Validation Across Mission Typologies.}
To conduct statistical validation (N=30 trials per condition) across three mission types---surveillance, search-and-rescue, and delivery---quantifying coverage recovery, adaptation latency, constraint violations, and escalation appropriateness relative to baseline and greedy strategies.

\textbf{SO6: Real-Time Response Performance.}
To demonstrate OODA cycle execution within 6-second latency (OBSERVE <500ms, ORIENT <500ms, DECIDE <1200ms, ACT <300ms), ensuring negligible computational overhead in time-critical contexts such as golden-hour search operations.

\textbf{SO7: Safety-Critical Constraint Satisfaction.}
To validate zero safety constraint violations across all trials regarding battery reserves, payload limits, spatial separation, and regulatory compliance, demonstrating that the architecture maintains safety invariants without compromising coverage objectives.

\section{Document Organization}

This monograph is organized as follows. Chapter~\ref{ch:literature} reviews theoretical foundations and prior work, identifying the research gap. Chapter~3 presents the system architecture, detailing the centralized OODA engine and distributed execution model. Chapter~4 describes the core algorithms, including constraint-aware reallocation. Chapter~5 introduces the three mission scenarios used for validation. Chapter~6 details experimental methodology. Chapter~7 presents results and statistical validation. Chapter~8 discusses limitations and future work. Chapter~9 concludes with a summary of contributions.

By grounding UAV mission control in the OODA decision cycle, this research introduces a system capable of adaptive, explainable, and regulatorily compliant autonomy—bridging the gap between theoretical multi-agent fault tolerance and real-world deployability.

\chapter{Literature Review}
\label{ch:literature}

This chapter examines prior work relevant to fault-tolerant multi-UAV coordination under operational constraints. The review covers fault tolerance at the vehicle level (Section~\ref{sec:lit-single-uav}), multi-agent coordination architectures (Section~\ref{sec:lit-multi-agent}), task allocation algorithms (Section~\ref{sec:lit-task-allocation}), coverage control (Section~\ref{sec:lit-coverage}), communication-aware coordination (Section~\ref{sec:lit-communication}), and the OODA loop in autonomous systems (Section~\ref{sec:lit-ooda}). Section~\ref{sec:lit-gap} synthesizes these threads to identify the research gap.

\section{Fault Tolerance in Single Unmanned Aerial Vehicles}
\label{sec:lit-single-uav}

Vehicle-level fault tolerance has achieved substantial maturity through control-theoretic approaches. Mueller and D'Andrea \cite{mueller2014} demonstrated that quadrotors can maintain controlled flight despite loss of one to three propellers by exploiting remaining actuator authority through control reallocation, achieving stable hovering with a single propeller albeit with reduced maneuverability.

Sun et al. \cite{sun2022} extended this to scenarios where the failed rotor is unknown a priori, using model predictive control (MPC) to estimate failure configurations online and adapt allocation accordingly, achieving stable recovery within 2.3 seconds.

\textbf{Critical Assessment:} These contributions establish that individual vehicles can survive actuator failures. However, both assume (i) an operator or system to issue high-level commands post-recovery, (ii) no coordination with other vehicles, and (iii) unlimited recovery time. In fleet operations where failed vehicles hold assigned tasks and battery constraints limit duration, vehicle-level recovery alone proves insufficient.

\section{Multi-Agent Coordination Architectures}
\label{sec:lit-multi-agent}

Parker's \cite{parker1998} ALLIANCE architecture represents foundational work in fault-tolerant multi-robot cooperation, employing behavior-based control with motivational dynamics: each robot maintains motivation levels for available tasks, with impatience parameters causing robots to assume tasks when teammates fail. Experimental validation demonstrated graceful degradation under robot failures.

\textbf{Limitations:} ALLIANCE assumes tasks are interchangeable among robots and does not model consumable resources. The motivational dynamics operate on implicit timeouts rather than explicit failure detection, introducing response latency.

Li et al. \cite{li2017} developed sliding-mode cooperative control for multiple UAVs, proving Lyapunov stability for the fleet under bounded actuator faults. Yang et al. \cite{yang2011} generalized this to nonlinear multi-agent systems, establishing theoretical fault tolerance through adaptive control laws that compensate for unknown fault magnitudes.

\textbf{Critical Assessment:} Both provide rigorous stability proofs but assume continuous communication, known dynamics, and no resource constraints. Battery depletion, payload consumption, and regulatory boundaries do not appear in their formulations.

\section{Task Allocation Under Resource Constraints}
\label{sec:lit-task-allocation}

Gerkey and Matarić \cite{gerkey2004} established the canonical taxonomy for multi-robot task allocation (MRTA), classifying problems along three dimensions: single-task versus multi-task robots, single-robot versus multi-robot tasks, and instantaneous versus time-extended assignment. Their complexity analysis proved that most MRTA variants are NP-hard, motivating heuristic approaches.

The Consensus-Based Bundle Algorithm (CBBA) of Choi et al. \cite{choi2009} enables decentralized task allocation through iterated auctions, with robots bidding on task bundles and a consensus phase resolving conflicts. CBBA guarantees convergence within $O(NM)$ communication rounds for $N$ robots and $M$ tasks.

Market-based coordination, surveyed by Dias et al. \cite{dias2006}, offers scalable allocation through economic mechanisms. Zlot and Stentz \cite{zlot2006} extended market approaches to complex tasks, demonstrating allocation among 80 simulated robots.

\textbf{Critical Assessment:} These algorithms optimize allocation efficiency but model robot capabilities as static. Dias et al. note that ``incorporating resource consumption and task deadlines remains an open challenge.'' CBBA can theoretically encode battery constraints, but most applications assume fixed capabilities throughout the mission.

\section{Coverage Control and Spatial Coordination}
\label{sec:lit-coverage}

Voronoi-based coverage control, formalized by Cortés et al. \cite{cortes2004}, partitions the operational area among robots such that each is responsible for the region closest to its position, with gradient-descent convergence to locally optimal configurations.

Schwager et al. \cite{schwager2009} extended this to adaptive coverage, learning density functions online from sensor measurements with a decentralized algorithm requiring only local neighbor information.

Elmaliach et al. \cite{elmaliach2009} addressed multi-robot patrol under frequency constraints with cyclic strategies guaranteeing visit intervals for convex environments.

\textbf{Critical Assessment:} Coverage algorithms assume continuous operation without resource depletion. When robots return to base for battery or payload replenishment, coverage guarantees break. None address mid-mission robot loss or coverage gap quantification.

\section{Communication-Aware Multi-Vehicle Control}
\label{sec:lit-communication}

Communication constraints fundamentally affect coordination feasibility. Abdessameud and Tayebi \cite{abdessameud2011} analyzed formation control for VTOL vehicles under communication delays, proving stability for delays below thresholds dependent on formation geometry.

Izadi et al. \cite{izadi2009} developed decentralized receding horizon control for cooperative vehicles with communication delays, partitioning fleets into subgroups with designated leaders. Izadi et al. \cite{izadi2013} extended this to communication failures, showing hierarchical architectures maintain stability despite intermittent link losses.

Beard et al. \cite{beard2006} presented one of the few complete systems integrating cooperative surveillance with communication constraints, deploying fixed-wing UAVs for decentralized target tracking with prioritized information sharing.

\textbf{Critical Assessment:} These works establish that coordination remains possible under degraded communication but assume eventual recovery or tolerance for permanent link losses. The interaction between communication failures and resource constraints---where delayed coordination may exhaust battery before updated commands arrive---remains unaddressed.

\section{The OODA Loop in Autonomous Systems}
\label{sec:lit-ooda}

Boyd's \cite{boyd1987} OODA loop---Observe, Orient, Decide, Act---originated from analysis of aerial combat, arguing that faster cycling through this loop yields decisive advantage. The framework emphasizes ``Orient'' as the critical phase where observations are filtered through experience to create situational awareness.

Bala et al. \cite{bala2025} applied OODA to cloudlet-based autonomous drones, mapping phases to edge computing primitives and demonstrating 340ms average decision latency for obstacle avoidance.

\textbf{Critical Assessment:} Bala et al. demonstrate OODA's applicability to single-drone decisions but do not address fleet-level coordination, failure recovery across vehicles, or constraint satisfaction. The phases map to individual operations rather than fleet-wide situational awareness.

\section{Research Gap Analysis}
\label{sec:lit-gap}

Table~\ref{tab:literature-gap} synthesizes the coverage of key concerns across the reviewed literature.

\begin{table}[htbp]
\centering
\caption{Coverage of operational concerns across reviewed literature.}
\label{tab:literature-gap}
\small
\begin{tabular}{lcccccc}
\toprule
\textbf{Work} & \textbf{Fleet} & \textbf{Battery} & \textbf{Payload} & \textbf{Comm.} & \textbf{Failure} & \textbf{Operator} \\
 & \textbf{Coord.} & & & \textbf{Delay} & \textbf{Recov.} & \textbf{Loop} \\
\midrule
Mueller \& D'Andrea (2014) & -- & -- & -- & -- & Vehicle & -- \\
Sun et al. (2022) & -- & -- & -- & -- & Vehicle & -- \\
Parker (1998) & \checkmark & -- & -- & -- & Implicit & -- \\
Li et al. (2017) & \checkmark & -- & -- & -- & Theoretical & -- \\
Gerkey \& Matarić (2004) & \checkmark & -- & -- & -- & -- & -- \\
Choi et al. (2009) & \checkmark & -- & -- & \checkmark & -- & -- \\
Cortés et al. (2004) & \checkmark & -- & -- & -- & -- & -- \\
Izadi et al. (2013) & \checkmark & -- & -- & \checkmark & Comm. & -- \\
Bala et al. (2025) & -- & -- & -- & \checkmark & -- & -- \\
\textbf{This Work} & \checkmark & \checkmark & \checkmark & \checkmark & \checkmark & \checkmark \\
\bottomrule
\end{tabular}
\end{table}

The reviewed literature reveals three distinct research trajectories that have not converged:

\begin{enumerate}
    \item \textbf{Vehicle-level fault tolerance} achieves robust recovery for individual UAVs but ignores fleet-level mission impact.
    \item \textbf{Multi-agent coordination} demonstrates scalable allocation and coverage but assumes static capabilities and unlimited resources.
    \item \textbf{Communication-aware control} addresses delays and link failures but does not integrate resource depletion or mission-level recovery.
\end{enumerate}

No prior work integrates:
\begin{itemize}
    \item Real-time failure detection with explicit latency bounds
    \item Constraint-aware task reallocation respecting battery reserves, payload capacity, and regulatory boundaries
    \item Graduated operator escalation distinguishing autonomously compensable failures from those requiring human judgment
    \item Coverage gap quantification enabling informed operator decisions
\end{itemize}

This research addresses that gap through a hybrid OODA-based architecture that unifies rapid failure detection, constraint-aware reallocation, and operator-in-the-loop supervision within a formally structured decision cycle.

\chapter{System Architecture}

\section{Centralized OODA Architecture with Distributed Execution}

The system implements a hierarchical control structure: the Ground Control Station hosts the OODA Loop Engine for centralized decision-making while UAVs execute tasks autonomously. This architecture balances global fleet visibility for optimization against distributed execution for responsiveness. Figure~\ref{fig:architecture} presents this layered architecture, illustrating information flow between the GCS decision engine, communication infrastructure, and UAV fleet---separating centralized planning (OODA loop processing fleet-wide telemetry for task allocation) from decentralized execution (vehicles maintaining local autonomy for navigation and obstacle avoidance).

\begin{figure}[h!tbp]
    \centering
    \includegraphics[width=\textwidth, height=0.9\textheight, keepaspectratio]{images/architecture.png}
    \caption{Architecture Overview}
    \label{fig:architecture}
\end{figure}

\subsection{Design Rationale and Trade-offs}

The centralized architecture derives from convergent factors: BVLOS regulatory compliance mandates operator oversight, global fleet visibility enables superior optimization versus distributed consensus, and single-point decision-making eliminates complex inter-UAV coordination protocols. The hybrid autonomy model recognizes scenarios where complete failure compensation proves physically impossible, maintaining safety through human-in-the-loop oversight.

Architectural trade-offs require mitigation. The GCS single point of failure is addressed through autonomous Return-to-Launch triggered by communication timeout. Communication bandwidth scales linearly with fleet size, though 2 Hz telemetry maintains manageable overhead for up to twelve UAVs. The bidirectional architecture exhibits asymmetric data flow: telemetry uplink (~2 KB) exceeds command downlink (~1 KB).

The 2 Hz telemetry rate reflects practical deployment constraints. Commercial systems typically employ 1-2 Hz; higher frequencies increase bandwidth and packet collision probability without proportional benefit given 3-6 second OODA response latency. This ensures field deployability while maintaining responsive fault detection.

\section{OODA Loop Execution Flow}

The OODA loop implements continuous monitoring and reactive decision-making through four sequential phases. \textbf{Statistical validation (N=30 runs per scenario) demonstrates algorithmic computation times of 0.1--0.9 ms (mean)}, with 95\% confidence intervals confirming sub-millisecond performance. This ensures algorithmic processing never becomes the bottleneck, leaving the 4-6 second end-to-end budget for communication latency.

\subsection{Computation Time vs. End-to-End System Latency}

A key distinction exists between \textbf{OODA algorithm computation time} and \textbf{total system response latency}. The 4-6 second design target specifies end-to-end response including communication; sub-millisecond measurements represent pure algorithmic computation. These are complementary metrics:

\textbf{OODA computation time (measured):} The complete OODA cycle---encompassing all four phases with failure detection, capacity analysis, constraint validation, and reallocation optimization---executes in 0.2 to 1.2 milliseconds. The core greedy allocation (Algorithm 2) executes in 0.1-0.3 ms. This represents pure computation in software-in-the-loop simulation without network delays, validating tractable algorithmic complexity for real-time deployment.

\textbf{End-to-end system latency (design target):} The 4-6 second estimate accounts for bidirectional RF communication delays (ABDESSAMEUD; TAYEBI, 2011; IZADI; GORDON; ZHANG, 2009): 0.5-1.0s uplink for failure notification, 0.5-1.0s downlink for command dispatch, telemetry aggregation at 2 Hz (up to 0.5s), and acknowledgment verification (0.2s). In field deployment:

\begin{equation}
T_{total} = T_{uplink} + T_{compute} + T_{downlink} + T_{ack} \approx 1.0 + 0.001 + 1.0 + 0.2 = 2.2 \text{ seconds}
\end{equation}

This 2-3 second response is faster than the conservative 4-6 second estimate, with time consumption dominated by RF propagation rather than computation. Millisecond-scale algorithmic performance ensures computation never becomes the limiting factor.

\textbf{Practical implication:} Real-world deployments experience 2-3 second adaptation times dominated by communication latency. Network optimization provides greater latency reduction potential than algorithmic speedup. The 2-3 second response validates OODA for time-critical applications such as search and rescue.

Figure~\ref{fig:ooda_loop} illustrates the OODA loop execution flow. Each Act phase feeds telemetry back into Observe, creating continuous feedback that maintains situational awareness. Failure events trigger immediate cycle activation.

\begin{figure}[h!tbp]
    \centering
    \includegraphics[width=\textwidth, height=0.9\textheight, keepaspectratio]{images/ooda_loop.png}
    \caption{OODA Loop Execution Flow}
    \label{fig:ooda_loop}
\end{figure}

\subsection{OBSERVE Phase: Failure Detection and State Aggregation}

The OBSERVE phase (1.0-1.5s budget) establishes situational awareness through multi-modal failure detection: timeout detection for telemetry gaps >1.5s, explicit fault messages from UAV systems, and statistical anomaly detection (battery discharge >5\%/30s, position jumps >100m, altitude violations). Upon failure confirmation, the system aggregates fleet state from all operational UAVs, identifies failed vehicle last-known state, enumerates lost tasks with priorities and deadlines, and calculates mission impact percentage.

\subsection{ORIENT Phase: Situation Assessment and Capacity Analysis}

The ORIENT phase (1.0-1.5s budget) transforms observations into actionable intelligence, drawing on coverage control theory for mobile sensing networks (CORTÉS et al., 2004; SCHWAGER; RUS; SLOTINE, 2009). Mission impact evaluation quantifies coverage loss, affected zones, and deadline pressure. Fleet capacity analysis inventories spare resources: battery capacity (subtracting committed energy and 15-20\% safety reserves, yielding ~3min flight per 10\% spare), payload capacity for delivery missions, and temporal margins relative to deadlines. Task prioritization applies Algorithm 1 to generate 0-1 priority scores from temporal urgency, mission criticality, and spatial cost. Feasibility classification: feasible (>75\% tasks reallocable, >90\% completion), partial (50-75\% reallocable, 75-90\% completion), or infeasible (<50\% reallocable, <75\% completion).

\subsection{DECIDE Phase: Strategic Planning and Algorithmic Optimization}

The DECIDE phase (1.0-1.5s budget) selects strategies via three-tier hierarchy. \textbf{Full Reallocation:} Executes Algorithm 2 for constraint-aware assignment to nearest UAVs with collision avoidance, optimizing path integration for 90-100\% completion. \textbf{Partial Reallocation:} Filters tasks with priority >0.7, allocates high-priority tasks first, generates coverage gap alerts with operator recommendations for 75-90\% completion. \textbf{Operator Escalation:} Generates critical alerts (urgency: high if coverage <50\% or critical tasks lost, medium if 50-75\%, low if >75\%), presents alternatives (backup UAV, mission abort, degraded coverage), implements 30s countdown timer with automatic safe-default execution.

\subsection{ACT Phase: Command Execution and System Update}

The ACT phase (0.5-1.5s budget) dispatches mission updates to affected UAVs (new waypoints, task assignments, collision parameters) via 2 Hz uplink with 200ms acknowledgment verification and 3-attempt retry logic. Dashboard updates display fleet status, mission progress, coverage heatmaps, and alerts. Performance logging captures failure details, phase durations, reallocation results, and escalation status. Following completion, the system returns to continuous 2 Hz monitoring, supporting cascading failure handling through iterative OODA cycles.

\section{Sequential Constraint Validation Process}

The constraint validation process implements fail-fast sequential checking, enabling early termination when fundamental limitations preclude autonomous compensation. Figure~\ref{fig:constraints} presents the decision flowchart: each constraint category serves as a gate before subsequent checks proceed. The ordering---battery first, payload second, temporal last---reflects criticality and computational efficiency of early rejection.

\begin{figure}[h!tbp]
    \centering
    \includegraphics[width=\textwidth, height=0.9\textheight, keepaspectratio]{images/constraints_checking.png}
    \caption{Constraint Checking Process}
    \label{fig:constraints}
\end{figure}


\subsection{Battery Constraint Verification}

Battery constraint evaluation serves as the primary gate due to safety-criticality. For each lost task, the system calculates Euclidean distance from candidate UAVs, determines required battery via efficiency factors, and computes spare capacity as current charge minus committed energy and 15--20\% reserves. Pass: every lost task finds a candidate UAV within spare capacity. Fail: any task unreachable without violating safety margins triggers operator escalation.

\subsection{Payload Constraint Verification}

Payload constraint evaluation (conditional on battery satisfaction) applies to cargo operations. Spare payload equals maximum capacity minus current load. Pass: all payload-requiring tasks find adequate capacity. Fail: removes payload-heavy tasks while reallocating feasible ones. Payload constraints are hard physical limits---mid-air transfers are impossible, constraining reallocation to base station cargo swaps.

\subsection{Time Constraint Verification}

Time constraint evaluation (final validation) verifies tasks complete before deadlines via cumulative calculation: transit + execution + current task remainder. Pass: 90--100\% completion plans. Fail: partial reallocation prioritizing deadline urgency, accepting delays for low-priority tasks (75--90\% completion). Final positioning reflects flexibility---violations produce mission degradation rather than safety risks.

\subsection{Design Rationale}

Sequential checking provides: (1) early exit efficiency, saving ~60\% computation when battery constraints fail; (2) clear failure attribution for diagnostics; (3) prioritized ordering---safety-critical battery first, physical payload second, flexible time last.

\section{Communication Sequence and Information Flow}

The communication sequence and pipeline architecture demonstrate end-to-end fault response, transforming sensor measurements into executable commands through OODA processing.

\begin{figure}[h!tbp]
    \centering
    \includegraphics[width=\textwidth, height=0.9\textheight, keepaspectratio]{images/sequence_diagram.png}
    \caption{Mission Execution Sequence}
    \label{fig:sequence}
\end{figure}

\subsection{Temporal Execution Analysis}

Figure~\ref{fig:sequence} illustrates communication flow during failure events. At time T, a UAV fails, transmitting a fault message (0.4-1.0s latency). OBSERVE confirms and aggregates fleet state by T+1.5s. ORIENT completes impact assessment by T+2.5s. DECIDE generates reallocation plans by T+3.5s. ACT dispatches commands by T+4.0s, establishing a 4-second response timeline.

\subsection{Information Pipeline Architecture}

\begin{figure}[H]
\centering
\includegraphics[width=0.9\textwidth]{images/processing_pipeline.png}
\caption{Data Flow Pipeline}
\label{fig:pipeline}
\end{figure}

The data flow architecture (Figure~\ref{fig:pipeline}) implements unidirectional transformation with clear phase boundaries. The input layer ingests telemetry at 2 Hz and static mission definitions. The processing layer transforms sequentially: OBSERVE produces fleet states and failure lists, ORIENT generates impact assessments, DECIDE creates strategies and commands, ACT executes and logs. The output layer distributes commands to UAVs, alerts to operators, and logs for analysis.

This unidirectional pattern provides traceability, testability through phase isolation, and maintainability. The feedback loop manifests as ACT commands update UAV states, with telemetry flowing back into OBSERVE in subsequent cycles.

\subsection{Scalability Considerations}

Complexity analysis: OBSERVE and ORIENT scale linearly with fleet size. DECIDE exhibits O(N$\times$M) for task allocation across N UAVs and M tasks, with collision avoidance scaling O(N$^2$) for pairwise verification. Expected execution times: 5-6 seconds for twelve-UAV fleets, 8-10 seconds for twenty-UAV fleets, potentially exceeding acceptable thresholds for larger deployments.

\section{Scenario-Specific Adaptations}

The architecture demonstrates flexibility through scenario-specific workflow patterns while maintaining consistent OODA mechanics. Three mission types illustrate adaptation to varying requirements and constraint priorities (Figures~\ref{fig:surveillance}, \ref{fig:sar}, \ref{fig:delivery}).

\begin{figure}[h!tbp]
    \centering
    \includegraphics[width=\textwidth, height=0.9\textheight, keepaspectratio]{images/surveillance_loop.png}
    \caption{Surveillance Mission OODA Workflow}
    \label{fig:surveillance}
\end{figure}

\textbf{Surveillance missions} (Figure~\ref{fig:surveillance}) emphasize continuous area coverage with failures requiring coverage gap minimization. The workflow depicts a 120m $\times$ 120m perimeter divided into 9 zones (40m $\times$ 40m) monitored by 6 UAVs with 30-minute endurance. Priorities differentiate critical zones ($P=0.9$ for top row) from routine areas ($P=0.4$ for bottom row). Battery limitations dominate constraint considerations. When UAV-3 experiences battery anomaly (8\% vs.\ expected 55\%), the OODA cycle evaluates spare capacity (UAV-2: 15\%, UAV-4: 12\%) and executes split-zone reallocation---Zone 5 divides between UAV-2 and UAV-4, achieving 100\% coverage in 0.7 milliseconds.

\begin{figure}[H]
    \centering
    \includegraphics[width=\textwidth, height=0.9\textheight, keepaspectratio]{images/sar_loop.png}
    \caption{Search and Rescue Mission OODA Workflow}
    \label{fig:sar}
\end{figure}

\textbf{Search and rescue missions} (Figure~\ref{fig:sar}) emphasize time-critical coverage through systematic grid search under golden hour constraints. The 120m $\times$ 120m area with 9 zones is searched by 4 thermal-equipped UAVs. Zone priorities reflect distance from last known position: $P=0.9$ (closest), $P=0.6$ (corridors), $P=0.4$ (remote). Battery and time constraints dominate. When UAV-2 loses GPS in a ravine, the system reallocates Zone 3 to UAV-1 and Zone 4 to UAV-3, achieving 100\% coverage in 1.2 milliseconds. Target: 75-90\% coverage acceptable when full compensation proves infeasible.

\begin{figure}[h!tbp]
    \centering
    \includegraphics[width=\textwidth, height=0.9\textheight, keepaspectratio]{images/delivery_loop.png}
    \caption{Delivery Mission OODA Workflow}
    \label{fig:delivery}
\end{figure}

\textbf{Delivery missions} (Figure~\ref{fig:delivery}) emphasize reliable cargo transport with payload-aware reallocation. The heterogeneous 3-UAV fleet---UAV-1 (1.0 kg capacity), UAV-2 and UAV-3 (0.5 kg each)---delivers 5 packages (0.2-0.5 kg). Payload and battery constraints dominate. When UAV-1 experiences battery anomaly while carrying 0.9 kg, Package B (0.4 kg) exceeds spare capacity (UAV-2: 0.05 kg, UAV-3: 0.15 kg), triggering operator escalation. The OODA cycle completes in 0.2 milliseconds, preserving 80\% autonomous coverage while deferring the infeasible task to human judgment.

These variations demonstrate architectural flexibility through consistent OODA mechanics with mission-specific constraint emphasis.

\chapter{Core Algorithms and Technical Contributions}

\section{Priority-Based Task Scoring Algorithm}

Task reallocation following UAV failure constitutes a resource-constrained scheduling problem. Algorithm~1 formalizes multi-objective prioritization through a weighted scoring function reconciling temporal urgency, task criticality, and resource expenditure.

We first define the data structures encoding system state.

\textbf{Definition 1 (Fleet State).} The fleet state $\mathcal{F}$ is a tuple $\mathcal{F} = (\mathcal{U}, \mathcal{U}_{\text{healthy}}, \mathcal{U}_{\text{failed}})$ where:
\begin{itemize}
\item $\mathcal{U} = \{u_1, u_2, \ldots, u_n\}$ is the set of all UAVs in the fleet
\item $\mathcal{U}_{\text{healthy}} \subseteq \mathcal{U}$ is the subset of operational UAVs available for task assignment
\item $\mathcal{U}_{\text{failed}} \subseteq \mathcal{U}$ is the subset of UAVs that have experienced failures
\end{itemize}
Each UAV $u \in \mathcal{U}$ is characterized by a state vector:
\begin{equation}
u = (\text{pos}, \text{battery}, \text{committed}, \text{payload}, \text{speed}, \eta, \text{status}, \text{permissions})
\end{equation}
where $\text{pos} \in \mathbb{R}^3$ is position, $\text{battery} \in [0, 100]$ is charge percentage, $\text{committed} \in [0, 100]$ is battery allocated to pending tasks, $\text{payload} \in \mathbb{R}_{\geq 0}$ is cargo weight, $\text{speed} \in \mathbb{R}_{> 0}$ is cruise velocity, $\eta \in \mathbb{R}_{> 0}$ is energy efficiency (meters per percent battery), $\text{status} \in \{\text{idle}, \text{active}, \text{failed}, \text{returning}\}$, and $\text{permissions}$ encodes regulatory authorizations.

\textbf{Definition 2 (Mission Context).} The mission context $\mathcal{M}$ encapsulates operational parameters:
\begin{equation}
\mathcal{M} = (\text{type}, \mathcal{T}, \mathcal{G}, W, \Theta)
\end{equation}
where:
\begin{itemize}
\item $\text{type} \in \{\text{surveillance}, \text{SAR}, \text{delivery}\}$ specifies the mission category
\item $\mathcal{T} = \{t_1, t_2, \ldots, t_k\}$ is the set of tasks, where each task $t_i$ contains position, type, start time, deadline, base priority, and payload weight
\item $\mathcal{G} \subset \mathbb{R}^2$ defines the authorized operational grid boundaries, with $R_{\max} = \text{diag}(\mathcal{G})$ denoting the maximum operational range (grid diagonal, e.g., $\sqrt{120^2 + 120^2} \approx 170$m)
\item $W: \text{type} \times \text{task\_type} \to [0, 1]$ is the criticality weight lookup table
\item $\Theta = (B_{\text{reserve}}, D_{\text{safe}}, T_{\text{safe}})$ contains safety thresholds: battery reserve (20\%), spatial separation (15m), and temporal buffer (10s)
\end{itemize}

Additionally, we define the clamping function:
\begin{equation}
\text{clamp}(x, a, b) = \max(a, \min(b, x))
\end{equation}
constraining $x$ to $[a, b]$. When written as $\text{clamp}(x)$, the default interval $[0, 1]$ is assumed.

The priority scoring function uses configurable weights $w_\tau, w_c, w_\sigma \in [0,1]$ that balance temporal urgency, mission criticality, and spatial cost respectively.

\begin{table}[H]
\centering
\begin{tabular}{|p{12cm}|}
\hline
\textbf{Algorithm 1: Task Priority Calculation} \\
\hline
\textbf{Input:} Task $t_i$ with position, type, and deadline; fleet state $\mathcal{F}$; mission context $\mathcal{M}$ \\
\textbf{Output:} Priority score $P_i \in [0, 1]$ \\[6pt]
\textbf{1.} Compute temporal urgency: $\tau \gets 1 - \dfrac{t_i^{\text{deadline}} - t_{\text{now}}}{t_i^{\text{deadline}} - t_i^{\text{start}}}$ \\[6pt]
\textbf{2.} Look up mission criticality weight: $c \gets W_{\mathcal{M}.\text{type}}[t_i.\text{type}]$ \\[6pt]
\textbf{3.} Find minimum distance to healthy UAV: $d_{\min} \gets \min_{u \in \mathcal{F}.\text{healthy}} \| u.\text{pos} - t_i.\text{pos} \|$ \\[6pt]
\textbf{4.} Normalize spatial cost: $\sigma \gets d_{\min} \,/\, R_{\max}$ \\[6pt]
\textbf{5.} Combine components: $P_i \gets w_\tau \cdot \text{clamp}(\tau) + w_c \cdot c - w_\sigma \cdot \text{clamp}(\sigma)$ \\[6pt]
\textbf{6.} \textbf{return} $\text{clamp}(P_i, 0, 1)$ \\
\hline
\end{tabular}
\end{table}

The temporal urgency term $\tau$ increases as deadlines approach. The criticality weight $c$ encodes domain-specific importance---critical packages receive weight 1.0 in delivery while routine packages score 0.4; in SAR, high-probability victim zones supersede peripheral patrols. The spatial cost $\sigma$ penalizes distant tasks, incorporating battery expenditure into priority.

The default weights ($w_\tau = 0.3$, $w_c = 0.5$, $w_\sigma = 0.2$) emphasize criticality while maintaining temporal responsiveness; parameters admit mission-specific customization (Section~4.4). This extends auction-based allocation (CHOI; BRUNET; HOW, 2009) and utility-theoretic coordination (DIAS et al., 2006), with explicit integration of resource costs within the priority calculus. The foundations derive from Gerkey and Matarić's (2004) ST-SR-IA problem class.

\section{Constraint-Aware Task Reallocation with Collision Avoidance}

With tasks ranked by Algorithm~1, the challenge involves constructing feasible assignments while satisfying interdependent constraints. This task allocation problem is NP-hard, rendering exact optimization intractable for real-time fault recovery.

Algorithm~2 resolves this through a greedy strategy processing tasks in priority order, assigning each to the nearest constraint-satisfying UAV. This guarantees constraint satisfaction by construction with polynomial-time execution---a promptly executed suboptimal allocation provides greater value than an optimal solution computed after deadlines elapse.

\begin{table}[H]
\centering
\begin{tabular}{|p{12cm}|}
\hline
\textbf{Algorithm 2: Constraint-Aware Task Reallocation} \\
\hline
\textbf{Input:} Failed UAV's task queue $\mathcal{T}$; healthy UAVs $\mathcal{U}$; mission context $\mathcal{M}$ \\
\textbf{Output:} Allocation $\mathcal{A}$; coverage percentage; operator alerts \\[6pt]
\textbf{1.} \textbf{for each} $u \in \mathcal{U}$ \textbf{do} \\
\quad Compute spare battery: $b_u \gets u.\text{battery} - B_{\text{reserve}} - u.\text{committed}$ \\
\quad Compute spare payload: $p_u \gets u.\text{max\_payload} - u.\text{current\_payload}$ \\[4pt]
\textbf{2.} $\mathcal{T}_{\text{ranked}} \gets \textsc{RankByPriority}(\mathcal{T}, \mathcal{U}, \mathcal{M})$ \hfill \textit{// Algorithm 1} \\[4pt]
\textbf{3.} $\mathcal{A} \gets \emptyset$; \quad $\mathcal{T}_{\text{unalloc}} \gets \emptyset$ \\[4pt]
\textbf{4.} \textbf{for each} $(t_i, P_i) \in \mathcal{T}_{\text{ranked}}$ \textbf{do} \\
\quad Sort candidates by distance: $\mathcal{U}_{\text{sorted}} \gets \text{sort}(\mathcal{U}, \text{key}=\|u.\text{pos} - t_i.\text{pos}\|)$ \\
\quad \textbf{for each} $u \in \mathcal{U}_{\text{sorted}}$ \textbf{do} \\
\quad\quad $d \gets \|u.\text{pos} - t_i.\text{pos}\|$; \quad $b_{\text{req}} \gets d \,/\, \eta_u$ \\
\quad\quad \textbf{if} $b_u < b_{\text{req}}$ \textbf{then continue} \hfill \textit{// Battery constraint} \\
\quad\quad \textbf{if} $\mathcal{M}.\text{type} = \text{delivery} \land p_u < t_i.\text{payload}$ \textbf{then continue} \hfill \textit{// Payload constraint} \\
\quad\quad \textbf{if} $\neg\textsc{CollisionFree}(u, t_i, \mathcal{A})$ \textbf{then continue} \hfill \textit{// Algorithm 3} \\
\quad\quad $\mathcal{A} \gets \mathcal{A} \cup \{(t_i, u)\}$; \quad $b_u \gets b_u - b_{\text{req}}$; \quad \textbf{break} \\
\quad \textbf{if} $t_i \notin \mathcal{A}$ \textbf{then} $\mathcal{T}_{\text{unalloc}} \gets \mathcal{T}_{\text{unalloc}} \cup \{(t_i, P_i)\}$ \\[4pt]
\textbf{5.} coverage $\gets |\mathcal{A}| \,/\, |\mathcal{T}| \times 100\%$ \\[4pt]
\textbf{6.} \textbf{return} $\mathcal{A}$, coverage, $\textsc{GenerateAlerts}(\mathcal{T}_{\text{unalloc}})$ \\
\hline
\end{tabular}
\end{table}

The algorithm enforces constraints through hierarchical validation. Battery constraints verify sufficient energy with 20\% safety margin. Payload constraints (delivery only) ensure lift capacity. Collision avoidance (Algorithm~3) evaluates spatiotemporal compatibility between trajectories.

\begin{table}[H]
\centering
\begin{tabular}{|p{12cm}|}
\hline
\textbf{Algorithm 3: Collision-Free Path Verification} \\
\hline
\textbf{Input:} UAV $u$; candidate task $t$; current allocation $\mathcal{A}$ \\
\textbf{Output:} Boolean indicating path safety \\[6pt]
\textbf{Constants:} $D_{\text{safe}} = 15\text{m}$ (spatial buffer); $T_{\text{safe}} = 10\text{s}$ (temporal buffer) \\[6pt]
\textbf{1.} $\pi_{\text{new}} \gets \textsc{GeneratePath}(u.\text{pos}, t.\text{pos})$ \\
\textbf{2.} $\tau_{\text{new}} \gets \textsc{EstimateTimeline}(\pi_{\text{new}}, u.\text{speed})$ \\[4pt]
\textbf{3.} \textbf{for each} $(t', u') \in \mathcal{A}$ where $u' \neq u$ \textbf{do} \\
\quad $\pi_{\text{other}} \gets \textsc{GetPlannedPath}(u')$ \\
\quad $\tau_{\text{other}} \gets \textsc{GetTimeline}(u')$ \\
\quad \textbf{for each} $(p_1, t_1) \in (\pi_{\text{new}}, \tau_{\text{new}})$ \textbf{do} \\
\quad\quad \textbf{for each} $(p_2, t_2) \in (\pi_{\text{other}}, \tau_{\text{other}})$ \textbf{do} \\
\quad\quad\quad \textbf{if} $\|p_1 - p_2\| < D_{\text{safe}} \land |t_1 - t_2| < T_{\text{safe}}$ \textbf{then return} \textsc{False} \\[4pt]
\textbf{4.} \textbf{return} \textsc{True} \\
\hline
\end{tabular}
\end{table}

Algorithm~3 implements spatiotemporal conflict detection derived from velocity obstacle methods (VAN DEN BERG; LIN; MANOCHA, 2008). Conflicts arise when trajectories bring UAVs within 15 meters during overlapping time windows. Upon conflict, the candidate is rejected and the next-nearest UAV evaluated; when no conflict-free assignment exists, departure delay scheduling provides deconfliction.

\section{Operator Escalation Decision Rules}

A distinguishing feature is honest self-assessment: the system recognizes when human judgment is required. Algorithm~4 encodes escalation logic as a decision tree evaluating degradation severity.

\begin{table}[H]
\centering
\begin{tabular}{|p{12cm}|}
\hline
\textbf{Algorithm 4: Operator Escalation Decision} \\
\hline
\textbf{Input:} Coverage percentage $\rho$; unallocated tasks $\mathcal{T}_{\text{unalloc}}$ with priorities \\
\textbf{Output:} Escalation decision with urgency level and recommendation \\[6pt]
\textbf{1.} \textbf{if} $\rho < 50\%$ \textbf{then} \\
\quad \textbf{return} (escalate=\textsc{True}, urgency=\textsc{High}, \\
\quad\quad reason=``Coverage below 50\% threshold'', \\
\quad\quad recommendation=``Deploy backup UAV or abort mission'') \\[4pt]
\textbf{2.} $n_{\text{critical}} \gets |\{t \in \mathcal{T}_{\text{unalloc}} : P_t > 0.7\}|$ \\
\quad \textbf{if} $n_{\text{critical}} > 0$ \textbf{then} \\
\quad \textbf{return} (escalate=\textsc{True}, urgency=\textsc{High}, \\
\quad\quad reason=``$n_{\text{critical}}$ critical tasks unassignable'', \\
\quad\quad recommendation=``Manual prioritization required'') \\[4pt]
\textbf{3.} \textbf{if} $\rho < 75\%$ \textbf{then} \\
\quad \textbf{return} (escalate=\textsc{True}, urgency=\textsc{Medium}, \\
\quad\quad reason=``Moderate degradation (50--75\% coverage)'', \\
\quad\quad recommendation=``Monitor; consider manual reallocation'') \\[4pt]
\textbf{4.} \textbf{return} (escalate=\textsc{False}, urgency=\textsc{Low}, \\
\quad reason=``Acceptable autonomous compensation ($>$75\%)'', \\
\quad recommendation=``Continue autonomous operation'') \\
\hline
\end{tabular}
\end{table}

Coverage below 50\% indicates the failure exceeds fleet capacity---continuing autonomously would produce unacceptable outcomes. Any unallocated high-priority task ($P > 0.7$) triggers immediate escalation. Moderate degradation (50--75\%) warrants operator awareness without demanding intervention. Above 75\%, the system proceeds autonomously.

This graduated response reserves operator attention for consequential decisions while handling routine failures autonomously.

\section{Objective Function and Optimization Strategy}

The preceding algorithms define \textit{how} tasks are allocated but not \textit{what constitutes a good allocation}. This section formalizes the objective function guiding the DECIDE phase.

\subsection{Allocation Quality Metric}

The quality of a task reallocation is quantified through objective function $J(\mathcal{A})$:

\textbf{Problem Statement.} Given orphaned tasks $\mathcal{T}_{\text{failed}} = \{t_1, \ldots, t_k\}$ and healthy UAVs $\mathcal{U}_{\text{healthy}} = \{u_1, \ldots, u_m\}$, find allocation $\mathcal{A}^* \subseteq \mathcal{T}_{\text{failed}} \times \mathcal{U}_{\text{healthy}}$ maximizing mission value under constraints.

\textbf{Decision Variable.} The allocation $\mathcal{A} = \{(t_i, u_j)\}$ assigns tasks to UAVs. Each task is assigned to at most one UAV.

\textbf{Objective Function.} The allocation quality is quantified as:

\begin{equation}
\mathcal{A}^* = \arg\max_{\mathcal{A}} J(\mathcal{A}) = \arg\max_{\mathcal{A}} \left\{ \sum_{(t_i, u_j) \in \mathcal{A}} \left[ P_i \cdot \phi_m(t_i, u_j) \right] - \lambda \cdot |\mathcal{T}_{\text{unalloc}}| \right\}
\label{eq:objective}
\end{equation}

\noindent where:
\begin{itemize}
\item $P_i \in [0,1]$ is the priority score of task $t_i$ computed by Algorithm 1
\item $\phi_m(t_i, u_j) \in [0,1]$ is a mission-specific modifier function (defined in Equation~\ref{eq:modifier})
\item $\lambda > 0$ is the penalty weight for unallocated tasks
\item $\mathcal{T}_{\text{unalloc}} = \mathcal{T}_{\text{failed}} \setminus \{t_i : (t_i, u_j) \in \mathcal{A}\}$ is the set of tasks that could not be feasibly assigned
\end{itemize}

\textbf{Constraint Set.} The allocation $\mathcal{A}$ must satisfy the following constraints for each assignment $(t_i, u_j) \in \mathcal{A}$:

\begin{enumerate}
\item \textbf{Battery constraint:} The UAV must retain sufficient energy to reach the task and return safely:
\begin{equation}
b_{u_j} - \frac{\|u_j.\text{pos} - t_i.\text{pos}\|}{\eta_{u_j}} \geq B_{\text{reserve}}
\end{equation}
where $b_{u_j}$ is the current battery level, $\eta_{u_j}$ is energy efficiency, and $B_{\text{reserve}} = 20\%$ is the safety margin.

\item \textbf{Payload constraint} (delivery missions only): The UAV must have sufficient spare capacity:
\begin{equation}
p_{u_j}^{\text{max}} - p_{u_j}^{\text{current}} \geq t_i.\text{payload}
\end{equation}

\item \textbf{Spatial constraint:} The task location must be within authorized boundaries or the UAV must hold appropriate permissions:
\begin{equation}
t_i.\text{pos} \in \mathcal{G} \quad \lor \quad \text{``out-of-grid''} \in u_j.\text{permissions}
\end{equation}

\item \textbf{Collision avoidance constraint:} The proposed trajectory must maintain safe separation from all other UAVs (verified by Algorithm 3).

\item \textbf{Uniqueness constraint:} Each task is assigned to at most one UAV:
\begin{equation}
\forall t_i \in \mathcal{T}_{\text{failed}}: |\{u_j : (t_i, u_j) \in \mathcal{A}\}| \leq 1
\end{equation}
\end{enumerate}

\textbf{Solution Method.} This NP-hard problem precludes exact optimization within 1.0--1.5 seconds. The system employs \textbf{greedy heuristic with local search} (Algorithm 5), achieving 70--85\% of optimal value in sub-millisecond time.

The mission-specific modifier $\phi_m$ adjusts task value based on context:
\begin{itemize}
\item $\Delta t_{\text{gap}}(t_i)$ --- time elapsed since the last coverage of task $t_i$'s zone (normalized to $[0, 1]$)
\item $t_{\text{golden}}$ --- the golden hour duration for SAR missions (typically 60 minutes)
\item $t_{\text{completion}}(t_i, u_j)$ --- estimated completion time if task $t_i$ is assigned to UAV $u_j$
\item $t_{\text{deadline}}$ --- the delivery deadline for task $t_i$
\end{itemize}

\begin{equation}
\phi_m(t_i, u_j) =
\begin{cases}
1 - \gamma \cdot \Delta t_{\text{gap}}(t_i) & \text{if } m = \text{surveillance} \\[6pt]
1 + \beta \cdot \dfrac{t_{\text{golden}} - t_{\text{completion}}(t_i, u_j)}{t_{\text{golden}}} & \text{if } m = \text{SAR} \\[6pt]
\begin{cases}
1.0 & \text{if } t_{\text{completion}} \leq t_{\text{deadline}} \\
0.5 & \text{otherwise}
\end{cases} & \text{if } m = \text{delivery}
\end{cases}
\label{eq:modifier}
\end{equation}

For surveillance, $\gamma$ penalizes coverage gaps. For SAR, $\beta$ provides a bonus for completion before golden hour. For delivery, a binary penalty applies for deadline violations.

\subsection{Optimization Strategy}

The DECIDE phase operates under 1.0--1.5 second constraints, precluding exact methods like MILP. The system employs two-stage optimization:

\textbf{Stage 1: Greedy Initialization.} Algorithm 2 generates a feasible allocation in $O(n \cdot m)$ time, achieving 70--85\% of optimal value.

\textbf{Stage 2: Local Search Refinement.} When time permits (~500ms remaining), iterative local search explores pairwise task swaps between UAVs.

\begin{table}[H]
\centering
\begin{tabular}{|p{12cm}|}
\hline
\textbf{Algorithm 5: Two-Stage Optimization} \\
\hline
\textbf{Input:} Failed tasks $\mathcal{T}$; healthy UAVs $\mathcal{U}$; mission context $\mathcal{M}$; time budget $T_{\max}$ \\
\textbf{Output:} Optimized allocation $\mathcal{A}^*$ with objective score $J^*$ \\[6pt]
\textbf{Stage 1: Greedy Initialization} \\
\textbf{1.} $t_0 \gets \textsc{CurrentTime}()$ \\
\textbf{2.} $\mathcal{A} \gets \textsc{GreedyAllocate}(\mathcal{T}, \mathcal{U}, \mathcal{M})$ \hfill \textit{// Algorithm 2} \\
\textbf{3.} $J_{\text{best}} \gets \textsc{ComputeObjective}(\mathcal{A}, \mathcal{M})$ \\[4pt]
\textbf{Stage 2: Local Search Refinement} \\
\textbf{4.} \textbf{while} $\textsc{CurrentTime}() - t_0 < T_{\max} - T_{\text{reserve}}$ \textbf{do} \\
\quad $\text{improved} \gets \textsc{False}$ \\
\quad \textbf{for each} $(t_1, u_1), (t_2, u_2) \in \mathcal{A} \times \mathcal{A}$ where $u_1 \neq u_2$ \textbf{do} \\
\quad\quad $\mathcal{A}' \gets \mathcal{A}$ with swap: $t_1 \to u_2$, $t_2 \to u_1$ \\
\quad\quad \textbf{if} $\textsc{IsFeasible}(\mathcal{A}', \mathcal{M})$ \textbf{then} \\
\quad\quad\quad $J' \gets \textsc{ComputeObjective}(\mathcal{A}', \mathcal{M})$ \\
\quad\quad\quad \textbf{if} $J' > J_{\text{best}}$ \textbf{then} $\mathcal{A} \gets \mathcal{A}'$; $J_{\text{best}} \gets J'$; $\text{improved} \gets \textsc{True}$; \textbf{break} \\
\quad \textbf{if} $\neg\text{improved}$ \textbf{then break} \hfill \textit{// Local optimum reached} \\[4pt]
\textbf{5.} \textbf{return} $\mathcal{A}$, $J_{\text{best}}$ \\
\hline
\end{tabular}
\end{table}

The algorithm terminates upon reaching a local optimum or exhausting the time budget. A 200ms reserve ensures time for finalization.

\subsection{Optimality Gap Analysis}

The strategy does not guarantee global optimality. The optimality gap is:

\begin{equation}
\text{Gap} = \frac{J^* - J(\mathcal{A}_{\text{heuristic}})}{J^*} \times 100\%
\label{eq:gap}
\end{equation}

\noindent where $J^*$ is optimal value computed offline. Analysis indicates expected gaps of 5--15\% for typical failures (3--6 lost tasks, 4--8 healthy UAVs). A 10\% suboptimal solution in 1.2 seconds provides greater value than optimal requiring 30+ seconds.

\subsection{Mission-Specific Parameter Configuration}

Objective function parameters are configured per mission type (Table~\ref{tab:objective_params}):

\begin{table}[H]
\centering
\caption{Objective Function Parameters by Mission Type}
\label{tab:objective_params}
\begin{tabular}{|l|c|c|c|c|}
\hline
\rowcolor{gray!30}
\textbf{Parameter} & \textbf{Symbol} & \textbf{Surveillance} & \textbf{SAR} & \textbf{Delivery} \\
\hline
Unallocated penalty & $\lambda$ & 0.3 & 0.5 & 0.4 \\
\rowcolor{gray!15}
Coverage gap weight & $\gamma$ & 0.2 & --- & --- \\
Golden hour bonus & $\beta$ & --- & 0.5 & --- \\
\rowcolor{gray!15}
Priority weight (temporal) & $w_{\text{temporal}}$ & 0.3 & 0.5 & 0.2 \\
Priority weight (criticality) & $w_{\text{criticality}}$ & 0.5 & 0.3 & 0.6 \\
\rowcolor{gray!15}
Priority weight (spatial) & $w_{\text{spatial}}$ & 0.2 & 0.2 & 0.2 \\
\hline
\end{tabular}
\end{table}

This parameterization enables the OODA loop to adapt optimization behavior based on mission context without algorithmic modifications.

\chapter{Mission Scenarios and Performance Analysis}

This chapter presents three mission scenarios demonstrating OODA-based fault tolerance across diverse operational contexts.

\section{Scenario Selection Rationale}

The three scenarios were selected to span the constraint space of real-world UAV operations:

\begin{itemize}
\item \textbf{Surveillance:} Battery-constrained with time-critical coverage requirements
\item \textbf{Search \& Rescue:} Time-critical with priority-driven partial coverage acceptance
\item \textbf{Delivery:} Payload-constrained with heterogeneous fleet capabilities
\end{itemize}

Each scenario represents a distinct operational domain with real-world deployment precedents.

\section{SCENARIO 1: Long-Duration Perimeter Surveillance}

Perimeter surveillance represents a canonical application of aerial monitoring with established multi-UAV coordination methodologies (BEARD et al., 2006).

\subsection{Mission Context}

\begin{itemize}
    \item \textbf{Application:} Critical infrastructure monitoring (port, airport, border)
    \item \textbf{Duration:} 2-hour continuous coverage requirement
    \item \textbf{Fleet:} 6 UAVs with 30-minute flight time each (requires rotation strategy)
    \item \textbf{Operational Area:} 120m $\times$ 120m secured perimeter
    \item \textbf{Coverage Strategy:} 9 patrol zones (40m $\times$ 40m each) in a 3$\times$3 grid
\end{itemize}

\subsection{Mission Setup}

The mission partitions the 120m $\times$ 120m area into nine patrol zones (3$\times$3 grid, 40m $\times$ 40m each) as shown in Figure~\ref{fig:operational_environment}. The mismatch between mission duration (2 hours) and vehicle endurance (30 minutes) necessitates coordinated rotation with 12 UAVs in alternating shifts.

\begin{figure}[H]
\centering
\includegraphics[width=0.7\textwidth]{images/uav_zone_assignement.png}
\caption{Operational Environment for Experimental Scenarios}
\label{fig:operational_environment}
\end{figure}

\subsection{Patrol Zone Specifications}

Zone prioritization reflects threat assessment: Zones 1--3 (northern perimeter, main entry points) receive $P = 0.9$; Zones 4--6 (central access routes) have $P = 0.6$; Zones 7--9 (low-risk southern boundary) operate at $P = 0.4$.

Priority differentiation manifests in patrol circuit complexity (ELMALIACH; AGMON; KAMINKA, 2009): high-priority zones require eight waypoints, medium six, and standard four. Waypoint density influences energy expenditure---critical during failure recovery.

\subsection{Rotation Strategy}

Sustaining 2-hour coverage with 30-minute endurance requires four consecutive waves with overlapping transitions. Wave 1 deploys UAVs 1--6; as vehicles approach 20\% battery (~25 minutes), Wave 2 (UAVs 7--12) assumes responsibilities. This pattern alternates through Waves 3--4. The 5-minute transition overlap prevents coverage gaps while introducing collision avoidance complexity.

\subsection{Failure Scenario: Mid-Mission Battery Depletion}

At T+45 minutes, UAV-3's battery drops to 8\% (expected: 35\%)---below safety threshold. The vehicle is mid-patrol in Zone 5; returning leaves the zone uncovered, continuing risks forced landing.

% \begin{figure}[H]
% \centering
% \includegraphics[width=0.8\textwidth]{images/surveillance_loop.png}
% \caption{Surveillance Mission OODA Loop Execution}
% \label{fig:surveillance_loop}
% \end{figure}

\subsubsection{OBSERVE Phase (T+0 to T+1.5s)}

\textbf{Failure Detection:}

The system detects battery anomalies through statistical monitoring. When discharge rate exceeds 1.5$\times$ expected threshold, the OODA cycle triggers (Table~\ref{tab:s5_anomaly}).

\begin{table}[H]
\centering
\small
\begin{tabular}{|l|c|c|c|}
\hline
\textbf{Parameter} & \textbf{Expected} & \textbf{Observed} & \textbf{Status} \\
\hline
Discharge Rate & 3\%/min & 8\%/min & \cellcolor{red!25}ANOMALY \\
Battery Level & 55\% & 8\% & \cellcolor{red!25}CRITICAL \\
Threshold & --- & 1.5$\times$ baseline & \cellcolor{red!25}EXCEEDED \\
\hline
\end{tabular}
\caption{UAV-3 Battery Anomaly Detection}
\label{tab:s5_anomaly}
\end{table}

\textbf{Fleet State Aggregation:}

The system aggregates fleet state to assess available resources (Table~\ref{tab:s5_fleet_state}).

\begin{table}[H]
\centering
\small
\begin{tabular}{|l|c|c|c|c|}
\hline
\textbf{UAV} & \textbf{Battery} & \textbf{Zone} & \textbf{Spare Capacity} & \textbf{Status} \\
\hline
UAV-3 & 8\% & 5 & 0\% & \cellcolor{red!25}FAILED \\
UAV-2 & 45\% & 2 & 15\% & \cellcolor{green!25}AVAILABLE \\
UAV-4 & 40\% & 6 & 12\% & \cellcolor{green!25}AVAILABLE \\
UAV-5 & 80\% & 8 & 35\% & \cellcolor{blue!25}AVAILABLE \\
\hline
\multicolumn{5}{|l|}{\textit{Lost Coverage: Zone 5 (11.1\% of mission area)}} \\
\hline
\end{tabular}
\caption{Fleet State at Failure Detection (T+1.5s)}
\label{tab:s5_fleet_state}
\end{table}

\subsubsection{ORIENT Phase (T+1.5s to T+3.0s)}

\textbf{Impact Assessment:}

The system evaluates mission impact (Table~\ref{tab:s5_impact}).

\begin{table}[H]
\centering
\small
\begin{tabular}{|l|p{8cm}|}
\hline
\textbf{Dimension} & \textbf{Assessment} \\
\hline
Mission Criticality & Medium (Zone 5 is medium-priority area, $P = 0.6$) \\
Coverage Gap & 11.1\% of total mission area lost \\
Temporal Urgency & $P_{\text{time}} = 0.8$ (continuous coverage mandate) \\
Response Required & Immediate reallocation (0-second tolerance) \\
\hline
\end{tabular}
\caption{Mission Impact Assessment}
\label{tab:s5_impact}
\end{table}

\textbf{Capacity Analysis:}

Spare capacity = current battery $-$ 20\% reserve $-$ committed energy (Table~\ref{tab:s5_capacity}).

\begin{table}[H]
\centering
\small
\begin{tabular}{|l|c|c|c|c|c|}
\hline
\textbf{UAV} & \textbf{Current} & \textbf{Safety} & \textbf{Committed} & \textbf{Spare} & \textbf{Distance} \\
 & \textbf{Battery} & \textbf{Reserve} & \textbf{Energy} & \textbf{Capacity} & \textbf{to Zone 5} \\
\hline
UAV-2 & 45\% & 20\% & 10\% & \textbf{15\%} & 40m (req: 3\%) \\
UAV-4 & 40\% & 20\% & 8\% & \textbf{12\%} & 40m (req: 3\%) \\
\hline
\end{tabular}
\caption{Spare Capacity Analysis for Zone 5 Reallocation}
\label{tab:s5_capacity}
\end{table}

\textbf{Feasibility Determination:}

Both candidates satisfy energy requirements (Table~\ref{tab:s5_feasibility}).

\begin{table}[H]
\centering
\small
\begin{tabular}{|l|c|c|c|}
\hline
\textbf{UAV} & \textbf{Spare Capacity} & \textbf{Required Energy} & \textbf{Feasible?} \\
\hline
UAV-2 & 15\% & 3\% & \cellcolor{green!25}\textbf{YES} (12\% margin) \\
UAV-4 & 12\% & 3\% & \cellcolor{green!25}\textbf{YES} (9\% margin) \\
\hline
\end{tabular}
\caption{Constraint Feasibility Check}
\label{tab:s5_feasibility}
\end{table}

\textbf{Reallocation Strategy:}

The system adopts a split-zone strategy:

\begin{itemize}
\item Split Zone 5 into two sub-zones: 5A (north 20m) and 5B (south 20m)
\item Assign 5A to UAV-2 (adjacent from Zone 2, 40m distance)
\item Assign 5B to UAV-4 (adjacent from Zone 6, 40m distance)
\item Accept reduced waypoint density: 3 waypoints per sub-zone vs. 6 original
\item Coverage degradation: 8.3\% surveillance quality reduction in Zone 5
\end{itemize}

\subsubsection{DECIDE Phase (T+3.0s to T+4.5s)}

\textbf{Strategy Selection:} Partial Reallocation (coverage degradation accepted)

\textbf{Task Allocation Plan:}

Extended patrol routes integrate new waypoints (Table~\ref{tab:s5_allocation}).

\begin{table}[H]
\centering
\small
\begin{tabular}{|l|c|c|c|c|}
\hline
\textbf{UAV} & \textbf{Original Zone} & \textbf{Added Zone} & \textbf{Total Waypoints} & \textbf{Circuit Time} \\
\hline
UAV-2 & Zone 2 (8 pts) & Zone 5A (3 pts) & 11 points & 120s (\textasciitilde2 min) \\
UAV-4 & Zone 6 (6 pts) & Zone 5B (3 pts) & 9 points & 105s (\textasciitilde1.7 min) \\
\hline
\multicolumn{5}{|l|}{\textit{Total distance UAV-2: 100m | Total distance UAV-4: 100m}} \\
\hline
\end{tabular}
\caption{Extended Patrol Route Allocation}
\label{tab:s5_allocation}
\end{table}

\textbf{Constraint Verification:}

Safety constraint verification (Table~\ref{tab:s5_constraints}):

\begin{table}[H]
\centering
\small
\begin{tabular}{|l|c|c|}
\hline
\textbf{Constraint Type} & \textbf{Verification Result} & \textbf{Details} \\
\hline
Battery Safety & \cellcolor{green!25}PASS & Both UAVs maintain $>$20\% reserve \\
Spatial Separation & \cellcolor{green!25}PASS & 15m buffer between 5A/5B boundary \\
Collision Avoidance & \cellcolor{green!25}PASS & Opposite patrol directions (temporal deconfliction) \\
Zone Conflicts & \cellcolor{green!25}PASS & No overlap with Zones 1, 2, 3, 4, 6, 7, 8, 9 \\
\hline
\end{tabular}
\caption{Safety Constraint Verification}
\label{tab:s5_constraints}
\end{table}

\subsubsection{ACT Phase (T+4.5s to T+5.5s)}

\textbf{Command Execution:}

Commands dispatched (Table~\ref{tab:s5_commands}):

\begin{table}[H]
\centering
\small
\begin{tabular}{|l|l|p{6cm}|}
\hline
\textbf{Target} & \textbf{Command Type} & \textbf{Payload} \\
\hline
UAV-2 & Mission Update & Extended waypoints: Zone 2 + 5A (11 points) \\
 & & Priority: HIGH | Timeout: 200ms \\
\hline
UAV-4 & Mission Update & Extended waypoints: Zone 6 + 5B (9 points) \\
 & & Priority: HIGH | Timeout: 200ms \\
\hline
UAV-3 & Emergency RTL & Immediate return-to-launch command \\
 & & Reason: Battery critical (8\%) \\
\hline
\end{tabular}
\caption{Command Dispatch Summary}
\label{tab:s5_commands}
\end{table}

\textbf{System State Update:}

Post-adaptation status (Table~\ref{tab:s5_status}):

\begin{table}[H]
\centering
\small
\begin{tabular}{|l|c|}
\hline
\textbf{Status Dimension} & \textbf{Value} \\
\hline
Coverage Recovery & 100\% (Zone 5 split between UAV-2 and UAV-4) \\
Mission Status & ADAPTED -- Partial Reallocation \\
Surveillance Quality & 91.7\% (reduced waypoint density in Zone 5) \\
Alert Level & \cellcolor{yellow!25}CAUTION -- Coverage degraded \\
Operator Action & None required (autonomous recovery successful) \\
\hline
\end{tabular}
\caption{Post-Adaptation Mission Status}
\label{tab:s5_status}
\end{table}

\textbf{Impact Summary:}
\begin{itemize}
\item Zone 5 coverage degraded: 3 waypoints per sub-zone (previously 6)
\item Estimated surveillance quality reduction: 8.3\% in Zone 5 only
\item Overall mission completion: 100\% spatial coverage maintained
\item No operator escalation required
\end{itemize}

\subsection{Scenario Outcome}

The OODA system achieved complete coverage recovery in 0.7 milliseconds while maintaining all safety constraints. Without adaptation, Zone 5 would remain unmonitored---88.9\% coverage degradation.

Limitations: merging Zone 5 reduces waypoint density, potentially degrading detection probability. The system remains vulnerable to cascading failures; beyond two simultaneous failures, manual intervention becomes unavoidable.

\section{SCENARIO 2: Emergency Search \& Rescue}

\subsection{Mission Context}

SAR operations are time-critical: the ``golden hour'' principle means survival probability decreases with elapsed time. This scenario models a missing person search across 120m $\times$ 120m (3$\times$3 grid) with four thermal-equipped UAVs and a 3-hour search window.

\subsection{Mission Setup}

SAR prioritization follows probabilistic reasoning based on terrain analysis, last known position (LKP), and probability-of-area (POA) calculations. Behavioral studies indicate predictable patterns (seeking water, following trails), concentrating resources where success is most likely.

\subsection{Search Grid Prioritization}

Algorithm 1 adapts to SAR by weighting temporal urgency heavily. Priority tiers based on LKP distance:

\begin{itemize}
\item \textbf{High ($P = 0.9$)}: Zones 1--3 (top row, closest to LKP)
\item \textbf{Medium ($P = 0.6$)}: Zones 4--6 (middle row, travel corridors)
\item \textbf{Low ($P = 0.4$)}: Zones 7--9 (bottom row, farthest from LKP)
\end{itemize}

\subsection{Initial Task Allocation}

The four-UAV fleet divides the 9-zone grid by priority (Table~\ref{tab:sar_initial_allocation}).

\begin{table}[H]
\centering
\small
\begin{tabular}{|l|c|c|c|}
\hline
\textbf{UAV} & \textbf{Assigned Zones} & \textbf{Priority} & \textbf{Est. Time} \\
\hline
UAV-1 & Zones 1, 2 (top-left, top-center) & \cellcolor{red!25}HIGH (0.9) & 8 min \\
UAV-2 & Zones 3, 4 (top-right, mid-left) & \cellcolor{red!25}HIGH/MED (0.9/0.6) & 8 min \\
UAV-3 & Zones 5, 6 (mid-center, mid-right) & \cellcolor{yellow!25}MEDIUM (0.6) & 8 min \\
UAV-4 & Zones 7, 8, 9 (bottom row) & \cellcolor{green!25}LOW (0.4) & 12 min \\
\hline
\multicolumn{4}{|l|}{\textit{Coverage rate: 1 zone per 4 minutes (thermal scan + image capture)}} \\
\hline
\end{tabular}
\caption{Initial Search Grid Allocation by Priority}
\label{tab:sar_initial_allocation}
\end{table}

% \begin{figure}[H]
% \centering
% \includegraphics[width=0.8\textwidth]{images/sar_loop.png}
% \caption{Search and Rescue Mission OODA Loop}
% \label{fig:sar_loop}
% \end{figure}

\subsection{Failure Scenario: Critical Zone UAV Loss}

At T+8 minutes, UAV-2 drops off network while sweeping Zones 3 (HIGH) and 4 (MEDIUM). GPS vanished in a ravine; after 90 seconds without updates, failsafe activates RTL. Two unsearched zones including one high-priority area are abandoned. The remaining fleet must absorb this loss immediately.

\subsubsection{OODA Cycle Execution}

The OODA cycle adapts for time-critical operations where golden hour constraint dominates.

\textbf{OBSERVE:} GPS signal loss triggers failsafe after 90-second timeout. Fleet state: UAV-1 at 75\% (20\% spare), UAV-3 at 80\% (35\% spare), UAV-4 at 78\% (30\% spare). Two zones unsearched (Zones 3 and 4).

\textbf{ORIENT:} 22\% coverage degradation with 52 minutes remaining in golden hour. UAV-1 absorbs Zone 3 (HIGH), UAV-3 absorbs Zone 4 (MEDIUM).

\textbf{DECIDE:} Full reallocation---UAV-1 expands to 3 zones, UAV-3 to 3 zones. All UAVs maintain >20\% reserves.

\textbf{ACT:} 100\% zone coverage recovered (9/9 zones assigned).

\subsection{Scenario Outcome}

The OODA system achieved 100\% coverage recovery in 1.2 milliseconds (R5) and 0.3 milliseconds (R6), consuming only 0.00003\% of golden hour. Without adaptation, coverage degrades to 77.8\%. Scenario R6 validates the permission system: UAV-4's out-of-grid capability is correctly leveraged when required.

\section{SCENARIO 3: Package Delivery}

\subsection{Mission Context}

\begin{itemize}
    \item \textbf{Application:} Package delivery to distributed destinations with payload constraints
    \item \textbf{Duration:} 1-hour delivery window
    \item \textbf{Fleet:} 3 UAVs (heterogeneous payload capacities)
    \item \textbf{Operational Area:} 120m $\times$ 120m (3$\times$3 grid, 40m $\times$ 40m zones)
    \item \textbf{Delivery Points:} 5 destinations distributed across grid zones
\end{itemize}

\subsection{Mission Setup}

The heterogeneous fleet comprises one heavy-lifter and two standard vehicles (Table~\ref{tab:delivery_fleet}).

\begin{table}[H]
\centering
\small
\begin{tabular}{|l|c|c|c|c|c|}
\hline
\textbf{UAV} & \textbf{Type} & \textbf{Payload} & \textbf{Battery} & \textbf{Speed} & \textbf{Route} \\
\hline
UAV-1 & Heavy Lifter & 1.0 kg & 25 min & 12 m/s & Dest. 1-2 \\
UAV-2 & Standard & 0.5 kg & 30 min & 15 m/s & Dest. 3-4 \\
UAV-3 & Standard & 0.5 kg & 30 min & 15 m/s & Dest. 5 \\
\hline
\end{tabular}
\caption{Delivery Fleet Configuration}
\label{tab:delivery_fleet}
\end{table}

Package priorities based on urgency and deadlines (Table~\ref{tab:delivery_packages}).

\begin{table}[H]
\centering
\small
\begin{tabular}{|c|c|c|c|c|c|}
\hline
\textbf{Pkg} & \textbf{Contents} & \textbf{Weight} & \textbf{Destination} & \textbf{Deadline} & \textbf{Priority} \\
\hline
A & Electronics & 0.5 kg & Dest. 1, Zone 1 (20, 100) & 30 min & \cellcolor{red!25}1.0 (CRITICAL) \\
B & Components & 0.4 kg & Dest. 2, Zone 3 (100, 100) & 45 min & \cellcolor{yellow!25}0.7 (HIGH) \\
C & Tools & 0.25 kg & Dest. 3, Zone 2 (60, 100) & 60 min & 0.4 (MEDIUM) \\
D & Supplies & 0.2 kg & Dest. 4, Zone 6 (100, 60) & 60 min & 0.4 (MEDIUM) \\
E & Parts & 0.35 kg & Dest. 5, Zone 8 (60, 20) & 90 min & 0.2 (LOW) \\
\hline
\end{tabular}
\caption{Package Prioritization by Urgency Level}
\label{tab:delivery_packages}
\end{table}

\subsection{Failure Scenario: Heavy Lifter Battery Anomaly}

At T+15 minutes, UAV-1 fights unexpected headwind while carrying 0.9 kg of time-sensitive cargo. Payload mass and wind drain battery nearly twice as fast as planned.

Battery at 40\% (projected: 55\%). UAV-1 cannot complete its route safely. It carries Package A (priority 1.0) and Package B (priority 0.7).

% \begin{figure}[H]
% \centering
% \includegraphics[width=0.8\textwidth]{images/delivery_loop.png}
% \caption{Delivery Mission OODA Loop}
% \label{fig:delivery_loop}
% \end{figure}

\subsubsection{OODA Cycle Execution}

This scenario demonstrates intelligent escalation when autonomous recovery is physically impossible.

\textbf{OBSERVE:} Battery anomaly triggers when discharge rate (5\%/min) exceeds baseline by 1.5$\times$. Current 40\% cannot support full route---Package A deliverable, Package B infeasible (return would deplete to 16.6\%).

\textbf{ORIENT:} No autonomous solution exists. UAV-2 has 0.05 kg spare, UAV-3 has 0.15 kg---neither can absorb Package B's 0.4 kg. Alternatives: return-to-base swap (10--15 min), backup UAV-4 (13 min, recommended).

\textbf{DECIDE:} Algorithm 4 determines autonomous recovery infeasible (Table~\ref{tab:delivery_escalation_decision}). Coverage 80\%, but one critical task lost. Operator escalation with 30-second auto-failsafe.

\textbf{ACT:} UAV-1 receives modified route (Package A only, then RTL). Operator interface presents backup options. 80\% autonomous coverage preserved, zero violations.

\begin{table}[H]
\centering
\small
\begin{tabular}{|l|c|c|}
\hline
\textbf{Decision Criterion} & \textbf{Threshold} & \textbf{Status} \\
\hline
Coverage Recovery & >75\% & \cellcolor{green!25}80\% (4/5 packages) \\
Critical Tasks Lost ($P > 0.7$) & 0 & \cellcolor{red!25}1 (Package B) \\
Constraint Violations & 0 & \cellcolor{red!25}1 (payload) \\
Autonomous Feasibility & TRUE & \cellcolor{red!25}\textbf{FALSE} \\
\hline
\multicolumn{3}{|l|}{\textbf{Decision:} Escalate to operator (critical task unrecoverable)} \\
\hline
\end{tabular}
\caption{Operator Escalation Decision Matrix}
\label{tab:delivery_escalation_decision}
\end{table}

\subsection{Scenario Outcome}

The delivery scenarios demonstrate constraint-aware design. In D6, Package B (0.4 kg) exceeds spare capacity (maximum 0.15 kg); in D7, destination lies outside operational grid. Both identified as infeasible in 0.2--0.4 ms, escalating while preserving zero violations.

A greedy baseline would overload UAV-2 to 0.85 kg against 0.5 kg capacity---causing mid-flight failure. The OODA system's 0\% autonomous coverage represents refusing plans that physics cannot execute. The hybrid approach (80\% autonomous, 20\% supervised) proves more deployable.

\section{Cross-Scenario Comparative Analysis}

Table~\ref{tab:scenario_performance} synthesizes performance metrics across all three mission types.

\begin{table}[H]
\centering
\small
\caption{Measured Performance by Scenario}
\label{tab:scenario_performance}
\begin{tabular}{|l|c|c|c|}
\hline
\textbf{Metric} & \textbf{Surveillance (S5)} & \textbf{SAR (R5/R6)} & \textbf{Delivery (D6/D7)} \\
\hline
Coverage Recovery & \textbf{100\%} & \textbf{100\%} & \textbf{0\% (escalated)} \\
OODA Time R5/D6 (mean $\pm$ std) & \textbf{0.42 $\pm$ 0.15 ms} & \textbf{0.88 $\pm$ 0.56 ms} & \textbf{0.18 $\pm$ 0.21 ms} \\
OODA Time R6/D7 (mean $\pm$ std) & --- & \textbf{0.24 $\pm$ 0.15 ms} & \textbf{0.10 $\pm$ 0.03 ms} \\
Operator Escalation & 0\% (autonomous) & 0\% (autonomous) & 100\% (correct) \\
Constraint Violations & \textbf{0} (N=30) & \textbf{0} (N=30) & \textbf{0} (N=30) \\
Dominant Constraint & Battery & Time (golden hour) & Payload \\
\hline
\end{tabular}
\end{table}

Surveillance and SAR achieved full autonomous recovery within fleet capacity. Delivery correctly escalated when payload constraints made recovery impossible---intelligent escalation is a feature, not a failure mode.

Statistical validation (N=30 per scenario) confirms sub-millisecond computation (0.10--0.88 ms range, 95\% CI stable). Total end-to-end response ~2.2 seconds, within 4-6 second target. Zero constraint violations across all 150 runs.

\chapter{Methodology and Experimental Design}

This chapter describes the validation methodology: software-in-the-loop simulation with deterministic failure injection, comparative evaluation against baseline strategies, and statistical validation (N=30 runs per scenario).

\section{Simulation Environment}

Real-world UAV failures are dangerous and difficult to reproduce. This research employs software-in-the-loop simulation (SOARES et al., 2025) with physically-grounded dynamics and deterministic failure injection.

The framework implements six-degree-of-freedom quaternion dynamics with cascaded PID control. Vehicle parameters: 1.5 kg mass, 100 Wh battery, 0.5--1.0 kg payload capacity (Quad SimCon reference, adapted for multi-vehicle coordination). Communication uses TCP/IP with JSON-RPC 2.0; GCS (port 5555) aggregates state at 2 Hz; Flask dashboard (port 8085) provides WebSocket visualization.

The implementation distributes functionality across purpose-specific modules:
\begin{itemize}
    \item \texttt{gcs/ooda\_engine.py} --- phase timing
    \item \texttt{gcs/fleet\_monitor.py} --- failure detection
    \item \texttt{gcs/constraint\_validator.py} --- safety margins
    \item \texttt{gcs/mission\_manager.py} --- task tracking
    \item \texttt{gcs/objective\_function.py} --- optimization
    \item \texttt{uav/simulation.py} --- dynamics
    \item \texttt{uav/client.py} --- communication
\end{itemize}

\noindent The complete source code is publicly available to enable replication and extension of this research (EULÁLIO REIS, 2025).

\section{Validation Metrics}

Four primary metrics anchor the validation framework.

\textbf{Coverage Recovery} measures orphaned tasks successfully reassigned:

\begin{equation}
\rho = \frac{|\mathcal{T}_{\text{reallocated}}|}{|\mathcal{T}_{\text{failed}}|} \times 100\%
\label{eq:coverage}
\end{equation}

Target: $\rho > 65\%$ for single-UAV failures, $\rho > 50\%$ for double failures.

\textbf{Adaptation Time} captures end-to-end responsiveness:

\begin{equation}
T_{\text{adapt}} = T_{\text{ACT\_complete}} - T_{\text{failure\_detected}}
\label{eq:adaptation}
\end{equation}

Target: $T_{\text{adapt}} < 5$ seconds.

\textbf{Battery Efficiency} evaluates fleet capacity utilization:

\begin{equation}
\eta_{\text{battery}} = \frac{\sum_{u \in \mathcal{U}_{\text{healthy}}} \Delta B_u}{\sum_{u \in \mathcal{U}_{\text{healthy}}} B_{\text{spare},u}} \times 100\%
\label{eq:efficiency}
\end{equation}

Efficiency below 80\% suggests conservative allocation; approaching 100\% indicates near-capacity operation.

\textbf{Mission Completion Rate}:

\begin{equation}
\xi = \frac{N_{\text{completed}}}{N_{\text{total}}} \times 100\%
\label{eq:completion}
\end{equation}

Target: $\xi > 90\%$ for single-failure scenarios. Secondary metrics: escalation frequency, bandwidth consumption, decision quality versus offline optimal.

\section{Test Scenarios}

The test suite comprises 169 automated cases (~0.22 seconds via pytest).

\textbf{Unit tests} (53 cases): battery management, grid boundaries, state machine transitions, constraint validation, priority scoring against hand-calculated values.

\textbf{Integration tests} (81 cases): surveillance coverage with rotation, SAR with golden hour enforcement, delivery with payload tracking, multi-UAV coordination, complete OODA cycle execution.

\textbf{Regression tests} (15 cases): edge cases, race conditions, numerical precision issues.

Five experimental scenarios (Chapter~7): \textbf{S5} (surveillance battery depletion), \textbf{R5/R6} (SAR GPS loss, R6 with out-of-grid permission), \textbf{D6/D7} (delivery payload violations and out-of-grid destinations).

\section{Baseline Comparisons}

Three baseline strategies:

\textbf{No Adaptation}: fixed assignments, permanent task loss (0\% recovery).

\textbf{Greedy Nearest-Neighbor}: assign to nearest UAV without constraint verification. May achieve 100\% coverage but risks safety violations.

\textbf{Hybrid OODA} (this work): priority-based allocation with constraint validation. Key differentiator is safety: refuses infeasible allocations and escalates appropriately.

\section{Visualization and Logging}

The web dashboard renders fleet state via Flask/WebSocket: UAV positions, battery bars, OODA timeline, and alert log.

\chapter{Experimental Results and Validation}

\section{Quantitative Performance Results}

The system was validated through five scenarios comparing OODA against three baselines. Results exceed initial expectations.

\subsection{Executive Summary}

Table~\ref{tab:executive_summary} provides a comprehensive overview of all experimental scenarios, enabling quick comparison across mission types. Table~\ref{tab:targets_achieved} summarizes achievement against design targets, confirming that all performance objectives were met or exceeded. The decision pathways underlying these outcomes are visualized in Figure~\ref{fig:ch6_decision_flow}, which traces the OODA cycle through each scenario type, while Figure~\ref{fig:ch6_radar} presents a multi-dimensional performance comparison across coverage, speed, safety, and autonomy metrics.

\begin{table}[H]
\centering
\caption{Executive Summary of Experimental Results}
\label{tab:executive_summary}
\small
\begin{tabular}{|l|c|c|c|c|c|}
\hline
\textbf{Scenario} & \textbf{Mission} & \textbf{Coverage} & \textbf{Time} & \textbf{Safety} & \textbf{Outcome} \\
 & \textbf{Type} & \textbf{Recovery} & \textbf{(ms)} & \textbf{Violations} & \\
\hline
\textbf{S5} & Surveillance & 100\% & 0.34 & 0 & Full Autonomous \\
\hline
\textbf{R5} & SAR & 100\% & 0.50 & 0 & Full Autonomous \\
\hline
\textbf{R6} & SAR (OOG) & 100\% & 0.16 & 0 & Full Autonomous \\
\hline
\textbf{D6} & Delivery & 0\% (esc.) & 0.11 & 0 & Intelligent Escalation \\
\hline
\textbf{D7} & Delivery & 0\% (esc.) & 0.23 & 0 & Intelligent Escalation \\
\hline
\multicolumn{6}{|l|}{\textit{OOG = Out-of-Grid; esc. = escalated to operator}} \\
\hline
\end{tabular}
\end{table}

\begin{table}[H]
\centering
\caption{Target Achievement Summary}
\label{tab:targets_achieved}
\begin{tabular}{|p{5cm}|c|c|c|}
\hline
\textbf{Metric} & \textbf{Target} & \textbf{Achieved} & \textbf{Improvement} \\
\hline
OODA Computation Time & --- & 0.10--0.88 ms (N=30) & Sub-ms achieved \\
\hline
End-to-End Response (projected) & <6000 ms & $\sim$2200 ms & 2.7$\times$ faster \\
\hline
Coverage Recovery (S5, R5/R6) & >75\% & 100\% (N=30) & Exceeded by 25\% \\
\hline
Constraint Violations & 0 & 0 (N=150) & Perfect \\
\hline
Golden Hour Impact (SAR) & Minimize & 0.00003\% & Negligible \\
\hline
Safe Escalation Rate & >90\% & 100\% & Perfect \\
\hline
\end{tabular}
\end{table}

\begin{figure}[H]
\centering
\includegraphics[width=0.9\textwidth]{images/chapter6_decision_flow.png}
\caption{OODA Decision Flow in Experimental Scenarios}
\label{fig:ch6_decision_flow}
\end{figure}

\begin{figure}[H]
\centering
\includegraphics[width=0.9\textwidth]{images/chapter6_performance_radar.png}
\caption{Multi-Dimensional Performance Comparison}
\label{fig:ch6_radar}
\end{figure}

\subsection{Detailed Performance Metrics}

Measured OODA timings (Figure~\ref{fig:ch6_time}) exceed design expectations: S5 completed in 0.7 ms (7,000$\times$ faster than 5-second target), R5 in 1.2 ms, R6 in 0.3 ms, D6/D7 in 0.2--0.4 ms. Constraint checking imposes negligible overhead.

Coverage recovery: surveillance and SAR achieved 100\% task recovery, exceeding 75--95\% target. Delivery scenarios achieved 0\% autonomous coverage with intelligent escalation---correct behavior when physical constraints preclude safe reallocation.

Computational performance (0.2--1.2 ms) exceeds the 4--6 second target by four orders of magnitude, ensuring computation never becomes the bottleneck. For SAR under golden-hour constraints, sub-millisecond recovery preserves maximum search time.

Safety validation (Figure~\ref{fig:ch6_safety}): Table~\ref{tab:coverage_matrix} shows OODA and greedy both achieving 100\% coverage for S5/R5/R6, but OODA correctly escalates D6/D7 while greedy claims unsafe 100\%. Table~\ref{tab:safety_violations} details the greedy approach's two constraint violations: one payload overload (D6) and one boundary transgression (D7). OODA maintained zero violations across all scenarios while achieving equivalent speed.

\begin{table}[H]
\centering
\caption{Coverage Recovery Matrix (\%)}
\label{tab:coverage_matrix}
\begin{tabular}{|l|c|c|c|c|}
\hline
\textbf{Approach} & \textbf{S5} & \textbf{R5} & \textbf{R6} & \textbf{D6/D7} \\
 & \textbf{Surveillance} & \textbf{SAR} & \textbf{SAR-OOG} & \textbf{Delivery} \\
\hline
\textbf{OODA} & \textbf{100} & \textbf{100} & \textbf{100} & \textbf{0 (esc.)}* \\
\hline
Greedy & 100 & 100 & 100 & 100\textsuperscript{\textdagger} \\
\hline
No Adaptation & 88.9 & 77.8 & 88.9 & 80 \\
\hline
\multicolumn{5}{|l|}{* Intelligent escalation is correct behavior; \textsuperscript{\textdagger} Unsafe (constraint violations)} \\
\hline
\end{tabular}
\end{table}

\begin{table}[H]
\centering
\caption{Constraint Violations by Approach}
\label{tab:safety_violations}
\begin{tabular}{|l|c|c|c|c|}
\hline
\textbf{Approach} & \textbf{Battery} & \textbf{Payload} & \textbf{Boundary} & \textbf{Total} \\
\hline
\textbf{OODA} & \textbf{0} & \textbf{0} & \textbf{0} & \textbf{0} \\
\hline
Greedy & 0 & 1 (D6) & 1 (D7) & \textbf{2} \\
\hline
No Adaptation & 0 & 0 & 0 & 0 \\
\hline
\end{tabular}
\end{table}

\begin{figure}[H]
\centering
\includegraphics[width=0.9\textwidth]{images/chapter6_time_comparison.png}
\caption{Computation Time Comparison Across Approaches (Log Scale)}
\label{fig:ch6_time}
\end{figure}

\begin{figure}[H]
\centering
\includegraphics[width=0.85\textwidth]{images/chapter6_safety_violations.png}
\caption{Safety Validation: Constraint Violations by Approach}
\label{fig:ch6_safety}
\end{figure}

\section{Experimental Scenario Analysis}

\subsection{S5: Surveillance Mission Recovery}

S5 compared three strategies: no-adaptation achieved 88.9\% coverage; greedy recovered 100\% in 1.5 ms (succeeding only because constraints weren't violated); OODA achieved 100\% in 0.7 ms with verified constraint satisfaction.

The OODA approach guaranteed safety through sequential battery, payload, and collision verification. All reserves remained above 20\%, spatial separation exceeded 15 m. Figure~\ref{fig:surveillance_trajectory} visualizes pre/post-failure trajectories.

\begin{figure}[H]
\centering
\includegraphics[width=0.95\textwidth]{images/surveillance_dashboard.png}
\caption{Surveillance Mission Dashboard (S5)}
\label{fig:surveillance_trajectory}
\end{figure}

\subsection{R5 \& R6: Search \& Rescue with Time Criticality}

R5 and R6 examined performance under golden hour constraints. R5: no-adaptation abandoned 22.2\% of search area; OODA recovered 100\% in 1.2 ms, consuming 0.00003\% of golden hour (Figure~\ref{fig:ch6_golden_hour}).

R6 positioned one zone 10 m outside the 120m $\times$ 120m grid, testing regulatory compliance. UAV-4 (with out-of-grid authorization) correctly received the exterior zone; unauthorized units excluded. Decision completed in 0.3 ms---fastest across all scenarios. Figure~\ref{fig:sar_trajectory} depicts the zone-based search pattern.

\begin{figure}[H]
\centering
\includegraphics[width=0.95\textwidth]{images/sar_dashboard.png}
\caption{Search and Rescue Mission Dashboard (R5/R6)}
\label{fig:sar_trajectory}
\end{figure}

\begin{figure}[H]
\centering
\includegraphics[width=\textwidth]{images/chapter6_golden_hour.png}
\caption{Search \& Rescue: Golden Hour Time Consumption}
\label{fig:ch6_golden_hour}
\end{figure}

\subsection{D6 \& D7: Delivery with Intelligent Escalation}

D6 and D7 validated the system's capacity to recognize limitations and defer to human judgment. D6 configured deliberate infeasibility: Package B weighed 0.4 kg while maximum spare payload capacity reached only 0.15 kg---a physical impossibility illustrated in Figure~\ref{fig:ch6_constraint}.

The greedy algorithm reported 100\% coverage by assigning the package regardless, overloading the UAV by 3$\times$ its spare capacity. The OODA system identified the infeasibility in 0.2 ms and escalated: "Package B (0.4 kg) exceeds maximum available spare capacity (0.15 kg). Recommend backup UAV deployment or ground vehicle dispatch."

D7 positioned a delivery destination at (150, 100) m---outside the authorized 120m $\times$ 120m grid---with no UAV carrying out-of-bounds authorization (unlike R6). The greedy algorithm violated the boundary; OODA refused in 0.4 ms, escalating with regulatory justification.

These results validate a key principle: intelligent refusal represents successful operation. Zero autonomous reallocation in infeasible scenarios demonstrates correct constraint detection, not algorithmic inadequacy. A system claiming 100\% autonomy while violating safety constraints provides less value than one achieving lower autonomous coverage while correctly identifying when human judgment is necessary. Figure~\ref{fig:delivery_trajectory} illustrates this escalation behavior.

\begin{figure}[H]
\centering
\includegraphics[width=0.95\textwidth]{images/delivery_dashboard.png}
\caption{Delivery Mission Dashboard (D6/D7)}
\label{fig:delivery_trajectory}
\end{figure}

Figure~\ref{fig:delivery_trajectory} shows delivery route visualization: UAV-2 approaches the drop-off; UAV-3 proceeds to an out-of-bounds position after operator authorization; UAV-4 triggered escalation at the operational boundary. The dashboard displays fleet status with active operator alerts.

\begin{figure}[H]
\centering
\includegraphics[width=0.9\textwidth]{images/chapter6_constraint_space.png}
\caption{D6 Payload Constraint Violation: Geometric Illustration of Escalation Necessity}
\label{fig:ch6_constraint}
\end{figure}

\begin{figure}[H]
\centering
\includegraphics[width=0.9\textwidth]{images/chapter6_coverage_heatmap.png}
\caption{Coverage Recovery Matrix Across All Approaches and Scenarios}
\label{fig:ch6_coverage}
\end{figure}

\section{Synthesis and Contributions}

Table~\ref{tab:scenario_comparison} synthesizes performance characteristics across all experimental scenarios, with Figure~\ref{fig:ch6_coverage} providing a visual coverage recovery matrix across all approaches and scenarios.

\begin{table}[H]
\centering
\small
\caption{Statistical Performance by Scenario (N=30 runs each)}
\label{tab:scenario_comparison}
\begin{tabular}{|l|c|c|c|}
\hline
\textbf{Metric} & \textbf{Surveillance (S5)} & \textbf{SAR (R5/R6)} & \textbf{Delivery (D6/D7)} \\
\hline
Coverage Recovery & \textbf{100\%} & \textbf{100\%} & \textbf{0\% (escalated)} \\
\hline
OODA Time R5/D6 (mean $\pm$ std) & \textbf{0.42 $\pm$ 0.15 ms} & \textbf{0.88 $\pm$ 0.56 ms} & \textbf{0.18 $\pm$ 0.21 ms} \\
\hline
OODA Time R6/D7 (mean $\pm$ std) & --- & \textbf{0.24 $\pm$ 0.15 ms} & \textbf{0.10 $\pm$ 0.03 ms} \\
\hline
Operator Escalation & 0\% (autonomous) & 0\% (autonomous) & 100\% (correct) \\
\hline
Constraint Violations & \textbf{0} (all runs) & \textbf{0} (all runs) & \textbf{0} (all runs) \\
\hline
\end{tabular}
\end{table}

\begin{table}[H]
\centering
\caption{Comprehensive Approach Comparison}
\label{tab:approach_comparison}
\small
\begin{tabular}{|p{3.5cm}|p{3.5cm}|p{3.5cm}|p{3.5cm}|}
\hline
\textbf{Criterion} & \textbf{OODA} & \textbf{Greedy} & \textbf{No Adapt} \\
\hline
\textbf{Coverage} & 100\% (S5, R5/R6), Escalate (D6/D7) & 100\% all scenarios & 77.8--88.9\% \\
\hline
\textbf{Speed} & 0.2--1.2 ms & 0.1--1.5 ms & 0 ms \\
\hline
\textbf{Safety} & \textbf{0 violations} & \textbf{2 violations} & 0 violations \\
\hline
\textbf{Constraint Awareness} & \textbf{Full} & None & N/A \\
\hline
\textbf{Deployability} & \textbf{YES} & \textbf{NO (unsafe)} & NO (degrades) \\
\hline
\textbf{Regulatory Compliance} & YES & NO & NO \\
\hline
\textbf{Scalability} & Good & Good & N/A \\
\hline
\end{tabular}
\end{table}

\begin{table}[H]
\centering
\caption{Speed vs. Safety vs. Coverage Trade-offs}
\label{tab:tradeoffs}
\begin{tabular}{|l|c|c|c|c|}
\hline
\textbf{Approach} & \textbf{Speed} & \textbf{Safety} & \textbf{Coverage} & \textbf{Overall Score} \\
 & \textbf{(0--10)} & \textbf{(0--10)} & \textbf{(0--10)} & \textbf{(0--30)} \\
\hline
\textbf{OODA} & 10 & \textbf{10} & 8 & \textbf{28} \\
\hline
Greedy & 10 & \textbf{2} & 10 & 22 \\
\hline
No Adaptation & 10 & 10 & \textbf{3} & 23 \\
\hline
\multicolumn{5}{|l|}{\textit{Speed: 10 = <10ms; Safety: 10 = zero violations; Coverage: 10 = 100\% recovery}} \\
\hline
\end{tabular}
\end{table}

Table~\ref{tab:approach_comparison} compares approaches across seven criteria: OODA achieves full constraint awareness and regulatory compliance with zero violations, greedy produces two violations rendering it non-deployable, and no-adaptation degrades to 77.8--88.9\% coverage. Table~\ref{tab:tradeoffs} quantifies trade-offs using normalized 0--10 scores: OODA scores 28/30 (10 speed, 10 safety, 8 coverage), greedy scores 22/30 (penalized for safety violations), no-adaptation scores 23/30 (penalized for coverage loss). Four patterns emerge: (1) computational speed uniformly under 1.5 ms validates greedy heuristics over unpredictable optimization; (2) zero constraint violations despite varying failure conditions; (3) selective autonomy---aggressive adaptation when feasible, intelligent refusal when not; (4) sub-millisecond responses ensure fault recovery never bottlenecks missions.

\subsection{Claims Validated}

The experimental program validated all primary thesis claims:

\begin{itemize}
\item \textbf{Rapid adaptation:} Statistical validation (N=30) confirms mean computation times of 0.10--0.88 ms for OODA cycles, ensuring algorithmic processing contributes negligibly to the 2-3 second projected end-to-end response (well within the 4-6 second design target)
\item \textbf{Safety guarantee:} Zero constraint violations across all 150 experimental runs (battery, payload, spatial boundaries)
\item \textbf{Intelligent escalation:} D6 and D7 correctly refused infeasible reallocations rather than producing unsafe plans
\item \textbf{Coverage recovery:} 100\% task recovery in surveillance and SAR missions, matching greedy baseline while maintaining safety
\end{itemize}

\subsection{Research Contributions}

The work advances three contributions: (1) a \textbf{unified constraint verification framework} enforcing battery reserves, payload limits, and temporal deadlines through fail-fast sequential checking---the first multi-UAV fault tolerance system addressing all three categories; (2) a \textbf{quantified degradation framework} with explicit decision rules for when autonomous reallocation becomes infeasible; (3) a \textbf{hybrid autonomy architecture} balancing machine speed with human oversight for BVLOS regulatory compliance.

Practical deliverables include commercial platform compatibility, autonomous recovery in three scenarios (S5, R5, R6) with appropriate escalation in two, and a comprehensive test suite (169 tests, 0.22s). The open-source simulation platform enables future research without physical hardware.

The core insight: realism-first design---modeling constraints explicitly---produces systems both academically rigorous and operationally deployable.

\chapter{Limitations and Future Work}

\section{Architectural Constraints}

\subsection{Centralized Control Architecture}

The centralized GCS creates a single point of failure. Upon GCS timeout, UAVs implement autonomous Return-to-Launch protocols, preserving vehicle safety while sacrificing mission completion. Distributed OODA architectures using consensus algorithms could eliminate this vulnerability while introducing communication complexity and Byzantine fault tolerance requirements.

\subsection{Fleet Scalability}

The system supports 3--12 UAVs. Beyond this, communication bandwidth scales linearly (48 KB/s for 12 UAVs at 2 Hz) while collision avoidance complexity scales quadratically. At 20 UAVs, collision checking increases 2.8-fold, potentially exceeding the 6-second OODA target. Hierarchical architectures or spatial hashing could enable coordination of larger fleets.

\subsection{Validation Approach and Limitations}

\subsubsection{Experimental Methodology}

Validation employs software-in-the-loop simulation with deterministic failure injection:

\begin{enumerate}
\item \textbf{Scenario Definition:} Mission type, fleet size, task distribution, and failure events with timing.

\item \textbf{Baseline Comparison:} Three strategies under identical conditions: (a) \textit{No Adaptation}---coverage loss baseline; (b) \textit{Greedy Nearest}---no pre-validation; (c) \textit{OODA}---constraint-aware pre-allocation.

\item \textbf{Statistical Validation:} N=30 runs per scenario with mean $\pm$ std and 95\% confidence intervals.

\item \textbf{Metrics Collection:} Coverage recovery, OODA time, constraint violations, and escalation events.
\end{enumerate}

\subsubsection{Simulation Limitations}

Current validation relies on software-in-the-loop simulation with physics-based quadrotor models. While appropriate for algorithm development, simulation abstracts GPS multipath, packet corruption, sensor noise, and environmental disturbances. Hardware-in-the-loop testing and field trials would establish operational reliability bounds.

\section{Algorithmic Limitations}

\subsection{Greedy Task Reallocation}

The greedy heuristic assigns tasks to the nearest constraint-satisfying UAV, achieving rapid execution but not globally optimal allocation. Mixed-integer linear programming could guarantee optimality, though solution times may exceed OODA budgets. Auction-based algorithms (Consensus-Based Bundle Algorithm) offer near-optimal solutions with polynomial complexity.

\subsection{Simplified Collision Avoidance}

The 15-meter spatial separation with temporal deconfliction suffices for sparse densities but exhibits limitations in dense patterns: the fixed buffer ignores relative velocities, and pairwise verification scales poorly. Velocity obstacle approaches and model predictive control could extend applicability to urban air mobility.

\section{Application Scope}

The system addresses surveillance, SAR, and delivery---missions sharing waypoint navigation, quantifiable priorities, and tolerance for degradation. Excluded are formation flying (requires tighter coordination than 2 Hz telemetry), adversarial scenarios (requires game-theoretic reasoning), and time-critical interception (requires trajectory optimization beyond waypoint following). Extensions could involve augmented OODA loops or game-theoretic allocation.

\section{Future Research Directions}

\textbf{Near-term:} Validation across all 27 scenarios, battery reserve sensitivity analysis (5--20\%), turbulence modeling.

\textbf{Medium-term:} Hardware-in-the-loop via PX4 SITL, field demonstrations exposing GPS/RF interference, distributed OODA with consensus decision-making, formal verification via model checking.

\textbf{Long-term:} Machine learning for priority optimization (neural battery prediction, RL for OODA tuning), game-theoretic adversarial analysis, fleet heterogeneity, and swarm coordination (50+ vehicles) requiring hierarchical architectures.

\section{Concluding Perspective}

These limitations reflect design choices prioritizing deployability over theoretical completeness. Centralized architecture enables regulatory compliance; greedy allocation trades optimality for real-time responsiveness; constraint awareness acknowledges physical limitations.

A system achieving 65--95\% autonomous recovery within realistic bounds provides more value than theoretical frameworks promising perfection under idealized assumptions. As multi-agent coordination matures, the community must prioritize systems bridging laboratory demonstration and operational deployment through explicit constraint modeling and hybrid human-machine architectures. This work demonstrates that constraint-aware, operator-supervised fault tolerance is the appropriate architecture for near-term UAV deployments in regulated, safety-critical applications.

\chapter{Conclusion}

This research addresses the gap between theoretical fault tolerance and deployable UAV systems through a constraint-aware multi-agent coordination framework. By explicitly modeling energy, payload, and regulatory constraints, the hybrid OODA architecture provides quantified mission assistance within realistic bounds, advancing beyond idealized assumptions in existing literature.

\section{Summary of Contributions}

The research objectives from Chapter 1 have been systematically addressed:

\textbf{SO1 (Hybrid OODA-FSM Architecture):} Centralized OODA Engine with deterministic FSM enables verifiable state transitions and probabilistic failure identification, maintaining BVLOS compliance through regulatory boundary enforcement.

\textbf{SO2 (Multi-Modal Failure Detection):} 2 Hz telemetry validated across communication timeout (1.5s), fault code recognition, and anomaly detection (battery discharge >5\%/30s, position discontinuities >100m, altitude violations).

\textbf{SO3 (Constraint-Aware Task Reallocation):} Priority-based allocation jointly optimizes energy reserves (20\% margin), payload limits, deadlines, and collision avoidance (15m). Fail-fast verification ensures realizable plans; infeasible reallocations escalate to operators.

\textbf{SO4 (Operator Escalation Framework):} Graduated escalation validated: operator involvement below 50\% recovery, partial reallocation at 50--75\%, autonomous above 75\%.

\textbf{SO5 (Statistical Validation):} N=30 trials per condition across mission types. Complete recovery for surveillance and SAR; appropriate escalation for delivery.

\textbf{SO6 (Real-Time Performance):} Mean OODA computation 0.10--0.88 ms with 95\% CI confirming sub-millisecond performance. End-to-end ~2.2s satisfies 6-second target.

\textbf{SO7 (Safety Constraint Satisfaction):} Zero violations across 150 runs and 169 tests, validating constraint maintenance without compromising coverage.

\section{Principal Findings}

The central insight: \textit{selective autonomy}---aggressive adaptation when feasible, intelligent refusal when constraints preclude safety---demonstrates superior practical value over unconstrained systems risking violations. Sub-millisecond recovery ensures adaptation overhead remains negligible in golden-hour SAR scenarios.

The three contributions---resource-aware reallocation, priority-based partial coverage, and graduated escalation---synergistically balance autonomous response with human oversight for regulatory compliance.

\section{Limitations and Threats to Validity}

Findings must be interpreted within methodological constraints. The centralized GCS introduces single-point-of-failure vulnerability; fleet coordination ceases during GCS failure (UAVs implement Return-to-Launch). Greedy allocation prioritizes responsiveness over global optimality.

Validation remains software-in-the-loop without field exposure to GPS multipath, RF interference, turbulence, and sensor noise. Hardware-in-the-loop and field trials are essential prior to deployment.

\section{Future Research Directions}

Research trajectories: (i) distributed OODA eliminating single-point failures, (ii) ML for predictive failure detection, (iii) formal safety verification, (iv) field validation under realistic conditions.

\section{Concluding Remarks}

This research demonstrates that constraint-aware, operator-supervised fault tolerance constitutes a viable paradigm for near-term UAV fleet deployments. The contribution lies not in claiming unrestricted autonomy, but in designing hybrid human-machine systems acknowledging physical and regulatory limitations while maximizing autonomous capability within those bounds---bridging theoretical research and practical deployment toward dependable autonomous aerial systems.

% ====================================
% REFERENCES
% ====================================
\newpage
\addcontentsline{toc}{chapter}{References}
\begin{thebibliography}{99}

\bibitem{mueller2014} 
Mueller, M. W., \& D'Andrea, R. (2014). Stability and control of a quadrocopter despite the complete loss of one, two, or three propellers. \textit{IEEE ICRA}.


\bibitem{sun2022}
Sun, Z., et al. (2022). Fault-Tolerant Model Predictive Control of a Quadrotor with an Unknown Complete Rotor Failure. \textit{IEEE ICRA}.

\bibitem{li2017}
Li, P., Yu, X., Peng, X., Zheng, Z., \& Zhang, Y. (2017). Fault-tolerant cooperative control for multiple UAVs based on sliding mode techniques. \textit{Science China Information Sciences}, 60(7).


\bibitem{yang2011}
Yang, H., Staroswiecki, M., Jiang, B., et al. (2011). Fault tolerant cooperative control for a class of nonlinear multi-agent systems. \textit{Systems \& Control Letters}, 60(4), 271-277.

\bibitem{gerkey2004}
Gerkey, B. P., \& Mataric, M. J. (2004). A formal analysis and taxonomy of task allocation in multi-robot systems. \textit{International Journal of Robotics Research}, 23(9), 939-954.

\bibitem{choi2009}
Choi, H. L., Brunet, L., \& How, J. P. (2009). Consensus-based decentralized auctions for robust task allocation. \textit{IEEE Transactions on Robotics}, 25(4), 912-926.

\bibitem{dias2006}
Dias, M. B., Zlot, R., Kalra, N., \& Stentz, A. (2006). Market-based multirobot coordination: A survey and analysis. \textit{Proceedings of the IEEE}, 94(7), 1257-1270.

\bibitem{zlot2006}
Zlot, R., \& Stentz, A. (2006). Market-based multirobot coordination for complex tasks. \textit{International Journal of Robotics Research}, 25(1), 73-101.

\bibitem{cortes2004}
Cortes, J., Martinez, S., Karatas, T., \& Bullo, F. (2004). Coverage control for mobile sensing networks. \textit{IEEE Transactions on Robotics and Automation}, 20(2), 243-255.

\bibitem{schwager2009}
Schwager, M., Rus, D., \& Slotine, J. J. (2009). Decentralized, adaptive coverage control for networked robots. \textit{International Journal of Robotics Research}, 28(3), 357-375.

\bibitem{elmaliach2009}
Elmaliach, Y., Agmon, N., \& Kaminka, G. A. (2009). Multi-robot area patrol under frequency constraints. \textit{Annals of Mathematics and Artificial Intelligence}, 57(3-4), 293-320.

\bibitem{abdessameud2011}
Abdessameud, A., \& Tayebi, A. (2011). Formation control of VTOL unmanned aerial vehicles with communication delays. \textit{Automatica}, 47(11), 2383-2394.

\bibitem{izadi2009}
Izadi, H. A., Gordon, B. W., \& Zhang, Y. M. (2009). Decentralized receding horizon control for cooperative multiple vehicles subject to communication delay. \textit{Journal of Guidance, Control, and Dynamics}, 32(6), 1959-1965.

\bibitem{izadi2013}
Izadi, H. A., Gordon, B. W., \& Zhang, Y. M. (2013). Hierarchical decentralized receding horizon control of multiple vehicles with communication failures. \textit{IEEE Transactions on Aerospace and Electronic Systems}, 49(2), 744-759.

\bibitem{beard2006}
Beard, R. W., McLain, T. W., Nelson, D. B., et al. (2006). Decentralized cooperative aerial surveillance using fixed-wing miniature UAVs. \textit{Proceedings of the IEEE}, 94(7), 1306-1324.

\bibitem{boyd1987}
Boyd, J. R. (1987). \textit{A Discourse on Winning and Losing}. [OODA Loop framework]

\bibitem{bala2025}
Bala, M., et al. (2025). The OODA Loop of Cloudlet-Based Autonomous Drones. \textit{IEEE/ACM Symposium on Edge Computing (SEC)}.

\bibitem{soares2025}
Soares, V. M. D., et al. (2025). UAV Simulation Environment for Fault Detection in Wind Farm Electrical Distribution Systems. \textit{IEEE Conference Proceedings}.

\bibitem{vandenberg2008}
van den Berg, J., Lin, M., \& Manocha, D. (2008). Reciprocal velocity obstacles for real-time multi-agent navigation. \textit{IEEE ICRA}, 1928-1935.

\bibitem{zhang2008}
Zhang, Y. M., \& Jiang, J. (2008). Bibliographical review on reconfigurable fault-tolerant control systems. \textit{Annual Reviews in Control}, 32(2), 229-252.

\bibitem{yu2015}
Yu, X., \& Jiang, J. (2015). A survey of fault-tolerant controllers based on safety-related issues. \textit{Annual Reviews in Control}, 39, 46-57.

\bibitem{parker1998}
Parker, L. E. (1998). ALLIANCE: An architecture for fault tolerant multirobot cooperation. \textit{IEEE Transactions on Robotics and Automation}, 14(2), 220-240.

\bibitem{repo}
Eulálio Reis, V. (2025). \textit{Multi-UAV OODA System: Constraint-Aware Fault-Tolerant Multi-Agent Coordination}. GitHub repository. \url{https://github.com/vriez/multi_uav_ooda_system}

\bibitem{quadref}
QUAD SIMCON. (2020). \textit{Quadcopter Simulation and Control}. GitHub repository. \url{https://github.com/bobzwik/Quadcopter_SimCon}

% \bibitem{repo}
% Project Repository: \url{https://github.com/vriez/multi_uav_ooda_system}
% \bibitem{guo2018}
% Guo, J., Zhang, Y., \& Li, W. (2018). Fault-tolerant control of quadrotor UAVs with actuator faults using adaptive backstepping. \textit{International Journal of Control, Automation and Systems}, 16(4), 1572-1583.

% \bibitem{wang2021}
% Wang, W., Zhang, Y., \& Xu, B. (2021). Fault and Failure Tolerant Model Predictive Control of Quadrotor UAV. \textit{IEEE ROBIO}.

% \bibitem{zhou2021}
% Zhou, B., Su, W., \& Han, J. (2021). Model predictive fault-tolerant control for quadrotor UAV subject to actuator faults. \textit{Aerospace Science and Technology}, 110, e106497.

% \bibitem{chang2024}
% Chang, Y., et al. (2024). Reinforcement Learning–Based Adaptive Fault-Tolerant Antidisturbance Control for UAVs. \textit{Journal of Aerospace Engineering}, 38(1).

% \bibitem{zhang2016}
% Zhang, X. Y., \& Duan, H. B. (2016). Altitude consensus based 3D flocking control for fixed-wing unmanned aerial vehicle swarm trajectory tracking. \textit{Journal of Aerospace Engineering}, 230(14), 2628-2638.

% \bibitem{liu2016}
% Liu, Z. X., Yuan, C., Yu, X., et al. (2016). Leader-follower formation control of unmanned aerial vehicles in the presence of obstacles and actuator faults. \textit{Unmanned Systems}, 4(3), 197-211.

% \bibitem{yu2016}
% Yu, X., Liu, Z. X., \& Zhang, Y. M. (2016). Fault-tolerant formation control of multiple UAVs in the presence of actuator faults. \textit{International Journal of Robust and Nonlinear Control}, 26(12), 2668-2685.

% \bibitem{korsah2013}
% Korsah, G. A., Stentz, A., \& Dias, M. B. (2013). A comprehensive taxonomy for multi-robot task allocation. \textit{International Journal of Robotics Research}, 32(12), 1495-1512.

% \bibitem{innocenti2004}
% Innocenti, M., Pollini, L., \& Giulietti, F. (2004). Management of communication failures in formation flight. \textit{Journal of Aerospace Computing, Information, and Communication}, 1(1), 19-35.

% \bibitem{franco2007}
% Franco, E., Parisini, T., \& Polycarpou, M. M. (2007). Design and stability analysis of cooperative receding-horizon control of linear discrete-time agents. \textit{International Journal of Robust and Nonlinear Control}, 17(10-11), 982-1001.

% Flight Formation
% \bibitem{pachter2001}
% Pachter, M., D'Azzo, J. J., \& Proud, A. W. (2001). Tight formation flight control. \textit{Journal of Guidance, Control, and Dynamics}, 24(2), 246-254.

% \bibitem{gu2006}
% Gu, Y., Seanor, B., Campa, G., et al. (2006). Design and flight testing evaluation of formation control laws. \textit{IEEE Transactions on Control Systems Technology}, 14(6), 1105-1112.

% \bibitem{lin2014}
% Lin, W. (2014). Distributed UAV formation control using differential game approach. \textit{Aerospace Science and Technology}, 35, 54-62.


\end{thebibliography}

\end{document}